%add all your local new commands to this file

\newcommand{\smiley}{:)}

\renewbibmacro*{index:name}[5]{%
  \usebibmacro{index:entry}{#1}
    {\iffieldundef{usera}{}{\thefield{usera}\actualoperator}\mkbibindexname{#2}{#3}{#4}{#5}}}

\makeatletter
\def\blx@maxline{77}
\makeatother

\newcommand{\appref}[1]{Appendix \ref{#1}}
\newcommand{\fnref}[1]{Appendix \ref{#1}}
\newcommand{\regel}[1]{#1}
\newcommand{\vernacular}[1]{\emph{#1}}
\newcommand{\gloss}[1]{#1}
\newcommand{\stem}[1]{\textbackslash#1/}


%%%%%%%%%%%%%%%%%%%%%%%%%%%%%%%%%%%%%%%%%%%%%%%%%%%%
%%%                                              %%%
%%%                  My commands	             %%%
%%%                                              %%%
%%%%%%%%%%%%%%%%%%%%%%%%%%%%%%%%%%%%%%%%%%%%%%%%%%%% 
\newcommand{\Corpus}[2]{\hspace*{1pt}\hfill{\small\mbox{\hyperlink{#1}{\textnormal{\footnotesize{[#1 #2]}}}}}}%allows to create a link to the corpus overview table %copied and adapted from Joshua Wilbur's Pite Saami grammar.
\renewcommand{\eachwordone}{\upshape}%turns second line (morpheme boundaries) in gb4e examples back to roman font
\newleipzig{stemmo}{\textbackslash.../}{verbstem, e.g. y\textbackslash fath/wr (\S\ref{verbprelim}, \S\ref{roots-and-temp})}
\newleipzig{tatueo}{(.)}{speech pause}
\newleipzig{fullstap}{.}{multiword gloss, e.g. old.man, be.standing}
\newleipzig{alph}{$\alpha{}$}{alpha prefix series (\S\ref{personprefsection}, \S\ref{tamprefixseries}) }					%alpha
\newleipzig{bet}{$\beta{}$}{beta prefix series (\S\ref{personprefsection}, \S\ref{tamprefixseries})}						%beta
\newleipzig{gam}{$\gamma{}$}{gamma prefix series (\S\ref{personprefsection}, \S\ref{tamprefixseries})}					%gamma
\newleipzig{betaone}{$\beta{}$1}{beta 1 prefix series (\S\ref{personprefsection}, \S\ref{tamprefixseries})}		%beta1
\newleipzig{betatwo}{$\beta{}$2}{beta 2 prefix series (\S\ref{personprefsection}, \S\ref{tamprefixseries})}		%beta2
\newleipzig{a}{a}{agent (\S\ref{alignmtemplates})}                 		%agent-like argument of canonical transitive verb
\newleipzig{abl}{abl}{ablative case (\S\ref{ablativecase})}         		%ablative
\newleipzig{abs}{abs}{absolutive case (\S\ref{abscase})}      			%absolutive
\newleipzig{adj}{adj}{adjective}        		%adjective
\newleipzig{adlzr}{adjzr}{adjectivaliser (\S\ref{nominals-intro})}		%adjectivalizer
\newleipzig{adv}{adv}{adverbial}        		%adverb(ial)
\newleipzig{all}{all}{allative case (\S\ref{allativecase})}         		%allative
\newleipzig{iam}{alr}{iamitive (`already'), (\S\ref{tamparticles}, \S\ref{iamitivez})}     			%iamitive
\newleipzig{andat}{and}{andative (\S\ref{directionalinflection})}				%andative = go = away
\newleipzig{anim}{anim}{animate (\S\ref{formfunccase}, \S\ref{casenotes})}
\newleipzig{appr}{appr}{apprehensive (\S\ref{verbproclitics}, \S\ref{deicticcliticssection}, \S\ref{apprehensivem} )}			%apprehensive
\newleipzig{assoc}{assoc}{associative case (\S\ref{comcase}, \S\ref{inclusorycontruction})}			%associative
\newleipzig{bg}{bg}{backgrounded (\S\ref{durativesuffixm})}%backgrounding the event
\newleipzig{char}{char}{characteristic case (\S\ref{charcase})}			%source or characteristic
\newleipzig{cop}{cop}{copula (\S\ref{copclause})}           		%copula
\newleipzig{dat}{dat}{dative case (\S\ref{datcase})}         		%dative
\newleipzig{dem}{dem}{demonstrative (\S\ref{demonstratives})}    		%demonstrative
\newleipzig{dim}{dim}{diminutive (\S\ref{diminuitivefaeth})}				%diminutive
\newleipzig{distr}{distr}{distributive (\S\ref{distributivekak})}		%distributive
\newleipzig{dist}{dist}{distal demonstrative (\S\ref{demonstratives})}         		%distal
\newleipzig{dur}{dur}{durative (\S\ref{durativesuffixm})}     			%durative
\newleipzig{du}{du}{dual (\S\ref{numbersubsec})}               		%dual
\newleipzig{emph}{emph}{emphatic (\S\ref{emphathicwae})}				%empahatic
\newleipzig{erg}{erg}{ergative case (\S\ref{ergcase})}       		%ergative
\newleipzig{etc}{etc}{et cetera (`and all'), (\S\ref{etceterasue})}
\newleipzig{ext}{ext}{extended verb stem (\S\ref{verbprelim}, \S\ref{roots-and-temp})}		%extended verb root / aspectual
\newleipzig{futimp}{futimp}{future imperative (\S\ref{combitam}, \S\ref{durativesuffixm})}
\newleipzig{fut}{fut}{future (\S\ref{futurekwa}, \S\ref{TAMsemtense})}      				%future
\newleipzig{hab}{hab}{habitual (\S\ref{habitualnomai})}					%habitual
\newleipzig{immpst}{ipst}{immediate past (\S\ref{verbproclitics}, \S\ref{combitam}, \S\ref{imminentm}, \S\ref{TAMsemtense})}    %immediate past
\newleipzig{imm}{imm}{immediate demonstrative (\S\ref{immediate-dem})}				%immediate marker
\newleipzig{imn}{imn}{imminent (\S\ref{verbproclitics}, \S\ref{imminentm})}
\newleipzig{imp}{imp}{imperative (\S\ref{personsuffsection}, \S\ref{imperativesuffix}, \S\ref{TAMsemmood})}       		%imperative
\newleipzig{indf}{indf}{indefinite (\S\ref{indefnae})}     		%indefinite
\newleipzig{ins}{ins}{instrumental case (\S\ref{inscase})} 		%instrumental
\newleipzig{io}{io}{indirect object (\S\ref{alignmtemplates})}
\newleipzig{ipfv}{ipfv}{imperfective (\S\ref{combitam}, \S\ref{TAMsemaspect})}   		%imperfective
\newleipzig{irr}{irr}{irrealis (\S\ref{irrealisra}, \S\ref{TAMsemmood}, \S\ref{info-tam-event})}         		%irrealis
\newleipzig{iter}{iter}{iterative  (\S\ref{combitam})}         		%iterative
\newleipzig{lk}{lk}{linking consonant (\S\ref{personsuffsection})}	    	%linking element
\newleipzig{loc}{loc}{locative case (\S\ref{locativecase})}       		%locative
\newleipzig{lpl}{lpl}{large plural (\S\ref{positonalnumber})}				%largeplural
\newleipzig{masc}{masc}{masculine (\S\ref{wordclasses-thegendersystem}, \S\ref{gendersubsec})}          	%masculine
\newleipzig{med}{med}{medial demonstrative (\S\ref{demonstratives})}					%medial deictic / demonstrative
\newleipzig{m}{m}{middle (\S\ref{alignmtemplates}, \S\ref{middletemplatesubsection})}						%middle
\newleipzig{ndu}{nd}{non-dual (\S\ref{numbersubsec})}					%non-dual
\newleipzig{neg}{neg}{negator (\S\ref{particles}, \S\ref{negationclause})}         		%negation, negative
\newleipzig{nmlz}{nmlz}{nominaliser (\S\ref{verbs}, \S\ref{valencyalternations}, \S\ref{interclausintro})}    %nominalizer/nominalization
\newleipzig{nonpast}{npst}{non-past (\S\ref{combitam}, \S\ref{TAMsemtense})}
\newleipzig{npl}{npl}{non-plural (\S\ref{numbersubsec})}
\newleipzig{nsg}{nsg}{non-singular (\S\ref{numbersubsec})}				%non-singular
\newleipzig{obj}{obj}{object (\S\ref{alignmtemplates})}          		%object
\newleipzig{only}{only}{exclusive marker (`only', `just'), (\S\ref{exclusivenzo})}        			%only
\newleipzig{pfv}{pfv}{perfective (\S\ref{combitam}, \S\ref{TAMsemaspect})}       		%perfective
\newleipzig{pl}{pl}{plural (\S\ref{numbersubsec})}             		%plural
\newleipzig{poss}{poss}{possessive (\S\ref{possessivemarking})}     		%possessive
\newleipzig{pos}{pos}{positional verb stem (\S\ref{positionalverbs})}
\newleipzig{pot}{pot}{potential (\S\ref{verbproclitics}, \S\ref{potentialkma})}     			%potential
\newleipzig{priv}{priv}{privative case (\S\ref{privcase})}    		%privative case (without)
\newleipzig{prop}{prop}{proprietive case (\S\ref{propcase})}    		%proprietive case (having)
\newleipzig{prox}{prox}{proximal demonstrative (\S\ref{demonstratives})}       		%proximal/proximate
\newleipzig{pst}{pst}{past (\S\ref{combitam}, \S\ref{pastsuffixa}, \S\ref{TAMsemtense})}             		%past
\newleipzig{purp}{purp}{purposive case (\S\ref{purpcase})}      	%purposive
\newleipzig{quot}{quot}{quotative (\S\ref{manner-den-adv-nima}, \S\ref{directspeechthought})}      		%quotative
\newleipzig{recog}{recog}{recognitional pronoun (\S\ref{recognitional-pronoun})}%recognitional pronoun
\newleipzig{redup}{redup}{reduplication (\S\ref{nouns})}		%reduplication
\newleipzig{rpst}{rpst}{recent past (\S\ref{combitam}, \S\ref{TAMsemtense})}
\newleipzig{rs}{rs}{restricted verb stem (\S\ref{verbprelim}, \S\ref{roots-and-temp}, \S\ref{prerootdual})}		%restricted verb root/aspectual
\newleipzig{sbj}{sbj}{subject (\S\ref{alignmtemplates})}    				%subject
\newleipzig{sg}{sg}{singular (\S\ref{numbersubsec})}           		%singular
\newleipzig{simil}{simil}{similative (\S\ref{similcase})}			%similative
\newleipzig{single}{s}{single argument of an intransitive verb}    		%subjunctive
\newleipzig{stat}{stat}{stative (\S\ref{positionalverbs})}				%stative,positional
\newleipzig{temp}{temp}{temporal case (\S\ref{temporalcase})}				%temporal case
\newleipzig{u}{u}{undergoer (\S\ref{alignmtemplates})}					%undergoer
\newleipzig{vc}{vc}{valency change (\S\ref{morphologicaltemplates}, \S\ref{valencyalternations}, \S\ref{allomorphdualsuffix})}
\newleipzig{venit}{vent}{ventive (\S\ref{directionalinflection})}				%venitive = come = towards

\renewleipzig{first}{\liningnums{1}}{first person (\S\ref{personsection})}%
\renewleipzig{second}{\liningnums{2}}{second person (\S\ref{personsection})}%
\renewleipzig{third}{\liningnums{3}}{third person (\S\ref{personsection})}%
\renewleipzig{f}{fem}{feminine (\S\ref{wordclasses-thegendersystem}, \S\ref{gendersubsec})}           		%feminine

\newcommand{\Zero}{$\varnothing{}$}%
% \newcommand{\Fsg}{{\First}{\Sg}}%
% \newcommand{\Fdu}{{\First}{\Du}}%
	\newcommand{\Fnsg}{{\First}{\Nsg}}%
% \newcommand{\Fpl}{{\First}{\Pl}}%
% \newcommand{\Ssg}{{\Second}{\Sg}}%
% \newcommand{\Sdu}{{\Second}{\Du}}%
	\newcommand{\Snsg}{{\Second}{\Nsg}}%
% \newcommand{\Spl}{{\Second}{\Pl}}%
% \newcommand{\Tsg}{{\Third}{\Sg}}%
% \newcommand{\Tdu}{{\Third}{\Du}}%
	\newcommand{\Tnsg}{{\Third}{\Nsg}}%
% \newcommand{\Tpl}{{\Third}{\Pl}}%
	\newcommand{\Stsg}{${\Second}$\textbar${\Third}{\Sg}$}%
	\newcommand{\Stdu}{${\Second}$\textbar${\Third}{\Du}$}%
	\newcommand{\Stnsg}{${\Second}$\textbar${\Third}{\Nsg}$}%
	\newcommand{\Stpl}{${\Second}$\textbar${\Third}{\Pl}$}%
	
\newcommand{\ilit}[1]{#1\il{#1}}
\newcommand{\isit}[1]{#1\is{#1}}	

\renewcommand{\checkmark}{✔}


%  \newcommand{\ᵑ}{\textsf{{\small\hspace{.5pt}ᵑ}}}
 \newcommand{\ᵑ}{{\small\textsuperscript{ŋ}}}
 \newcommand{\fixgll}{\hspace*{2cm}}
 
 \let\eachwordthree=\footnotesize