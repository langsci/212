%!TEX root = ../main.tex

\chapter{Verb morphology} \label{cha:verb morphology}

\section{Introduction} \label{verbmorphintro}

This chapter describes the verbal morphology of Komnzo, which is by far the most complex subsystem in the language, and reaches a scale of complexity which is found in polysynthetic languages.\footnote{Most definitions of polysynthesis stress two main criteria: noun incorporation and the expression of syntactic relations by pronominal affixes (\citealt[16]{Baker:1996poly}; \citealt[2]{Evans:2002sasse}; and \citealt{Mithun:2009wh}). Komnzo lacks noun incorporation, but cross-references up to two participants with pronominal affixes. Typically, a verb consists of 3 up to 9 morphs.} Morphological complexity in Komnzo verbs arises not only from the number of affixes which the \isi{verb} may host, but also from the way these combine to encode grammatical categories (see \S{}\ref{verbprelim}). In its simplest form a \isi{verb} exists as an \isi{infinitive}, that is the stem plus a \isi{nominaliser} suffix. At their most complex, verbs may host a large number of affixes and clitics. Table \ref{verbtemplate} gives an overview of the \isi{verb} template, the inflectional categories and the formatives to be discussed in this chapter.\\

The central feature that reverberates throughout Komnzo \isi{verb} morphology is its cumulative and distributed \isi{combinatorics}. The particular values of most grammatical categories are only arrived at after unifying information from several morphological slots within the \isi{verb} structure. This feature has shaped my descriptive approach which bounces back and forth between a functional and a formal perspective. I address \isi{alignment} and \isi{valency} in \S{}\ref{alignmtemplates}, \isi{person}, \isi{gender} and \isi{number} in \S{}\ref{persgendnumber}, \isi{deixis} and \isi{directionality} in \S{}\ref{deixisanddirectionality}. At the same time, the functional perspective is interspersed with the description of structural phenomena like the two stem types in \S{}\ref{roots-and-temp} or the suffixing subsystem in \S{}\ref{personsuffsection}. Tense, \isi{aspect} and mood will be described in Chapter \ref{TAMpalooza}. I describe the formatives and the possible combinations thereof in \S{}\ref{combitam}, the contribution of TAM particles in \S{}\ref{TAMparticlessection}, and the semantic nuances of the TAM categories in \S{}\ref{TAMsemantics}. In order to avoid too much repetition, many cross-references in the text link related topics.

\clearpage
\begin{landscape}
\begin{table}
\caption{Templatic representation of verb inflection}
{\footnotesize%
{\renewcommand{\tabcolsep}{2pt}
\begin{tabularx}{\textwidth}{lllllllllll}
\label{verbtemplate}\\
	\lsptoprule
	&\textsc{clitic} & \multicolumn{3}{l}{\textsc{prefix slots}}	&\textsc{stem}	&\multicolumn{5}{l}{\textsc{suffix slots}}\\
	&& \multicolumn{3}{l}{ }&&\multicolumn{5}{l}{}\\
	&-4&\multicolumn{1}{l}{-3}&-2&\multicolumn{1}{l}{-1}&$\sqrt{}$&\multicolumn{1}{l}{1}&2&3&4&\multicolumn{1}{l}{5}\\
	&& \multicolumn{3}{l}{ }&&\multicolumn{5}{l}{ }\\\midrule
	\multirow{2}{*}{\textsc{valency}}&&&\textsc{val change:}	&&&&&&&\\
	&&&\emph{a-}&&&&&&&\\\midrule
	&&\textsc{\isi{undergoer}:} &&&&&&&&\textsc{actor:}\\
	\textsc{person}&&\First, \Second, \Third &&&&&&&&\First, \Second\textbar\Third\\
	&&or \textsc{middle}&&&&&&&&or $\varnothing$\\\midrule
	\multirow{2}{*}{\textsc{gender}}&&\textsc{undergoer}:&&&&&&&&\\
	&&\Tsg.\F, \Tsg.\Masc&&&&&&&&\\\midrule
	&&\textsc{\isi{undergoer}:} &\textsc{dual}: $\varnothing$-&&&&\textsc{dual}: \emph{-n}&&&\textsc{actor:}\\
	\textsc{number}&&\Sg, \Nsg &\textsc{non-dual}: \emph{a-}&&&&\textsc{non-dual}:&&& \Sg, \Nsg\\
	&&&&&&&\emph{-nzr, -wr, -r}&&&\\\midrule
	\textsc{deixis}& \Prox: \emph{z=}&&&\textsc{\isi{ventive}:}&&&&&\textsc{andative:}\\
	\&& \Med: \emph{b=}	&&&\emph{n-}&&&&&\emph{-o}&\\
	\textsc{direction}& \Dist: \emph{f=}&&&&&&&&&\\\midrule
	\multirow{4}{*}{\textsc{tam}}&\textsc{apprehensive}:& \textsc{\isi{prefix series}:}&\textsc{irrealis:}&&\textsc{stem type:}&\textsc{\isi{stative}:}&&\textsc{\isi{past}:}&&\textsc{imperative:}\\
	&\emph{m=}& \Alph{}, \Bet{}, \Betaone{}, \Betatwo{}, \Gam&\emph{ra-}	&&\Ext{} (extended)&\emph{-thgr}&&\emph{-a}&&actor\\
	&\textsc{imminent}:& &&&\Rs{} (restricted)&&&\textsc{durative}&&suffixes\\
	&\emph{n=}& &&&&&&\emph{-m}&&\\
	\lspbottomrule
\end{tabularx}}}%
\end{table} 

\end{landscape}%verb template

\section{Morphological complexity} \label{verbprelim}

The relationship between the value of a grammatical category and its exponents exhibits varying degrees of complexity in Komnzo verbs. At its simplest, we find a one-to-one mapping between function and form, which exists for the \isi{directional} affixes. For the most part, however, Komnzo verbs are characterised by complexity of \isi{exponence} of the type one-to-many and many-to-many. Concerning the former, we find what Matthews (\citeyear[147-149]{Matthews:1979vm}) calls ``cumulative \isi{exponence}'', whereby one exponent expresses several grammatical categories, as well as ``extended \isi{exponence}'', whereby several exponents express one grammatical category. Note that the latter has also been called ``\isi{multiple exponence}'' in the literature (\citealt[163]{Caballero:2012vr}). For example, the Komnzo verb prefixes are \isi{portmanteau} realisations of the categories \isi{person}, \isi{gender}, \isi{number}, \isi{tense} and \isi{aspect}. Conversely, a category like \isi{tense} is encoded in three different slots on the \isi{verb}. These slots can be independently manipulated, which results in a many-to-many mapping. Complex \isi{exponence} of this type is a feature found in all \ili{Yam languages}.\\

Let us take one inflected verb form to illustrate these types of \isi{exponence}. Example (\ref{ex138}) gives the inflected \isi{verb} form \emph{yfathwroth} `they hold him away'.\footnote{This verb form can have a stative as well as a dynamic reading: someone is holding a baby moving it away from the deictic centre (dynamic), or someone holds the baby in such a way that the toddler is facing away from the deictic centre (stative).}

\begin{exe}
	\ex \emph{yfathwroth}\\
	\gll y\stem{fath}wroth\\
	\Stpl:\Sbj>\Tsg.\Masc:\Obj:\Nonpast:\Ipfv:\Andat/hold\\
	\trans `They hold him away.'
	\label{ex138}
\end{exe}

Here we find a one-to-one mapping between the \isi{directional} value (\isi{andative}) and the suffix \emph{-o}. This is expressed in Figure \ref{one-to-one} below where the verb form has been segmented into morphs. A line indicates the \isi{exponence} relationship between the value (\Andat) and the formative (\emph{-o}).

\begin{figure}
\begin{center}
\fbox{%
\begin{tikzpicture}
	\draw (7,1) --(5.1,0);
	\node[above] at (0,0.9) {[\Second\textbar\Third};
	\node[above] at (0.9,0.9) {\Pl]};
	\node[above] at (1.55,1) {>};
	\node[above] at (2,0.9) {[\Third};
	\node[above] at (2.6,1) {\Sg};
	\node[above] at (3.6,0.9) {\Masc]};
	\node[above] at (4.8,1) {\Nonpast};
	\node[above] at (6,1) {\Ipfv};
	\node[above] at (7.3,1) {\Andat};
	\node[below] at (0,0) {\emph{y-}};
	\node[below] at (1.7,0.1) {\emph{fath}};
	\node[below] at (3.4,0) {\emph{-wr}};
	\node[below] at (5.1,0) {\emph{-o}};
	\node[below] at (7,0.1) {\emph{-th}};
\end{tikzpicture}}
\end{center}
\caption{One-to-one mapping for the directional}
\label{one-to-one}
\end{figure}%one-to-one mapping with fathasi

Cumulative \isi{exponence} is found in the verb prefix \emph{y-} which fuses information on person (\Third), number (\Sg), and gender (\Masc) of the object argument. In addition, \emph{y-} contains information on tense (\Nonpast) and aspect (\Ipfv{}). This is schematised in Figure \ref{cumulfath}.

\begin{figure}
\begin{center}
\fbox{%
\begin{tikzpicture}
	\draw (2,1)   --(0,0);
	\draw (3.6,1) --(0,0);
	\draw (2.6,1) --(0,0);
	\draw (4.8,1) --(0,0);
	\draw (6,1) --(0,0);
	\node[above] at (0,0.9) {[\Second\textbar\Third};
	\node[above] at (0.9,0.9) {\Pl]};
	\node[above] at (1.55,1) {>};
	\node[above] at (2,0.9) {[\Third};
	\node[above] at (2.6,1) {\Sg};
	\node[above] at (3.6,0.9) {\Masc]};
	\node[above] at (4.8,1) {\Nonpast};
	\node[above] at (6,1) {\Ipfv};
	\node[above] at (7.3,1) {\Andat};
	\node[below] at (0,0) {\emph{y-}};
	\node[below] at (1.7,0.1) {\emph{fath}};
	\node[below] at (3.4,0) {\emph{-wr}};
	\node[below] at (5.1,0) {\emph{-o}};
	\node[below] at (7,0.1) {\emph{-th}};
\end{tikzpicture}}
\end{center}
\caption{Cumulative exponence of person, number, gender, tense and aspect}
\label{cumulfath}
\end{figure}%cumulative \isi{exponence} with fathasi

Note that the prefix \emph{y-} is necessary, but not sufficient, to establish the values for some of these categories. For example, the aspectual value of the \isi{verb} (\Ipfv) is not expressed solely by \emph{y-}. This is what Matthews calls ``extended \isi{exponence}'' (\citeyear[147-149]{Matthews:1979vm}) and Caballero \& Harris refer to as ``\isi{multiple exponence}'' (\citeyear[163]{Caballero:2012vr}). It is essentially the mirror image of Figure \ref{cumulfath}. Thus, Figure \ref{extendfath} below shows that \isi{aspect} is distributed over three exponents in \emph{yfathwroth}.

\begin{figure}
\begin{center}%
\fbox{
\begin{tikzpicture}
	\draw (6,1) --(0,0);
	\draw (6,1) --(1.7,0);
	\draw (6,1) --(3.4,0);
	\node[above] at (0,0.9) {[\Second\textbar\Third};
	\node[above] at (0.9,0.9) {\Pl]};
	\node[above] at (1.55,1) {>};
	\node[above] at (2,0.9) {[\Third};
	\node[above] at (2.6,1) {\Sg};
	\node[above] at (3.6,0.9) {\Masc]};
	\node[above] at (4.8,1) {\Nonpast};
	\node[above] at (6,1) {\Ipfv};
	\node[above] at (7.3,1) {\Andat};
	\node[below] at (0,0) {\emph{y-}};
	\node[below] at (1.7,0.1) {\emph{fath}};
	\node[below] at (3.4,0) {\emph{-wr}};
	\node[below] at (5.1,0) {\emph{-o}};
	\node[below] at (7,0.1) {\emph{-th}};
\end{tikzpicture}}
\end{center}
\caption{Extended exponence of aspect}
\label{extendfath}
\end{figure}%extended \isi{exponence} with fathasi

A change in any one of the three slots above will cause a change in the TAM value. For example, the prefix \emph{y-} can be replaced by \emph{su-} to form a \isi{recent past} \isi{imperfective} (\emph{sufathwroth}) or a suffix \emph{-m} can be added after \emph{-wr} to form a \isi{recent past} durative (\emph{yfathwrmoth}). If both of these changes are made at the same time, we get a \isi{past} durative (\emph{sufathwrmoth}). It follows that we are not dealing with a circumfix where separated formatives always occur together, but rather with a circumfixal paradigm where the formatives in the different slots are quite independent. Although there are some combinatorial restrictions, it would be a distortion to describe this as a circumfix. The essence of the system is that only by unifying the information from each slot are we in a position to calculate the correct value of a given grammatical category.\\

Thus, the overall complexity of Komnzo verbs results from the co-ocurrence of different types of \isi{exponence} relationships. Figure \ref{recipfath} below captures all the dependencies between the values of a grammatical category and the morphs that make up \emph{yfathwroth}. Quite literally, we find a web of tightly interwoven dependencies.

\begin{figure}
\begin{center}
\fbox{%
\begin{tikzpicture}
	\draw (0,1)   --(6.9,0);
	\draw (0.9,1) --(6.9,0);
	\draw (0.9,1) --(3.4,0);
	\draw (3.6,1) --(0,0);
	\draw (2.6,1) --(3.4,0);
	\draw (4.8,1) --(1.7,0);
	\draw (6,1) --(0,0);
	\draw (6,1) --(1.7,0);
	\draw (6,1) --(3.4,0);
	\draw (2,1)   --(0,0);
	\draw (2.6,1) --(0,0);
	\draw (4.8,1) --(0,0);
	\draw (7,1) --(5.1,0);
	\node[above] at (0,0.9) {[\Second\textbar\Third};
	\node[above] at (0.9,0.9) {\Pl]};
	\node[above] at (1.55,1) {>};
	\node[above] at (2,0.9) {[\Third};
	\node[above] at (2.6,1) {\Sg};
	\node[above] at (3.6,0.9) {\Masc]};
	\node[above] at (4.8,1) {\Nonpast};
	\node[above] at (6,1) {\Ipfv};
	\node[above] at (7.3,1) {\Andat};
	\node[below] at (0,0) {\emph{y-}};
	\node[below] at (1.7,0.1) {\emph{fath}};
	\node[below] at (3.4,0) {\emph{-wr}};
	\node[below] at (5.1,0) {\emph{-o}};
	\node[below] at (7,0.1) {\emph{-th}};
\end{tikzpicture}}
\end{center}
\caption{Reciprocal conditioning}\label{recipfath}
\end{figure}%\isi{reciprocal} conditioning with fathasi

Anderson uses the term ``\isi{reciprocal} conditioning'' (\citeyear[55]{Anderson:1992uw}) for this phenomenon, whereby exponents depend on several grammatical categories, while being underspecified for a single grammatical category.\footnote{Morpheme underspecifiation does not stop at the word boundary in Komnzo. For example, the actor argument in \emph{yfathwroth} can be either second or third person. Without context, this ambiguity can only be resolved by the personal pronouns. The same is true for future tense or event completion, which are expressed periphrastically with the preverbal particles \emph{kwa} and \emph{z} respectively.} I adopt the term ``\isi{distributed exponence}'' from Caballero \& Harris (\citeyear[170]{Caballero:2012vr}), who point out that it may be related to multiple/extended \isi{exponence}. Although it is excluded from the survey, Caballero \& Harris mention \isi{distributed exponence} in the theoretical discussion by explaining some aspects of Georgian \isi{verb} morphology (\citealt{Gurevich2006:geo}). Baerman (\citeyear{Baerman:2012ko}) describes a phenomenon that could also be called \isi{distributed exponence} for \ili{Nuer}, a Western Nilotic language. The complexity of marking case and \isi{number} in \ili{Nuer} builds on suffixes and stem alternations, which are independently manipulated and give rise to inflectional classes. Baerman stresses the noniconicity of the system ``in that these operations characterise simply a contrast of meaning, without being linked to any particular meaning'' (\citeyear[490]{Baerman:2012ko}). Similarly, Komnzo \isi{verb} morphology must be understood as a system where morphs contribute to a grammatical category, but a specific value of a given grammatical category requires information from several slots. Caroll provides the most detailed study of \isi{distributed exponence} in his grammar on \ili{Ngkolmpu} (\citeyear{Carroll:Ngkolmpu}), a related \ili{Tonda} language.\\

There are practical consequences for the description of such a system. I have used a glossing style which follows the Word-and-Paradigm model (\citealt[67]{Matthews:1979vm}) throughout this grammar to give the reader effortless access to the morphosyntactic features of an inflected \isi{verb} form. Since this chapter addresses verbal morphology, I will employ a double glossing and a verb like \emph{yfathwroth} will be glossed as in (\ref{ex139}) below. The first line gives a maximally segmented \isi{gloss} in the Item-and-Arrangement style, while the second line in smaller print gives a unified \isi{gloss} in the Word-and-Paradigm style.\footnote{Elsewhere in the grammar - where there is no double glossing, but only the unified gloss - the verb stem is shown by slanted lines \stem{...} on the segmentation line.}

\begin{exe}
	\ex \emph{yfathwroth}\\
	\glll y-fath-w-r-o-th\\
	\Tsg.\Masc:\Alph-hold.\Ext-\Ndu-\Lk-\Andat-\Stnsg\\
	\footnotesize{\Stpl:\Sbj>\Tsg.\Masc:\Obj:\Nonpast:\Ipfv:\Andat/hold}\\
	\trans `They hold him away.'
	\label{ex139}
\end{exe}

The Item-and-Arrangement style provides more transparency in the morphological structure which is the aim of this chapter. In spite of that, widespread underspecification means that the gain in structural transparency comes at the cost of somewhat opaque glossing labels. For example, while we find in (\ref{ex139}) established labels like \Sg{} (\isi{singular}) and \Nsg{} (\isi{non-singular}) to encode \isi{number}, we also need to recruit \Ndu{} (\isi{non-dual}). As for \isi{tense} and \isi{aspect}, we have to introduce even more abstract labels like \Alph{} (alpha) in the prefixes or \Ext{} (extended) with the verb stem. These will be explained in the following sections. A further drawback of the Item-and-Arrangement style is that some of the grammatical values like non-past (\Nonpast) or imperfective (\Ipfv) as well as \isi{subject} (\Sbj), \isi{object} (\Obj) and \isi{indirect object} (\Io) cannot be shown on the \isi{gloss} line because they can be inferred only after integrating several exponents.

\section{Stem types} \label{roots-and-temp}

Komnzo verbal stems have two forms; an `\textsc{extended stem}' (\Ext) and a `\textsc{re-stricted stem}' (\Rs). As these labels indicate, the distinction is sensitive to \isi{aspect} without encoding a specific aspectual category. For now it is sufficient to state that the labels refer to the \isi{temporal} structure of the event, i.e. `extended in time' and `restricted in time'. The two stems differ (i) in their form, (ii) in the order of slots with respect to \isi{dual} marking and (iii) in their combinatorial possibilities with the \isi{prefix series}. I describe each point below.

\subsection{The formal relationship of extended and restricted stems} \label{formalrelationshipextrs}

Komnzo has pairs of \isi{verb} stems whose relationship is often unpredictable from any formal or semantic criteria. Nevertheless, there is a cline of similarity in form between the two stems which allows us to divide the verbal lexicon into seven classes (Table \ref{frbearr}). For thirty percent, there is a rule-based relation between the shapes of the two stems. At the other end of the spectrum, we find suppletive pairs of stems in five percent of the verbal lexicon. For more than two thirds of the lexicon the shape of the stems is unpredictable.\\

\begin{table}
\caption{The formal relationship between \Ext{} and \Rs{} stem}
\begin{tabularx}{\textwidth}{lllllc}
\label{frbearr}\\
	\lsptoprule
	\textsc{class}&\textsc{rule} &\Ext{} &\Rs{} &\textsc{gloss} &\textsc{count}\\\midrule
	%\endfirsthead
	\textsc{class}&\textsc{rule} &\Ext{} &\Rs{} &\textsc{gloss} &\textsc{count}\\\midrule
	%\endhead
	\multirow{5}{*}{\textsc{i}}	&\multirow{5}{*}{\Ext{}=\Rs{}} &\multicolumn{2}{c}{\emph{mar-}} &see &\multirow{5}{*}{\textsc{42}}\\
	&&\multicolumn{2}{c}{\emph{zik-}} &turn off &\\
	&&\multicolumn{2}{c}{\emph{rikn-}} &destroy &\\
	&&\multicolumn{2}{c}{\emph{rmän-}} &close &\\
	&&\multicolumn{2}{c}{\emph{matukn-}} &shake &\\\midrule
	\multirow{5}{*}{\textsc{ii}} &\multirow{5}{*}{\Ext{}$\Leftarrow$\Rs-ak} &\emph{rfitfak}- &\emph{rfitf}- &answer &\multirow{5}{*}{\textsc{52}}\\
	&&\emph{morak}- &\emph{mor}- &lean &\\
	&&\emph{bthak}- &\emph{bth}- &finish &\\
	&&\emph{ritak}-	&\emph{rit}- &cross &\\
	&&\emph{msak}- &\emph{ms}-	&sit &\\\midrule
	\multirow{5}{*}{\textsc{iii}} &\multirow{5}{*}{\Ext-\textsc{c}$\Rightarrow$\Rs} &\emph{gar}- &\emph{garf}- &break &\multirow{5}{*}{\textsc{81}}\\
	&&\emph{fsi}- &\emph{fsir}- &count &\\
	&&\emph{tri}- &\emph{trinz}- &scratch &\\
	&&\emph{rni}- &\emph{rnith}- &smile &\\
	&&\emph{thari}- &\emph{tharif}-	&dig &\\\midrule
	\multirow{5}{*}{\textsc{iv}} &\multirow{5}{*}{\textsc{mutation}} &\emph{thwek}- &\emph{thweth}- &be glad &\multirow{5}{*}{\textsc{96}}\\
	&&\emph{mthek}- &\emph{mthef}- &lift up &\\
	&&\emph{moneg}-	&\emph{mones}- &wait &\\
	&&\emph{trakumg}- &\emph{trakumth}- &smash &\\
	&&\emph{bnaz}- &\emph{bnaf}- &wake up\\\midrule
	\multirow{5}{*}{\textsc{v}}	&\multirow{5}{*}{\textsc{irregular}} &\emph{rsör}- &\emph{rsöfäth}- & descend &\multirow{5}{*}{\textsc{26}}\\
	&&\emph{thorak}- &\emph{thothm}- &search &\\
	&&\emph{myukn}-	&\emph{myuf}- &twist &\\
	&&\emph{rirkn}-	&\emph{rirkfth}- &avoid	&\\
	&&\emph{tur}- &\emph{turam}- &kiss &\\\midrule
	\multirow{6}{*}{\textsc{vi}} &\multirow{6}{*}{\textsc{suppletive}} &\emph{re}- &\emph{zigrthm}- &look around &\multirow{6}{*}{\textsc{15}}\\
	&&\emph{ru}- &\emph{mg}-&shoot, spear &\\
	&&\emph{fn}- &\emph{kwr}-&hit, kill &\\
	&&\emph{na}- &\emph{znob}-&drink &\\
	&& \emph{zä}- & \emph{thor}-&carry &\\
	&&\emph{si}- &\emph{füs}- &cook &\\\midrule
	&\Rs{} \textsc{only} &- & \emph{-kuk}\textsuperscript{a} &stand &1 \\
	\textsc{vii}&\multirow{2}{*}{\Ext{} \textsc{only}} &\emph{rug}- &- &sleep&\multirow{2}{*}{6}\\
	&&\emph{rmug}- &- &envy &\\\midrule
	\textsc{total}&&&&&319\\
	\lspbottomrule
	\multicolumn{6}{l}{{\footnotesize \textsuperscript{a} This verb has a second stem \emph{-kogr}, which I analyse as a \isi{positional} stem (see \S\ref{positionalverbs}}).}\\
\end{tabularx}%The formal relationship between \Ext{} and \Rs{} stem
\end{table}


In class I, which makes up 13\% of verbs, the two stems are identical (\Ext{}=\Rs{}). Class II verbs (16\%) derive their extended stems from the \isi{restricted stem} with a suffix (\Ext{}=\Rs-\emph{ak}). Thus class I and class II make up that portion of the verb lexicon with a rule-based relationship between the two stems. However, only a few generalisations can be made about the scope of the rule, i.e. given a particular lexeme, one cannot decide straightforwardly which class it belongs to. Amongst those few generalisations is the fact that most verbs in class I end in /n/, but this is not true of all. Moreover, verbs ending in /n/ are also found in the other classes.\\

The majority of verbs involve unpredictable changes at the right edge of the stem. In class III, which makes up 25\% of verbs, a consonant is added to the \isi{extended stem} in order to form the \isi{restricted stem} (\Rs=\Ext-\textsc{c}). The stem pairs of class IV verbs (30\%) involve final consonant mutation. In class III and IV, the affected consonants are not conditioned by the phonological environment. Class V verbs (8\%) are irregular, i.e. the difference involves more than the last consonant. The stems of class VI (5\%) are fully suppletive. Finally, a handful of verbs in class VII are defective, and have only one of the two stems.\\

We can make a few observations from Table \ref{frbearr}. First, we find a cline of similarity which ranges from identity of the two stems to suppletive pairs with the bulk of verbs between the two extremes. Classes II\textendash{}V all have in common that the difference in form occurs the right edge of stem. Secondly, the classes and processes (consonant mutation, consonant addition, suffixation of \emph{-ak}) are neither phonologically conditioned, nor can we detect a semantic basis for them. Thirdly, the system shows little productivity, which I take as evidence for lexicalisation. In Table \ref{frbearr}, it is only class II for which a regular process can be formulated; the suffixation of \emph{-ak}. Finally, we find that for almost all verbs, both stems are attested. As a result, virtually all verbs can be inflected for the entire range of TAM categories, which leaves little role to play for lexical \isi{aspect} (or Aktionsart) in Komnzo.\\

I will offer a historical explanation below (see \S{}\ref{comparativenoteextrs}) as to how the two stems have evolved in Komnzo and in the \ili{Tonda} subgroup more generally.

\subsection{Dual marking with extended and restricted stems} \label{dualextrs}

The most salient difference between the two stems is the location of the \isi{dual} marker, which follows the \isi{extended stem} but precedes the \isi{restricted stem}. I describe number marking in detail in \S{}\ref{numbersubsec}. In the examples (\ref{ex141}-\ref{ex143}) and (\ref{ex145}-\ref{ex147}), I contrast the imperfective and perfective imperatives of `hit'. The former often has a continuative interpretation (`keep on x-ing!') while the latter points to inception (`start doing x!'). In (\ref{ex140}) and (\ref{ex144}), all grammatical categories are held constant, and only the actor argument is cycled through the three number values. In (\ref{ex141}-\ref{ex143}), \isi{dual} is shown by a suffix (\emph{-n}), which contrasts with a \isi{non-dual} (\emph{-z}). In (\ref{ex145}-\ref{ex147}), \isi{dual} is expressed by a \isi{zero} which contrasts with a \isi{non-dual} prefix (\emph{a-}).

\begin{exe}
\ex
\label{ex140}
\begin{xlist}
	\ex %\textit{be fi sfnzé!}\\
	\glll \emph{be} \emph{fi} \emph{s-fn-z-é}\\
	 \Ssg{}.\Erg{} \Third{}.\Abs{} \Tsg{}.\Masc{}:\Bet{}-hit.\Ext{}-\Ndu{}-\Ssg{}.\Imp{}\\
	 {} {} \footnotesize{\Ssg:\Sbj>\Tsg.\Masc:\Obj:\Imp:\Ipfv/hit}\\
	\trans `You keep hitting him!'
	\label{ex141}
	\ex %\textit{bné fi sfnne!}\\
	\glll \emph{bné} \emph{fi} \emph{s-fn-n-e}\\
	 \Snsg{}.\Erg{} \Third{}.\Abs{} \Tsg{}.\Masc{}:\Bet{}-hit.\Ext{}-\Du{}-\Snsg{}.\Imp{}\\
	  {} {} \footnotesize{\Sdu:\Sbj>\Tsg.\Masc:\Obj:\Imp:\Ipfv/hit}\\
	\trans `You (2) keep hitting him!'
	\label{ex142}
	\ex %\textit{bné fi sfnze!}\\
	\glll \emph{bné} \emph{fi} \emph{s-fn-z-e}\\
	 \Snsg{}.\Erg{} \Third{}.\Abs{} \Tsg{}.\Masc{}:\Bet{}-hit.\Ext{}-\Ndu{}-\Snsg{}.\Imp{}\\
	  {} {} \footnotesize{\Spl:\Sbj>\Tsg.\Masc:\Obj:\Imp:\Ipfv/hit}\\
	\trans `You (3+) keep hitting him!'
	\label{ex143}
\end{xlist}
\end{exe}%ex140a-c
\begin{exe}
\ex
\label{ex144}
\begin{xlist}
	\ex %\textit{be fi sakwer!}\\
	\glll \emph{be} \emph{fi} \emph{s-a-kwr-\Zero{}}\\
	\Ssg{}.\Erg{} \Third{}.\Abs{} \Tsg{}.\Masc{}:\Bet{}-\Ndu{}-hit.\Rs{}-\Ssg{}.\Imp{}\\
	{} {} \footnotesize{\Ssg:\Sbj>\Tsg.\Masc:\Obj:\Imp:\Pfv/hit}\\
	\trans `You hit him!'
	\label{ex145}
	\ex %\textit{bné fi skwre!}\\
	\glll \emph{bné} \emph{fi} \emph{s-\Zero{}-kwr-e}\\
	\Snsg{}.\Erg{} \Third{}.\Abs{} \Tsg{}.\Masc{}:\Bet{}-\Du{}-hit.\Rs{}-\Snsg{}.\Imp{}\\
	{} {} \footnotesize{\Sdu:\Sbj>\Tsg.\Masc:\Obj:\Imp:\Pfv/hit}\\
	\trans `You (2) hit him!'
	\label{ex146}
	\ex %\textit{bné fi sakwre!}\\
	\glll \emph{bné} \emph{fi} \emph{s-a-kwr-e}\\
	\Snsg{}.\Erg{} \Third{}.\Abs{} \Tsg{}.\Masc{}:\Bet{}-\Ndu{}-hit.\Rs{}-\Snsg{}.\Imp{}\\
	{} {} \footnotesize{\Spl:\Sbj>\Tsg.\Masc:\Obj:\Imp:\Pfv/hit}\\
	\trans `You (3+) hit him!'
	\label{ex147}
\end{xlist}
\end{exe}%ex144a-c

The post-stem \isi{non-dual} marker, \emph{-z} in (\ref{ex140}), has a number of phonologically conditioned allomorphs (see \S{}\ref{allomorphdualsuffix}). The \isi{dual} marker is always \emph{-n}. In terms of segmentation, the post-stem slot is simple to recognise. This is not the case with the pre-stem duality marker which is \isi{zero} for \isi{dual} and \emph{a-} for \isi{non-dual} in (\ref{ex144}). For purposes of illustration, I have selected the imperatives here because the segmentation is clearest. In other parts of the paradigm, segmentation is messier because the \isi{dual} marker fuses with the \isi{valency change} prefix resulting in an ablaut contrast; \emph{a-} for \isi{dual} and \emph{ä-} for \isi{non-dual} (see \S{}\ref{prerootdual}). From a historical perspective, this structural split between a pre-stem and a post-stem slot is a way of preserving \isi{dual} marking after the original suffix had fused with the stem (see \S{}\ref{comparativenoteextrs}).

\subsection{The combinatorics of extended and restricted stems} \label{combinatoricsextrs}

Extended and restricted stems taken alone are underspecified for a particular TAM value and information from other morphological sites is required. With respect to the five \isi{prefix series} \Alph, \Bet{}, \Betaone{}, \Betatwo, \Gam{} (see \S{}\ref{personsuffsection}), the two stems differ in their combinatorial possibilities. For example, the \Alph{} prefixes combine with the \isi{extended stem} and the \Gam{} prefixes combine with the \isi{restricted stem}, but not vice versa. The \Alph{} series is recruited to form \isi{non-past}, \isi{immediate past}, \isi{recent past} or \isi{past} in \isi{imperfective} or \isi{durative} \isi{aspect} depending on suffixal material. The \Gam{} series is employed for \isi{recent past} or \isi{past}, both \isi{perfective}. The \Bet{} prefixes combine with both stems to form imperatives and irrealis with \isi{imperfective} and \isi{perfective} \isi{aspect}. The \Betaone{} and \Betatwo{} prefixes combine with the \isi{extended stem} (the latter exclusively so) to form \isi{recent past} and \isi{past} in \isi{imperfective} or \isi{durative} \isi{aspect}, again depending on suffixal material. The \Betaone{} prefixes combine with the \isi{restricted stem} to form an \isi{iterative}. The details of the five \isi{prefix series} as well as the aspectual distinctions will be addressed in \S{}\ref{combitam}. For present purposes, it is sufficient to stress that there are some limitations on the combinatorial possibilities for extended and restricted stems.

\subsection{A comparative note on multiple stems} \label{comparativenoteextrs}

Verb stem pairs which are sensitive to \isi{aspect} are known from other Papuan languages, for example \ili{Mian} (\citealt[245]{Fedden:2011wu}). In the Southern New Guinea region, \ili{Marind} shows striking architectural similarities to the Komnzo system. Drabbe reports on 4 verb classes in \ili{Marind} (\citeyear[31]{Drabbe:1955tm}). The first two classes which make up the main distinction are labelled ``momentaan'' versus ``duratief.'' Members of a third class can be both, and only the affixes signal the aspectual value of an inflected verb form. The fourth class is characterised as ``momentaan,'' but it can be turned into ``duratief'' by suffixing \emph{-a(t)}. The overall design of the \ili{Marind} system looks similar once we equate ``duratief'' with extended and ``momentaan'' with restricted. Drabbe's third class in \ili{Marind} bears resemblance to that group of Komnzo verbs where only one form is attested (class I in Table \ref{frbearr}). The fourth class is very close to those stem pairs in Komnzo which add the suffix \emph{-ak} to the \isi{restricted stem} in order to form the \isi{extended stem} (class II in Table \ref{frbearr}). Moreover, the two suffixes, \emph{-a(t)} in \ili{Marind} and \emph{-ak} in Komnzo, are formally similar. However, neglecting Drabbe's group three and four, the \ili{Marind} system differs in that most verbs fall into either ``momentaan'' or ``duratief.'' As we have seen above, almost all verbs in Komnzo have both stems.\\

Within the Yam family, multiple verb stems are found in the \ili{Nambu} as well as the \ili{Tonda} subgroup. However, the system as laid out here seems to be more developed in the \ili{Tonda} languages. Pairs of verb stems are attested in \ili{Arammba}, where Boevé \& Boevé (\citeyear{Bouve:2003ar}) label them ``common root'' and ``limited action root.'' In my own fieldwork, I found stem pairs in \ili{Anta}, \ili{Wára}, \ili{Wèré}, Kámá, \ili{Kánchá}, \ili{Blafe}, Ránmo and \ili{Wartha} Thuntai. As for \ili{Ngkolmpu}\footnote{\ili{Ngkolmpu}, as well as Bädi, Smerky and Sota, were classified as varieties of \ili{Kanum} in the past.}, there are up to three stems for some verbs and these are sensitive to \isi{aspect} as well as verbal number (\citealt{Carroll:Ngkolmpu}). More descriptive work is needed to understand how the two stems are employed in the respective TAM systems of these languages.\\

I will offer a first tentative historical explanation based on the comparison of duality/TAM marking and multiple stems within the Yam family. In the \ili{Nambu} subgroup, aspect-sensitive stems are only a marginal phenomenon. However, part of the verb inflection is a suffix which combines aspectual information with dual marking. For example, in \ili{Nen} (\citealt{Evans:2015to}) and \ili{Nama} (\citealt{Siegel:2015bp}) a thematic suffix follows the verb stem encoding TAM plus dual versus non-dual. In Komnzo, the suffix following the stem encodes only duality, but the presence versus absence of this suffix is determined by the stem type. Thus, it is involved in marking \isi{aspect} (see \S{}\ref{dualextrs}).\\

I have shown above that the differences in form between the two stem types are located at the right edge. It is therefore a likely scenario that multiple stems have emerged through a process of demorphologisation (\citealt[154]{Hopper:1990vm}), i.e. through a fusion of suffixal material with the stem. Until more decriptive material is available, we are left to speculate on the nature of the original system. Logically, there are at least two possibilities: (i) the original suffix followed the \ili{Nambu} pattern encoding TAM and duality simultaneously or (ii) there were separate suffixes for each category. Since both the occurrence of multiple stems as well as cognate forms are attested in all varieties of the \ili{Tonda} languages, demorphologisation would constitute an innovation, which supports \ili{Tonda} as a subgroup of the Yam family. This is of some importance, because other systematic differences between \ili{Nambu} and \ili{Tonda}, like word-initial \isi{velar} \isi{nasals}\footnote{The \ili{Nambu} language Dre which is spoken close to other \ili{Tonda} languages has preserved initial velar nasals.} or gender marking on verbs, can be explained by assuming the loss of a particular feature in \ili{Nambu} rather than assuming an innovation in \ili{Tonda}.\\

The historical scenario advanced above gave rise to different inflectional patterns within the \ili{Tonda} subgroup. Languages further to the west including \ili{Blafe}, Ránmo, \ili{Wartha} Thuntai and to some extent \ili{Kánchá} have lost \isi{dual} marking except in some high \isi{frequency} verbs like the copula. Other varieties like \ili{Wára}, \ili{Anta} and Komnzo have kept post-stem \isi{dual} marking for one stem type, but requisitioned a different slot in the template for the other stem type. This would explain why, in terms of morphological segmentation, the pre-stem \isi{dual} marking with restricted stems is much messier than post-stem \isi{dual} marking with extended stems (compare \S{}\ref{dualextrs} and \S{}\ref{prerootdual}). We could say that in a historical process, \isi{dual} marking has ``hijacked'' a slot which was hitherto solely employed for marking \isi{valency}. A third pattern is attested in \ili{Wèré}, where \isi{dual} marking is consistently post-stem for both stem types. However, irregularities involving a vowel change in the prefixes for some parts of the paradigm show that the \ili{Wèré} pattern is a case of regularisation of the Komnzo system rather than an independent development.\\

The scenario developed here has to be treated with some caution, as there are exceptions to the generalisations made above. For example, \ili{Nen} has multiple stems for a few verbs like $\sqrt{}$\emph{waram} versus $\sqrt{}$\emph{warama} `give', encoding \isi{imperfective} and \isi{perfective} \isi{aspect} respectively (\citealt{Evans:nen}). Another exception is the \ili{Nambu} language \ili{Nä}, which has pre-stem \isi{dual} marking for some middle verbs. Much more comparative work needs to be done to fully account for the emergence of multiple verb stems in these languages.

\section{Alignment and verb templates} \label{alignmtemplates}

\subsection{Grammatical relations} \label{grammrel}

This section describes the argument structure in Komnzo. The term is understood as ``the configuration of arguments that are governed by a particular lexical item'' (\citealt[1130]{HaspelmathBardey:2004}). For the purpose of defining argument structure, we need to take into account particular constructions (\citealt[433]{Bickel:2011wo}). In Komnzo, these are \isi{case} and agreement (i.e. verb indexing). There are no other constructions restricted to a set of arguments (e.g. control, relativisation, \isi{coordination}, \isi{nominalisation} of verbs).\\

First, I identify generalised semantic roles (\textsc{gsr}s) for each verb form. Following Bickel (\citeyear{Bickel:2011wo}), these roles are labelled as follows: A is the most agent-like argument and P is the most patient-like argument of a \isi{transitive} \isi{verb}, S is the sole argument of an \isi{intransitive} verb. For \isi{ditransitive} verbs, T is the most theme-like argument and R the most recipient-like argument.\\

In the following, I will outline the two parameters of argument structure in Komnzo. In (\ref{ex752}-c), I show the basic structure for one-argument and two-argument predicates in a reduced glossing style. A is assigned \isi{ergative} \isi{case}, while S and P are assigned \isi{absolutive} \isi{case}. Example (\ref{ex754}) shows that A is indexed in the suffix and P is indexed in the prefix. S has to be split into S\textsubscript{P}, which is indexed in the prefix (\ref{ex752}), and S\textsubscript{A}, which is indexed in the suffix (\ref{ex753}). The underlying factor is the \isi{dynamicity} of the predicate (see \S\ref{prefixingverbsec}).

\begin{exe}
\ex
\label{ex751}
\begin{xlist}
	\ex %\textit{fi ykogr.}\\
	\gll \emph{fi} \emph{y-kogr}\\
	\Third(\Abs) \Tsg.\Masc-stand\\
	\trans `He stands.'
	\label{ex752}
	\ex %\textit{fi ŋamränzrth.}\\
	\gll \emph{fi} \emph{ŋamränzr-th}\\
	\Third(\Abs) stroll-\Tpl\\
	\trans `They stroll around.'
	\label{ex753}
	\ex %\textit{nafa fi yfnzrth.}\\
	\gll \emph{nafa} \emph{fi} \emph{y-fnzr-th}\\
	\Tpl{}.\Erg{} \Third(\Abs) \Tsg.\Masc-hit-\Tpl{}\\
	\trans `They hit him.'
	\label{ex754}
\end{xlist}
\end{exe}

Examples (\ref{ex756}-c) show the argument structure for three-argument predicates. Note that I discuss why there are some problems in describing ditransitives in \S\ref{ambifixingtemp}. For \isi{case} assignment, the examples show that P and T are marked by the \isi{absolutive} \isi{case} and R by the \isi{dative} \isi{case}. The R is always indexed in the prefix, not P nor T. Furthermore, the verb form is inflected with the \emph{a-} prefix, which I label \Vc{} for \isi{valency change}.

\begin{exe}
\ex
\label{ex755}
\begin{xlist}
	\ex %\textit{nafa fi yfnzrth.}\\
	\gll \emph{nafa} \emph{giri} \emph{nafan} \emph{y-a-rithr-th}\\
	\Tpl{}.\Erg{} knife(\Abs) \Tsg.\Dat{} \Tsg.\Masc-\Vc-give-\Tpl{}\\
	\trans `They give him the knife.'
	\label{ex756}
	\ex %\textit{nafa fi yfnzrth.}\\
	\gll \emph{nafa} \emph{bone} \emph{zokwasi} \emph{nzun} \emph{w-a-rbänzr-th}\\
	\Tpl{}.\Erg{} \Ssg.\Poss{} speech(\Abs) \Fsg.\Dat{} \Fsg-\Vc-explain-\Tpl{}\\
	\trans `They explain your message to me.'
	\label{ex757}
	\ex %\textit{nafa fi yfnzrth.}\\
	\gll \emph{nafa} \emph{srak} \emph{nafan} \emph{y-a-brigwr-th}\\
	\Tpl{}.\Erg{} boy(\Abs) \Tsg.\Dat{} \Tsg.\Masc-\Vc-return-\Tpl{}\\
	\trans `They return the boy for/to him.'
	\label{ex758}
\end{xlist}
\end{exe}

From the types of argument structure shown above, we can define the following \isi{grammatical relations} in Komnzo:

\begin{enumerate}
	\item The \isi{subject} relation is characterised by either \isi{ergative} or \isi{absolutive} case assignment.
	\begin{enumerate}
		\item If the noun phrase is in the \isi{ergative}, it will always be indexed in the suffix.
		\item If the noun phrase is in the \isi{absolutive}, it may be indexed in the suffix or the prefix. It is considered to be a \isi{subject}, iff the clause contains no ergative-marked noun phrase.
	\end{enumerate}
	\item The \isi{object} relation is characterised by \isi{absolutive} case assignment and indexation in the prefix. This only applies in the presence of another \isi{ergative} noun phrase which is indexed in the suffix.
	\item The \isi{indirect object} relation is characterised by \isi{dative} (or \isi{possessive}) case assignment and indexation in the prefix. Additionally, the verb form receives the \isi{valency change} prefix \emph{a-}.
\end{enumerate}%\isi{grammatical relations}

Similar to other grammatical categories, for example TAM and \isi{number}, \isi{grammatical relations} are constructed by unifying information from different sites. These are the \isi{person} marking affixes and the diathetic prefix, but also the \isi{case} assignment on the respective noun phrases. I describe the \isi{person} marking affixes on the verb as the actor suffix and the \isi{undergoer} prefix.\footnote{I use a semantic definition of the term undergoer as that argument which is affected by the event.} However, in the unified \isi{gloss}, which is employed throughout this grammar, I use \Sbj{} (\isi{subject}), \Obj{} (\isi{object}) and \Io{} (\isi{indirect object}). A reviewer suggested to use \A{} (actor) und \U{} (\isi{undergoer}) and avoid using categories like \isi{subject} and \isi{object}. I agree that there is no strong evidence for a \isi{subject} category in Komnzo. Nevertheless, I employ the terms \isi{subject}, \isi{object} and \isi{indirect object} as metalinguistic labels that I find useful in communicating with other linguists. I do not claim that these play an overly important role in the grammar of Komnzo. In addition, there are practical reasons for using \Sbj{} (\isi{subject}), \Obj{} (\isi{object}) and \Io{} (\isi{indirect object}) in the \isi{gloss} line. If I employ \A{} (actor) und \U{} (\isi{undergoer}), it would be impossible to show the distinction between an \isi{object} and an \isi{indirect object} in the unified \isi{gloss} line.

\subsection{Morphological templates}\label{morphologicaltemplates}

This section describes the structure of verbs by looking at the slots involved in the indexation of arguments. More precisely, I describe the arrangement of slots, the presence vs. absence of slots, as well as their content.\\

Based on the inflectional pattern, Komnzo verbs can be classified into prefixing, \isi{middle} and ambifixing verbs, depending on whether prefix, suffix or both are employed. I use the term ``\isi{template}'' for the different inflectional patterns. Hence, we can say that a \isi{verb} form occurs in ``a prefixing \isi{template}'' or in ``an ambifixing \isi{template}''. These templates are lexically determined for some verb lexemes, which means we can speak of ``a prefixing verb'' or ``a \isi{middle} verb''. For the majority of verb lexemes, the system is flexible and verbs occur in different templates. We can describe a particular verb lexeme by stating that ``it occurs in the \isi{middle} template and the ambifixing template, but not in the prefixing \isi{template}''.\\

The slots involved in the definition of templates are the following: (i) the \isi{undergoer} prefix, (ii) the diathetic prefix, and (iii) the actor suffix. The \isi{undergoer} prefix can index an argument, or it can be filled by the \isi{middle} prefix, which is person-invariant. The diathetic prefix can be absent or be filled by the \isi{valency change} prefix.\footnote{I use the neutral term ``valency change'' because its function is to either increase or decrease the valency of a verb.} The actor suffix can be either absent or present. Figure \ref{verbtemplatearg} provides a first schematic overview of the possible templates. Note that there are more than the three templates mentioned above. This is because the prefixing and the ambifixing template can be further subdivided depending on the presence versus absence of the \isi{valency change} prefix. Hence, there is a prefixing template and an \isi{indirect object} prefixing template; and there is a \isi{transitive} ambifixing template and a \isi{ditransitive} ambifixing template.

\begin{figure}

	\begin{tabular}{r|l|l|l|l|}
		\cline{2-2}\cline{4-4}
		\textsc{prefixing}:&\isi{undergoer} prefix &  & stem & \multicolumn{1}{l}{}\\ \cline{2-2}\cline{4-4}
		\multicolumn{4}{l}{}\\\cline{2-4}
		\textsc{\isi{indirect object} prefixing}:&\isi{undergoer} prefix & \Vc & stem & \multicolumn{1}{l}{}\\ \cline{2-4}
		\multicolumn{4}{l}{}\\\cline{2-5}
		\textsc{middle}:&\isi{middle} prefix & \Vc{} & stem & actor suffix\footnotemark\\ \cline{2-5}
		\multicolumn{4}{l}{}\\\cline{2-2}\cline{4-5}
		\textsc{\isi{transitive} ambifixing}:&\isi{undergoer} prefix & & stem & actor suffix\\ \cline{2-2}\cline{4-5}
		\multicolumn{4}{l}{}\\\cline{2-5}
		\textsc{\isi{ditransitive} ambifixing}:&\isi{undergoer} prefix & \Vc{} & stem & actor suffix\\ \cline{2-5}
		\multicolumn{4}{l}{}\\
	\end{tabular}
\caption{Morphological templates and argument structure}
\label{verbtemplatearg}
\end{figure}%Morphological templates and argument structure
\footnotetext{The label `actor suffix' is problematic with some lexemes which employ the middle template for a passive function. In this case, the suffix encodes a patient argument (see \S{}\ref{middletemplatesubsection}).}

I briefly describe each template here and refer the reader to the subsequent sections in which a detailed description follows (\S{}\ref{prefixingverbsec}-6). In the prefixing template, only the \isi{undergoer} prefix is used for \isi{person} indexing. In the \isi{indirect object} prefixing template also, only the \isi{undergoer} prefix is used for \isi{person} indexing. However, the \isi{undergoer} prefix indexes an \isi{indirect object} (\isi{beneficiary} or \isi{possessor}). This is formally marked by the \isi{valency change} prefix \emph{a-}. In the \isi{middle} template, the prefix is filled by a \isi{middle} marker which is invariant for \isi{person} and \isi{number}. The sole argument is indexed in the suffix. The \isi{middle} marker is always followed by the \isi{valency change} prefix \emph{a-}. The \isi{middle} template is used for a variety of functions, and depending on the function of the argument in the suffix it may index an \isi{agent} or \isi{patient}. The ambifixing \isi{transitive} template uses both affixes for \isi{person} indexing. The prefix encodes the \isi{object} (\isi{patient}, \isi{theme}, \isi{experiencer}) and the suffix encodes the \isi{subject} (\isi{agent}, \isi{stimulus}). The \isi{ditransitive} ambifixing template follows the pattern of the \isi{transitive} template with the addition of the \isi{valency change} prefix \emph{a-}. The \isi{undergoer} prefix indexes the \isi{indirect object} (\isi{goal}, \isi{beneficiary}, \isi{possessor}).\\

I illustrate the five templates with the verb \emph{migsi} `hang' in examples (\ref{ex760}-e). Note that although the system is flexible, i.e. verbs occur in different templates, there is only a small amount of \isi{verb} lexemes which can occur in all five templates. I choose the \isi{positional} verb \emph{migsi} `hang' in (\ref{ex759}). Positional verbs have a number of peculiarities, for example a special verbstem and \isi{stative} suffix, which also encodes \isi{number} (see \S\ref{positionalverbs}). This can be seen in (\ref{ex759}a) and  (\ref{ex759}b).

\begin{exe}
\ex
\label{ex759}
\begin{xlist}
	\ex \textsc{prefixing}:\\
	\gll \emph{y-mi-thgr}\\
	 	\Tsg.\Masc-hang.\Pos-\Stat.\Ndu\\
	\trans `He is hanging.'
	\label{ex760}

	\ex \textsc{\isi{indirect object} prefixing}:\\
	\gll \emph{y-a-mi-thgr}\\
	 	\Tsg.\Masc-\Vc-hang.\Pos-\Stat.\Ndu\\
	\trans `(Something) is hanging for him.'
	\label{ex761}

	\ex \textsc{middle}:\\
	\gll \emph{ŋ-a-mig-wr-\Zero}\\
	 	\M-\Vc-hang.\Ext-\Ndu-\Stsg\\
	\trans `It hangs itself up.'
	\label{ex762}

	\ex \textsc{\isi{transitive} ambifixing}:\\
	\gll \emph{y-mig-wr-\Zero}\\
	 	\Tsg.\Masc-hang.\Ext-\Ndu-\Stsg\\
	\trans `S/He hangs him up.'
	\label{ex763}

	\ex \textsc{\isi{ditransitive} ambifixing}:\\
	\gll \emph{y-a-mig-wr-\Zero}\\
	 	\Tsg.\Masc-\Vc-hang.\Ext-\Ndu-\Stsg\\
	\trans `S/He hangs it up for him.'
	\label{ex764}
\end{xlist}
\end{exe}

The templates do not align neatly with transitivity. For example, only a small minority of \isi{intransitive} verbs are prefixing (\ref{ex149}), while most employ a middle template (\ref{ex150}). The underlying semantic factor is the \isi{dynamicity} of the event (see \S{}\ref{prefixingverbsec}). On the other hand, the middle template covers a wide range of functions including reflexives and reciprocals, passives, as well as antipassives (see \S{}\ref{middletemplatesubsection}). Transitive verbs are usually expressed in the ambifixing template (\ref{ex151}). Ditransitive verbs occur in the ambifixing template with the addition of the \isi{valency change} prefix \emph{a-}, whereby an \isi{indirect object} is introduced to the clause. The corresponding noun phrase is flagged with \isi{dative} (\ref{ex152}) or \isi{possessive} \isi{case}, and it is indexed in the \isi{undergoer} prefix (see \S{}\ref{ambifixingtemp}).

\begin{exe}
\ex
\label{ex148}
\begin{xlist}
	\ex \textit{ktktme erfikwr.}\\
	\glll kt-kt=me e-rfik-wr\\
	 \Redup{}-group=\Ins{} \Stnsg{}:\Alph{}-grow.\Ext{}-\Ndu{}\\
	 {} \footnotesize{\Stpl:\Sbj:\Nonpast:\Ipfv/grow}\\
	\trans `They grow in groups.'
	\label{ex149}

	\ex \textit{nagayé ŋakwinth.}\\
	\glll nagayé ŋ-a-kwi-n-th\\
	 children \M{}:\Alph{}-\Vc{}-run.\Ext{}-\Du{}-\Stnsg{}\\
	  {} \footnotesize{\Stdu:\Sbj:\Nonpast:\Ipfv/run}\\
	\trans `The two children run.'
	\label{ex150}

	\ex \textit{nafa ŋad yrbänzrth.}\\
	\glll nafa ŋad y-rbä-nzr-th\\
	 \Tnsg{}.\Erg{} rope \Tsg.\Masc:\Alph{}-untie.\Ext{}-\Ndu{}-\Stnsg{}\\
	  {} {} \footnotesize{\Stpl:\Sbj>\Tsg.\Masc:\Obj:\Nonpast:\Ipfv/untie}\\
	\trans `They untie the rope.'
	\label{ex151}

	\ex \textit{nze nafan wawa yarithé.}\\
	\glll nze nafan wawa y-a-ri-th-é.\\
	 \Fsg{}.\Erg{} \Tsg.\Dat{} yam \Tsg{}.\Masc:\Alph{}-\Vc-give.\Ext{}-\Ndu{}-\Fsg{}\\
	  {} {} {} \footnotesize{\Fsg:\Sbj>\Tsg.\Masc:\Io:\Nonpast:\Ipfv/give}\\
	\trans `I give him the yam(s).'
	\label{ex152}
\end{xlist}
\end{exe}

{\renewcommand{\tabcolsep}{2pt}%
\begin{table}
\caption{Argument marking}
\label{argalignverbs}
	{\footnotesize%
	\begin{tabular}{p{1,6cm}p{2,5cm}p{1,4cm}p{2,1cm}lp{2,3cm}}
		\lsptoprule
		\textsc{template} &\textsc{semantic role} &\textsc{diathetic} &\textsc{semantic role} &\textsc{case} & \textsc{construction}\\
		&\textsc{in the prefix} &\textsc{prefix} &\textsc{in the suffix} &\textsc{frame} &\\\midrule
		% %\endfirsthead
		% \textsc{template} &\textsc{semantic role} &\textsc{diathetic} &\textsc{semantic role} &\textsc{case} & \textsc{construction}\\
		% &\textsc{in the prefix} &\textsc{prefix} &\textsc{in the suffix} &\textsc{frame} &\\\midrule
		% %\endhead
		prefixing&\isi{experiencer}, &\Zero{} &n/a &\Abs &intransitive\\
		&(agent)\textsuperscript{a} &&&&(stative)\\
		&&&&&\\
		indirect &beneficiary or &\emph{a-}	&n/a &\Dat{} or &intransitive\\
		\isi{object} &\isi{possessor} &&&\Poss &(stative)\\
		prefixing &&&&&\\
		&&&&&\\
		\isi{middle} &n/a &\emph{a-} &agent &\Abs	&intransitive\\
		&&&&&(dynamic)\\
		&&&&&\\
		\isi{middle} &n/a &\emph{a-} &agent &\Abs	&\isi{impersonal}\\
		&&&&&\\
		\isi{middle} &n/a &\emph{a-} &\isi{patient} &\Abs &\isi{passive}\\
		&&&&&\\
		\isi{middle} &n/a &\emph{a-} &agent &\Abs	&reflex. \& recip.\\
		&&&&&\\
		\isi{middle} &n/a &\emph{a-} &agent &\Erg{} (\Abs)\textsuperscript{b}	& suppressed-\\
		&&&&&\isi{object}\\
		&&&&&\\
		\isi{transitive} ambifixing &\isi{patient}, \isi{theme} &\Zero &agent	&\Erg{} \Abs{} &\isi{transitive}\\
		&&&&&\\
		\isi{transitive} ambifixing &\isi{experiencer} &\Zero &\isi{stimulus}	&\Abs{} \Erg{} &\isi{experiencer-object}\\
		&&&&&\\
		\isi{ditransitive} ambifixing &beneficiary, \isi{goal} &\emph{a-} &agent &\Erg{} \Abs{} \Dat &\isi{ditransitive}\\
		&&&&&\\
		\isi{ditransitive} ambifixing &\isi{possessor} &\emph{a-} &agent &\Erg{} \Abs{} \Poss &\isi{ditransitive}\\
		\lspbottomrule
		\multicolumn{6}{l}{\footnotesize{\textsuperscript{a} This is a marginal pattern as almost all prefixing verbs have stative semantics.}}\\
		\multicolumn{6}{l}{\footnotesize{\textsuperscript{b} In \isi{suppressed-object} clauses, the \isi{object} is suppressed from the indexation in the verb.}}\\
	\end{tabular}}
\end{table}}%Argument marking

It follows that the \isi{valency change} prefix \emph{a-} (\Vc) has a double function. It increases and decreases the \isi{valency} of a \isi{verb}. This is exemplified with \emph{migsi} `hang' in examples (\ref{ex760}-e) above. There are a number of \isi{deponent} verbs attested, for example prefixing verbs or \isi{transitive} ambifixing verbs which obligatorily take the \emph{a-} prefix. I analyse them as \isi{deponent} in the sense of Baerman et al (\citeyear{Baerman:2006depo}) because in these cases the \isi{undergoer} prefix indexes a direct \isi{object}, although the presence of the \Vc{} prefix suggests an \isi{indirect object}.\footnote{Deponency is defined as a ``mismatch between morphology and morpho-syntax'' (\citealt{Baerman:2006depo}).}\\

Table \ref{argalignverbs} provides a fine-grained overview of the templates. I show the semantic roles of the arguments indexed in the affixes, the presence/absence of the \isi{valency change} prefix, the \isi{case} frame and the name of the corresponding construction. These constructions are described in the section on clause types (\S\ref{clause types}).

\subsection{Valency alternations} \label{valencyalternations}

In Komnzo, \isi{valency alternations} are achieved by placing the \isi{verb} in different templates. There is only a handful of verbs which occur in all the templates. I choose the verb \emph{msaksi} `sit, dwell' to show its possibilities below with text examples (\ref{ex320}-\ref{ex323}). Note that \emph{msaksi} deviates in two ways from other verbs. First, it takes the \isi{valency change} prefix obligatorily when it occurs in a prefixing template, as can be seen in (\ref{ex320}). Secondly, there is a special verbstem for the prefixing template: \emph{m}. In other templates, \emph{msaksi} has the \isi{extended stem} \emph{msak} and the \isi{restricted stem} \emph{ms}, i.e. it is a class II verb (compare Table \ref{frbearr}).\\

In example (\ref{ex320}), the speaker showed me a place which used to be inhabited by a spirit. He states that nobody knows where the spirit lives nowadays. Hence, the verb \emph{msaksi} has a \isi{stative} meaning in the prefixing template and can be translated into \ili{English} with `dwell, live, stay', or `be sitting'.

\begin{exe}
	\ex \emph{watik ŋafäniza ... ni miyamr mä zena \textbf{yamnzr}.}\\
	\glll watik ŋ-a-fäni-z-a-\Zero{} (.) ni miyamr mä zena y-a-m-nzr\\
	then \M.\Alph{}-\Vc-shift.place.\Ext-\Ndu-\Pst-\Stsg{} (.) \Fnsg{} ignorance where today \Tsg.\Masc.\Alph-\Vc-dwell.\Ext-\Ndu\\
	{} \footnotesize{\Stsg:\Sbj:\Ipfv:\Pst/shift.place} {} {} {} {} {} \footnotesize{\Tsg.\Masc:\Sbj:\Nonpast:\Ipfv/dwell}\\
	\trans `Then he shifted (location). We don't know where he lives today.'\\\Corpus{tci20120922-19}{DAK \#37}
	\label{ex320}
\end{exe}

Example (\ref{ex321}) was uttered in the context of me visiting a garden place in the forest, where I was accompanied by the owner of the garden. The speaker happened to cycle past the garden place catching sight of me and the owner. The speaker comments on how he saw the two of us sitting down. Thus, \emph{msaksi} in the \isi{middle} template encodes a dynamic event and can be translated into \ili{English} with `sit down' or `assume a sitting position'.

\begin{exe}
	\ex \emph{nze nimäwä! boba thnmaré \textbf{ŋamsakrnmth}.}\\
	\glll nze nima=wä boba th-\Zero{}-n-mar-é\\
	\Fsg.\Erg{} like.this=\Emph{} \Med:\Abl{} \Stnsg.\Gam-\Du-\Venit-see.\Rs-\Fsg{}\\
	{} {} {} \footnotesize{\Fsg:\Sbj>\Stdu:\Obj:\Rpst:\Pfv:\Venit/see}\\
	\sn
	\glll ŋ-a-msak-rn-m-th\\
	\M.\Alph-\Vc-sit.\Ext-\Du-\Dur-\Stnsg{}\\
	\footnotesize{\Stdu:\Sbj:\Rpst:\Dur/sit}\\
	\trans `Me too! I saw you two from there and you were just sitting down.'\\\Corpus{tci20130823-06}{STK \#90}
	\label{ex321}
\end{exe}

Example (\ref{ex322}) shows \emph{msaksi} in a \isi{transitive} ambifixing template. The example comes from a narrative, in which an angry man is forcefully seated and calmed down by giving him kava to drink.

\begin{exe}
	\ex \emph{wati \textbf{ymsakwrth} fof krär \textbf{yarinakwrth} bänemr fof nafane noku frazsir.}\\
	\glll wati y-msak-wr-th fof krär\\
	then \Tsg.\Masc.\Alph-sit.\Ext-\Ndu-\Stnsg{} \Emph{} kava\\
	{} \footnotesize{\Stpl:\Sbj>\Tsg.\Masc:\Obj:\Nonpast:\Ipfv/sit} {} {}\\
	\sn
	\glll y-a-rinak-wr-th bän=mr fof nafane noku\\
	\Tsg.\Masc.\Alph-\Vc-pour.\Ext-\Ndu-\Stnsg{} \Dem:\Med=\Purp{} \Emph{} \Tsg.\Poss{} anger\\
	\footnotesize{\Stpl:\Sbj>\Tsg.\Masc:\Io:\Nonpast:\Ipfv/pour} {} {} {}\\
	\sn
	\gll fraz-si=r\\
	extinguish-\Nmlz=\Purp{}\\
	\trans `So they sit him down properly and pour kava for him to cool down his anger.'\Corpus{tci20120909-06}{KAB 93-94}
	\label{ex322}
\end{exe}

Example (\ref{ex323}) is an elicited example showing \emph{msaksi} in a \isi{ditransitive} ambifixing template, where the \isi{undergoer} prefix indexes the \isi{possessor} (`his child'). Note that the same template is found in the second verb in (\ref{ex322}), where the \isi{undergoer} prefix indexes a \isi{beneficiary} (`pour kava for him').

\begin{exe}
	\ex \textit{nze nafange \textbf{yamsakwé}.}\\
	\glll nze nafa-nge y-a-msak-w-é.\\
	 \Fsg{}.\Erg{} \Third.\Poss-child \Tsg{}.\Masc{}:\Alph{}-\Vc{}-sit.\Ext{}-\Ndu{}-\Fsg{}\\
	  {} {} \footnotesize{\Fsg:\Sbj>\Tsg.\Masc:\Io:\Nonpast:\Ipfv/sit}\\
	\trans `I sit his child down.'
	\label{ex323}
\end{exe}

The above examples show that \isi{valency alternations} are achieved by using the same verb in different templates. It is important to note that all the inflected verb forms share the same \isi{infinitive}, which is formed by suffixing the \isi{nominaliser} \emph{-si} to the stem. In (\ref{ex324}) and (\ref{ex325}) I show the \isi{infinitive} with a \isi{stative} and a dynamic interpretation. Example (\ref{ex324}) is the conclusion of a short narrative about taboos and customs that involve the bird of paradise. The speaker uses \emph{msaksi} with a \isi{locative} \isi{case} suffix in a \isi{possessive} construction to express `in our life'. In example (\ref{ex325}), the speaker showed me a beautiful place on the bank of Morehead river. She comments that this is a good place to sit down and rest. Hence, the \isi{infinitive} \emph{msaksi} is used for both interpretations, a timeless state in (\ref{ex324}) and a dynamic event in (\ref{ex325}).

\begin{exe}
	\ex \emph{nzenme trtha mrmren nzenme \textbf{msaksin} ... wtrikarä anema fof ŋamränzre.}\\
	\gll nzenme trtha mrmr=en nzenme msak-si=n (.) wtri=karä\\
	\Fnsg.\Poss{} life inside=\Loc{} \Fnsg.\Poss{} sit-\Nmlz=\Loc{} (.) fear=\Prop{}\\
	\sn
	\glll ane=ma fof ŋ-a-mrä-nzr-e\\
	\Dem=\Char{} \Emph{} \M.\Alph-\Vc-stroll.\Ext-\Ndu-\Fnsg{}\\
	{} {} \footnotesize{\Fpl:\Sbj:\Nonpast:\Ipfv/stroll}\\
	\trans `In our way of life ... in our living ... we walk about with fear because of this.'\Corpus{tci20120817-02}{ABB \#40-43}
	\label{ex324}
\end{exe}
\begin{exe}
	\ex \emph{camp rä ... zmbo fthé nanyak \textbf{msaksir}.}\\
	\glll camp rä (.) zmbo fthé n-a-n-yak\\
	camp \Tsg.\F.\Cop.\Ndu{} (.) \Prox.\All{} when \Fnsg.\Alph-\Vc-\Venit-walk.\Ext.\Ndu{}\\
	{} \footnotesize{\Tsg.\F:\Sbj:\Nonpast:\Ipfv/be} {} {} {} \footnotesize{\Fpl:\Sbj:\Nonpast:\Ipfv/come}\\
	\sn
	\gll msak-si=r\\
	sit-\Nmlz=\Purp{}\\
	\trans `This is a camp ... We come here to sit down (and rest).'\\\Corpus{tci20130907-02}{RNA \#331-333}
	\label{ex325}
\end{exe}

The meaning of a \isi{verb} in one template may differ substantially when used in another template. For example, the verb \emph{rfiksi} `grow' occurs in the prefixing template (\ref{ex149}), but it can be used in a \isi{transitive} ambifixing template with the meaning `nurture' (Lit. `grow somebody'). A second example is the verb \emph{rbänzsi} `untie' which usually occurs in a \isi{transitive} ambifixing template (\ref{ex151}). Used in a \isi{ditransitive} ambifixing template it has the meaning `explain' (Lit. `untie for somebody'). Nevertheless, inflected verbs in different templates all share the same \isi{infinitive}. In this aspect Komnzo differs from other \ili{Yam languages}. For example in \ili{Nen}, there are no infinitives for prefixing verbs, but instead valency-altered forms have distinct infinitives which include the relevant formatives from a set of diathetic prefixes (\citealt{Evans:2015wy}). For example, one pair of infinitives is: \emph{amzs} `sit (v.i.)' versus \emph{wamzs} `set, sit (v.t.)'. There are even triplets: \emph{an\={g}ws} `return (v.i.)' versus \emph{wan\={g}ws} `return (v.t.)' versus \emph{wawan\={g}ws} `return to/for (v.t.)'. In Komnzo, there are no distinct infinitives for valency-altered forms. Hence, \emph{rfiksi} is the \isi{infinitive} of both `grow' and `nuture', and \emph{rbänzsi} is the \isi{infinitive} of `untie' and `explain'.\\

There are two ways of analyzing shared infinitives in Komnzo and I argue that both are needed. On the one hand, we can understand it as a system where \isi{valency} is fluid and lexemes are flexible. Under this analysis a lexeme can alter its \isi{valency} by occuring in different templates. On the other hand, we could adopt the notion of \isi{heterosemy} (\citealt{Lichtenberk:1991ic} and \citealt[524]{Evans:2012we}) to capture that different lexical items and meanings are expressed by different templates.\footnote{This assumes a definition of the linguistic sign as having three parts: form, meaning and combinatorics (or syntax) as put forward by (\citealt{Melcuk:1973vu}) and (\citealt[51]{Pollard:1987wu}).} A verb like \emph{msaksi} shows that we need both perspectives. On the one hand, \emph{msaksi}\textsubscript{1} means `dwell, live' in a prefixing template, while \emph{msaksi}\textsubscript{2} means `sit down' in a middle/ambifixing template. We would understand \emph{msaksi}\textsubscript{1} as being heterosemous to \emph{msaksi}\textsubscript{2} because there is a significant shift in meaning due to the template. The same holds for pairs like \emph{rfiksi} meaning `grow' or `nuture' and \emph{rbänzsi} meaning `untie' or `explain'. On the other hand, the system of \isi{valency alternations} in Komnze is very productive. Especially the \isi{middle} template and the \isi{ditransitive} ambifixing template can be used for almost every verb which can also occur in the \isi{transitive} ambifixing template. Thus, describing the alternation between \emph{msaksi} in (\ref{ex322}) `sit someone down' and (\ref{ex323}) `sit down someone's (child)' in terms of \isi{heterosemy} would fall short of an exhaustive description. It would not adequately capture the productivity of the system, nor would it fully explain shared \isi{infinitive}s for verb forms of different templates.

\subsection{The prefixing template} \label{prefixingverbsec}

\subsubsection{Introduction}

Prefixing verbs are a small class with around 20 lexical items attested so far. Some of them can occur in other templates, but most occur only in the prefixing template. Table \ref{pref.verbs} lists all the members of the prefixing class. Furthermore, there is a class of 41 \isi{positional} verbs, which can occur in the prefixing template (see \S{}\ref{positionalverbs}).

{\renewcommand{\tabcolsep}{4pt}
\begin{table}
\caption{Prefixing verbs}
\label{pref.verbs}
	\begin{tabular}{llll}
		\lsptoprule
		\textsc{infinitive} &\textsc{gloss} &\textsc{possible} &\textsc{gloss}\\
		or \textsc{stem}\footnotemark &&\textsc{templates}&\\\midrule
		\emph{-rug}	&`sleep' &pref. only & -\\
		\emph{-yak}	&`walk, go'	&pref. only & -\\
		\textsuperscript{a}\emph{-nyak} &`come' &pref. only & -\\
		\textsuperscript{a}\emph{yathizsi} &`suffer' &pref. only & -\\
		\textsuperscript{a}\emph{mthizsi} &`rest' &pref. only & -\\
		\textsuperscript{a}\emph{-nor} &`shout, emit sound' &pref. only & -\\
		\emph{wäksi} &`be caught by daybreak' &pref. only & -\\
		\emph{fogsi} &`be caught by nightfall' &pref. only & -\\
		\emph{rmigfaksi} &`be in the middle of (doing) sth.'& pref. only &-\\
		\emph{-thn}	&`be lying'	&pref. only & -\\
		\textsuperscript{a}\emph{yarenzsi} &`look around' &pref. only & -\\
		\emph{-ythk} &`be finished' &pref. only & -\\
		\textsuperscript{a}\emph{namgsi}	&`be panting, gasping' &pref. only & -\\
		\emph{thfäsi} &`jump' &pref./middle	&`fly'\\
		\textsuperscript{a}\emph{thgusi}	&`forget' & pref./trans.  &`confuse sth.'\\
		\emph{thoraksi}	&`appear, arrive' & pref./trans.  &`find, search'\\
		\emph{wokraksi}	&`float' &pref./trans. &`make sth. float'\\
		\emph{-rä} &`be' &all templates &`do'\\
		\textsuperscript{a}\emph{msaksi}	&`dwell, live' &all templates &`sit (self or sb.)'\\
		\emph{sufaksi} &`grow old' &all templates &`bring to an end'\\
		\emph{ziksi} &`turn off, be on the side' &all templates &`put to the side'\\
		\emph{rfiksi} &`grow' &all templates &`nurture'\\
		\lspbottomrule
		\multicolumn{4}{l}{\footnotesize{\textsuperscript{a} These verbs are \isi{deponent}, i.e. they use the \Vc{} prefix obligatorily.}}\\
	\end{tabular}
\end{table}}%Prefixing verbs
\footnotetext{Infinitives are marked with the nominaliser suffix \emph{-si}. Prefixing verbs are irregular in many respects. Some of the verbs listed here lack an infinitive and only the extended stem is given, while others employ a common noun as their infinitive, for example \emph{etfth} `sleep,' \emph{moth} `path, walk, come' and \emph{kwan} `noise, shout.' This does not correlate with whether there are other templates available. Where a nominalised form with \emph{-si} is lacking, I give the extended stem. Another irregularity are verbs where the stem is sensitive to the dual versus non-dual distinction, for example `walk' \emph{-yak} (\Ndu) versus \emph{-yan} (\Du) or `shout' \emph{-nor} (\Ndu) versus \emph{-rn} (\Du). In these cases, the non-dual stem is listed.}

Prefixing verbs are special in their morphology in that they can encode a fourth \isi{number} value. The somewhat odd combination of a \isi{non-singular} prefix and a \isi{dual} suffix yields a \isi{large plural}. This is attested in other \ili{Yam languages}, for example for \isi{positional} verbs in \ili{Nen} and \ili{Nä} (\citealt{Evans:2014bz}). I describe the four-way \isi{number} contrast in \S{}\ref{positonalnumber}.\\

Prefixing verbs are mostly \isi{stative} in their semantics. Comparative work on split intransitivity has shown that differences in \isi{alignment} are often semantically motivated ( \citealt{Merlan:1985tu}, \citealt{Mithun:1991wu} and \citealt{Arkadiev:2008vq}). In Komnzo, the semantic parameters involved are the \isi{dynamicity} of the event and the \isi{volitionality} of the \isi{participant}, the former plays the dominant role. As we have seen in \S\ref{valencyalternations}, predicates in a prefixing template tend to be more \isi{stative} (\ref{ex320}), while predicates in \isi{middle} or ambifixing templates tend to be more dynamic (\ref{ex321}-\ref{ex323}). In other languages of the Yam family, the split between \isi{stative} and dynamic event types is congruent with the distinction between prefixing and \isi{middle} intransitives, for example in \ili{Nen} (\citealt{Evans:2015to}) and \ili{Nama} (\citealt{Siegel:2015bp}).\footnote{Siegel uses different terminology in his description of \ili{Nama}. What I call the prefixing template or stative intransitives equals ``patientive intransitives'', and what I label the middle template or dynamic intransitives equals ``agentive intransitives'' (\citealt[213]{Siegel:2015bp}).}\\

In Komnzo, although all verbs in a middle or ambifixing template depict dynamic event types, we find a somewhat mixed picture with prefixing verbs. Table \ref{pref.verbs} contains a few dynamic events, for example \emph{-nor} `shout', \emph{thoraksi} `appear, arrive' and \emph{rfiksi} `grow'. In some cases, \isi{volitionality} is the semantic parameter involved in the prefixing/middle/ambifixing alternation: \emph{thoraksi} and \emph{rfiksi} in an ambifixing \isi{transitive} template mean `find' and `nurture' respectively.\footnote{In ambifixing templates, the case marking of a more agent-like argument is ergative. This is also found in middle templates with an suppressed-object function.} The verb \emph{-nor} `shout' allows no alternation, but occurs only in a prefixing template. Interestingly, \emph{-nor} is often used in a \isi{pseudo-cognate object} construction: \emph{kwan yannor}\footnote{\emph{-nor} lacks a nominalised infinitive and instead the common noun \emph{kwan} `shout, call' is used.} `He shouts (the shout)' or \emph{ya yannor} `He cries (the tears)'. Hence, with this verb a less volitional meaning like `emit a sound' might be licenced. Pseudo-\isi{cognate object} constructions are described in \S{}\ref{pseudocognate}. Nevertheless, with other predicates in Table \ref{pref.verbs} such an explanation fails, for example \emph{ziksi} `turn off, go in' or \emph{thfäsi} `jump'. Keeping the unusually small size of the prefixing class in mind, I interpret these cases as exceptions to the overall rule. Furthermore, the existence of a class of \isi{positional} verbs (\S{}\ref{positionalverbs}) underscores the split along the lines of event \isi{dynamicity} and \isi{volitionality}.\\

All prefixing verbs can take the \isi{valency change} prefix \emph{a-}. This template was labelled \isi{indirect object} prefixing in Table \ref{argalignverbs}. However, in doing so they remain monovalent in their cross-referencing. The `additional argument', usually a Beneficiary or Possessor, replaces the `original argument', usually an Experiencer. However, the event itself remains to `be about' the original argument. A common usage of this pattern involves the \isi{copula}: When handing something to a person, one would say \emph{bnarä!} `There you are!' (literally: `(It) is there for you!'). A textual example comes from a stimulus task in which two speakers are describing the content of picture cards (\ref{ex160}). The picture in the example shows a policeman who hands some personal belongings to another man. After describing the scene, one of the two speakers points to a few things on the side asking what these were. The first verb in (\ref{ex160}) `be lying down' indexes the (assumed) \isi{possessor} and not the things on the ground. The second clause is accompanied by a pointing gesture in order to draw the interlocutor's attention to the objects. Here, the copula indexes the things on the ground.

\begin{exe}
	\ex \emph{mrmr ra \textbf{yathn}? zane \textbf{zerä}!}\\
	\glll mrmr ra y-a-thn zane\\
	inside what.\Abs{} \Tsg.\Masc.\Alph-\Vc-lie.\Ext.\Ndu{} \Dem:\Prox{}\\
	{} {} {\footnotesize \Tsg.\Masc:\Io:\Nonpast:\Ipfv/lie} {}\\
	\sn
	\glll z=e-rä\\
	\Prox=\Stnsg.\Alph-be.\Ext.\Ndu\\
	{\footnotesize \Prox=\Stpl:\Sbj:\Nonpast:\Ipfv/be}\\
	\trans `What are these (of his) inside? These ones here!'\Corpus{tci20111004}{TSA \#29-30}
	\label{ex160}
\end{exe}

Table \ref{pref.verbs} indicates that eight out of 20 prefixing verbs obligatorily use the \emph{a-} prefix without introducing an argument. I analyse these verbs as \isi{deponent} (\citealt{Baerman:2006depo}).

\subsubsection{Positional verbs} \label{positionalverbs}

The class of 41 \isi{positional} or postural verbs underscores the role of \isi{dynamicity} in the \isi{alignment} of S. Positional verbs express states of the type `be in position X' (`be leaning,' `be standing,' `be submerged' etc). Example (\ref{ex328}) shows the verb \emph{migsi} `hang'.\\

\begin{exe}
	\ex \emph{bidrthatha zbo \textbf{sumithgrm} wämnen.}\\
	\glll bidr=thatha zbo su-mi-thgr-m wämne=n\\
	{{flying.fox}=\Simil{}} {\Prox.\All{}} \Tsg.\Masc.\Betaone{}-be.hanging-\Stat.\Ndu-\Dur{} tree=\Loc\\
	{} {} \footnotesize{\Tsg.\Masc:\Sbj:\Pst:\Dur:\Stat/be.hanging} {}\\
	\trans `He was hanging like a flying fox on the tree.'\Corpus{tci20130901-04}{RNA \#48}
	\label{ex328}
\end{exe}

Like most \isi{positional} verbs, \emph{migsi} can enter into other templates, for example a \isi{middle} template (`assume a hanging position') or a \isi{transitive} template (`hang something'). This is shown below in examples (\ref{ex326}) and (\ref{ex327}) respectively. Example (\ref{ex326}) is part of a plant walk around Rouku village. The speaker shows me a plant in the part of the land which is inundated during the rainy season. Example (\ref{ex327}) comes from a procedural text in which the speaker shows me around his yam storage house. He remarks that small yam suckers are called \emph{sagusagu} and they are stored by tying several yams into bundles.\\

\begin{exe}
	\ex \emph{bubukr zä zf kwa \textbf{ŋamigwrth} ... watik kofäyé zbo zf kwa erkunzrth.}\\
	\glll bubukr zä zf kwa ŋ-a-mig-wr-th (.) watik kofä=é zbo zf kwa e-rku-nzr-th\\
	insect \Prox{} \Imm{} \Fut{} \M.\Alph-\Vc-hang.\Ext-\Ndu-\Stnsg{} (.) then fish=\Erg.\Nsg{} \Prox.\All{} \Imm{} \Fut{} \Stnsg.\Alph-knock.down.\Ext-\Ndu-\Stnsg\\
	{} {} {} {} \footnotesize{\Stpl:\Sbj:\Nonpast:\Ipfv/hang} {} {} {} {} {} {} \footnotesize{\Stpl:\Sbj>\Stpl:\Obj:\Nonpast:\Ipfv/knock.down}\\
	\trans `The insects will hang (themselves) from here and the fish will knock them down right here.'\Corpus{tci20130907-02}{RNA \#657}
	\label{ex326}
\end{exe}
\begin{exe}
	\ex \emph{nima yamme ane fof ŋafrmnzre bnrä ... \textbf{bemigwre} ane sagusagu.}\\
	\glll nima yam-me ane fof ŋ-a-frm-nzr-e \\
	like.this custom-\Ins{} \Dem{} \Emph{} \M.\Alph-\Vc-prepare.\Ext-\Ndu-\Fnsg{}\\
	{} {} {} {} \footnotesize{\Fpl:\Sbj:\Nonpast:\Ipfv/prepare}\\
	\sn
	\glll b=n-rä (.) b=e-mig-wr-e ane sagusagu\\
	\Med=\Fnsg.\Alph-\Cop.\Ndu{} (.) \Med=\Stnsg.\Alph-hang.\Ext-\Ndu-\Fnsg{} \Dem{} sagusagu\\
	\footnotesize{\Med=\Fpl:\Sbj:\Nonpast:\Ipfv/be} {} \footnotesize{\Med=\Fpl:\Sbj>\Stpl:\Obj:\Nonpast:\Ipfv/hang} {} {}\\
	\trans `We prepare them in this way ... We hang up those \emph{sagusagu}.'\\\Corpus{tci20121001}{ABB \#38}
	\label{ex327}
\end{exe}

Positionals are attested in languages throughout the Yam family (\citealt{Evans:2014bz}). For Komnzo, I define them as a class of lexemes with \isi{positional} or postural semantics which share the following morphosyntactic properties: (i) the ability to occur in the prefixing template, (ii) the ability to take the \isi{stative} suffix \emph{-thgr}, (iii) the ability to form related \isi{middle} and \isi{transitive} \isi{verb} forms, and (iv) to inflect only for a subset of TAM categories when used in a prefixing template. Table \ref{positional.verbs} lists the 41 members of the class which are currently attested. We find both very general meanings (\emph{rzarsi} `be tied', \emph{yufaksi} `be bent over') and very specific meanings (\emph{rngthksi} `be stuck in a tree fork', \emph{mgthksi} `be in the mouth'). Some of these verbs occur with prototypical participants, for example \emph{zaksi} `be anchored' with \emph{garda} `canoe' or \emph{thamsaksi} `be spread out' with \emph{yame} `mat'.\\

Table \ref{positional.verbs} compares the extended (\Ext) and \isi{restricted stem} (\Rs) and shows that for some verbs a \isi{positional} stem (\Pos) can be postulated. The \isi{positional} stem is the lexical base to which the \isi{stative} suffix \emph{-thgr} attaches. In the first two groups of Table \ref{positional.verbs}, the base is formally identical to the extended or \isi{restricted stem}. Only in the third group, is the base different from both, in that it is always shorter. The last group contains three lexemes which are irregular in a number of ways: (i) they take a slightly different form of the \isi{stative} suffix, which is given in parentheses for each, (ii) the last two lexemes in this group occur only as positionals, (iii) the second lexeme in the group lacks an infinitive.\\

The data from Table \ref{positional.verbs} shows that for some of the verbs we need to posit a third stem type, the \isi{positional} stem, in addition to the extended and restricted stems we already encountered. The formal difference or similarity between the \isi{positional} stem and the other two stem types for a given lexeme cannot be predicted on semantic or phonological grounds, but must be seen as lexicalisation in a specific morphosyntactic context. Furthermore, one should keep in mind that \isi{positional} stems are not in a paradigmatic relationship of the kind we have seen with extended and restricted stems (\S{}\ref{roots-and-temp}). For example, the \isi{stative} semantics of positionals blocks all \isi{perfective} TAM categories.


\begin{table}
\caption{Positional verbs}
\label{positional.verbs}
{\small%
\begin{tabular}{lllll}
	\lsptoprule
	\textsc{infinitive} & \Pos{} \textsc{stem} & \Ext{} \textsc{stem} 	& \Rs{} \textsc{stem} 	& \textsc{gloss} \\\midrule
	\emph{mosisi} &\emph{mosi-} &\emph{mosi-} &\emph{mosir-} &be gathered, piled\\
	\emph{moyusi} &\emph{moyu-} &\emph{moyu-} &\emph{moyuth-} &be shrunk\\
	\emph{rfakusi} &\emph{rfaku-} &\emph{rfaku-} &\emph{rfakuth-} &be sprinkled\\
	\emph{ttüsi} &\emph{ttü-} &\emph{ttü-} &\emph{ttüth-} &be printed, carved\\
	\emph{tharasi} &\emph{thar-} &\emph{thar-} &\emph{tharf-} &be underneath\\
	\emph{worsi} &\emph{wor-} &\emph{wor-} &\emph{won-} &be planted\\\midrule
	\emph{brüzsi} &\emph{brüs-} &\emph{brüz-} &\emph{brüs-} &be submerged\\
	\emph{krsi} &\emph{kr-} &\emph{krth-} &\emph{kr-} &be blocked off\\
	\emph{räzsi} &\emph{räs-} &\emph{räz-} &\emph{räs-} &be erected\\
	\textsuperscript{a}\emph{rfuthraksi} &\emph{rfuth-} &\emph{rfuthrak-} &\emph{rfuthr-} &be piled up\\
	\emph{rmithraksi} &\emph{rmithr-} &\emph{rmithrak-}	&\emph{rmithr-} &be joined together\\
	\emph{rmnzüfaksi} &\emph{rmnzüf-} &\emph{rmnzüfak-}	&\emph{rmnzüf-} &be side by side / parallel\\
	\emph{rthbraksi} &\emph{rthbr-} &\emph{rthbrak-} &\emph{rthbr-} &be sticking (on sth.)\\
	\emph{rzarsi} &\emph{rzaf-} &\emph{rzar-} &\emph{rzaf-} &be tied\\
	\emph{thamsaksi} &\emph{thams-} &\emph{thamsak-} &\emph{thams-} &be spread out\\
	\textsuperscript{a}\emph{yufaksi} &\emph{yuf-} &\emph{yufak-} &\emph{yuf-} &be bent\\
	\emph{zaksi} &\emph{z-} &\emph{zak-} &\emph{z-} &be anchored\\\midrule
	\emph{fätfaksi} &\emph{fät-} &\emph{fätfak-} &\emph{fätf-} &be across sth.\\
	\emph{fethaksi} &\emph{fe-} &\emph{fethak-} &\emph{feth-} &be dipped in water\\
	\emph{fifthaksi} &\emph{fif-} &\emph{fifthak-} &\emph{fifth-} &be lying straight\\
	\emph{migsi} &\emph{mi-} &\emph{mig-} &\emph{mir-} &be hanging\\
	\emph{moraksi} &\emph{mo-} &\emph{morak-} &\emph{mor-} &be leaning\\
	\textsuperscript{a}\emph{mgthksi} &\emph{mg-} &\emph{mgthk-} &\emph{mgthm-} &be in the mouth\\
	\emph{mreznsi} &\emph{mre-} &\emph{mrezn-} &\emph{mrezn-} &be straight\\
	\textsuperscript{a}\emph{mtheksi} &\emph{mthe-} &\emph{mthek-} &\emph{mthef-} &be lifted up \\
	\emph{myuknsi} &\emph{myu-}& \emph{myukn-} &\emph{myuf-} &be twisted\\
	\emph{nänzüthzsi} &\emph{nänzü-}& \emph{nänzüthz-} &\emph{nänzütham-} &be covered with soil\\
	\emph{rafigsi} &\emph{rafi-}&\emph{rafig-} &\emph{rafinz-} &be on top of sth.\\
	\emph{rakthksi}	&\emph{rak-}&\emph{rakthk-} &\emph{rakthm-} &be on top of sth.\\
	\emph{rinaksi} &\emph{ri-}& \emph{rinak-} &\emph{rin-} &be poured into\\
	\emph{rngthksi}	&\emph{rng-}& \emph{rngthk-} &\emph{rngthm-} &be in a tree fork\\
	\textsuperscript{a}\emph{rgsi}	&\emph{rk-} &\emph{rg-} &\emph{rg-} &be wearing clothes\\
	\emph{sisraksi} &\emph{si-}& \emph{sisrak-}	&\emph{sisr-} &be sticking out of sth.\\
	\emph{sümraksi} &\emph{süm-}& \emph{sümrak-} &\emph{sümr-} &be widened, be open\\
	\emph{thäfrsi} &\emph{thäfrs-}& \emph{thäf-} &\emph{thäfrs-} &be covered\\
	\emph{tharuksi} &\emph{tharu-}& \emph{tharuk-} &\emph{tharuf-} &be inside (open container)\\
	\emph{ththaksi} &\emph{th-}& \emph{ththak-}	&\emph{ththm-} &be pinned on sth.\\
	\emph{wäthsi} &\emph{wä-}& \emph{wäth-}	&\emph{wäf-} &be wrapped\\\midrule
	\emph{thorsi} &\emph{th-(kgr)}& \emph{thor-} &\emph{thb-} &be inside (closed container)\\
	n/a &\emph{wä-(gr)} &n/a &n/a &be up high\\
	\emph{yukrasi} &\emph{ko-(gr)} &n/a &\emph{-kuk} &be standing\\
	\lspbottomrule
	\multicolumn{5}{l}{{\footnotesize{\textsuperscript{a} These verbs are \isi{deponent}, i.e. they use the \Vc{} prefix obligatorily.}}}\\
\end{tabular}}
\end{table}%prefixing verbs

Just like other verbs in the prefixing template, positionals may add a \isi{possessor} or \isi{beneficiary} by using the \isi{valency change} prefix \emph{a-}. An example of this is given in (\ref{ex158}) where the speaker describes how he carried two fish up from the river. The first verb in (\ref{ex158}) indexes the two catfish, but the second verb indexes a first singular, in this case the \isi{possessor} (`my shoulder'). Thus, although the predicate is about the two fish (`They were on top.'), the verb only indexes the first singular.

\begin{exe}
	\ex \emph{thwä \textbf{femithgrn} zane zazame \textbf{nwanwägr} ... fatren.}\\
	\glll thwä f-e-mi-thgrn zane {zaza=me}\\
	catfish \Dist=\Stnsg{}:\Alph-be.hanging-\Stat.\Du{} \Prox{} {carrying stick}=\Ins{}\\
	{} \footnotesize{\Dist{}=\Stdu:\Sbj:\Nonpast:\Stat/be.hanging} {} {}\\
	\sn
	\glll n=wo-a-n-wä-gr (.) fatr=en\\
	\Immpst=\Fsg-\Vc-\Venit-be.on.top-\Stat.\Ndu{} (.) shoulder=\Loc\\
	\footnotesize{\Fsg:\Io{}:\Immpst:\Stat:\Venit/be.on.top} {} {}\\
	\trans `Those two catfish are hanging there. I just brought them here on my shoulder with the carrying stick.'\Corpus{tci20121008-03}{MAB \#13}
	\label{ex158}
\end{exe}

As Table \ref{positional.verbs} shows, there are a five out of 41 \isi{positional} verbs which I analyse as \isi{deponent}, i.e. they take the \emph{a-} prefix obligatorily without adding an additional argument to the clause.

\subsection{The middle template} \label{middletemplatesubsection}

The majority of verb stems can enter into what I call the \isi{middle} template. In the \isi{middle} template, the prefix slot is filled by a person-invariant \isi{middle} marker (glossed as \M{}) and the single argument is cross-referenced in the suffix. In addition, the \isi{valency change} prefix \emph{a-} is employed. As we will see below, the suffix in this template may cross-reference an A, S or P argument. The distinction is signalled by the \isi{case} marking on the \textsc{np} (\isi{ergative} vs. \isi{absolutive}).\\

I employ the term ``\isi{middle}'', as defined by Kemmer (\citeyear[207-210]{Kemmer:1993wda}) for situation types with a low degree of elaboration. Low degree of elaboration may refer to the event and/or to the participants involved in the event. The \isi{middle} template in Komnzo covers a range of functions: intransitives, passive-impersonals, reflexives and reciprocals as well as \isi{suppressed-object} middles (or antipassives). Kemmer describes these events as typical ``\isi{middle} situation types'' (\citeyear[15]{Kemmer:1993wda}).

\begin{table}
\caption{Intrinsic middle verbs}
\label{intrinsicmiddleverbs}
	\begin{tabular}{lll}
		\lsptoprule
		\textsc{infinitive} & \Ext{} \textsc{stem}		& \textsc{gloss}\\  \midrule
		\textsuperscript{a}\emph{moth}		& \emph{kwi-}				& `run'\\
		\emph{mränzsi}		& \emph{mränz-}				& `stroll'\\
		\emph{sogsi}		& \emph{sog-}				& `ascend, climb up'\\
		\emph{rsörsi}		& \emph{rsör-}				& `descend, climb down'\\
		\textsuperscript{a}\emph{mni}		& \emph{rsir-}				& `burn, cook' (v.i.)\\
		\emph{müsinzsi}		& \emph{müsinz-}			& `glow'\\
		\emph{rfeksi}		& \emph{rfek-}				& `limp'\\
		\emph{frezsi}		& \emph{frez-}				& `come up (from river)'\\
		\emph{risoksi}		& \emph{risok-}				& `look down'\\
		\emph{rnäthsi}		& \emph{rnäth-}				& `get stuck'\\
		\emph{rninzsi}		& \emph{rninz-}				& `smile'\\
		\textsuperscript{a}\emph{wath}		& \emph{rnzür-}				& `dance'\\
		%\emph{yonasi}		& \emph{na-}				& `drink'\\
		\emph{rüsi}			& \emph{rü-}				& `rain'\\
		\emph{sufaksi}		& \emph{sufak-}				& `gulp down, guzzle'\\
		\emph{fänizsi}		& \emph{fäniz-}				& `shift location'\\
		%\emph{fsisi}		& \emph{fsi-}				& `count'\\
		\emph{bznsi}		& \emph{bzn-}				& `work'\\
		\emph{thärkusi}		& \emph{thärku-}			& `crawl'\\
		\emph{farksi}		& \emph{fark-}				& `set off'\\
		%\emph{kwthenzsi}	& \emph{kwthe-}				& `change'\\
		\emph{fsknsi}		& \emph{fskn-}				& `doze'\\
		\emph{borsi}		& \emph{bor-}				& `laugh, play'\\
		\emph{thweksi}		& \emph{thwek-}				& `rejoice'\\
		n/a					& \emph{ko-}				& `become'\\
		n/a					& \emph{rä-}				& `do, think'\\
		\lspbottomrule
		\multicolumn{3}{l}{{\footnotesize \textsuperscript{a} These verbs employ a common noun as their infinitive}}\\
	\end{tabular}
\end{table}%intrinsic middle verbs

In\isi{transitive} event types in Komnzo are distributed over the prefixing and the \isi{middle} template (see \S{}\ref{prefixingverbsec}). The majority of syntactically \isi{intransitive} verbs employ the middle template. As a consequence for the description of the middle template, we have to draw a distinction between intrinsic middle verbs and derived middle verbs. Intrinsic middles can only occur in the middle template. Derived middle verbs are derived from \isi{transitive} verbs, whereby the middle template is used for different \isi{valency} decreasing functions. There is a third group of verb stems, which almost always occur in the middle template, but with which a derived \isi{transitive} or \isi{ditransitive} is possible. These groups will be discussed below. For now, the main distinction is between verbs, for which the middle template is one strategy amongst others and verbs, which only occur in the middle template. I call the latter intrinsic middle verbs.\\

Some intrinsic middle verbs are listed in Table \ref{intrinsicmiddleverbs}. In her cross-linguistic survey, Kemmer identifies a number of situation types which commonly occur with \isi{middle} morphology (\citeyear[16-21]{Kemmer:1993wda}). In Komnzo these are: translational motion (`run', `climb up', `climb down', `shift location'), emotion \isi{middle} (`laugh', `rejoice', `smile'), cognition \isi{middle} (`think') and spontaneous events (`change', `become'). The tendency to encode intransitive verbs with a dynamic event type in the \isi{middle} template has been discussed above in \S \ref{prefixingverbsec}.\\

In addition to intrinsic \isi{middle} verbs, most \isi{verb} stems can occur in the \isi{middle} template with various related functions. One such verb is \emph{brigsi} `return'. In the examples (\ref{ex161}) and (\ref{ex162}), the S argument is indexed in the suffix, while the prefix is filled with the middle morpheme. Since there is no formal difference in the \isi{middle} template between intransitives, impersonals and reflexives, these should be understood as reflexiva tanta (\citealt{Geniusienie:1987refl}) and example (\ref{ex161}) could also be translated as `I return myself'.

\begin{exe}
	\ex \emph{oh nzä karfo zena zf \textbf{ŋabrigwé}.}\\
	\glll oh nzä kar=fo zena zf ŋ-a-brig-w-é\\
	oh \Fsg.\Abs{} village=\All{} today \Imm{} \M.\Alph-\Vc-return.\Ext-\Ndu-\Fsg\\
	{} {} {} {} {} \footnotesize{\Fsg:\Sbj:\Nonpast:\Ipfv/return}\\
	\trans `Oh, now I will go back to the village.'\Corpus{tci20111004}{RMA 437}
	\label{ex161}
\end{exe}
\begin{exe}
	\ex \emph{oh kaimätdbo fam \textbf{ŋabrigwrth}.}\\
	\glll oh kaimät=dbo fam ŋ-a-brig-w-r-th\\
	oh sister.in.law=\All.\Anim{} thoughts \M.\Alph-\Vc-return.\Ext-\Ndu-\Lk-\Stnsg\\
	{} {} {} \footnotesize{\Stpl:\Sbj:\Nonpast:\Ipfv/return}\\
	\trans `Oh, (my) thoughts are returning to my sister-in-law.'\Corpus{tci20130907-02}{JAA 665}
	\label{ex162}
\end{exe}

Examples (\ref{ex159}-\ref{ex329}) show \emph{brigsi} in different templates. Both examples are taken from the same story about a headhunt which took place in the narrator's village Firra. In (\ref{ex159}), the ambifixing \isi{transitive} template is used (Lit. `They returned the payback'). Just a few clauses later, the narrator concludes this part of the story in (\ref{ex329}) where the same referent, which was indexed in the prefix in (\ref{ex159}), is now indexed in the suffix with a \isi{passive} or \isi{impersonal} interpretation (Lit. `Revenge (was) returned').

\begin{exe}
	\ex
	\begin{xlist}
	\ex	\emph{okay, nafa nezä z faw \textbf{wbrigrnath} ... bänema nafanme mayawa kakafar z bramöwä thäkwrath firran.}\\
	\glll okay nafa nezä z faw w-brig-r-n-a-th (.) bäne=ma nafanme mayawa ka-kafar z bramöwä\\
	okay \Tnsg.\Erg{} revenge \Iam{} payment \Tsg.\F.\Alph-return.\Ext-\Lk-\Du-\Pst-\Stnsg{} (.) \Dem:\Med=\Char{} \Tnsg.\Poss{} mayawa \Redup-big \Iam{} all\\
	{} {} {} {} {} \footnotesize{\Stdu:\Sbj>\Tsg.\F:\Obj:\Pst:\Ipfv/return} {} {} {} {} {} {} {}\\
	\sn
	\glll th-ä-kwr-a-th firra=n\\
	\Stnsg.\Gam-\Vc\textbar\Ndu-hit.\Rs-\Pst-\Stnsg{} firra=\Loc\\
	\footnotesize{\Stpl:\Sbj>\Stpl:\Obj:\Pst:\Pfv/kill} {}\\
	\trans `Okay, then the two took revenge, because all their Mayawa elders had been killed in Firra.'\Corpus{tci20111107-01}{MAK 126-127}
	\label{ex159}

	\ex	\emph{watik, faw z \textbf{ŋabrigwa} ane ... ane ebar nimame firran rera fof.}\\
	\glll watik faw z ŋ-a-brig-w-a-\Zero{} ane (.) ane ebar nima=me firra=n rä-r-a fof\\
	then payment \Iam{} \M.\Alph-\Vc-return.\Ext-\Ndu-\Pst-\Stsg{} \Dem{} (.) \Dem{} head like.this=\Ins{} firra=\Loc{} \Tsg.\F.\Cop-\Lk-\Pst{} \Emph\\
	{} {} {} \footnotesize{\Stsg:\Sbj:\Pst:\Ipfv/return} {} {} {} {} {} {} \footnotesize{\Tsg.\F:\Sbj:\Pst:\Ipfv/be} {}\\
	\trans `Then, revenge was taken. This is really how the head(hunting) took place in Firra.'\Corpus{tci20111107-01}{MAK 134-135}
	\label{ex329}
	\end{xlist}
\end{exe}

Consequently, I refrain from using the terms `\isi{middle} voice' or `\isi{passive} voice'. It is more adequate to speak of a \isi{middle} template with a specific function. This function might be \isi{reflexive}, \isi{reciprocal}, \isi{passive} or \isi{impersonal}. Consider example (\ref{ex163}) below, in which the speaker describes how he got home after a hard day of work in his garden. The first two verbs in (\ref{ex163}) are prefixing verbs. The last three verbs occur in the \isi{middle} template and could be translated as either \isi{reflexive} (`wash self', `change self', `bring oneself up from river') or intransitives (`wash', `get changed', `come up from the river').\footnote{Note that `get changed' is expressed with a nominal \emph{sänis} (< \ili{English} `change') and a generic verb `do', literally `I do the change'. The nominal is not indexed in the verb. I describe light verb constructions in \S\ref{lightverb}.}

\begin{exe}
	\ex \emph{yoganai worärm, kwofiyak, \textbf{kwamaikwé}, sänis \textbf{kwaräré}, \textbf{zänfrefé}.}\\
	\glll yoganai wo-rä-r-m kwof-yak kw-a-mayk-w-é sänis kw-a-rä-r-é z-ä-n-fref-é\\
	tiredness \Fsg.\Alph-be-\Lk-\Dur{} \Fsg.\Betatwo-walk.\Ext.\Ndu{} \M.\Betaone-\Vc-wash.\Ext-\Ndu-\Fsg{} change \M.\Betaone-\Vc-do.\Ext-\Lk-\Fsg{} \M.\Gam-\Vc.\Ndu-\Venit-come.up.from.river.\Rs-\Fsg\\
	{} \footnotesize{\Fsg:\Sbj:\Rpst:\Dur/be} \footnotesize{\Fsg:\Sbj:\Rpst:\Ipfv/walk} \footnotesize{\Fsg:\Sbj:\Rpst:\Ipfv/wash} {} \footnotesize{\Fsg:\Sbj:\Rpst:\Ipfv/do} \footnotesize{\Fsg:\Sbj:\Rpst:\Pfv:\Venit/come-up-from-river}\\
	\trans `I was tired. I walked. I washed myself. I got changed and I came up here from the river.'\Corpus{tci20120922-24}{MAA 78-80}
	\label{ex163}
\end{exe}

We find the same ambiguity between \isi{reflexive} and \isi{reciprocal} interpretations. In (\ref{ex164}), the speaker describes how his ancestors used to live in small hamlets which comprised a clan or often a single patriline. The \isi{reciprocal} interpretation of the second verb only comes from the context. The verb form \emph{kwamarwrme} in a different context could equally be translated as a \isi{reflexive}: `We were looking at ourselves'.

\begin{exe}
	\ex \emph{mrnmenzo nzwamnzrm. zagr sime \textbf{kwamarwrme}.}\\
	\glll mrn=me=nzo nzu-a-m-nz-r-m zagr si=me\\
	clan=\Ins=\Only{} \Fnsg.\Betaone-\Vc-dwell.\Ext-\Ndu-\Lk-\Dur{} far eye=\Ins{}\\
	{} \footnotesize{\Fpl:\Sbj:\Pst:\Dur/dwell} {} {}\\
	\sn
	\glll kw-a-mar-w-r-m-e\\
	\M.\Betaone-\Vc-see-\Lk-\Dur-\Fnsg\\
	\footnotesize{\Fpl:\Sbj:\Pst:\Dur/see}\\
	\trans `We used to stay in our clans. We saw each other only from a distance.'\\\Corpus{tci20120922-08}{DAK 117-118}
	\label{ex164}
\end{exe}

We have seen an \isi{impersonal} usage of the \isi{middle} template in (\ref{ex329}) above. An example with a much clearer \isi{passive} reading is provided in (\ref{ex168}) below, where the speaker talks about sorting and selecting yam tubers in his storage house. The context reveals that it is the \isi{patient} argument of the verbs (`choose', `put down') which is indexed in the suffix. Keenan and Dryer include the entailment of an agent in their definition of passives setting them apart from middles (\citeyear[352]{Keenan:2007passives}). In Komnzo, this is dependent on the semantics of the verb. Prototypical \isi{transitive} verbs, like `choose' and `put down' in (\ref{ex168}), invite a \isi{passive} interpretation rather than an \isi{impersonal} one. However, in terms of morpho-syntax, there is no dedicated \isi{passive} marking. Furthermore, the agent noun phrase cannot be included in the clause, because it would have to be indexed in the suffix of the verb, which is already occupied by the \isi{patient} argument.

\begin{exe}
	\ex \emph{zane zf woksimär erä. gaba foba fof \textbf{kräwokthth} bobo we kwa \textbf{ŋanakwrth} a nima berä.}\\
	\glll zane zf wok-si=mär e-rä gaba foba fof k-ra-a-wokth-th bobo we kwa\\
	\Dem:\Prox{} \Imm{} choose-\Nmlz=\Priv{} \Stnsg.\Alph-\Cop.\Ndu{} {eating yam} \Dist.\Abl{} \Emph{} \M.\Bet{}-\Irr-\Vc\textbar\Ndu-choose.\Rs-\Stnsg{} \Med.\All{} also \Fut{}\\
	{} {} {} \footnotesize{\Stpl:\Sbj:\Nonpast:\Ipfv/be} {} {} {} \footnotesize{\Stpl:\Sbj:\Irr:\Pfv/choose} {} {} {}\\
	\sn
	\glll ŋ-a-nak-w-r-th a nima b=e-rä\\
	\M.\Alph-\Vc-put.down.\Ext-\Ndu-\Lk-\Stnsg{} and like.this \Med=\Stnsg.\Alph-\Cop.\Ndu{}\\
	\footnotesize{\Stpl:\Sbj:\Nonpast:\Ipfv/put.down} {} {} \footnotesize{\Med=\Stpl:\Sbj:\Nonpast:\Ipfv/be}\\
	\trans `These have not been selected. They will be selected over there and then put down there like those ones.'\Corpus{tci20121001}{ABB 41-42}
	\label{ex168}
\end{exe}

A somewhat different function of the \isi{middle} template is the \isi{suppressed-object} \isi{middle}. The formal difference with respect to the previous functions of the \isi{middle} template lies in the marking of the \textsc{np}, which receives an \isi{ergative}. Thus, the argument is an actor and the event is inherently \isi{transitive}. Consider example (\ref{ex165}), which is taken from a conversation between two young men. The speaker reports to his friend what his wife thinks about his plan to shift the garden place to another location. In (\ref{ex165}), the \isi{pronoun} \emph{naf} is in the \isi{ergative} \isi{case} and agrees with the verb \emph{ŋanafr} which is in the \isi{middle} template. The \isi{object} is suppressed from indexation and without context we are left to speculate what it might be: the \isi{goal} (`she said to me') or the clausal \isi{theme} (`to continue the old garden').

\begin{exe}
	\ex \emph{naf \textbf{ŋanafr} drdr mäyogsir.}\\
	\glll naf ŋ-a-na-f-r-\Zero{} drdr mäyog-si=r\\
	\Tsg.\Erg{} \M-\Vc-speak.\Rs-\Ndu-\Lk-\Stsg{} old.garden repeat-\Nmlz=\Purp\\
	{} \footnotesize{\Stsg:\Sbj:\Nonpast.\Ipfv/speak} {} {}\\
	\trans `She suggested/said to continue the old garden.'\Corpus{tci20130823-06}{STK 161}
	\label{ex165}
\end{exe}

The \isi{suppressed-object} \isi{middle} is obligatory for a few lexemes, for example \emph{na-} `speak (v.t.)' in (\ref{ex165}), \emph{karksi} `pull (v.t.)' or \emph{yonasi}\footnote{Interestingly, `drink' and `eat' share the same extended stem (\emph{na}), but `eat' almost always occurs in an ambifixing transitive template and it employs a common noun as its infinitive (\emph{dagon} `food'). The verb `drink' on the other hand employs the infinitive \emph{yonasi} with a regular nominaliser suffix and it always occurs in a (suppressed-object) middle template. The restricted stems of `drink' and `eat' are different: \emph{nob} and \emph{wob} respectively.} `drink (v.t.)'. For most verbs, the \isi{suppressed-object} \isi{middle} is a possible alternation and should be seen as derived from verbs which normally employ an ambifixing \isi{transitive} template.\\

There are pragmatic reasons for suppressing the \isi{object}, for example when the referent is common ground or when the event is somehow generic.\footnote{During the translation of texts, consultants would often rephrase the suppressed-object middle with a generic event (`He did the X-ing') instead of a specific event (`He X-ed it').} These motivations can be subsumed under Kemmer's criterion of low degree of (\isi{participant}) elaboration with \isi{middle} morphology. Consider example (\ref{ex167}), where the speaker talks about how yams are stored. He says that the yams are heaped and sorted into separate piles and that the spatial layout signals the use of the yams. This last proposition is expressed as \emph{naf ŋatrikwr} `it indicates'. The verb \emph{trikasi} `tell' is usually used for story telling or for reporting on something, but the event depicted in example (\ref{ex167}) is generic and less elaborated.

\begin{exe}
	\ex \emph{mnz mrmr fof enakwre zena monwä zane ethn zerä. \textbf{naf ŋatrikwr} zane zf ŋatr wawa erä zerä. zane gaba zf erä zerä.}\\
	\glll mnz mrmr fof e-nak-w-r-e zena mon-wä zane e-thn z=e-rä naf\\
	house inside \Emph{} \Stnsg.\Alph-put.down.\Ext-\Ndu-\Lk-\Fnsg{} now how-\Emph{} \Dem:\Prox{} \Stnsg.\Alph-lie.down.\Ext.\Ndu{} \Prox=\Stnsg.\Alph-\Cop.\Ndu{} \Tsg.\Erg{}\\
	{} {} {} \footnotesize{\Fpl:\Sbj>\Stpl:\Obj:\Nonpast{}:\Ipfv/put-down} {} {} {} \footnotesize{\Prox=\Stpl:\Nonpast{}:\Ipfv/lie.down} \footnotesize{\Prox=\Stpl:\Nonpast{}:\Ipfv/be} {}\\
	\sn
	\glll ŋ-a-trik-w-r-\Zero{} zane zf ŋatr wawa\\
	\M.\Alph-\Vc-tell.\Ext-\Ndu-\Lk-\Stsg{} \Dem:\Prox{} \Imm{} rattan.vine yam\\
	\footnotesize{\Stsg:\Sbj:\Nonpast:\Ipfv/tell} {} {} {} {}\\
	\sn
	\glll e-rä z=e-rä zane gaba zf\\
	\Stnsg.\Alph-\Cop.\Ndu{} \Prox=\Stnsg.\Alph-\Cop.\Ndu{} \Dem:\Prox{} {eating yam} \Imm{}\\
	\footnotesize{\Stpl:\Sbj:\Nonpast{}:\Ipfv/be} \footnotesize{\Prox=\Stpl:\Sbj:\Nonpast{}:\Ipfv/be} {} {} {}\\
	\sn
	\glll e-rä z=e-rä\\
	\Stnsg.\Alph-\Cop.\Ndu{} \Prox=\Stnsg.\Alph-\Cop.\Ndu{}\\
	\footnotesize{\Stpl:\Sbj:\Nonpast{}:\Ipfv/be} \footnotesize{\Prox=\Stpl:\Sbj:\Nonpast{}:\Ipfv/be}\\
	\trans `We put (the yams) down in the house, how these are laying here. That will indicate that these are measuring yams\footnote{\emph{ŋatr} is a rattan piece which is often used to measure the dimensions of a particularly big tuber. Large yams are used in competitions or as special gifts.} here and these are eating yams here.'\Corpus{tci20121001}{ABB 15-16}
	\label{ex167}
\end{exe}

Another motivation for suppressing the \isi{object}, partly relevant to the previous example, lies in the relative salience of the referent. There is a tendency for \isi{inanimate} referents not to be indexed, as we can see in example (\ref{ex166}). This example is taken from a stimulus task about domestic violence. The speaker takes over the role of one of the characters in the story. He uses the verb \emph{fiyoksi} `make' twice, first in a \isi{middle} template and then in a \isi{transitive} template.\footnote{As we will see in \S{}\ref{ambifixingtemp}, some transitive verbs like \emph{fiyoksi} obligatorily take the valency change prefix \emph{a-}. Since the argument is in absolutive case, one would expect the inflected verb to be \emph{wfiyokwr} (without the \emph{a-} prefix). But this is ungrammatical and \emph{fiyoksi} never occurs without the \emph{a-} prefix. Thus, I regard \emph{fiyoksi} and similar verbs as being deponent.} The crucial difference between the two situation types lies in the salience of the referent. In the first clause the referent is generic and \isi{inanimate} (\emph{yam} `custom, event'), but in the second clause it is a close relative (\emph{nzenme emoth} `our sister').

\begin{exe}
	\ex \emph{``be nima yam \textbf{ŋafiyokwr}. nzenme emoth be nima \textbf{wäfiyokwr}!''}\\
	\glll be nima yam ŋ-a-fiyok-w-r-\Zero{} nzenme emoth be nima w-a-fiyok-w-r-\Zero{}\\
	\Ssg.\Erg{} like.this event \M.\Alph-\Vc-make.\Ext-\Ndu-\Lk-\Stsg{} \Fnsg.\Poss{} sister \Ssg.\Erg{} like\.his \Tsg.\F.\Alph-\Vc-make.\Ext-\Ndu-\Lk-\Stsg\\
	{} {} {} \footnotesize{\Stsg:\Sbj:\Nonpast:\Ipfv/make} {} {} {} {} \footnotesize{\Stsg:\Sbj>\Tsg.\F:\Obj:\Nonpast:\Ipfv/make}\\
	\trans ``You are behaving like this. You are doing this to our sister.''\\\Corpus{tci20120925}{MAE 89}
	\label{ex166}
\end{exe}

We can conclude that intrinsic middles are \isi{intransitive} event types, but the \isi{middle} template is used for various functions. The uniting characteristic of these functions is a relatively low degree of elaboration. This may apply either to the participants (\ref{ex166}), i.e. they rank low in importance/salience, or to the event itself (\ref{ex167}), i.e. the event is less elaborated.

\subsection{The ambifixing template} \label{ambifixingtemp}

The ambifixing template employs both affixes to index referents. The \isi{subject} argument appears in the suffix, while the \isi{object} argument is indexed in the prefix (\ref{ex170}).

\begin{exe}
	\ex \emph{gwamf nafangth \textbf{sräkor}: ``muri zba känrit nzuzawe!''}\\
	\glll gwam=f nafa-ngth s-ra-a-kor-\Zero{} muri zba k-ä-n-rit-\Zero{} nzu-zawe\\
	gwam=\Erg{} \Third.\Poss-younger.sibling \Tsg.\Masc.\Bet-\Irr-\Ndu-say.\Rs-\Stsg{} muri \Prox.\Abl{} \M.\Bet-\Ndu-\Venit-cross.over.\Rs-\Ssg.\Imp{} \Fsg.\Poss-side\\
	{} {} \footnotesize{\Stsg:\Sbj>\Tsg.\Masc:\Obj:\Irr:\Pfv/says} {} {} \footnotesize{\Ssg:\Sbj:\Imp:\Pfv:\Venit/cross.over} {}\\
	\trans `Gwam said to his brother: ``Muri, come over here to my side!'''\\\Corpus{tci20131013-01}{ABB \#96}
	\label{ex170}
\end{exe}

In most cases, the suffix indexes an Agent, as in (\ref{ex170}) above. Example (\ref{ex169}) shows an \isi{experiencer-object} construction, in which the suffix encodes a Stimulus. After an evening of stories about sorcery, the speaker announces that she will go to sleep now because `fear has taken hold of her already'.

\begin{exe}
	\ex \emph{nze rokar kwa thräfrmsé. wtrif z \textbf{zwefaf}.}\\
	\glll nze rokar kwa th-ra-a-frms-é wtri=f z zu-ä-faf-\Zero\\
	\Fsg.\Erg{} thing \Fut{} \Stnsg.\Bet-\Irr-\Vc\textbar\Ndu-prepare.\Rs-\Fsg{} fear=\Erg.\Sg{} \Iam{} \Fsg.\Gam-\Ndu-hold.\Rs-\Stsg{}\\
	{} {} {} \footnotesize{\Fsg:\Sbj>\Stpl:\Obj:\Irr:\Pfv/prepare} {} {} \footnotesize{\Stsg:\Sbj>\Fsg:\Obj:\Rpst:\Pfv/hold}\\
	\trans `I will prepare (my) things. I am already scared.'\Corpus{tci20130901-04}{RNA \#164}
	\label{ex169}
\end{exe}

Since no more than two referents can be indexed on a \isi{verb}, the same ambifixing template encodes \isi{transitive} and \isi{ditransitive} events. The differences lie in the presence versus absence of the \isi{valency change} prefix \emph{a-} and the \isi{case} marking of that argument NP which is indexed in the prefix. In ambifixing transitives, the prefix encodes a Patient (`prepare' in \ref{ex169}), Theme (\ref{ex170}) or Experiencer (`hold' in \ref{ex169}), all in the \isi{absolutive}. The prefix in ambifixing ditransitives encodes a Goal (\ref{ex173}) in \isi{dative} \isi{case} or a Possessor (\ref{ex171}) marked with a \isi{possessive}.

\begin{exe}
	\ex \emph{nzun nafaemoth \textbf{zwärath} fof ... bänemr ... fäms ŋarer}\\
	\glll nzun nafa-emoth zu-ä-r-a-th fof (.) bäne=mr\\
	\Fsg.\Dat{} \Third.\Poss-sister \Fsg.\Gam-\Vc.\Ndu-give.\Rs-\Pst-\Stnsg{} \Emph{} (.) \Recog=\Purp{}\\
	{} {} \footnotesize{\Stpl:\Sbj>\Fsg:\Io:\Pst:\Pfv/give} {} {} {} {} {}\\
	\sn
	\gll (.) fäms ŋare=r\\
	(.) exchange woman=\Purp\\
	\trans `They gave me their sister as that ... as an exchange woman.'\\\Corpus{tci20120805-01}{ABB \#791-792}
	\label{ex173}
\end{exe}
\begin{exe}
	\ex \emph{nzone miyo kwa \textbf{wabthakwr}.}\\
	\glll nzone miyo kwa wo-a-bthak-w-r-\Zero{}\\
	\Fsg.\Poss{} desire \Fut{} \Fsg.\Alph-\Vc-finish.\Ext-\Lk-\Stsg{}\\
	{} {} {} \footnotesize{\Stsg:\Sbj>\Fsg:\Io:\Nonpast:\Ipfv/finish}\\
	\trans `You will fulfill my wish.'\Corpus{tci20130823-06}{CAM \#23}
	\label{ex171}
\end{exe}

Because the \isi{middle} template is used for reflexives, the two argument slots of the ambifixing template may not be coreferential. Thus, if we wanted to change example (\ref{ex171}) above to an auto-\isi{benefactive} (`I fulfill my wish / I fulfill the wish for me'), it would be ungrammatical to say \textsuperscript{$\ast$}\emph{nzone miyo wabthakw\underline{é}}. The underlined segment in the verb marks the actor as first singular. Instead, one would have to employ a \isi{middle} construction for the verb: \emph{nzone miyo ŋabthakwé.}\\

Example (\ref{ex172}) shows both a \isi{possessor} and a \isi{goal} in the first and second verb form respectively. The example is taken from a story about sorcerers, who \textendash{} according to local belief \textendash{} visit the grave sites of recently deceased people. The first clause shows that the \isi{possessor} noun phrase can be dropped. The noun \emph{mitafo} `spirit' is usually feminine, but the verb encodes a masculine referent (`his spirit').

\begin{exe}
	\ex \emph{befé mitafo \textbf{sabrim} nzun fefe \textbf{kwagathif}!}\\
	\glll be-wä mitafo s-a-brim-\Zero{} nzun fefe kw-a-gathif-\Zero{}\\
	\Ssg.\Erg-\Emph{} spirit \Tsg.\Masc.\Bet{}-\Vc.\Ndu-return.\Rs-\Ssg.\Imp{} \Fsg.\Dat{} body \Fsg.\Bet-\Vc.\Ndu-leave.behind.\Rs-\Ssg.\Imp{}\\
	{} {} \footnotesize{\Ssg:\Sbj>\Tsg.\Masc:\Io:\Imp:\Pfv/return} {} {} \footnotesize{\Ssg:\Sbj>\Fsg:\Io:\Imp:\Pfv/leave.behind}\\
	\trans `You take his spirit back and leave the body for me!'\Corpus{tci20130903-04}{RNA \#92-93}
	\label{ex172}
\end{exe}

Example (\ref{ex172}) highlights a problem that occurs with \isi{verb} forms using the restricted stem. As I have shown in \S{}\ref{dualextrs}, with restricted stems the dual versus non-dual contrast and the \isi{valency change} is expressed by a vowel change in the prefix. Although there are differences in the vowel pattern for different \isi{number} combinations, which show the absence versus presence of the \isi{valency change} prefix, there are a number of neutralisations (\S{}\ref{prerootdual}). The first verb \emph{sabrim} in example (\ref{ex172}) can mean both `return him' (with a direct \isi{object}) or `return X for him' / `return his X' (with an \isi{indirect object}). Only the fact that \emph{mitafo} `spirit' is feminine, while the prefix is governed by a masculine referent, indicates that the \isi{indirect object} is indexed (`return his spirit').\\

The \isi{valency change} prefix \emph{a-} attaches productively to almost all \isi{transitive} verbs introducing a third argument into the clause, usually a \isi{beneficiary} (\isi{dative}) or \isi{possessor} (\isi{possessive}). A number of lexemes are \isi{deponent} in the sense that they obligatorily take the \isi{valency change} prefix \emph{a-}, while the clause remains \isi{transitive} and the referent indexed in the prefix is flagged with the \isi{absolutive} \isi{case}. Such \isi{deponent} verbs are \emph{frmnzsi} `prepare' (\ref{ex169}) or \emph{fiyoksi} `make' (\ref{ex166}). Given the basic productivity of the \isi{ditransitive} alternation, we may ask whether the category `\isi{ditransitive}' exists in Komnzo at all or whether it is better to view the phenomenon merely as applicativisation, in other words whether all ditransitives are derived.\footnote{Please note that the \emph{a-} prefix cannot be called an applicative prefix because it fulfills both functions: increasing and decreasing the valency. Thus, I prefer to label it valency change or valency switch.} Two counterarguments can be brought forward. First, there are a few verbs which only exist in an ambifixing \isi{ditransitive} template, the obivous one being \emph{yarisi} `give'. Secondly, while the \isi{ditransitive} alternation simply introduces a \isi{beneficiary} for some verbs, there are rather idiosyncratic changes in meaning for other verbs. For example, \emph{säminzsi} means `whisper' in the ambifixing \isi{transitive} template, but `teach' in the ambifixing \isi{ditransitive} template. Another example was given above in (\ref{ex151}) where \emph{rbänzsi} means `untie' as a \isi{transitive}, but `explain' in a \isi{ditransitive} template. Although the meanings of the different templates share the same \isi{infinitive}/\isi{nominalisation} and are clearly related (`untie' $\rightarrow$ `untie for sb.' = `explain'), they often differ in idiosyncratic ways (`whisper' $\rightarrow$ `whisper for sb.' = `teach'). Thus, it is better to recognise \isi{ditransitive} verbs as an independent category.

\section{Person, gender and number} \label{persgendnumber}

\subsection{Person} \label{personsection}

Person marking in Komnzo verbs exhibits various patterns of syncretism and \isi{neutralisation} in certain contexts. These patterns differ in the two sites of \isi{person} marking: the prefix and the suffix. The suffixes show more complexity in their syntagmatic distribution: under certain conditions they are reduced to \isi{zero} morphemes, neutralise their \isi{person} values and, in addition, the status of the first singular as an independent morpheme is questionable. On the other hand, the suffixes show less paradigmatic complexity. They encode only two \isi{person} values and there is only one suffix series. As for the prefixes, the opposite seems to be the case. Although they can be neatly separated and recognised, the prefix slot is equipped with five \isi{prefix series} and widespread syncretism within the paradigm is a central characteristic. I will address each subsystem of \isi{person} marking below.

\subsubsection{Person suffixes} \label{personsuffsection}

The \isi{person} suffix differentiates two \isi{person} values: first and non-first \isi{person}. Thus, second and third \isi{person} are always neutralised and additional information from the personal pronouns or from context is required. As I will explain below, in certain morphological contexts, even this basic distinction is neutralised and only \isi{number} marking is retained. Table \ref{perssuff} lists the suffix forms in indicative and irrealis mood.

\begin{table}
\caption{Person suffixes}
\label{perssuff}
	\begin{tabular}{llll}
		\lsptoprule
		\textsc{gloss} & \textsc{formative} & \textsc{example} &\textsc{translation}\\\midrule
		\Fsg &\emph{-é} &\emph{ŋakwiré}	&`I run'\\
		\Fnsg &\emph{-e} &\emph{ŋakwire} &`we run'\\
		\Stsg &-\Zero &\emph{ŋakwir} &`you run' or `s/he runs'\\
		\Stnsg &\emph{-th} &\emph{ŋakwirth}	&`you run' or they run'\\
		\lspbottomrule
	\end{tabular}
\end{table}%Person suffixes

In middle and ambifixing templates, the \isi{person} suffixes are involved in marking \isi{imperative} mood. Table \ref{perssuffimp} below shows that the indexing of the addressee employs formatives which are identical to the first \isi{person} suffixes in indicative or irrealis mood. Evans (\citeyear{Evans:2012ue}) describes an inflectional category in \ili{Nen} called the assentive. The assentive is the second part of an adjacency pair (or dyadic sequence), and it follows an \isi{imperative} (`Boil the water!' > `I will boil the water.'). In the assentive, the \isi{person} suffix deviates from indicative inflection in that it is identical to the preceding \isi{imperative}; both being \isi{zero} in \isi{perfective} aspect. Although assentive inflections are not attested in Komnzo, the formal identity of first \isi{person} indicative and second \isi{person} \isi{imperative} suffixes can be explained by such conversational adjacency pairs.\\

Komnzo imperatives can be imperfective (`Keep on doing X!') or \isi{perfective} (`Do X!'). An example of this is shown in (\ref{ex766}) below. This distinction is signalled by the stem type, but also by the fact that the second singular suffix in perfective imperatives is \isi{zero}. The formatives are listed in Table \ref{perssuffimp} below.

\begin{table}
\caption{Imperative person suffixes}
\label{perssuffimp}
	\begin{tabular}{lllll}
		\lsptoprule
		&\textsc{gloss} &\textsc{formative} &\textsc{example} &\textsc{translation}\\\midrule
		\multirow{2}{*}{\Ext{} stem} &\Ssg.\Imp &\emph{-é}	&\emph{kakwiré} &`You keep running!'\\
		&\Snsg.\Imp	&\emph{-e} &\emph{kakwire} &`You (pl) keep running!'\\
		&&&&\\
		\multirow{2}{*}{\Rs{} stem} &\Ssg.\Imp &-\Zero &\emph{kamath} &`You run!'\\
		&\Snsg.\Imp	&\emph{-e} &\emph{kemathe} &`You (pl) run!'\\
		\lspbottomrule
	\end{tabular}
\end{table}%Imperative \isi{person} suffixes

In Table \ref{perssuffimp} above, the middle verb \emph{-kwi} `run' is shown. The distinction between second \isi{singular} and \isi{non-singular} is expressed in the suffix. Another quirk in the system, is that the suffix \emph{-é} is used even if the verb is a prefixing verb, despite the fact that the \isi{number} distinction is shown in the prefixes only: \emph{gn-} \Ssg{} vs. \emph{th-} \Snsg{} (see \S\ref{personprefsection}). A prefixing verb like \emph{-kogr} `stand' will be \emph{gnkogré} `You (\Sg) keep standing!' versus \emph{thkogré} `You (\Pl) keep standing!' In these cases I \isi{gloss} \emph{-é} as marking solely \isi{imperative} mood, as in (\ref{ex766}). However, prefixing verbs do follow the pattern in that only extended stems (imperfective \isi{imperative}) receive the \emph{-é} suffix, not the restricted stems (\isi{perfective} imperatives). I show this in example (\ref{ex766}), in which the speaker reports about the rough ways of going hunting with the \ili{Suki} people.\footnote{This verb is irregular in that it encodes dual versus non-dual in the positional stem, \emph{-kogr} \Ndu{} vs. \emph{-kogrn} \Du{}, but not in the restricted stem \emph{-kuk}.} See also \S\ref{imperativesuffix} for further discussion of \isi{imperative} marking.

\begin{exe}
	\ex \emph{fiwä we nima ane kwa änor: ``kwot fthé \textbf{gnäkuk} fathfathenwä \textbf{gnkogé}!''}\\
	\glll fi=wä we nima ane kwa e-a-nor kwot fthé gn-ä-kuk fath-fath=en=wä\\
	\Third.\Abs{}=\Emph{} also like.this \Dem{} \Fut{} \Stnsg-\Vc-shout.\Ext.\Ndu{} properly when \Ssg.\Bet.\Imp-\Ndu-stand.\Rs{} \Redup-clear.place=\Loc=\Emph{}\\
	{} {} {} {} {} {\footnotesize \Tpl:\Sbj:\Nonpast:\Ipfv/shout} {} {} {\footnotesize \Ssg:\Sbj:\Imp:\Pfv/stand} {}\\
	\sn
	\glll gn-kog-é\\
	\Ssg.\Bet.\Imp-stand.\Ext.\Ndu-\Imp{}\\
	{\footnotesize \Ssg:\Sbj:\Imp:\Ipfv/stand}\\
	\trans `They will also yell at one another like this ``You stand properly in the clearing! Keep on standing!'''\Corpus{tci20130927-06}{MAB \#52-53}
	\label{ex766}
\end{exe}

\noindent
\textbf{The morphemic status of the first singular \emph{-é}}\\

I want to discuss the morphemic status of \emph{-é} and provide evidence for the emergence of a marginal phoneme \emph{é} [ə̆]. Both tables above include a suffix \emph{-é} which for the purpose of the following discussion I will call `first \isi{person} singular suffix' disregarding that it may also signal a second singular in imperative mood without \isi{person} marking in the prefixing template. This suffix is realised as a short \isi{schwa} [ə̆] and I have argued in \S{}\ref{schwa-as-non-phoneme} that \isi{schwa} is the \isi{epenthetic vowel} whose distribution is predictable. Schwa is not predictable in word final position and, thus, has to be represented by a grapheme <é>. There are a handful of morphs in which \isi{schwa} is attested word-finally, for example \isi{nominal}s (\emph{kayé} `tomorrow, yesterday', \emph{megé} `green coconut leaf'), function words (\emph{fthé} `when') and suffixes (\emph{-thé} \Adlzr{}, \emph{-é} \Fsg{}). The following discussion puts forward the argument that \emph{-é} is the result of a truncation of the \isi{non-dual} suffix in extended stems, which might have originated in some verbs and was later generalised to all verbs. A possible historical explanation in terms of vowel reduction comes from neighboring varieties in which the first \isi{person} is marked by an \emph{-a} suffix, for example in \ili{Wára} and \ili{Anta}. In Komnzo, there exists a suffix \emph{-a}, but it is a \isi{past} marker.\\

As we can see in both tables above, \emph{-é} contrasts with \emph{-e} (\Fnsg) and -\Zero{} (\Stsg). The first singular \emph{-é} could be analysed either as a morpheme in its own right or as the result of a truncation process of the \isi{non-dual} suffix, which leaves no possible \isi{syllabification} other than \isi{schwa} in a word-final context. I am not claiming that truncation is a synchronic process, but I want to argue that truncation of the \isi{non-dual} suffix plays a role in the explanation. I draw on evidence from more general properties of the suffix subsystem such as the \isi{non-dual} suffix, the presence of a \isi{linking consonant} and the \isi{neutralisation} of \isi{person} distinctions. As we will see below, the argumentation is only applicable to inflected forms which build on the \isi{extended stem}. Restricted stems encode the duality contrast in pre-stem position. Hence, we have to assume that the result of the truncation process, the word final \isi{schwa} \emph{-é}, has been extended to other morphological contexts.\\

First, let us turn to the \isi{non-dual} marker for extended stems. The verb \emph{kwi-} `run' in Table \ref{perssuff} is irregular in that it employs \emph{-r} for signalling the non-dual. The regular pattern, attested for 90\% of \isi{verb} lexemes, involves one of the three non-dual allomorphs \emph{-wr}, \emph{-nzr} and \emph{-thr}. Consider the verb \emph{marasi} `see' in (\ref{ex195}-\ref{ex200}), which takes the \emph{-wr} \isi{allomorph}. In first \isi{person} singular (\ref{ex195}), the \isi{non-dual} suffix is \emph{-w} instead of \emph{-wr}.

\begin{exe}
\ex \label{ex204}
\begin{xlist}
	\ex
	\gll \emph{y-mar-w-é}\\
	\Tsg.\Masc-see-\Ndu-\Fsg\\
	\trans `I see him.'
	\label{ex195}
	\ex
	\gll \emph{y-mar-n-e}\\
	\Tsg.\Masc-see-\Du-\Fnsg\\
	\trans `We two see him.'
	\label{ex196}
	\ex
	\gll \emph{y-mar-wr-e}\\
	\Tsg.\Masc-see-\Ndu-\Fnsg\\
	\trans `We see him.'
	\label{ex197}
	\ex
	\gll \emph{y-mar-wr-\Zero}\\
	\Tsg.\Masc-see-\Ndu-\Stsg\\
	\trans `S/He sees him.' or `You see him.'
	\label{ex198}
	\ex
	\gll \emph{y-mar-n-th}\\
	\Tsg.\Masc-see-\Du-\Stsg\\
	\trans `They (two) see him.' or `You (two) see him.'
	\label{ex199}
	\ex
	\gll \emph{y-mar-wr-th}\\
	\Tsg.\Masc-see-\Ndu-\Stnsg\\
	\trans `They see him.' or `You see him.'
	\label{ex200}
\end{xlist}
\end{exe}

In the examples above, only the first singular (\ref{ex195}) deviates in that it takes a truncated form \emph{-w}, from which final \emph{-r} is cut. This truncation with the first singular is attested for all three allomorphs of the \isi{non-dual} suffix: \emph{-wr} $\rightarrow$ \emph{-w}, \emph{-nzr} $\rightarrow$ \emph{-nz} and \emph{-thr} $\rightarrow$ \emph{-th}. What weakens this particular piece of evidence is the fact that there is some variation between the non-truncated and the truncated formative even when other suffixal material follows like \Andat{} \emph{-o}, \Fnsg{} \emph{-e} or \Stnsg{} \emph{-th}. For example, looking at the token \isi{frequency} in the corpus of \Stnsg{} \emph{-th} preceded by \emph{-nzr} (non-truncated) versus \emph{-th} preceded by \emph{-nz} (truncated), we find 91 \isi{verb} forms with the non-truncated \isi{non-dual} \emph{-nzrth} and 13 with the truncated \isi{non-dual} \emph{-nzth}.\footnote{This search can be replicated by a simple search query: ``nzrth'' versus ``nzth'' in word final context (in \textsc{REGEX} syntax: ``nzrth\textbackslash b'' versus ``nzth\textbackslash b'').} A similar distribution is found with the first \isi{non-singular} \emph{-e} suffix. There is no variation with the \Stsg{}, which is a \isi{zero} morpheme. The \Stsg{} is never preceded by the truncated formative. In conclusion, the \isi{non-dual} is never truncated with the \Stsg{} \isi{zero}, it shows some variation with other suffixes (but the non-truncated formative has a much higher \isi{frequency}), and it is always truncated with the first singular.\\

Further evidence comes from \isi{person} \isi{neutralisation} patterns. The first singular \emph{-é} disappears when further suffixes are added, for example the \isi{past} suffix \emph{-a}, the durative suffix \emph{-m} or the andative suffix \emph{-o}. Consider examples (\ref{ex201}, \ref{ex202} and \ref{ex214}) which neutralise the \isi{person} value completely. In (\ref{ex204}), the distinction between first and second/third \isi{person} is basically a contrast between the surface result of a truncation process \emph{-é} (\ref{ex195}) and a \isi{zero} morpheme (\ref{ex198}). In (\ref{ex201}, \ref{ex202} and \ref{ex214}) below, we have to postulate a \isi{zero} marker, which now only encodes \isi{number} (\Sg) and contrasts with \Fnsg{} \emph{-e} (\ref{ex205}) and \Stnsg{} \emph{-th} (\ref{ex203}).

\begin{exe}
\ex
\begin{xlist}
	\ex
	\gll \emph{y-mar-wr-a-\Zero{}}\\
	\Tsg.\Masc-see-\Ndu-\Pst-\Sg\\
	\trans `I saw him.' or `You saw him.' or `S/He saw him.'
	\label{ex201}
	\ex
	\gll \emph{y-mar-wr-a-k-e}\\
	\Tsg.\Masc-see-\Ndu-\Pst-\Lk-\Fnsg\\
	\trans `We saw him.'
	\label{ex205}
	\ex
	\gll \emph{y-mar-wr-a-th}\\
	\Tsg.\Masc-see-\Ndu-\Pst-\Stnsg\\
	\trans `You saw him.' or `They saw him.'
	\label{ex203}
	\ex
	\gll \emph{y-mar-wr-m-\Zero{}}\\
	\Tsg.\Masc-see-\Ndu-\Dur-\Sg\\
	\trans `I was seeing him.' or `You were seeing him.' or `S/He was seeing him.'
	\label{ex202}
	\ex
	\gll \emph{y-mar-wr-o-\Zero{}}\\
	\Tsg.\Masc-see-\Ndu-\Andat-\Sg\\
	\trans `I see him that way.' or `You see him that way.' or `S/He sees him that way.'
	\label{ex214}
\end{xlist}
\end{exe}

A third piece of evidence comes from a \isi{linking consonant} in the suffix subsystem. Example (\ref{ex205}) above shows that the \isi{past} suffix \emph{-a} and the \Fnsg{} \emph{-e} are separated by \emph{-k}. We have seen in \S{}\ref{syllabificationandepenthesis}, that the \isi{phonology} of Komnzo allows strings of consonants which are broken up by \isi{epenthesis}. However, the phonological system does not tolerate strings of vowels, which is demonstrated by the appearance of the linker in (\ref{ex205}). This can be used to strengthen the argument that the first singular \emph{-é} deviates from other suffixes. We would expect (\ref{ex201}) not to neutralise the \isi{person} value, and instead to insert the linker between the \isi{past} suffix \emph{-a} and \emph{-é} analogous to (\ref{ex205}). However, the predicted inflection \textsuperscript{$\ast$}\emph{ymarwraké} is ungrammatical.\\

The first singular \emph{-é} occurs in other morphological contexts, where there is no truncated preceding element. As pointed out above, the template of restricted stems marks the \isi{dual} versus non-dual contrast in pre-stem position and, thus, there is no non-dual marker to truncate (\ref{ex207}).\footnote{The verb \emph{marasi} belongs to the class which has identical forms for restricted and extended stems (see Table \ref{frbearr}), and only the template and the affixal material signal the aspectual value.} Likewise, there is no truncation of the \isi{dual} marker \emph{-n} in the template of extended stems (\ref{ex206}). However, the \isi{person} neutralisations described above also occur in these contexts (\ref{ex208} and \ref{ex209}).

\begin{exe}
\ex
\begin{xlist}
	\ex
	\gll \emph{s-a-mar-é}\\
	\Tsg.\Masc-\Ndu-see(\Rs)-\Fnsg\\
	\trans `I saw him.'
	\label{ex207}
	\ex
	\gll \emph{e-mar-n-é}\\
	\Stnsg-see(\Ext)-\Ndu-\Pst-\Sg\\
	\trans `I see both of them.'
	\label{ex206}
	\ex
	\gll \emph{s-a-mar-a-\Zero{}}\\
	\Tsg.\Masc-\Ndu-see(\Rs)-\Pst-\Sg\\
	\trans `I saw him.' or `You saw him.' or `S/He saw him.'
	\label{ex208}
	\ex
	\gll \emph{e-mar-n-a}\\
	\Stnsg-see(\Ext)-\Ndu-\Pst-\Sg\\
	\trans `I saw both of them.' or `You saw both of them.' or `S/He saw both of them.'
	\label{ex209}
\end{xlist}
\end{exe}

We have to conclude that a case of truncation or a negative morpheme as a synchronic process can only be made for a very circumscribed morphological context: for non-dual inflected verbs built from the \isi{extended stem}. For other contexts, we have to postulate a suffix formative \emph{-é}. This is best explained by a historical process of vowel reduction or \isi{syllable} loss, which created a new marginal phoneme \emph{é}. This can be used to explain word-final \isi{schwa} in other items.\footnote{The adjectivaliser \emph{-thé} might be a reduced form of the similative case marker \emph{-thatha}.} As I mentioned in the beginning of this section, surrounding varieties like \ili{Wára} or \ili{Anta} mark the first \isi{person} singular with an \emph{-a} suffix. Comparative material from other \ili{Tonda} varieties is needed to settle this question.\\

\noindent
\textbf{Linking \emph{-k}, person neutralisation and morpheme slots}\\

In the preceding discussion, the \isi{linking consonant} \emph{-k} was introduced as a way of separating two adjacent vowel suffixes. This purely phonological explanation is insufficient and, on closer inspection, we find that the linker \emph{-k} helps to arrange the suffixal material into morpheme slots. In addition to the first singular \emph{-é}, the suffixal material includes the following morphemes: \isi{past} \emph{-a}, \isi{durative} \emph{-m}, \isi{andative} \emph{-o}, \Fnsg{} \emph{-e} and \Stnsg{} \emph{-th}. In the following section, I describe how these suffixes line up, which of them are mutually exclusive, and in which context \isi{person} neutralisations occur.\\

First, the \isi{past} suffix \emph{-a} and the \isi{durative} suffix \emph{-m} never co-occur. The combinatorial system of Komnzo verb morphology employs a different strategy to express a \isi{past} \isi{durative} category, discussed in \S{}\ref{combitam}.\\

Secondly, the \isi{andative} \emph{-o} and the \Fnsg{} \emph{-e} stand in syntagmatic opposition to each other occuping the same slot. Consider examples (\ref{ex210}-\ref{ex213}) below. In examples (\ref{ex211}) and (\ref{ex213}) the \isi{person} value is fully neutralised, because the suffix \emph{-th}, which was indexing \Stnsg{} in earlier examples (\ref{ex199}-\ref{ex200} and \ref{ex203}), can now only be glossed as \Nsg{}.\footnote{An alternative would be to analyse \emph{-th} as marking only number (\Nsg) not person. I reject this analysis, because (i) this would result in a system where only first person is marked overtly and (ii) the \Fnsg{} in examples like (\ref{ex210}) would be an exception to the regular non-singular (\emph{-th}).} The important observation in (\ref{ex211}) is that the linker \emph{-k} is not used. If its appearance could be predicted on purely phonological grounds, we would expect a form like \textsuperscript{$\ast$}\emph{ymarwroke}. But this is ungrammatical. Thus, I characterise the \isi{linking consonant} in the following way: \emph{-k} occurs (i) after the \isi{past} suffix \emph{-a}, (ii) if the following suffix consists of a vowel formative.

\begin{exe}
\ex
\begin{xlist}
	\ex
	\gll \emph{y-mar-wr-e}\\
	\Tsg.\Masc-see-\Ndu-\Fnsg\\
	\trans `We see him.'
	\label{ex210}
	\ex
	\gll \emph{y-mar-wr-o-th}\\
	\Tsg.\Masc-see-\Ndu-\Andat-\Nsg\\
	\trans `We see him that way.' or `You see him that way.' or `They see him that way.'
	\label{ex211}
	\ex
	\gll \emph{y-mar-wr-a-k-e}\\
	\Tsg.\Masc-see-\Ndu-\Pst-\Lk-\Fnsg\\
	\trans `We saw him.'
	\label{ex212}
	\ex
	\gll \emph{y-mar-wr-a-k-o-th}\\
	\Tsg.\Masc-see-\Ndu-\Pst-\Lk-\Andat-\Nsg\\
	\trans `We saw him that way.' or `You saw him that way.' or `They saw him that way.'
	\label{ex213}
\end{xlist}
\end{exe}

Examples (\ref{ex211}) and (\ref{ex213}) also show that amongst the three categories (\isi{person}, \isi{number}, \isi{direction}) it is \isi{person} which is neutralised first. In the discussion of examples (\ref{ex201}-\ref{ex214}), we found the same to be true for \isi{person} values of the singulars.\\

Below in (\ref{ex241}), we find a textual example of the \isi{person} \isi{neutralisation} in (\ref{ex213}). In the example, a woman talks about her marriage and how she and her husband prepared a feast for her brothers and uncles. In (\ref{ex241}) the first \isi{person} interpretation of the actor of \emph{tharakoth}\footnote{In \emph{tharakoth} the pre-stem marker operates on a plural versus non-plural opposition. This pattern of pre-stem marking is discussed in \S{}\ref{prerootdual}.} is clear from the preceding verb \emph{yafiyokrnake} which lacks the \isi{andative} \emph{-o} suffix and, thus, is inflected with the first \isi{non-singular} \emph{-e} suffix.

\begin{exe}
	\ex \emph{dagon yafiyokrnake. babainm ane \textbf{tharakoth}.}\\
	\glll dagon y-a-fiyok-rn-a-k-e babai=nm ane\\
	food \Tsg.\Masc-\Vc-make.\Ext-\Pst-\Lk-\Fnsg{} uncle=\Dat.\Nsg{} \Dem{}\\
	{} \footnotesize{\Fdu:\Sbj>\Tsg.\Masc:\Obj:\Pst:\Ipfv/make} {} {}\\
	\sn
	\glll th-a-r-a-k-o-th\\
	\Stnsg.\Gam-\Vc.\Du-give.\Rs-\Pst-\Lk-\Andat-\Nsg\\
	\footnotesize{\Fdu:\Sbj>\Stpl:\Io:\Pst:\Pfv:\Andat/give}\\
	\trans `We prepared the food. We gave that to the uncles.'\Corpus{tci20130823-08}{WAM \#66-67}
	\label{ex241}
\end{exe}

The suffix subsystem of Komnzo verbs is summarised in Figure \ref{suffsubsys}. The elements which share a column or an extended column in the figure are mutually exclusive. For example, if \emph{-é} occurs, all the other material will not appear or if the \isi{durative} suffix \emph{-m} occurs, the \isi{past} suffix \emph{-a} (along with the linker \emph{-k}) will not appear. The system as described here is applicable to both stem types. For the restricted stem the only difference lies in the fact that duality is marked in pre-stem position as in \emph{tharakoth} in (\ref{ex241}). Therefore, some of the morphemes in the suffix system are optional: the dual/non-dual morphemes, the two TAM markers (\Pst{} \emph{-a} and \Dur{} \emph{-m}) and the \isi{andative} \emph{-o}. Number (\Sg{} vs. \Nsg) is always marked.

\begin{figure}
\begin{center}%
	\begin{tabular}{|c|c|c|c|c|}
		\midrule
		\textsc{stem}&\textsc{(duality)}&\multicolumn{1}{l|}{\textsc{(tam)}}&\multicolumn{2}{l|}{\textsc{(direction, person), number}}\\ \midrule
		\multirow{4}{*}{$\sqrt{}$}&&\multicolumn{3}{c|}{\emph{-é}}\\ \cline{3-5}
		&\emph{-nzr, wr-, r-} & \multirow{2}{*}{\emph{-m}}&  \multicolumn{2}{c|}{\emph{-e}}\\\cline{4-5}
		&\emph{-n} &&\multirow{2}{*}{\emph{-o}} & \emph{-\Zero}\\\cline{3-3}\cline{5-5}
		&&\multicolumn{1}{c|}{\emph{-a}(\emph{-k})} & & \emph{-th}\\
		\midrule
	\end{tabular}
\end{center}
\caption{Suffix subsystem of Komnzo verbs}\label{suffsubsys}
\end{figure}%Suffix subsystem of Komnzo verbs

The suffixing system is thus characterised by syntagmatic complexity, i.e. the chain of suffixes does not allow a straightforward segmentation into slots and respective functions. Moreover, the presence versus absence of individual suffixes impacts on the form and function of other suffixes.

\subsubsection{Person prefixes} \label{personprefsection}

The \isi{person} prefixes are syntagmatically less complex than the \isi{person} suffixes. The prefix system comprises a single slot which is always filled with a formative, i.e. there are no \isi{zero} morphemes.\footnote{The only formative which occurs in the person marking slot, but does not encode person, is the middle marker, which is used for other purposes (\S{}\ref{middletemplatesubsection}).} On the other hand, the prefix system is paradigmatically more complex. The prefix fuses \isi{person} and \isi{number} marking with information relevant to TAM. However, we have to draw on abstract glossing labels because the five \isi{prefix series} are underspecified for a particular TAM value. Table \ref{perspref} lays out the five \isi{prefix series}: \Alph, \Bet, \Betaone, \Betatwo, and \Gam.

\begin{table}
\caption{Person prefixes}
\label{perspref}
	\begin{tabular}{llllll}
		\lsptoprule
		\textsc{gloss} &\Alph &\Bet &\Betaone &\Betatwo &\Gam\\\midrule
		\Fsg &\emph{wo-} &\emph{kw-} &\emph{ku-} &\emph{kwof-} &\emph{zu-}\\
		\Fnsg &\emph{n-} &\emph{nz-} / \emph{nzn-} &\emph{nzu-} &\emph{nzf-} &\emph{nzn-}\\
		\Ssg &\emph{n-}	&\emph{nz-} / \emph{gn-} &\emph{gu-} &\emph{gf-} &\emph{nzn-}\\
		\Tsg.\F &\emph{w-} &\emph{z-} &\emph{zu-} &\emph{zf-} &\emph{z-}\\
		\Tsg.\Masc &\emph{y-} &\emph{s-} &\emph{su-} &\emph{sf-} &\emph{s-}\\
		\Stnsg &\emph{e-} &\emph{th-} &\emph{thu-} &\emph{thf-} &\emph{th-}\\
		\M &\emph{ŋ-} &\emph{k-} &\emph{kw-} &\emph{kf-} &\emph{z-}\\
		\lspbottomrule
	\end{tabular}
\end{table}%Person prefixes

Before we look at the patterns of \isi{person} marking, I will provide some justification as to why there are five independent series. Table \ref{perspref} shows that there is widespread syncretism between the series, especially in the third \isi{person} between the \Bet{} and \Gam{} series. The formal difference between the \Alph, \Bet{} and \Gam{} series is clearest in the first \isi{person} \isi{singular} and the middle marker, each of which distinguishes overtly all five series. Furthermore, the table shows that we can speak of three main series: \Alph, \Bet, \Gam{}, plus two subseries: \Betaone{} and \Betatwo. These two subseries add an /\emph{u}/ and /\emph{f}/ element to the \Bet{} series. I will discuss in detail why I still treat them as independent series in \S{}\ref{tamprefixseries}. An additional quirk is added to the system by the fact that, within the \Bet{} series, the first nonsingular and the second \isi{singular} have two different formatives for the two modal categories: the \isi{imperative} and \isi{irrealis}.\footnote{The second singular differs in a number of ways which will be discussed in \S{}\ref{tamprefixseries}. Note that the second singular \emph{gn-} is only used in the imperatives of prefixing verbs where the addressee argument is encoded in the prefix. Verbs in middle and ambifixing templates on the other hand employ the suffix to encode the addressee argument in the imperatives, leaving the prefix \Bet{} series for the middle marker or the indexing of the undergoer argument.}\\

The prefixes differentiate three \isi{person} values in the \isi{singular}: first, second and third. The values of second and third \isi{person} in \isi{non-singular} are always neutralised, leaving this ambiguity for either context or the personal pronouns to resolve. The same holds true for the syncretism between the first \isi{non-singular} and the second \isi{singular} in the \Alph{} and the \Gam{} series.\footnote{Table \ref{perspref} also includes identical formatives \emph{nz-} for first non-singular and second singular in the \Bet{} series. The \Bet{} series is used for irrealis inflection. The neutralisation is there on an abstract paradigmatic level, but the inflected verbs are never identical, because - unlike all other person/number combinations - the second singular does not take the irrealis prefix \emph{ra-}. This will be further discussed in \S{}\ref{tamprefixseries}.} This pattern of syncretism is found in languages across the Yam family (\citealt{Evans:sng}).\\

The overview of the \isi{verb} template presented in the introduction of this chapter (Table \ref{verbtemplate}) shows that the \isi{person} prefix is followed by the \isi{valency change} prefix \emph{a-} whose presence impacts on the formatives of the \isi{person} prefixes in various ways. The \Alph{} series shows a number of irregularities given in Table \ref{persprefwithvalencychange}, for example with the first \isi{singular}: /\emph{wo-a-}/ $\rightarrow$ \emph{wa-}.

\begin{table}
\caption{Person prefixes: \Alph-series with valency change prefix \emph{a-}}
\label{persprefwithvalencychange}
	\begin{tabular}{lll}
		\lsptoprule
		\textsc{gloss} &\textsc{formative} &\textsc{segmentation}\\\midrule
		\Fsg &\emph{wa-} &\emph{wo-a-}\\
		\Fnsg &\emph{na-} &\emph{n-a-}\\
		\Ssg &\emph{na-} &\emph{n-a-}\\
		\Tsg.\F	&\emph{wä-} &\emph{w-a-}\\
		\Tsg.\Masc &\emph{ya-} &\emph{y-a-}\\
		\Stnsg &\emph{ä-} &\emph{e-a-}\footnotemark\\
		\M &\emph{ŋa-} &\emph{ŋ-a-}\\
		\lspbottomrule
	\end{tabular}
\end{table}%Person prefixes: \Alph-series with \isi{valency change} prefix \emph{a-}
\footnotetext{In a Komnzo recording from the 1980's made by the anthropologist Mary Ayres, I found a different realisation of this prefix, namely [eja-]. In terms of segmentation, this is a much more transparent realisation. The recording was made with an older man, maybe in his late 60's. In modern Komnzo, there is no variation and the prefix is realised as given in the table [æ-].}

The other \isi{prefix series} behave more regular in the presence of the \isi{valency change} prefix \emph{a-}, but there is some influence of the \isi{valency change} prefix. For example, the formatives of the \Betatwo{} series all end in a high back vowel [u], which turns into the corresponding glide when \emph{a-} is present: \Ssg{} \emph{gu-} $\rightarrow$ \emph{gwa-}. The \Bet{} and \Betatwo{} series end in consonants. For both series, the \emph{a-} prefix is simply added, for example \Stnsg{} \emph{th-} $\rightarrow$ \emph{tha-} for the \Bet{} series and \Stnsg{} \emph{thf-} $\rightarrow$ \emph{thfa-} for the \Betatwo{} series.\\

As I have discussed in \S{}\ref{combinatoricsextrs}, the \Bet{}, \Betaone{} and \Gam{} series may combine with the \isi{restricted stem}, the last of the three exclusively so. With the \isi{restricted stem}, dual marking takes place in pre-stem position (see \S{}\ref{dualextrs}) and the \emph{a-} prefix simultaneously encodes valency change and the dual vs. non-dual contrast. As the marking pattern does not impact on the formatives of the person prefixes, I will defer this topic to the discussion of number marking in \S{}\ref{prerootdual}.

\subsection{Gender} \label{gendersubsec}

The agreement target of \isi{gender} is the third \isi{singular} prefix of the verb. There is a \isi{feminine} and \isi{masculine} \isi{gender} category. Metalinguistic statements by speakers are often expressed as \emph{madema rä} `It is a girl' for \isi{feminine} or \emph{srak yé} `It is a boy' for \isi{masculine}. The formatives employed to encode \isi{gender} across the \isi{prefix series} are given in Table \ref{perspref} above.\\

The discussion in \S{}\ref{alignmtemplates} has shown that the prefix indexes the direct and indirect object in the ambifixing \isi{transitive} template, and the \isi{subject} of intransitives in the prefixing template. It follows that only those types of argument roles show agreement in \isi{gender}, whereas the more agent-like arguments never show \isi{gender} agreement.\\

The semantic perspective of \isi{gender} classification of the \isi{noun} lexicon is discussed in \S{}\ref{wordclasses-thegendersystem}.

\subsection{Number} \label{numbersubsec}

Komnzo verbs encode three \isi{number} values: \isi{singular}, \isi{dual} and \isi{plural}. There exists an additional \isi{large plural} which is available only for prefixing verbs or verbs in the prefixing template. I describe the fourth \isi{number} value in \S{}\ref{positonalnumber}.\\

The peculiarity of \isi{number} marking in Komnzo lies in the fact that it is distributed over two separate slots which, looked at individually, do not distinguish all three values, but operate on a binary opposition. Hence, the overall ternary \isi{number} opposition is reduced to a binary opposition in the respective slots on the verb. There are three logical possibilities for this reduction because each of the three \isi{number} values can be contrasted with its opposite: \isi{singular} vs. \isi{non-singular}; \isi{dual} vs. \isi{non-dual}; \isi{plural} vs. \isi{non-plural}. The combination of any two of the three binary oppositions is sufficient to encode all three \isi{number} values. Figure \ref{ternbinary} below shows the principle behind this reduction.

\begin{figure}

\begin{tikzpicture}
	\draw (1,4.5) --(7,4.5);
	\draw (1,5.5) --(7,5.5);
	\draw (1,4.5) --(1,5.5);
	\draw (3,4.5) --(3,5.5);
	\draw (5,4.5) --(5,5.5);
	\node at (2,5){\Sg};
	\node at (4,5){\Du};
	\node at (6,5){\Pl};
	\draw[densely dotted] (-4.6,4.25) --(7.3,4.25);
	\draw (1,3) --(7,3);
	\draw (1,4) --(7,4);
	\draw (1,3) --(1,4);
	\draw (3,3) --(3,4);
	\node at (2,3.5){\Sg};
	\node at (5,3.5){\Nsg};
	\draw (1,1.5) --(7,1.5);
	\draw (1,2.5) --(7,2.5);
	\draw (1,1.5) --(1,2.5);
	\draw (3,2) --(3,2.5);
	\draw (5,2) --(5,2.5);
	\draw (3,2) --(3,1.5);
	\draw (5,2) --(5,1.5);
	\node at (4,2){\Du};
	\node at (2,2){\Ndu};
	\node at (6,2){\Ndu};
	\draw (1,0) --(7,0);
	\draw (1,1) --(7,1);
	\draw (1,0) --(1,1);
	\draw (5,0) --(5,1);
	\node at (3,0.5){\Npl};
	\node at (6,0.5){\Pl};
	\node[left] at (0,5) {ternary:};
	\node[left] at (0,3.5) {binary (\Sg{} vs. \Nsg{}):};
	\node[left] at (0,2) {binary (\Du{} vs. \Ndu):};
	\node[left] at (0,0.5) {binary (\Pl{} vs. \Npl):};
\end{tikzpicture}
\caption{Three ways of breaking up a ternary opposition}\label{ternbinary}
\end{figure}%Three ways of breaking up a ternary opposition

Komnzo makes use of all three oppositions, but only two of the possible combinations. The \isi{person} affixes operate always on a \isi{singular} vs. \isi{non-singular} opposition. A separate affix, which I call the duality affix, makes a distinction between \isi{dual} vs. \isi{non-dual}. I will show below that under certain circumstances, the same affix encodes \isi{plural} vs. \isi{non-plural}, but this is a marginal pattern (\S{}\ref{prerootdual}). The basic system of distributed \isi{number} marking integrates a \Sg{}-\Nsg{} opposition in the \isi{person} affixes with a \Du-\Ndu{} opposition in the duality affix. Figure \ref{basicnumberm} provides an overview of this principle.

\begin{figure}

	\begin{tabular}{|c|c|p{1,5cm}p{1,5cm}|}
		\cline{3-4}
		\multicolumn{2}{c|}{}&\multicolumn{2}{c|}{\textsc{duality affix}}\\\cline{3-4}
		\multicolumn{2}{c|}{}&\multicolumn{1}{c}{\Du}&\multicolumn{1}{|c|}{\Ndu}\\\midrule
		\parbox[t]{2mm}{\multirow{6}{*}{\rotatebox[origin=c]{90}{\textsc{person affix}}}}&\parbox[t]{2mm}{\multirow{3}{*}{\rotatebox[origin=c]{90}{\Sg}}}&\multicolumn{1}{c}{\multirow{3}{*}{}}&\\
		&&&\multicolumn{1}{c|}{singular}\\
		&&&\\\cline{2-2}
		&\parbox[t]{2mm}{\multirow{3}{*}{\rotatebox[origin=c]{90}{\Nsg}}}&&\\
		&&\multicolumn{1}{c}{dual}&\multicolumn{1}{c|}{plural}\\
		&&&\\\midrule
		\multicolumn{4}{c}{}\\
	\end{tabular}
\caption{Basic principle of distributed number marking on verbs}\label{basicnumberm}
\end{figure}%Basic principle of distributed \isi{number} marking

Figure \ref{basicnumberm} shows that out of four possible combinations, in fact only three are normally put to use, namely those that are logically compatible. Prefixing verbs and stems in a prefixing template, which includes \isi{positional} verbs, are exceptional in that they utilise the fourth, seemingly non-sensical, combination \Sg{}-\Du{} to encode a \isi{large plural} (\S{}\ref{positonalnumber}).\\

The two sites involved in \isi{number} marking have very different properties. The binary opposition in the \isi{person} prefixes and suffixes is much more stable in the sense that (i) the encoded value can be straightforwardly associated with an argument, because \isi{person} and \isi{number} marking are fused into one morpheme, (ii) the position of these affixes with respect to the stem is fixed and (iii) the values encoded are always \Sg{} and \Nsg{}. The duality affix differs in all three points and the subsequent discussion of \isi{number} marking will focus on its peculiarities. But to give an overview here: first, if there are two participants indexed in the verb, the duality affix is ambiguous as to which of the two it is indexing. Secondly, duality is marked in a suffix with extended stems, but in a complex \isi{portmanteau} prefix with restricted stems. Finally, as was mentioned above, in part of the paradigm, the \Du{}-\Ndu{} opposition is replaced by a \Pl{}-\Npl{} opposition. I will discuss these points below.

\subsubsection{Ambiguities in the reference of the duality affix} \label{ambiguitiesdualref}

Examples (\ref{ex215}-g) show the verb \emph{fathasi} `hold' with different \isi{number} combinations of the two arguments.\footnote{Note, that the \ili{English} translations are all in third person, although some of the person indexing morphemes neutralise the distinction between second and third person and, thus, could also be translated as second person.} Only in example (\ref{ex219}), we find several possibilities with respect to \isi{number} marking because both \isi{person} affixes signal \isi{non-singular}. The ambiguity stems from the fact that the duality marker is ambiguous as to which of the two arguments it is referencing. In other words, the \isi{dual} morpheme \emph{n-} in (\ref{ex219}) signals that one the two participants is \isi{dual}, but not which one. This does not create any ambiguities in cases where one of the the two \isi{person} affixes is \isi{singular} (\ref{ex215}-e). Likewise, it is not a problem if both \isi{person} affixes are \isi{non-singular} and the duality affix in \isi{non-dual} (\ref{ex220}). Although examples (\ref{ex215}-g) show the \isi{extended stem} of the verb \emph{fathasi}, this ambiguity is also found with restricted stems where the duality affix occurs in pre-stem position.

\begin{exe}
\ex
\begin{xlist}
	\ex
	\gll \emph{y-fath-wr-\Zero}\\
	\Tsg.\Masc-hold.\Ext-\Ndu-\Stsg\\
	\trans `S/He holds him.'
	\label{ex215}
	\ex
	\gll \emph{y-fath-n-th}\\
	\Tsg.\Masc-hold.\Ext-\Du-\Stnsg\\
	\trans `They (2) hold him.'
	\label{ex216}
	\ex
	\gll \emph{y-fath-wr-th}\\
	\Tsg.\Masc-hold.\Ext-\Ndu-\Stnsg\\
	\trans `They (3+) hold him.'
	\label{ex217}
	\ex
	\gll \emph{e-fath-n-\Zero}\\
	\Stnsg-hold.\Ext-\Du-\Stsg\\
	\trans `S/He holds them (2).'
	\label{ex218}
	\ex
	\gll \emph{e-fath-wr-\Zero}\\
	\Stnsg-hold.\Ext-\Ndu-\Stsg\\
	\trans `S/He holds them (3+).'
	\label{ex221}
	\ex
	\gll \emph{e-fath-n-th}\\
	\Stnsg-hold.\Ext-\Du-\Stnsg\\
	\trans `They (2) hold them (3+).' or `They (2) hold them (2).' or `They (3+) hold them (2).'
	\label{ex219}
	\ex
	\gll \emph{e-fath-wr-th}\\
	\Stnsg-hold.\Ext-\Ndu-\Stnsg\\
	\trans `They (3+) hold them (3+).'
	\label{ex220}
\end{xlist}
\end{exe}

For verbs in the \isi{transitive} ambifixing and \isi{ditransitive} ambifixing template, the distribution of the \isi{dual} and \isi{non-dual} markers can be expressed in an abstract way as in Figure \ref{dualitymatrixfathasi}.
\vspace{-.1cm}
\begin{figure}

	\begin{tabular}{|ccccccc|}
		\cline{3-5}
		\multicolumn{2}{c|}{}&\multicolumn{3}{c|}{\textsc{actor}}&\\\cline{3-5}
		\multicolumn{2}{c|}{}&\multicolumn{1}{c}{\Sg}&\multicolumn{1}{|c|}{\Du}&\multicolumn{1}{c|}{\Pl}&\\\cline{1-5}
		{\parbox[t]{2mm}{\multirow{6}{*}{\rotatebox[origin=l]{90}{\textsc{undergoer}\hspace{0,2cm}}}}}&\multicolumn{1}{|c|}{\parbox[t]{2mm}{\multirow{2}{*}{\rotatebox[origin=c]{90}{\Sg}}}}&\multicolumn{1}{c}{\multirow{2}{*}{\hspace{0,1cm}wr\hspace{0,1cm}}}	&\multirow{2}{*}{\hspace{0,1cm}n\hspace{0,1cm}}	&\multicolumn{1}{c|}{\multirow{2}{*}{\hspace{0,1cm}wr\hspace{0,1cm}}}&\\
		&\multicolumn{1}{|c|}{}&&&\multicolumn{1}{c|}{}\\\cline{2-2}
		&\multicolumn{1}{|c|}{\parbox[t]{2mm}{\multirow{2}{*}{\rotatebox[origin=c]{90}{\Du}}}}&\multicolumn{1}{c}{\multirow{2}{*}{\hspace{0,1cm}n\hspace{0,1cm}}}&\multirow{2}{*}{\hspace{0,1cm}n\hspace{0,1cm}}&\multicolumn{1}{c|}{\multirow{2}{*}{\hspace{0,1cm}n\hspace{0,1cm}}}&\\
		&\multicolumn{1}{|c|}{}&&&\multicolumn{1}{c|}{}\\\cline{2-2}
		&\multicolumn{1}{|c|}{\parbox[t]{2mm}{\multirow{2}{*}{\rotatebox[origin=c]{90}{\Pl}}}}&\multicolumn{1}{c}{\multirow{2}{*}{\hspace{0,1cm}wr\hspace{0,1cm}}}&\multirow{2}{*}{\hspace{0,1cm}n\hspace{0,1cm}}&\multicolumn{1}{c|}{\multirow{2}{*}{\hspace{0,1cm}wr\hspace{0,1cm}}}&\\
		&\multicolumn{1}{|c|}{}&&&\multicolumn{1}{c|}{}\\\cline{1-5}
	\end{tabular}
\caption{The duality matrix with \emph{fathasi}}
\label{dualitymatrixfathasi}
\end{figure}%duality matrix post stem

For \isi{verb} forms which index only one argument the marking pattern is simpler, as there is no ambiguity in reference of the duality suffix. This is relevant for verbs in a prefixing or middle template. Examples (\ref{ex222}-c) show the verb \emph{thoraksi} `appear' in a prefixing template cycled through all three \isi{number} values.

\begin{exe}
\ex
\begin{xlist}
	\ex
	\gll \emph{wo-thorak-wr}\\
	\Fsg-appear.\Ext-\Ndu\\
	\trans `I arrive.'
	\label{ex222}
	\ex
	\gll \emph{n-thorak-n} ($\sim$ \emph{n-thorak-rn})\\
	\Fnsg-appear.\Ext-\Du{}\\
	\trans `We (2) arrive.'
	\label{ex223}
	\ex
	\gll \emph{n-thorak-wr}\\
	\Fnsg-appear.\Ext-\Ndu\\
	\trans `We (3+) arrive.'
	\label{ex224}
\end{xlist}
\end{exe}

Note that there are two variants for the \isi{dual} morpheme, \emph{-n} and \emph{-rn} in (\ref{ex223}), which are attested for almost all members of the small class of prefixing verbs. This variation is both intra-speaker and inter-speaker and, thus far, no patterning along social lines could be detected (e.g. age of the speaker, speaker's exposure to other varieties, etc).

\subsubsection{Large plurals with prefixing verbs} \label{positonalnumber}

The prefixing template indexes the sole argument of the \isi{verb} in the prefix, while the suffix slot is not used. We have seen that only a small number of verbs are inherently prefixing (\S{}\ref{prefixingverbsec}), and about fifty stems may enter into this template. The latter group includes \isi{positional} verbs (\S{}\ref{positionalverbs}). I show below that because there is no ambiguity in the reference of the duality marker, all four cells in the paradigm can be exploited. This allows for a fourth \isi{number} value, the \isi{large plural}, which is formed by combining the \isi{dual} marker with a \isi{singular}. Figure \ref{prefnumberm} illustrates the pattern.

\begin{figure}

	\begin{tabular}{|c|c|p{2cm}p{2cm}|}
		\cline{3-4}
		\multicolumn{2}{c|}{}&\multicolumn{2}{c|}{\textsc{duality affix}}\\\cline{3-4}
		\multicolumn{2}{c|}{}&\multicolumn{1}{c}{\Du}&\multicolumn{1}{|c|}{\Ndu}\\\midrule
		\parbox[t]{2mm}{\multirow{6}{*}{\rotatebox[origin=c]{90}{\textsc{person affix}}}}&\parbox[t]{2mm}{\multirow{3}{*}{\rotatebox[origin=c]{90}{\Sg}}}&\multicolumn{1}{c}{\multirow{3}{*}{}}&\\
		&&\multicolumn{1}{c}{large plural}&\multicolumn{1}{c|}{singular}\\
		&&&\\ \cline{2-2}
		&\parbox[t]{2mm}{\multirow{3}{*}{\rotatebox[origin=c]{90}{\Nsg}}}&&\\
		&&\multicolumn{1}{c}{dual}&\multicolumn{1}{c|}{plural}\\
		&&&\\ \midrule
		\multicolumn{4}{c}{}\\
	\end{tabular}
\caption{Principle of distributed number marking for prefixing verbs}
\label{prefnumberm}
\end{figure}%prefixing verb and large plurals

Consider example (\ref{ex715}) below. The speaker in the story has been away from Rouku for a long time. He asks his brother whether the palm wine containers are still hanging, and the brother replies `there are plenty'. This is expressed by the copula in \isi{dual} and the prefix in \isi{singular}. Note that the stem of the copula is sensitive to \isi{dual} versus \isi{non-dual}. I used the \isi{gloss} label \Lpl{} for \isi{large plural}.

\begin{exe}
	\ex \emph{``eh ngthé bana! sgeru komnzo emithgr?'' ``ah, segeru komnzo \textbf{yrn}''}\\
	\glll eh ngthé bana sgeru komnzo e-mi-thgr ah\\
	hey {brother} poor {palm.wine} still \Stnsg:\Alph-hang.\Ext-\Stat.\Ndu{} ah\\
	{} {} {} {} {} \footnotesize{\Stpl:\Sbj:\Nonpast:\Stat/hang} {}\\
	\sn
	\glll segeru komnzo y-rn\\
	{palm.wine} still \Tsg.\Masc:\Alph-\Cop.\Du\\
	{} {} \footnotesize{\Third\Lpl:\Sbj:\Nonpast:\Ipfv/be}\\
	\trans ``Hey brother, are the palm wine (containers) still hanging?'' ``Yes, there are still plenty.''\Corpus{tci20130927-06}{MAB \#189}
	\label{ex715}
\end{exe}

Examples (\ref{ex225}-d) are elicited forms showing the \isi{positional} verb \emph{räzsi} `erect, stand up' in all four \isi{number} values.\footnote{Note that we find the same variation in the dual morpheme (\emph{-n} and \emph{-rn}) as with other prefixing verbs. Compare with examples \ref{ex222}-c above.}

\begin{exe}
\ex
\begin{xlist}
	\ex
	\gll \emph{woz} \emph{w-räs-thg-r}\\
	bottle \Tsg.\F-erect-\Stat-\Ndu\\
	\trans `The bottle is standing.'
	\label{ex225}
	\ex
	\gll \emph{woz} \emph{e-räs-thg-n} ($\sim$ \emph{e-räs-thg-rn})\\
	bottle \Stnsg-erect-\Stat-\Du\\
	\trans `The two bottles are standing.'
	\label{ex226}
	\ex
	\gll \emph{woz} \emph{e-räs-thg-r}\\
	bottle \Stnsg-erect-\Stat-\Ndu\\
	\trans `The bottles are standing.'
	\label{ex227}
	\ex
	\gll \emph{woz} \emph{y-räs-thg-n} ($\sim$ \emph{y-räs-thg-rn})\\
	bottle \Tsg.\Masc-erect-\Stat-\Du\\
	\trans `All the bottles are standing.' or `Many bottles are standing.'
	\label{ex228}
\end{xlist}
\end{exe}

Example (\ref{ex228}) shows the \isi{large plural} construction in which the seemingly non-sensical combination of a \isi{singular} in the \isi{person} prefix and a \isi{dual} in the duality slot yields a \isi{large plural} or exhaustive \isi{plural} interpretation. There are some restrictions to the large pural. First, as we have seen, it only occurs in the prefixing template. Even though a stem like \emph{räz-} `erect' can appear in a middle or ambifixing template, it cannot form large plurals in these templates. Secondly, large plurals only occur in third \isi{person}, not in first or second. Note that it is always the \isi{masculine} prefix which is used in the \isi{large plural} construction, even if the referent is \isi{feminine}, as with \emph{woz} `bottle' (\ref{ex225}). In this way, the \isi{large plural} construction substantiates the principle of \isi{distributed exponence}, whereby the morphological material at the language's disposal is employed in ways that are not predictable by looking at individual morphemes.\\

Unfortunately, the \isi{large plural} construction is attested only once in the corpus (\ref{ex715}). The evidence presented above comes from eliciation.\footnote{I want to thank Nick Evans for pointing out the combinatorial possibility (\Sg+\Du) in \ili{Nen} (\citealt{Evans:2014bz}) which allowed me to test this pattern with Komnzo speakers.} Although the \isi{large plural} is readily understood and judged grammatical by all my informants, I have not overheard it in daily conversation. Speakers commonly refer to this construction as `a way the old people spoke'. Therefore, we have to assume that it will fade from the speakers' \isi{passive} knowledge eventually and disappear altogether. In fact, the speaker in example (\ref{ex715}) was an older man.\\

Although on different levels of comparison, \isi{dual} marking in pre-stem position and the formation of large plurals are not compatible. This is partly caused by the \isi{stative} semantics of verbs in the prefixing template. For example, positionals take the \isi{stative} suffix \emph{-thgr} which blocks all \isi{perfective} semantics. Pre-stem \isi{dual} marking on the other hand occurs only with restricted stems, and restricted stems are used to form perfectives. A \isi{positional} verb like \emph{räzsi} `erect', can occur outside the prefixing template and form perfectives, but in this case the \isi{large plural} does not apply. We saw in \S{}\ref{prefixingverbsec}, that there are some prefixing verbs, which are not \isi{stative}, for example \emph{yarenzsi} `look around' or \emph{ziksi} `turn to side'. These do form perfectives in the prefixing template. However, the \isi{large plural} combination results in an ungrammatical inflection.\\

I suggest that a historical perspective explains why this is the case. I show in \S{}\ref{prerootdual}, that pre-stem \isi{dual} marking is messier than post-stem \isi{dual} marking in the sense that it is less segmentable and there are more patterns of syncretism. I have argued in \S{}\ref{comparativenoteextrs} that pre-stem \isi{dual} marking is an innovation, and that post-stem \isi{dual} marking is an older pattern. Thus, the \isi{large plural} construction has not survived the change in the pattern shift. Therefore, prefixing verbs with dynamic semantics cannot form large plurals in their perfectives.

\subsubsection{Allomorphy in the post-stem duality slot} \label{allomorphdualsuffix}

Before I turn to the \isi{dual} marking in pre-stem position with restricted stems, I discuss the topic of allomorphy in post-stem position. The \isi{dual} morpheme in the duality slot shows little variation. The above described variation between \emph{-n} and \emph{-rn} is found with prefixing verbs only; elsewhere the \isi{dual} morpheme is always \emph{-n}. As for the \isi{non-dual} morpheme, the situation is different. There are three allomorphs (\emph{wr-}, \emph{nzr-}, \emph{-r}) and their distribution is phonologically conditioned by the final element of the verb stem. The conditioning rules layed out in Table \ref{allonondual} account for 85\% (275/322) of the attested \isi{verb} lexemes.

\begin{table}
	\caption{Allomorphs of the non-dual suffix}
\begin{tabularx}{\textwidth}{lllll}
	\label{allonondual}\\
	\lsptoprule
	\textsc{formatives} & \textsc{rule} &\textsc{count}& \multicolumn{1}{c}{\textsc{example}}& \textsc{gloss}\\
	\midrule
	%\endfirsthead
	\textsc{formative} & \textsc{rule} &\textsc{count}& \multicolumn{1}{c}{\textsc{example}}& \textsc{gloss}\\
	\midrule
	%\endhead
	\emph{-wr}& / k]\textsubscript{\tiny{stem}}\_&92& \emph{mätrak-}& `bring out'\\
	&&& \emph{wek-}&`invite'\\
	& / g]\textsubscript{\tiny{stem}}\_	&38& \emph{mäyog-}& `repeat'\\
	&&& \emph{brig-}&`return'\\
	& / n]\textsubscript{\tiny{stem}}\_	&34& \emph{wathkn-}& `pack up'\\
	&&& \emph{myukn-}&`twist'\\
	& / r]\textsubscript{\tiny{stem}}\_	&25& \emph{rsr-}& `fish (poison)'\\
	&&& \emph{wagr-}&`meet'\\\midrule
	\emph{-nzr}	& / V]\textsubscript{\tiny{stem}}\_&62& \emph{yagu-}& `pour out'\\
	&&& \emph{yafü-}& `open'\\
	&&& \emph{mrä-}& `stroll'\\
	&&& \emph{fsi-}& `count'\\
	&&& \emph{tha-}& `uncover'\\\midrule
	\emph{-r}& / z]\textsubscript{\tiny{stem}}\_&24& \emph{brüz-}& `submerge'\\
	&&& \emph{rifthz-}& `hide'\\
	&&& \emph{räz-}& `erect'\\\midrule
	\textsc{total}&&275&&\\
	\lspbottomrule
\end{tabularx}%Allomorphs of the \isi{non-dual} suffix
\end{table}


The remaining 15\% of \isi{verb} lexemes are irregular (i) in taking a different formative to mark \isi{non-dual} (e.g. \emph{-thr} or \emph{-\Zero}), (ii) in taking one of the three allomorphs under violation of the conditioning rules or (iii) in expressing the \isi{dual}/\isi{non-dual} contrast by irregular changes in the verb stem, for example \emph{moth} `walk' (\emph{-yak} \Ndu{} vs. \emph{-yan} \Du) or \emph{kwan} `shout' (\emph{-nor} \Ndu{} vs. \emph{-rn} \Du).

\subsubsection{Pre-stem dual marking with restricted stems} \label{prerootdual}

The previous discussion concentrated on \isi{dual} marking with extended stems. For restricted stems, this suffix slot is not available and the \isi{dual} vs. \isi{non-dual} contrast is marked in the vowel of the prefix, which changes to \emph{ä} for \isi{non-dual}. Pre-stem \isi{dual} marking is relevant only for those TAM categories which build their inflection on the \isi{restricted stem}. These are verbs inflected for \isi{iterative} and \isi{perfective} \isi{aspect}. The latter include indicative (\isi{recent past} and \isi{past} \isi{tense}), \isi{imperative} or \isi{irrealis} forms. In the following description, I use the \isi{irrealis} \isi{perfective} forms to explain the pattern and point to other TAM categories where they deviate.\\

Interestingly, it is the \isi{non-dual} that receives a marker (\emph{ä-}), while the \isi{dual} is \isi{zero} marked. At the same time, pre-stem \isi{dual} marking is less segmentable and harder to \isi{gloss} than post-stem \isi{dual} marking, because the \isi{non-dual} \emph{ä} vowel superposes vowels from other prefixal material, for example the \isi{valency} changer \emph{a-} or the \isi{irrealis} prefix \emph{ra-}. This leads to patterns of syncretism which span several grammatical dimensions (\isi{valency}, \isi{number}, \isi{aspect}, \isi{mood}, etc).\\

Ir\isi{realis} mood is expressed by the prefix \emph{ra-}, which directly follows the \isi{person}/\isi{number} prefix or the middle marker of the \Bet{} \isi{prefix series} (see Table \ref{perspref} in \S{}\ref{personprefsection}). The \isi{non-dual} marker \emph{ä} replaces the vowel of the \emph{ra-} prefix for all the \isi{person}/\isi{number} combinations which involve a \isi{non-dual} \isi{participant}. This pattern is uniform for prefixing as well as ambifixing verbs. Below in (\ref{ex243}-\ref{ex247}), I provide textual examples of the \isi{number} combinations with a third \isi{person} actor and a first \isi{person} \isi{undergoer}.\footnote{Irrealis mood may be used in narratives for pragmatic reasons (backgrounding) and refer to events which actually took place (\S{}\ref{TAMsemmood})} We find the \emph{ä} vowel for the following actor>undergoer combinations: \Sg>\Sg{} (\ref{ex243}), \Pl>\Sg{} (\ref{ex246}), \Sg>\Pl{} (\ref{ex249}) and \Pl>\Pl{} (\ref{ex250}).

\begin{exe}
	\ex \emph{adif nima \textbf{kwräs} ``ranzo?''}\\
	\glll adi=f nima kw-rä-s-\Zero{} ra=nzo\\
	aunt=\Erg.\Sg{} \Quot{} \Fsg.\Bet-\Irr.\Ndu-ask.\Rs-\Stsg{} what=\Only\\
	{} {} \footnotesize{\Stsg:\Sbj>\Fsg:\Obj:\Irr:\Pfv/ask} {}\\
	\trans `Aunt asked me: ``What is it?'''\Corpus{tci20120922-25}{ALK \#15-16}
	\label{ex243}
\end{exe}
\begin{exe}
	\ex \emph{yare kma nzä nafa \textbf{kwrakarth}.}\\
	\glll yare kma nzä nafa kw-ra-kar-th\\
	bag \Pot{} \Fsg.\Abs{} \Tnsg.\Erg{} \Fsg.\Bet-\Irr.\Du-pull.\Rs-\Stnsg\\
	{} {} {} {} \footnotesize{\Stdu:\Sbj>\Fsg:\Obj:\Irr:\Pfv/pull}\\
	\trans `They (2) should take the bag from me.'\Corpus{tci20130907-02}{JAA \#10}
	\label{ex245}
\end{exe}
\begin{exe}
	\ex \emph{ngatha fäth ferä nafa \textbf{kwränbrmth} e ...}\\
	\glll ngatha fäth f=e-rä nafa\\
	dog \Dim{} \Dist=\Stnsg.\Alph-\Cop.\Ndu{} \Tnsg.\Erg{}\\
	{} {} \footnotesize{\Dist=\Stpl:\Sbj:\Nonpast/be} {}\\
	\sn
	\glll kw-rä-n-brm-th e (.)\\
	\Fsg{}.\Bet{}-\Irr.\Ndu-\Venit-follow.\Rs-\Stnsg{} until (.)\\
	\footnotesize{\Stpl:\Sbj>\Fsg:\Obj:\Irr:\Pfv:\Venit/follow} {} {}\\
	\trans `The small dogs over there, they follow me until...'\Corpus{tci20111119-03}{ABB \#94}
	\label{ex246}
\end{exe}
\begin{exe}
	\ex \emph{foba \textbf{nzrans} ``bä mon ern?''}\\
	\glll foba nz-ra-n-s-\Zero{} bä mon e-rn\\
	\Dist.\Abl{} \Fnsg.\Bet-\Irr.\Du-\Venit-ask.\Rs-\Stsg{} \Second.\Abs{} how \Stnsg.\Alph-\Cop.\Du\\
	{} \footnotesize{\Stsg:\Sbj>\Fdu:\Obj:\Irr:\Pfv:\Venit/ask} {} {} \footnotesize{\Stdu:\Sbj:\Nonpast:\Ipfv/be}\\
	\trans `He asked us (2): ``Who are you?'''\Corpus{tci20120904-02}{MAB \#125}
	\label{ex248}
\end{exe}
\begin{exe}
	\ex \emph{paituaf \textbf{nzräkor} ``nzä fiyafr wiyak.''}\\
	\glll paitua=f nz-rä-kor-\Zero{} nzä fiyaf=r\\
	old.man=\Erg.\Sg{} \Fnsg.\Bet-\Irr.\Ndu-speak.\Rs-\Stsg{} \Fsg.\Abs{} hunting=\Purp{}\\
	{} \footnotesize{\Stsg:\Sbj>\Fpl:\Obj:\Irr:\Pfv/speak} {} {}\\
	\sn
	\glll wo-yak\\
	\Fsg.\Alph-walk.\Ext.\Ndu\\
	\footnotesize{\Fsg:\Nonpast:\Ipfv/walk}\\
	\trans `He said to us: ``I will go hunting.'''\Corpus{tci20120821-02}{LNA \#11-12}
	\label{ex249}
\end{exe}
\begin{exe}
	\ex \emph{kar zf rä zf masu ... manema \textbf{nzräkorth} masu kar.}\\
	\glll kar zf rä zf masu (.) mane=ma\\
	place \Imm{} \Tsg.\F.\Cop.\Ndu{} \Imm{} masu (.) which=\Char{}\\
	{} {} \footnotesize{\Tsg.\F:\Sbj:\Nonpast.\Ipfv/be} {} {} {} {}\\
	\sn
	\glll
	nz-rä-kor-th masu kar\\
	\Fnsg.\Bet-\Irr.\Ndu-speak.\Rs-\Stnsg{} masu place.\\
	\footnotesize{\Stpl:\Sbj>\Fpl:\Obj:\Irr:\Pfv/speak} {} {}\\
	\trans `This place right here is Masu, which is why they call us Masu people.'\\\Corpus{tci20120922-08}{DAK \#87}
	\label{ex250}
\end{exe}
\begin{exe}
	\ex \emph{ni \textbf{nzrakorth} ``bä!'' ... oroman babua ... ``bä kwa ŋakwinth zmbär aki kwayanen!''}\\
	\glll ni nz-ra-kor-th bä (.) oroman babua (.) bä kwa ŋ-a-kwi-n-th zmbär aki kwayan=en\\
	\Fnsg{} \Fnsg.\Bet-\Irr.\Du-speak.\Rs-\Stnsg{} \Second.\Abs{} (.) old.man babua (.) \Second.\Abs{} \Fut{} \M.\Alph-\Vc-run.\Ext-\Du-\Stnsg{} night moon light=\Loc\\
	{} \footnotesize{\Stpl:\Sbj>\Fdu:\Obj:\Irr:\Pfv/speak} {} {} {} {} {} {} {} \footnotesize{\Stdu:\Sbj:\Nonpast:\Ipfv/run} {} {} {}\\
	\trans `They said to us (2): ``You!'' to old man Babua ``You two will run at night in the moonlight'''\Corpus{tci20120904-01}{MAB \#135-137}
	\label{ex247}
\end{exe}

Note that just like in post-stem \isi{dual} marking (\S{}\ref{ambiguitiesdualref}), pre-stem \isi{dual} marking is ambiguous as to which of the two arguments is \isi{dual} or \isi{non-dual}. The verb \emph{nzrakorth} `they said to us' in (\ref{ex247}) could be any of the three possible actor>\isi{undergoer} combinations (\Pl>\Du, \Du>\Du{} or \Du>\Pl) because both \isi{person} affixes index a \isi{non-singular} \isi{participant}. Thus, the absence of the \emph{ä} vowel indicates that one of the two participants is \isi{dual}, but not which one. Only context may solve this structural ambiguity, which in (\ref{ex247}) is clear from the second verb \emph{ŋakwinth} `you two go'. For verbs in a prefixing template, there is no ambiguity since they index only one argument. Non-\isi{dual} participants receive the \emph{ä} vowel, while \isi{dual} participants do not. The same holds for verbs in the middle template.\\

The marking pattern can be expressed in an abstract matrix as in Figure \ref{raezerorae}. In terms of structure, not in its formatives, this matrix is identical to post-stem duality marking (see Figure \ref{dualitymatrixfathasi} above).

\begin{figure}

	\begin{tabular}{|ccccccc|}
		\cline{3-5}
		\multicolumn{2}{c|}{}&\multicolumn{3}{c|}{\textsc{actor}}&\\\cline{3-5}
		\multicolumn{2}{c|}{}&\multicolumn{1}{c}{\Sg}&\multicolumn{1}{|c|}{\Du}&\multicolumn{1}{c|}{\Pl}&\\\cline{1-5}
		{\parbox[t]{2mm}{\multirow{6}{*}{\rotatebox[origin=l]{90}{\textsc{undergoer}\hspace{0,1cm}}}}}&\multicolumn{1}{|c|}{\parbox[t]{2mm}{\multirow{2}{*}{\rotatebox[origin=c]{90}{\Sg}}}}&\multicolumn{1}{c}{\multirow{2}{*}{\hspace{0,1cm}ä\hspace{0,1cm}}}&\multirow{2}{*}{\hspace{0,1cm}\Zero{}\hspace{0,1cm}}	&\multicolumn{1}{c|}{\multirow{2}{*}{\hspace{0,1cm}ä{\hspace{0,1cm}}}}&\\
		&\multicolumn{1}{|c|}{}&&&\multicolumn{1}{c|}{}\\\cline{2-2}
		&\multicolumn{1}{|c|}{\parbox[t]{2mm}{\multirow{2}{*}{\rotatebox[origin=c]{90}{\Du}}}}&\multicolumn{1}{c}{\multirow{2}{*}{\hspace{0,1cm}\Zero{}\hspace{0,1cm}}}&\multirow{2}{*}{\hspace{0,1cm}\Zero{}\hspace{0,1cm}}&\multicolumn{1}{c|}{\multirow{2}{*}{\hspace{0,1cm}\Zero{}\hspace{0,1cm}}}&\\
		&\multicolumn{1}{|c|}{}&&&\multicolumn{1}{c|}{}\\\cline{2-2}
		&\multicolumn{1}{|c|}{\parbox[t]{2mm}{\multirow{2}{*}{\rotatebox[origin=c]{90}{\Pl}}}}&\multicolumn{1}{c}{\multirow{2}{*}{\hspace{0,1cm}ä\hspace{0,1cm}}}&\multirow{2}{*}{\hspace{0,1cm}\Zero{}\hspace{0,1cm}}&\multicolumn{1}{c|}{\multirow{2}{*}{\hspace{0,1cm}ä\hspace{0,1cm}}}&\\
		&\multicolumn{1}{|c|}{}&&&\multicolumn{1}{c|}{}\\\cline{1-5}
	\end{tabular}
\caption{The duality matrix without \Vc{} prefix}
\label{raezerorae}
\end{figure}%duality matrix without VC

There are some exceptions for the third singular prefixes (both feminine and masculine). The combination of \Sg>\Tsg{} in the ambifixing template and \Tsg{} in the prefixing template receive the vowel \emph{a} and not \emph{ä} in all relevant TAM categories. In the imperatives, it is \emph{a} for both combinations \Sg>\Tsg{} and \Pl>\Tsg{}. Inflections involving a \isi{dual} \isi{participant} would receive a \isi{zero} marker. In a discussion after listening to old recordings made by the anthropologist Mary Ayres in the 1980's, I was able to elicit one inflectional form that is relevant to this topic. The informant contrasted the modern Komnzo inflection \emph{santhor} `He arrived here' with an older form of the same verb \emph{snäthor}.\footnote{\parbox{0.02cm}{\hfill}\parbox{6cm}{\emph{s-a-n-thor}} \parbox{5cm}{\emph{s-n-ä-thor}}\\ \parbox{0.1cm}{\hfill}\parbox{6cm}{\Tsg.\Masc.\Gam-\Ndu-\Venit-arrive.\Rs{}}  \parbox{15cm}{\Tsg.\Masc.\Gam-\Venit-\Ndu-arrive.\Rs}} A first observation is that the \emph{ä} does occur in the older form. Interestingly, it occurs after the \isi{ventive} \emph{n-} prefix. At the current stage of documentation, not much can be said about the time frame during which this change has occured. The informant who provided this information is now in his mid-60's and he remembers `old people' using this form. I was not able to elicit a full paradigm of these older inflections and, thus, we are denied insight into the changes that took place in the verb template. As for now, we can only state that the \isi{non-dual} \emph{ä} vowel existed at some point in time with third singulars in the prefix.\\

As I mentioned above, since pre-stem duality marking involves the \emph{ä} vowel, it occupies a slot in the template which may be filled by other prefixal material, for example the irrealis prefix \emph{ra-} and the \isi{valency} changer \emph{a-}, or both. We saw in the examples above, that the \isi{non-dual} \emph{ä} vowel superposes the irrealis \emph{ra-} prefix which results in the form \emph{rä-}. This is not the case for the imperatives and indicative inflected verbs. As we have seen in \S{}\ref{combinatoricsextrs}, restricted stems combine only with prefixes of the \Bet{}, \Betatwo{} and \Gam{} series. Most formatives of these series are composed of only a consonant (See Table \ref{perspref} in \S{}\ref{personprefsection}). Only the \Fsg.\Gam{} (\emph{zu-}) and all formatives of the \Betatwo{} series end in /u/, which resyllabifies as part of a complex onset (\emph{zw-}) in the presence of \emph{ä} or \emph{a}. For example, the \Fsg.\Gam{} \emph{zu-} in (\ref{ex258}) is followed by a \isi{zero}. Therefore, the verb is inflected for \isi{dual}. In (\ref{ex257}), the \Fsg.\Gam{} is followed by the \isi{non-dual} \emph{ä} vowel and the prefix changes into \emph{zwä-}. Therefore, I analyse the distribution of the \emph{ä} vowel as was shown above in Figure \ref{raezerorae}.

\begin{exe}
	\ex \emph{nzä nima \textbf{zukorth}: ``be fafä zane nagayé fäth zä thamonegwé!''}\\
	\glll nzä nima zu-\Zero-kor-th be fafä zane\\
	\Fsg.\Abs{} \Quot{} \Fsg.\Gam-\Du-speak.\Rs-\Stnsg{} \Ssg.\Erg{} after.this \Dem:\Prox{}\\
	{} {} \footnotesize{\Stdu:\Sbj>\Fsg:\Obj:\Rpst:\Pfv/speak} {} {} {}\\
	\sn
	\glll nagayé fäth zä th-a-moneg-w-é\\
	children \Dim{} \Prox{} \Stnsg.\Bet-\Vc-wait.\Ext-\Ndu-\Ssg.\Imp{}\\
	{} {} {} \footnotesize{\Ssg:\Sbj>\Stpl:\Io:\Imp:\Ipfv/wait}\\
	\trans `They (2) said to me: ``You will look after these small children here later!'''\\\Corpus{tci20121019-04}{ABB \#97}
	\label{ex258}
\end{exe}
\begin{exe}
	\ex \emph{watik, naf \textbf{zwäkora}: ``watik, nzone efoth fof zefafth.''}\\
	\glll watik naf zu-ä-kor-a-\Zero{} watik nzone efoth fof z-ä-faf-th\\
	then \Tsg.\Erg{} \Fsg.\Gam-\Ndu-speak.\Rs-\Pst-\Stsg{} then \Fsg.\Poss{} sun|day \Emph{} \M.\Gam-\Ndu.\Vc-hold.\Rs-\Stnsg{}\\
	{} {} \footnotesize{\Stsg:\Sbj>\Fsg:\Obj:\Pst:\Pfv/speak} {} {} {} {} \footnotesize{\Stnsg:\Sbj:\Pst:\Pfv/hold}\\
	\trans `Then she said to me: ``Well, my days are over now.'''\Corpus{tci20130911-03}{MBR \#76}
	\label{ex257}
\end{exe}

Pre-stem duality marking co-occurs with the \isi{valency change} prefix \emph{a-}. The resulting vowel pattern is summarised in the matrix in Figure \ref{raezeroraevalchange}, which shows that the \isi{non-dual} \emph{ä} vowel (i) replaces the \emph{a-} prefix and (ii) that it patterns differently to the forms given so far. Compare Figure \ref{raezerorae} above with Figure \ref{raezeroraevalchange} below. Note that this neutralises the \isi{valency change} prefix \emph{a-} for some of the actor>\isi{undergoer} combinations: \Pl>\Sg{}, \Sg>\Pl{} and \Pl>\Pl{}. For these combinations, it is only the \isi{case} frame which identifies whether the \isi{undergoer} argument is a direct \isi{object} (\Abs{} \isi{case}) or an indirect object (\Dat{} or \Poss{} \isi{case}).

\begin{figure}

	\begin{tabular}{|ccccccc|}
		\cline{3-5}
		\multicolumn{2}{c|}{}&\multicolumn{3}{c|}{\textsc{actor}}&\\\cline{3-5}
		\multicolumn{2}{c|}{}&\multicolumn{1}{c}{\Sg}&\multicolumn{1}{|c|}{\Du}&\multicolumn{1}{c|}{\Pl}&\\\cline{1-5}
		{\parbox[t]{2mm}{\multirow{6}{*}{\rotatebox[origin=l]{90}{\textsc{undergoer}\hspace{0,15cm}}}}}&\multicolumn{1}{|c|}{\parbox[t]{2mm}{\multirow{2}{*}{\rotatebox[origin=c]{90}{\Sg}}}}&\multicolumn{1}{c}{\multirow{2}{*}{\hspace{0,15cm}a\hspace{0,15cm}}}	&\multirow{2}{*}{\hspace{0,15cm}a\hspace{0,15cm}}	&\multicolumn{1}{c|}{\multirow{2}{*}{\hspace{0,15cm}ä\hspace{0,15cm}}}&\\
		&\multicolumn{1}{|c|}{}&&&\multicolumn{1}{c|}{}\\\cline{2-2}
		&\multicolumn{1}{|c|}{\parbox[t]{2mm}{\multirow{2}{*}{\rotatebox[origin=c]{90}{\Du}}}}&\multicolumn{1}{c}{\multirow{2}{*}{\hspace{0,15cm}a\hspace{0,15cm}}}&\multirow{2}{*}{\hspace{0,15cm}a\hspace{0,15cm}}&\multicolumn{1}{c|}{\multirow{2}{*}{\hspace{0,15cm}a\hspace{0,15cm}}}&\\
		&\multicolumn{1}{|c|}{}&&&\multicolumn{1}{c|}{}\\\cline{2-2}
		&\multicolumn{1}{|c|}{\parbox[t]{2mm}{\multirow{2}{*}{\rotatebox[origin=c]{90}{\Pl}}}}&\multicolumn{1}{c}{\multirow{2}{*}{\hspace{0,15cm}ä\hspace{0,15cm}}}&\multirow{2}{*}{\hspace{0,15cm}a\hspace{0,15cm}}&\multicolumn{1}{c|}{\multirow{2}{*}{\hspace{0,15cm}ä\hspace{0,15cm}}}&\\
		&\multicolumn{1}{|c|}{}&&&\multicolumn{1}{c|}{}\\\cline{1-5}
	\end{tabular}
\caption{The duality matrix with \Vc{} prefix}
\label{raezeroraevalchange}
\end{figure}%duality matrix with VC

One exception is the combination of \Sg>\Sg{}. As we can see in Figure \ref{raezeroraevalchange}, this combination receives no \emph{ä} vowel although both participants are \isi{non-dual}. This pattern is regular for all persons. Thus, a \Pl>\Tsg{} would receive \emph{ä}, whereas \Du>\Tsg{} and \Sg>\Tsg{} would not receive it. For the last combination and all prefixing verbs with a \Tsg{} this means that the \isi{valency change} is neutralised and again only the \isi{case} frame shows what type of \isi{undergoer} is indexed. It is not neutralised for the other \isi{person} values (\Sg>\Fsg{}, \Sg>\Ssg{} and \Fsg{}, \Ssg{} on prefixing verbs) precisely because \Sg>\Sg{} (and the \Sg{} in prefixing verbs) does not take \emph{ä} but \emph{a}.\\

Note that prefixing verbs with the \isi{valency change} prefix \emph{a-} show a pattern where \emph{ä} only occurs on a \isi{plural}, while \emph{a} occurs with a singular and \isi{dual} \isi{participant}. At least on the surface, this results in the binary opposition of \isi{plural} vs. \isi{non-plural}. In (\ref{ex259}) below, the prefixing verb \emph{rfiksi} `grow' occurs in the inflected form \emph{zarfif} `sth. grew for/over it'. From the context, it is clear that the speaker is talking about the grass growing over the path. The verb encodes a \isi{feminine} \isi{undergoer}, which can only be interpreted as being the pathway (\emph{moth}), because \emph{yusi} `grass' is masculine. A \isi{dual} number of the \isi{undergoer} would be \emph{tharfif} and a \isi{plural} \emph{thärfif}. Thus, under several conditions (presence of \isi{valency change}, prefixing template, \isi{restricted stem}), the duality marker marks an opposition between \isi{plural} and \isi{non-plural}.

\begin{exe}
	\ex \emph{gathagatha moth rä ... z wrfrwake we ane \textbf{zarfif}.}\\
	\glll {gathagatha} moth rä (.) z\\
	bad path \Tsg.\F:\Cop:\Ndu{} (.) \Iam{}\\
	{} {} \footnotesize{\Tsg.\F:\Sbj:\Nonpast:\Ipfv/be} {} {}\\
	\sn
	\glll w-rfr-w-a-k-e we ane z-a-rfif\\
	\Tsg.\F.\Alph-trim.\Ext-\Ndu-\Pst-\Lk-\Fnsg{} also \Dem{} \Tsg.\F.\Gam-\Ndu.\Vc-grow.\Rs\\
	\footnotesize{\Fpl:\Sbj>\Tsg.\F:\Obj:\Pst:\Ipfv/trim} {} {} \footnotesize{\Tsg.\F:\Io:\Rpst:\Pfv/grow}\\
	\trans `This is a bad path. We cut it already, but (the grass) grew over it again.'\\\Corpus{tci20130907-02}{RNA \#39-41}
	\label{ex259}
\end{exe}

Before I conclude this section on \isi{number} marking, I want to look at the behaviour of the \emph{ä} vowel when the \isi{irrealis} prefix \emph{ra-} and \isi{valency change} prefix \emph{a-} come together. Since the \isi{irrealis} prefix includes a vowel, the \isi{valency change} prefix is neutralised in most parts of the paradigm. For extended stems, this \isi{neutralisation} is complete, i.e. only the \isi{case} frame indicates whether the \isi{undergoer} argument is a direct \isi{object} (\Abs{}) or an \isi{indirect object} (\Dat{} or \Poss{}). This will be further discussed in \S{}\ref{irrealisra}. For restricted stems, the \isi{valency change} prefix \emph{a-} is likewise neutralised, but the \isi{number} marking pattern differs in those actor>\isi{undergoer} combinations which involve \Sg>\Sg{} (Figure \ref{raezeroraevalchange}). Consider the vowel contrast between (\ref{ex243}) which was given above and (\ref{ex251}) below. The \isi{participant} combination is held constant: \Tsg>\Fsg{}. In (\ref{ex243}) we find the \emph{ä} vowel, because it is \isi{ditransitive} and the \isi{valency change} prefix \emph{a-} is employed, but in (\ref{ex251}) it is missing, because (\ref{ex243}) is \isi{transitive} and lacks the \emph{a-} prefix. Compare (\ref{ex251}) with (\ref{ex252}) where the same verb \emph{yarisi} `give' shows the \emph{ä} because the actor \isi{participant} is \isi{plural}.

\begin{exe}
	\ex \emph{nafane bärbärnzo keke \textbf{kwrar}.}\\
	\glll nafane {bärbär=nzo} keke kw-ra-r-\Zero\\
	\Tsg.\Poss{} {half=\Only} \Neg{} \Fsg.\Bet{}-\Irr.\Ndu.\Vc-give.\Rs-\Stsg\\
	{} {} {} \footnotesize{\Stsg:\Sbj>\Fsg:\Io:\Irr:\Pfv/give}\\
	\trans `She will not give me half of her (fish).'\Corpus{tci20120922-26}{DAK \#125}
	\label{ex251}
\end{exe}
\begin{exe}
	\ex \emph{nä kwot \textbf{kwrärth} fafä.}\\
	\glll nä kwot kw-rä-r-th fafä\\
	some again \Fsg.\Bet-\Irr.\Pl.\Vc-give.\Rs-\Stnsg{} after.that\\
	{} {} \footnotesize{\Stpl:\Sbj>\Fsg:\Io:\Irr:\Pfv/give} {}\\
	\trans `They might give me some more later.'\Corpus{tci20120805-01}{ABB \#226}
	\label{ex252}
\end{exe}

We can conclude from the examples that the \isi{irrealis} inflection complies with the \isi{number} marking patterns as they were shown in Figure \ref{raezeroraevalchange} above. The only difference lies in the fact that the \isi{irrealis} prefix \emph{ra-} creates neutralisations in more combinations (with regard to the \isi{valency change}) because \emph{ra-} contains a vowel. However, there is one important caveat to this conclusion. As I have pointed out in \S{}\ref{prefixingverbsec} and \S{}\ref{ambifixingtemp}, there are some verbs which are \isi{deponent} in the sense that they obligatorily take the \emph{a-} without a change in the \isi{valency}. Two examples are the \isi{transitive} verb \emph{fiyoksi} `make' and intransitive/prefixing verb \emph{yarenzsi} `look'. Consequently we would expect them to comply with the pattern in Figure \ref{raezeroraevalchange}. Consider example (\ref{ex254}) with a \Sg>\Sg{} \isi{participant} combination and example (\ref{ex255}) with its single referent in \Sg{}. Both show the \emph{ä} \isi{non-dual} vowel, i.e. they violate the pattern in Figure \ref{raezeroraevalchange} which predicts the vowel to be \emph{a} and not \emph{ä}. This violation occurs only with \isi{deponent} verbs and only in \isi{irrealis} \isi{mood}. The natural explanation is that, for \isi{deponent} verbs, the distinction between the presence vs. absence of the \isi{valency change} prefix is redundant.

\begin{exe}
	\ex \emph{katan kwa \textbf{sräfiyothé}. kafar minzü yé.}\\
	\glll katan kwa s-rä-fiyoth-é kafar minzü\\
	small \Fut{} \Tsg.\Masc.\Bet-\Irr.\Ndu.\Vc-make.\Rs-\Fsg{} big very\\
	{} {} \footnotesize{\Fsg:\Sbj>\Tsg.\Masc:\Obj:\Irr:\Pfv/make} {} {}\\
	\sn
	\glll \stem{yé}\\
	\Tsg.\Masc.\Cop.\Ndu\\
	\footnotesize{\Tsg.\Masc:\Sbj:\Nonpast:\Ipfv/be}\\
	\trans `I will make it smaller. It is very big.'\Corpus{tci20120914}{RNA \#41-42}
	\label{ex254}
\end{exe}
\begin{exe}
	\ex \emph{wati, we nima n \textbf{kwräzigrthm} ``eh, ra gru zane ŋamitwanzr nabi tutin?''}\\
	\glll wati we nima n kw-rä-zigrthm eh ra gru\\
	then also \Quot{} \Imn{} \Fsg.\Bet-\Irr.\Ndu.\Vc-look.\Rs{} eh what shooting.star\\
	{} {} {} {} \footnotesize{\Fsg:\Sbj:\Irr:\Pfv/look} {} {} {}\\
	\sn
	\glll zane ŋ-a-mitwa-nzr-\Zero{} nabi tuti=n\\
	\Dem.\Prox{} \M.\Alph-\Vc-swing.\Ext-\Ndu-\Stsg{} bamboo branch=\Loc{}\\
	{} \footnotesize{\Stsg:\Sbj:\Nonpast:\Ipfv/swing} {} {}\\
	\trans `Then, I was about to look around and thought: ``Hey, what is this shooting star swinging on the bamboo branch?'''\Corpus{tci20111119-03}{ABB \#126-127}
	\label{ex255}
\end{exe}

Another observation relevant for all TAM categories with pre-stem \isi{dual} marking is the fact that the \isi{middle} marker also obligatorily takes the \isi{valency change} prefix \emph{a-}. Likewise, a verb in the \isi{middle} template which indexes a \isi{singular} \isi{participant} does not pattern along the lines of Figure \ref{raezeroraevalchange}, and instead it employs the \emph{ä} vowel. Again, this can only be explained by taking into account that there is no need to make a distinction between the presence vs. absence of the \isi{valency change} prefix, because it always occurs with the \isi{middle} morpheme.\\

The patterning of \emph{ä}, \emph{a} and \emph{\Zero} in the prefixes cannot be adequately captured by the traditional notion of a morpheme with a distinct meaning. It seems to be the case that the vowel change is employed only to mark a difference in meaning without being easily linked to a specific meaning. The vowel change or the \emph{ä} vowel in the prefix can be glossed as a \isi{non-dual} for only part of the paradigm. In other parts of the paradigm, the distribution is employed to maximise the possible grammatical categories that can be encoded. Thus, pre-stem duality marking is much messier than post-stem duality marking. Both show some ambiguities and neutralisations, and in both cases the duality marker has to be integrated with the \isi{singular} vs. \isi{non-singular} opposition of the \isi{person} affixes. But at the same time, pre-stem \isi{dual} marking is sensitive to more grammatical categories and shows more idiosyncrasies.

\section{Deixis and directionality} \label{deixisanddirectionality}

Komnzo verbs may be inflected for \isi{deixis} and \isi{directionality}. Deictic inflection comprises the values of \isi{proximal}, \isi{medial}, \isi{distal} and \isi{interrogative}. Directionality comprises a \isi{ventive} (`hither') and an \isi{andative} (`thither') category. Both \isi{deixis} and \isi{directionality} operate from a \isi{deictic} center, which is usually the speaker, but may be extended to cover a particular character or place in a narrative, or a point in time. Morphologically, both sets are simple in that there is a one-to-one mapping between form and function.

\subsection{The directional affixes \emph{n-} and \emph{-o}} \label{directionalinflection}

Directional inflection takes place in two slots on the \isi{verb}: the \isi{ventive} prefix \emph{n-} precedes the verb stem, while the \isi{andative} suffix \emph{-o} occurs in the second last slot on the verb preceding the \isi{person}/\isi{number} suffixes. Although morphologically possible, the two morphemes may not co-occur, i.e. a verb is marked either \isi{ventive} or \isi{andative}. In other \ili{Yam languages}, the two morphemes share one slot in the verb template, for example in \ili{Nen} (\citealt{Evans:2015to}). I have described in {\S{}\ref{personsuffsection}} how the presence of the \isi{andative} suffix can lead to the \isi{neutralisation} of the \isi{person} value in the actor suffix. Example (\ref{ex241}) in that section provided a text example of this \isi{neutralisation}.\\

The use of \isi{directional} marking is shown below in example (\ref{ex262}). The sentence concludes a mythical story which explains why two particular clans do not intermarry, but instead `help each other out' with girls to be exchanged with other groups. The speaker assumes the position of one of the two clans, both spatially as well as in terms of kin relations. The verb \emph{yarisi} `give' is then marked with an \isi{andative} in the first clause (`give away') and a \isi{ventive} (`give towards') in the second clause. Additionally, both clauses contain a \isi{deictic} in ablative \isi{case} (\emph{zba} `from here', \emph{boba} `from there').

\begin{exe}
	\ex \emph{zba nezä \textbf{ärithroth} fäms ŋarer. boba nezä \textbf{änrithrth} fäms ŋarer}\\
	\glll zba nezä e-a-ri-thr-o-th fäms\\
	\Prox.\Abl{} in.return \Stnsg.\Alph-\Vc-give.\Ext-\Ndu-\Andat-\Nsg{} exchange\\
	{} {} \footnotesize{\Stpl:\Sbj>\Stpl:\Io:\Nonpast:\Ipfv:\Andat/give} {}\\
	\sn
	\glll ŋare=r boba nezä e-a-n-ri-thr-th\\
	woman=\Purp{} \Med.\Abl{} {in return} \Stnsg.\Alph-\Vc-\Venit-give.\Ext-\Ndu-\Stnsg{}\\
	{} {} {} \footnotesize{\Stpl:\Sbj>\Stpl:\Io:\Nonpast:\Ipfv:\Venit/give} {} {}\\
	\sn
	\gll fäms ŋare=r\\
	exchange woman=\Purp{}\\
	\trans `From here, they give them girls to exchange. In return, they give them girls to exchange from there.'\Corpus{tci20110802}{ABB \#159-161}
	\label{ex262}
\end{exe}

The \isi{directional} affixes can be used with dynamic events as in (\ref{ex262}) or with stative verbs as in (\ref{ex260}), which is taken from the description of a picture card. The image depicts an older man who is standing in the background watching what is happening. The \isi{ventive} inflection on `stand' refers to the direction of his posture, i.e. he is standing facing towards the \isi{deictic} centre.

\begin{exe}
	\ex \emph{wotukarä ane \textbf{ynkogr}. sinzo foba \textbf{ynrä}.}\\
	\glll wotu=karä ane y-n-kogr si=nzo foba\\
	stick=\Prop{} \Dem{} \Tsg.\Masc.\Alph-\Venit-stand.\Ndu{} eye=\Only{} \Dist:\Abl{}\\
	{} {} \footnotesize{\Tsg.\Masc:\Sbj:\Nonpast:\Ipfv:\Venit/stand} {} {}\\
	\sn
	\glll y-n-rä\\
	\Tsg.\Masc.\Alph-\Venit-\Cop.\Ndu\\
	\footnotesize{\Tsg.\Masc:\Sbj:\Nonpast:\Ipfv:\Venit/be}\\
	\trans `He stands there with his walking stick and he is just looking from there.'\\\Corpus{tci20111004}{RMA \#253}
	\label{ex260}
\end{exe}

The \isi{copula} may receive a \isi{directional} inflection, giving the interpretation of `come' (\ref{ex260}) and `go' (\ref{ex264}), literally translated as `be hither' and `be thither'.

\begin{exe}
	\ex \emph{watik, teacher zwäkor ``keke kayé kwa \textbf{nrno}.''}\\
	\glll watik teacher zu-ä-kor-\Zero{} keke kayé kwa\\
	then teacher \Fsg:\Gam-\Ndu-speak.\Rs-\Stsg{} \Neg{} tomorrow \Fut{}\\
	{} {} \footnotesize{\Stsg:\Sbj>\Fsg:\Obj:\Rpst:\Pfv/speak} {} {} {}\\
	\sn
	\glll n-rn-o\\
	\Fnsg:\Alph-\Cop.\Du-\Andat{}\\
	\footnotesize{\Fdu:\Sbj:\Nonpast:\Ipfv:\Andat/be}\\
	\trans `Then, the teacher said to me: ``No, we will go tomorrow.'''\\\Corpus{tci20130823-06}{STK \#67-68}
	\label{ex264}
\end{exe}

The spatial semantics of \isi{directional} inflection can be extended to cover metaphorical uses. Example (\ref{ex261}) shows a \isi{temporal} use where the speaker explains the old custom of tying a bowstring. Thus, he literally says that he `follows the custom hither'. Example (\ref{ex263}) is a description of a very old woman, who has outlived some of her own children. The speaker uses the \isi{andative} inflection on the verb \emph{yathizsi} `die' which is best translated into \ili{English} as `pass away'.

\begin{exe}
	\ex \emph{nzenme bada nimame zf ŋatr thuzirakwrmth. watik, ni ane \textbf{wänbragwre} zenathamar.}\\
	\gll nzenme bada nima=me zf ŋatr\\
	\Fnsg.\Poss{} ancestor like.this=\Ins{} \Imm{} bowstring\\
	\sn
	\glll thu-zirak-wr-m-th watik ni ane w-a-n-brag-wr-e zena=thamar\\
	\Stnsg.\Betaone{}-tie.\Ext-\Ndu-\Dur-\Stnsg{} then \Fnsg{} \Dem{} \Tsg.\F.\Alph-\Vc-\Venit-follow.\Ext-\Ndu-\Fnsg{} today=\Temp.\All{}\\
	\footnotesize{\Stpl:\Sbj>\Stpl:\Obj:\Pst:\Dur/tie} {} {} {} \footnotesize{\Fpl:\Sbj>\Tsg.\F:\Obj:\Nonpast:\Ipfv:\Venit/follow} {}\\
	\trans `Our ancestors where tying the bowstring this way. We have been following (this custom) until today.'\Corpus{tci20130914-01}{KAB \#1-3}
	\label{ex261}
\end{exe}
\begin{exe}
	\ex \emph{nagayé nafanemäwä nä z \textbf{äthizrako}.}\\
	\gll nagayé nafane=ma=wä nä z\\
	children \Tsg.\Poss=\Char=\Emph{} some \Iam{}\\
	\sn
	\glll e-a-thiz-r-a-k-o\\
	\Stnsg.\Alph-\Vc-die.\Ext-\Ndu-\Pst-\Lk-\Andat{}\\
	\footnotesize{\Stpl:\Sbj:\Pst:\Ipfv:\Andat/die}\\
	\trans `Some of her own children have already passed away.'\\\Corpus{tci20120922-26}{DAK \#54}
	\label{ex263}
\end{exe}

\subsection{The deictic clitics \emph{z=}, \emph{b=}, \emph{f=} and \emph{m=}} \label{deicticcliticssection}

Deictics include the three categories \isi{proximal} \emph{z=}, \isi{medial} \emph{b=} and \isi{distal} \emph{f=}. Additionally, there is an \isi{interrogative} form \emph{m=} which behaves slightly different. These morphemes are analysed as proclitics because they (i) attach to the outer layer of the verb, (ii) are not assigned \isi{stress} (if they create an initial \isi{syllable} through \isi{epenthesis}) and (iii) are reduced forms of the demonstratives. In \S{}\ref{demonstrative-identifiers} and \S{}\ref{clitics} I have labelled these clitic demonstratives.\\

Clitic demonstratives are always used situationally in order to point, direct or show the location of an event or a referent in relation to the \isi{deictic} center. Example (\ref{ex265})\footnote{The verb \emph{-nor} `shout' is deponent and takes the valency change prefix \emph{a-} prefix without an impact on the argument structure.} comes from a narrative. The \isi{deictic} center of that part of the story is a man who sits in his camp and happens to hear someone shouting from the river. Note that both verbs (`hear' and `shout') are inflected with a \isi{ventive} marker. Thus, we can translate the second verb \emph{byannor}, to which the \isi{medial} clitic \isi{demonstrative} (\emph{b=} \Med) is attached, as `He shouts there towards here'.

\begin{exe}
	\ex \emph{nafafämsf srenkaris ``oh, kabe \textbf{byannor} gardar.''}\\
	\glll nafa-fäms=f s-rä-n-karis-\Zero{} oh\\
	\Third.\Poss-exchange.man=\Erg.\Sg{} \Tsg.\Masc.\Bet-\Irr.\Ndu-\Venit-hear.\Rs-\Stsg{} oh\\
	{} \footnotesize{\Stsg:\Sbj>\Tsg.\Masc:\Obj:\Irr:\Pfv:\Venit/hear} {}\\
	\sn
	\glll kabe b=y-a-n-nor garda=r\\
	man \Med=\Tsg.\Masc.\Alph-\Vc-\Venit-shout.\Ext.\Ndu{} canoe=\Purp{}\\
	{} \footnotesize{\Med=\Tsg.\Masc:\Sbj:\Nonpast:\Ipfv:\Venit/shout} {}\\
	\trans `His exchange man heard him (and said:) ``Oh, there is a man calling out for the canoe.'''\Corpus{tci20111119-01}{ABB \#68}
	\label{ex265}
\end{exe}

If the inflected \isi{verb} is vowel initial or begins in a glide (only some formatives of the \Alph{} series), the \isi{clitic} \isi{demonstrative} simply attaches as an onset, for example in (\ref{ex266})\footnote{The verb \emph{msaksi} `sit|dwell' is deponent and takes the valency change prefix \emph{a-} without an impact on the argument structure} or (\ref{ex268}) below. Elsewhere, an initial \isi{syllable} is created through \isi{epenthesis}, as in (\ref{ex265}) and (\ref{ex267}).

\begin{exe}
	\ex \emph{frükakmenzo nzwamnzrm. ane mrn \textbf{fämnzr}. ane mrn \textbf{fämnzr}. ane mrn \textbf{fämnzr}.}\\
	\glll frü-kak=me=nzo nzu-a-m-nzr-m 3x[ane mrn\\
	alone-\Distr=\Ins=\Only{} \Fnsg.\Betatwo-\Vc-sit.\Ext-\Ndu-\Dur{} 3x[\Dem{} clan\\
	{} \footnotesize{\Fpl:\Sbj:\Pst:\Dur/sit} {} {}\\
	\sn
	\glll f=e-a-m-nzr]\\
	\Dist=\Stnsg.\Alph-\Vc-sit.\Ext-\Ndu]\\
	\footnotesize{\Stpl:\Sbj:\Nonpast:\Ipfv/sit}\\
	\trans `We used to live in groups. One clan lives over there, one clan lives over there and one clan lives over there.'\Corpus{tci20120922-08}{DAK \#114-117}
	\label{ex266}
\end{exe}
\begin{exe}
	\ex \emph{ane bä \textbf{bkwaruthrmth} büdisnen mnz znen}.\\
	\glll ane bä b=kw-a-ru-thr-m-th büdisn=en mnz\\
	\Dem{} \Med{} \Med=\M.\Betaone-\Vc-bark.\Ext-\Ndu-\Dur-\Stnsg{} büdisn=\Loc{} house\\
	{} {} \footnotesize{\Med=\Stpl:\Sbj:\Pst:\Dur/bark} {} {} {}\\
	\sn
	\gll zn=en\\
	place=\Loc{}\\
	\trans `Those (dogs) were barking there in Büdisn at the house.'\Corpus{tci20111119-03}{ABB \#95}
	\label{ex267}
\end{exe}

Clitic demonstratives are found most frequently attached to the \isi{copula} which then follows the main verb of a clause. In the discussion of demonstratives, I have labelled this construction \isi{demonstrative} \isi{identifier} (see \S{}\ref{demonstrative-identifiers}). In (\ref{ex268}), the speaker points to another person cutting off the branches of a tree. Note that the \isi{deictic} value (\Med) is held constant on the \isi{demonstrative} \isi{pronoun} \emph{bäne}, the \isi{clitic} \isi{demonstrative} on \emph{rtmaksi} `cut' and the \isi{demonstrative} \isi{identifier} \emph{byé}.

\begin{exe}
	\ex \emph{nima \textbf{bäne} \textbf{birtmakwr} \textbf{byé}.}\\
	\glll nima bäne b=y-rtmak-wr-\Zero{}\\
	like.this \Dem:\Med{} \Med=\Tsg.\Masc.\Alph-cut.\Ext-\Ndu-\Stsg{}\\
	{} {} \footnotesize{\Med=\Stsg:\Sbj>\Tsg.\Masc:\Obj:\Nonpast.\Ipfv/cut}\\
	\sn
	\glll b=\stem{yé}\\
	\Med=\Tsg.\Masc.\Cop.\Ndu{}\\
	\footnotesize{\Med=\Tsg.\Masc:\Sbj:\Nonpast:\Ipfv/be}\\
	\trans `She cuts off that one there.'\Corpus{tci20130907-02}{JAA \#441}
	\label{ex268}
\end{exe}

I choose the label \isi{demonstrative} \isi{identifier} for the whole construction (\isi{clitic} \isi{demonstrative} plus \isi{copula}), because the copula is inert to \isi{tense} marking, i.e. it always occurs in non-past. In example (\ref{ex270}), the speaker took me to a place on the riverbank which used to be a `story place' a long time ago. Story places are always inhabited by spiritual beings and, therefore, they must not be disturbed by people. The verbs \emph{rafisi} `paddle' and \emph{yak} `walk, go' are in \isi{past} \isi{tense} and only the copula is in non-past.

\begin{exe}
	\ex \emph{gardame fthé kwarafinzrmth, boba wozinzo thfiyakm \textbf{berä}.}\\
	\glll garda=me fthé kw-a-rafi-nzr-m-th boba wozi=nzo thf-yak-m b=e-rä\\
	canoe=\Ins{} when \M.\Betaone-\Vc-paddle.\Ext-\Ndu-\Dur-\Stnsg{} \Med.\Abl{} side=\Only{} \Stnsg.\Betatwo-walk.\Ext-\Dur{} \Med=\Stnsg.\Alph-\Cop.\Ndu\\
	{} {} \footnotesize{\Stpl:\Sbj:\Pst:\Dur/paddle} {} {} \footnotesize{\Stpl:\Sbj:\Pst:\Dur/walk} \footnotesize{\Med=\Stpl:\Sbj:\Nonpast:\Ipfv/be}\\
	\trans `When paddling with the canoe, they only went there on the side there.'\\\Corpus{tci20120922-19}{DAK \#8}
	\label{ex270}
\end{exe}

Naturally, \isi{deictic} markers are found mostly in situations where visual identification is important. Example (\ref{ex272}) is taken from a plant walk where the speaker points out two different kinds of trees: \emph{mni bäwzö} and \emph{fothr} (sometimes called \emph{fothr bäwzö}).\footnote{The words \emph{bäwzö} and \emph{fothr} are proper nouns. However, \emph{mni} means `fire' and the name \emph{mni bäwzö} `fire bäwzö' is used because the bark of this tree is hardened over the fire and later used for house walls.} In the recording, \emph{fothr bäwzö} trees stood between the speaker and some \emph{mni bäwzö} trees. Hence, the latter are marked as being further away and all \isi{deictic} markers are \isi{medial}: the \isi{deictic} (\emph{bä} `there'), the \isi{proclitic} on the verb (\emph{bikogro} `it stands there') and the \isi{deictic} in \isi{ablative} \isi{case} (\emph{bobafa} `from there'). Note that the verb is also inflected with an \isi{andative} because more trees of the \emph{mni bäwzö} kind were growing in that direction. As for the other tree, \emph{fothr bäwzö}, it is marked by a \isi{proximal} \isi{deictic} (\emph{zä} `here'), a \isi{proximal} \isi{demonstrative} \isi{identifier} (\emph{zyé} `it is here') and another \isi{proximal} \isi{deictic} in \isi{ablative} \isi{case} (\emph{zbafa} `from here').\footnote{Both deictics \emph{bobafa} and \emph{zbafa} are doubly ablative, i.e. \emph{boba} is already ablative and contrasts with allative \emph{bobo}. This is the only example in the corpus of doubly marked deictics.}

\begin{exe}
	\ex \emph{\textbf{bä} ane mni bäwzö \textbf{bikogro}. \textbf{zä} yé \textbf{zyé} fothr \textbf{zbafa}. \textbf{bobafa} mni bäwzö.}\\
	\glll bä ane mni bäwzö b=y-kogr-o zä\\
	\Med{} \Dem{} fire bäwzö \Med=\Tsg.\Masc.\Alph-stand.\Ndu-\Andat{} \Prox{}\\
	{} {} {} {} \footnotesize{\Med=\Tsg.\Masc:\Sbj:\Nonpast:\Ipfv/stand} {}\\
	\sn
	\glll \stem{yé} z=\stem{yé} fothr zba=fa\\
	\Tsg.\Masc.\Cop.\Ndu{} \Prox=\Tsg.\Masc.\Cop.\Ndu{} fothr \Prox.\Abl=\Abl{}\\
	\footnotesize{\Tsg.\Masc:\Sbj:\Nonpast:\Ipfv/be} \footnotesize{\Prox=\Tsg.\Masc:\Sbj:\Nonpast:\Ipfv/be} {} {} {} {} {} {}\\
	\sn
	\gll boba=fa mni bäwzö\\
	\Med.\Abl=\Abl{} fire bäwzö\\
	\trans `There, \emph{mni bäwzö} is standing there. From here it is \emph{fothr bäwzö} and from there (it is) \emph{mni bäwzö}.'\Corpus{tci20130907-02}{RNA \#166-168}
	\label{ex272}
\end{exe}

The three proclitics \emph{z=}, \emph{b=} and \emph{f=} can in principle attach to verb forms of all TAM categories. For example in (\ref{ex267}), the \isi{medial} \emph{b=} is cliticised to a verb in \isi{past} durative. Nevertheless, they occur most frequently with verbs in present \isi{tense} because of their situational use.\\

The \isi{clitic} \emph{m=} only occurs with the \isi{copula} and the meaning `where is X?' as in (\ref{ex271}). As I will discuss in \S{}\ref{TAMparticlessection}, \emph{m=} can attach to verbs in \isi{irrealis} or \isi{imperative} \isi{mood} with an \isi{apprehensive} (`you might do X!') and \isi{prohibitve} interpretation (`you must not do X!') respectively. Formally, the \emph{m=} \isi{clitic} patterns with the other demonstratives (See Table \ref{demonstratives-table} in \S{}\ref{demonstratives}).

\begin{exe}
	\ex \emph{\textbf{mern}? ni wmägne zöbthé.}\\
	\glll m=e-rn ni w-mäg-n-e zöbthé\\
	where=\Stnsg.\Alph-\Cop.\Du{} \Fnsg{} \Tsg.\F.\Alph-lead.\Ext-\Du-\Fnsg{} first\\
	\footnotesize{where=\Stdu:\Sbj:\Nonpast:\Ipfv/be} {} \footnotesize{\Fdu:\Sbj>\Tsg.\F:\Obj:\Nonpast:\Ipfv/lead} {}\\
	\trans `Where are they? We will lead (the path) first.'\Corpus{tci20130907-02}{JAA \#12}
	\label{ex271}
\end{exe}