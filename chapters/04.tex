%!TEX root = ../main.tex
\chapter{Nominal morphology}
\label{cha:nominals-chapter}

\section{Introduction} \label{nommorphintro}

This chapter describes the \isi{nominal} morphology of Komnzo. With the exception of the close \isi{possessive} construction, all \isi{nominal} morphology is encliticised or suffixed to the element over which it has scope, which is almost always the \isi{noun phrase}. There is little to no allomorphy in the \isi{enclitic} and affix formatives. There are no declension classes. There are special marking patterns for \isi{animate} referents, which include a \isi{number} distinction.%\\

I begin by a description of \isi{reduplication}, which is only found with \isi{nominal}s (\S{}\ref{nomreduplication}). The remainder and bulk of this chapter describes \isi{case} and further morphological markers. I introduce the reader to the 17 cases and their respective functions in \S{}\ref{formfunccase}. After this, each case is discussed in turn (\S{}\ref{ergcase} - \ref{similcase}). In \S{}\ref{furthernommorph}, I describe three enclitics and one suffix which are not related to \isi{case}. Finally, in \S{}\ref{casenotes}, I offer a few concluding remarks on the formal and functional overlap between particular \isi{case} markers.

\section{Reduplication}\label{nomreduplication}

There are two \isi{reduplication} patterns in Komnzo. They differ only formally, not in their meaning, and words for which \isi{reduplication} is a productive morphological process can form both patterns. I use the terms partial \isi{reduplication} and full \isi{reduplication}. In the former, the reduplicant is only the first consonant of the word. In the latter, the whole word is reduplicated.%\\

Semantically, \isi{reduplication} expresses non-prototypicality, plurality, or both. In (\ref{ex707}), \emph{ttrikasi} `stories' is formed from \emph{trikasi} `story', and \isi{reduplication} expresses plurality. In (\ref{ex708}), the \isi{reduplication} of \emph{yawi} `seed' refers to `coins', i.e. it expresses non-prototypicality in addition to plurality. 

\begin{exe}
	\ex \emph{komnzo nima fä zämnzerake nä \textbf{ttrikasi} keke.}\\
	\gll komnzo nima fä zä\stem{mnzer}ake nä t-trika-si keke\\
	only {like.this} \Dist{} \Fnsg:\Sbj:\Pst:\Pfv/fall.asleep \Indf{} \Redup-tell-\Nmlz{} \Neg{}\\
	\trans `We just fell asleep there, no more stories.'\Corpus{tci20120922-25}{ALK \#45}
	\label{ex707}
\end{exe}
\begin{exe}
	\ex \emph{ŋareane \textbf{yawiyawime} kwa ŋonathr ane kambef.}\\
	\gll ŋare=ane yawi-yawi=me kwa ŋo\stem{na}thr ane kambe=f\\
	woman=\Poss.\Sg{} \Redup-seed=\Ins{} \Fut{} \Stsg:\Sbj:\Nonpast:\Ipfv/drink \Dem{} man=\Erg.\Sg{}\\
	\trans `That guy is going to drink with his wife's money.'\Corpus{tci20111004}{TSA \#182}
	\label{ex708}
\end{exe}

The \isi{nominal} subclasses which can be reduplicated are nouns, adjectives, property nouns and quantifiers. Example (\ref{ex709}) shows the \isi{quantifier} \emph{tüfr} expressing that many different jobs are involved in raising a pig. In (\ref{ex710}), the \isi{adjective} \emph{tnz} `short' is reduplicated, meaning that the man was just a bit short. In (\ref{ex721}), the \isi{adjective} \emph{kafar} `big' is reduplicated, meaning that the elders of the Mayawas of Firra had been killed in the headhunting raid.

\begin{exe}
	\ex \emph{zena keke miyo worä ruga mgthksi ... znsä \textbf{ttüfr}.}\\
	\gll zena keke miyo wo\stem{rä} ruga mgthk-si (.) znsä t-tüfr\\
	today \Neg{} desire \Fsg.\Sbj:\Nonpast:\Ipfv/be pig feed-\Nmlz{} (.) work \Redup-plenty\\
	\trans `Today, I do not want to feed pigs ... (too) much work.'\\\Corpus{tci20120805-01}{ABB \#819-820}
	\label{ex709}
\end{exe}
\begin{exe}
	\ex \emph{nafafis yf nagawa ... \textbf{tnztnz} kabe sfrärm.}\\
	\gll nafa-fis yf nagawa (.) tnz-tnz kabe sf\stem{rä}rm\\
	\Third.\Poss-husband name nagawa (.) \Redup-short man \Tsg.\Masc:\Sbj:\Pst:\Dur/be\\
	\trans `Her husband's name (was) \emph{Nagawa} ... he was a bit short guy.'\\\Corpus{tci20120901-01}{MAK \#17-18}
	\label{ex710}
\end{exe}
\begin{exe}
	\ex	\emph{nafanme mayawa \textbf{kkafar} z bramöwä thäkwrath firran.}\\
	\gll nafanme mayawa k-kafar z bramöwä thä\stem{kwr}ath firra=n\\
	\Tnsg.\Poss{} mayawa \Redup-big \Iam{} all \Stpl:\Sbj>\Stpl:\Obj:\Pst:\Pfv/kill firra=\Loc\\
	\trans `All their Mayawa elders had been killed in Firra.'\Corpus{tci20111107-01}{MAK 127}
	\label{ex721}
\end{exe}
	
In addition to productive \isi{reduplication} with the above meanings, reduplications are found across the lexicon to form new meanings. There is a high number of reduplications in plant names and in the names for animals, especially bird and fish species. Often the pattern of \isi{reduplication} establishes a semantic link between biota of different species, families or even kingdoms. I describe this phenomenon under label ``sign \isi{metonymy}'' in \S\ref{redupmeton}.%\\

Lastly, I want to mention that there are some reduplicative orphans which lack a corresponding simplex, for example \emph{gwargwar} `mud' or \emph{ŋarŋar} `bamboo paddle'.

\section{The form and function of case markers}\label{formfunccase}
\largerpage
I follow Blake (\citeyear{Blake:case}) in making a distinction between core cases and peripheral cases. Core cases in Blake's typology ``encode complements of typical one-place and two-place \isi{transitive} verbs'' (\citeyear[32]{Blake:case}), i.e. they are required by the verb's argument structure. I define core cases in Komnzo as those cases whose referent can be indexed in the verb. Thus, core cases are the \isi{absolutive}, \isi{ergative} and \isi{dative} \isi{case}. Note that the \isi{absolutive} is zero-marked. The \isi{possessive} is also counted as core \isi{case}, because the \isi{possessor} can be raised and indexed in the verb. Peripheral cases are those cases whose referents are not required by the structure of the verb, nor can they be indexed in the verb. I will use the term semantic cases for these.%%\\

Following (\citealt{Andrews:2007nounphrase}), I understand semantic roles to refer to `thematic relations' or `deep cases' (\citealt{Fillmore:1968case}). From these, one can derive grammatical functions such as A, S, and P (\citealt{Dixon:1972dyirbal}).\footnote{``I will use the term ‘semantic role’ to refer to both the specific roles imposed on \textsc{np}s by a given predicate (..) and to the more general classes of roles, such as ‘agent’ and ‘patient’. Semantic roles are important in the study of grammatical functions [A, S and P] since grammatical functions usually express semantic roles in a highly systematic way'' (\citealt[136]{Andrews:2007nounphrase}).} In the following, the terms core \isi{case} and semantic \isi{case} are used to refer to the cases, while the term semantic role is used to refer to the underlying semantics.%\\

\begin{table}[t]
\caption{The Komnzo case system} 
\label{caseoverview}
\centering
	\small{%
	\begin{tabularx}{\textwidth}{|p{2mm}|p{2mm}|p{2cm}|l|>{\raggedright}p{3cm}|>{\raggedright\arraybackslash}p{2,6cm}|}
		%\lsptoprule
		\cline{3-6}
		\multicolumn{2}{c|}{}&\multicolumn{1}{c|}{\multirow{2}{*}{{case label}}}&\multicolumn{3}{c|}{{semantic roles by function}}\\
		\cline{4-6}
		\multicolumn{2}{c|}{}&&{adnominal}&{clausal}&{cross-clausal}\\
		\cline{1-6}
		\multicolumn{2}{|c|}{}&\multirow{2}{*}{{absolutive}}&&Agent, Experiencer, Theme, Patient&Agent, Experiencer, Theme, Patient\\\cline{3-6}
		\multicolumn{2}{|c|}{{core}}&{ergative}&&Agent&Agent\\\cline{3-6}
		\multicolumn{2}{|c|}{{cases}}&{dative}&&Recipient, Beneficiary	&\\\cline{3-6}
		\multicolumn{2}{|c|}{}&{possessive}&Possessor&&\\\hhline{|======|}
		\parbox[t]{4mm}{\multirow{20}{*}{
		  \rotatebox[origin=l]{90}{
		    {semantic cases}}
		    }
						  }&\multirow{3}{*}{
						      \rotatebox[origin=l]{90}{
							{spatial}
							}
										}&{locative}&&Location	&Simultaneity\\\cline{3-6}
		&&{allative}&&Goal of motion&\\\cline{3-6}
		&&{ablative}&&Source of motion&\\\hhline{|~=====|}
		&\multirow{3}{*}{\rotatebox[origin=l]{90}{{temporal}\hspace{0,5cm}}}&{\isi{temporal} locative}&&Location in time&\\\cline{3-6}
		&&{\isi{temporal} purposive}&&Goal in time&\\\cline{3-6}
		&&{\isi{temporal} possessive}&Origin&Origin&\\\hhline{|~=====|}
		&\multirow{9}{*}{\rotatebox[origin=l]{90}{{other}}}&{instrumental}&&Instrument, Manner&Result, Manner\\\cline{3-6}
		&&{purposive}&&Purpose&Purpose\\\cline{3-6}
		&&{characteristic}&Origin& \mbox{Source, Reason, Purpose}&Reason, Purpose\\\cline{3-6}
		&&{proprietive}&&Association&Association, Manner\\\cline{3-6}
		&&{privative}&&Absence&\\\cline{3-6}
		&&{associative}&&Association, Inclusion&Association\\\cline{3-6}
		&&{similative}&&Comparison&\\		
		\cline{1-6}
		%\lspbottomrule
	\end{tabularx}}%\quad
\end{table}%Komnzo case system

Following Evans and Dench (\citeyear{Evans:1988ge}), who discuss the ways in which case can be used to establish three levels in Australian languages, I recognise three distinct levels at which cases operate in Komnzo. First, there is the adnominal level which relates one \isi{noun phrase} within a matrix \isi{noun phrase}. Secondly, there is the clausal level which operates directly below the \isi{clause} level. Thirdly, there is the cross-clausal level which indicates that one \isi{clause} is the argument of another \isi{clause}. \tabref{caseoverview} provides an overview of the cases and their functions. Note that semantic cases can be subdivided into spatial, \isi{temporal} and other.

As mentioned above, there is little allomorphy with the \isi{case} markers. Examples are the \isi{locative}, \isi{allative} and \isi{ablative} \isi{case} which have different formatives for \isi{animate} and \isi{inanimate} referents. It can be said that Komnzo \isi{nominal} morphology is relatively simple, especially when compared to its verb morphology (see chapter \ref{cha:verb morphology}). The formatives are given in \tabref{caseformatives} below. 

\begin{table} 
\caption{Case markers in Komnzo} 
\label{caseformatives}
	\begin{tabularx}{\textwidth}{Xlll}
		\lsptoprule
		 {{case}}& {{inanimate}}&\multicolumn{2}{c}{{animate}}\\
		&&\multicolumn{1}{l}{{singular}}&\multicolumn{1}{l}{{non-singular}}\\ \midrule
		\Abs{}&\Zero{}&\Zero{}&\emph{=é} (\emph{=yé})\textsuperscript{a}\\
		\Erg{}&\emph{=f}&\emph{=f}&\emph{=é} (\emph{=yé})\\									
		\Dat{}&n/a&\emph{=n}&\emph{=nm}\\
		\Poss{}&\emph{=ane}&\emph{=ane}&\emph{=aneme}\\
		\Loc{}&\emph{=en} (\emph{=n})&\emph{=dben}& \emph{=medben}\\
		\All{}&\emph{=fo}&\emph{=dbo}& \emph{=medbo}\\
		\Abl{}&\emph{=fa}&\emph{=dba}& \emph{=medba}\\									
		\Temp.\Loc&\emph{=thamen}&n/a&n/a\\
		\Temp.\Poss&\emph{=thamane}&n/a&n/a\\
		\Temp.\Purp&\emph{=thamar}&n/a&n/a\\
		\Purp{}&\emph{=r}&n/a&n/a\\
		\Ins&\emph{=me}&n/a&n/a\\
		\Char{}&\emph{=ma}&\emph{=ane=ma}&\emph{=aneme=ma}\\
		\Prop{}&\emph{=karä} / \emph{=kaf}&\emph{=karä} / \emph{=kaf}&\emph{=karä} / \emph{=kaf}\\
		\Priv{}&\emph{=märe}&\emph{=märe}&\emph{=märe}\\
		\Assoc{}\textsuperscript{b}&\emph{=ä}&\emph{=r}&\emph{=ä}\\
		\Simil&\emph{=thatha}&n/a&n/a\\
		\lspbottomrule
		\multicolumn{4}{l}{\footnotesize{\textsuperscript{a} Overt marking of the ablative (\Nsg) is very rare.}}\\
		\multicolumn{4}{l}{\footnotesize{\textsuperscript{b} The associative forms encode \Du~ versus \Pl~ (\S~\ref{inclusorycontruction}).}}\\
	\end{tabularx}
\end{table}%Komnzo case markers

We find that \isi{case} markers make a distinction between \isi{animate} and \isi{inanimate} referents. For certain cases, there are designated formatives for \isi{animate} referents, for example all the spatial cases. Only with \isi{animate} referents is there a \isi{number} distinction (\Sg{} vs. \Nsg{}) in the \isi{case} markers. Consider examples (\ref{ex722}-\ref{ex724}) below. The first example shows the \isi{locative} \isi{case} on \emph{mnz} `house', and the context of the story reveals that this is about several houses. The \isi{case} marker, however, does not encode \isi{number}. Examples (\ref{ex723}) and (\ref{ex724}) show that this is different for \isi{animate} referents, and the \isi{case} markers make a \isi{singular} versus \isi{non-singular} distinction.

\begin{exe}
	\ex \emph{kwot namäme thfanakwrm ... \textbf{mnzen} thwarakthkwramo.}\\
	\gll kwot namä=me thfa\stem{nak}wrm (.) mnz=en thwa\stem{rakthk}wramo\\
	properly good=\Ins{} \Stsg:\Sbj>\Stpl:\Obj:\Pst:\Dur/put.down (.) house=\Loc{} \Stsg:\Sbj>\Stpl:\Io:\Pst:\Dur:\Andat/put.on.top\\ 
	\trans `She was sorting (the things) properly ... She put the things back in the houses.'\Corpus{tci20120901-01}{MAK \#38-39}
	\label{ex722}
\end{exe}
\begin{exe}
	\ex \emph{\textbf{mizidben} sokoro zewära.}\\
	\gll mizi=dben sokoro ze\stem{wär}a\\
	pastor=\Loc.\Anim.\Sg{} school \Sg:\Sbj:\Pst:\Pfv/happen\\
	\trans `The school was at the pastor('s place).'\Corpus{tci20120904-02}{MAB \#16}
	\label{ex723}
\end{exe}	
\begin{exe}
	\ex \emph{\textbf{nafangthmedben} byamnzr.}\\
	\gll nafa-ngth=medben b=ya\stem{m}nzr\\
	\Third.\Poss-younger.sibling=\Loc.\Anim.\Nsg{} \Med=\Tsg.\Masc:\Sbj:\Nonpast:\Ipfv/dwell\\
	\trans `He stays at his small brothers' place.'\Corpus{tci20120814}{ABB \#216}
	\label{ex724}
\end{exe}
		 	
As \tabref{caseformatives} above shows, most \isi{case} markers employ an /m/ or /me/ element to mark \isi{non-singular} \isi{number}. I refrain from segmenting this element as a separate morph for two reasons. First, the /m/ or /me/ does not occur in all cases, for example not the \isi{ergative} \isi{case}. Secondly, its position is not fixed. With the \isi{possessive}, /me/ follows the \isi{possessive} marker \emph{=ane}. With the \isi{dative}, only /m/ follows the \isi{dative} marker \emph{=n}. With spatial cases /me/ precedes the \isi{locative}, \isi{allative} and \isi{ablative} marker. I will offer an explanation of this in the final section of this chapter (\S{}\ref{casenotes}).%\\

These formatives attach to the rightmost element of the phrase, but have scope over the whole \isi{noun phrase}. In example (\ref{ex485}), the adjective \emph{katan} `small' precedes noun \emph{nzram} `flower' and the \isi{case} marker attaches to the latter. Example (\ref{ex484}) shows the same adjective postposed to the noun \emph{yfö} `hole'. Again, the \isi{case} marker attaches to the rightmost element.

\begin{exe}
	\ex \emph{\textbf{katan nzramma} emarwr.}\\
	\gll katan nzram=ma e\stem{mar}wr\\
	small flower=\Char{} \Stsg:\Sbj>\Stpl:\Obj:\Nonpast:\Ipfv/see\\
	\trans `You (can) identify them from the small flowers.'\Corpus{tci20130907-02}{JAA \#211}
	\label{ex485}
\end{exe}
\begin{exe}
	\ex \emph{watik \textbf{yfö katanr} kwa yarenzr.}\\
	\gll watik yfö katan=r kwa ya\stem{re}nzr\\
	then hole small=\Purp{} \Fut{} \Tsg.\Masc:\Sbj:\Nonpast:\Ipfv/look\\
	\trans `Then, he will look around for a small hole.'\Corpus{tci20130903-04}{RNA \#26}
	\label{ex484}
\end{exe}

\section{Absolutive}\label{abscase}

The \isi{absolutive} \isi{case} is almost always unmarked. The \isi{non-singular} \isi{clitic} (\emph{=é}), (\emph{=yé}) when it follows a vowel, is rarely used. On the clausal level the \isi{absolutive} encodes the single argument of \isi{intransitive} verbs (\ref{ex486}), or the \isi{patient} argument of \isi{transitive} verbs (\ref{ex487}).

\begin{exe}
	\ex \emph{\textbf{nzä} zä zf wamnzr.}\\
	\gll nzä zä zf wa\stem{m}nzr\\
	\Fsg.\Abs{} \Prox{} \Imm{} \Fsg.\Sbj:\Nonpast:\Ipfv/dwell\\
	\trans `I live right here.'\Corpus{tci20130823-08}{WAM \#85}
	\label{ex486}
\end{exe}
\begin{exe}
	\ex \emph{\textbf{nzä} fthé fof afaf schoolen zwäthba.}\\
	\gll nzä fthé fof afa=f school=en zwä\stem{thb}a\\
	\Fsg.\Abs{} when \Emph{} father=\Erg{} school=\Loc{} \Stsg:\Sbj>\Fsg:\Obj:\Pst:\Pfv/put.inside\\
	\trans `That was when father put me in school.'\Corpus{tci20120924-01}{TRK \#5}
	\label{ex487}
\end{exe}
	
When a nominalised verb functions as the \isi{patient} of a matrix clause, it appears with no overt \isi{case} marking. It could be analysed as being marked with \isi{absolutive} \isi{case}, though for reasons of parsimony I will not \isi{gloss} it as such. This commonly occurs with phasal verbs, like in (\ref{ex488}), where the speaker shows me how to make a whistle from a coconut leaf.

\begin{exe}
	\ex \emph{\textbf{myuknsi} srethkäfe. zane zf ymyuknwé.}\\
	\gll myukn-si sre\stem{thkäf}e zane zf y\stem{myukn}wé\\
	roll-\Nmlz{} \Fpl:\Sbj>\Tsg.\Masc:\Obj:\Irr:\Pfv/start \Dem:\Prox{} \Imm{} \Fsg:\Sbj>\Tsg.\Masc:\Obj:\Nonpast:\Ipfv/roll\\
	\trans `We would start twisting it. I am twisting this here.'\Corpus{tci20120914}{RNA \#45}
	\label{ex488}
\end{exe}

Overt marking of \isi{non-singular} \isi{number} is possible if the referent is \isi{animate}. The formative is \emph{=é} if the host is consonant final, and \emph{=yé} if it is vowel final. Hence, there is a syncretism between \isi{absolutive} and \isi{ergative} \isi{non-singular}. This pattern of syncretism is also found in the first \isi{person} pronouns (\S{}\ref{personalpron}), where \emph{ni} is used for both \isi{ergative} and \isi{absolutive} \isi{non-singular}. As a \isi{case} marker on \isi{absolutive} noun phrases it is very rare. One example is given below in (\ref{ex609}).

\begin{exe}
	\ex \emph{\textbf{nzone amayé} bä thfamrnm ksi karen.}\\
	\gll nzone ama=é bä thfa\stem{m}rnm ksi kar=en\\
	\Fsg.\Poss{} mother=\Abs.\Nsg{} \Med{} \Stdu:\Sbj:\Pst:\Dur/dwell bush place=\Loc\\
	\trans `My two mothers lived there in the bush.'\Corpus{tci20150919-05}{LNA \#240}
	\label{ex609}
\end{exe}
	
Only in the syntactic context of the \isi{inclusory} construction is the \isi{absolutive} \isi{non-singular} obligatory (\S{}\ref{inclusorycontruction}). Elsewhere it is optional, and tokens in the corpus are infrequent.
	
\section{Ergative \emph{=f}, \emph{=è}}\label{ergcase}

The \isi{ergative} \isi{case} marker is \emph{=f} (\Sg) or \emph{=é} (\Nsg). The \isi{ergative} usually operates at the clausal level. It encodes the semantic role of actor or \isi{stimulus}. Example (\ref{ex463}) is taken from a ``\emph{Nzürna} story''. These stories are widespread in the Morehead region. The main character \emph{nzürna}, but also the plot of the stories, bears some resemblance to the classic European witch stories.

\begin{exe}
	\ex \emph{okay, ausi zakora ``ŋame, \textbf{nzürna ŋaryf} wanmrinzr!" ... \textbf{ausif} sakora ``anema fof gukonzé nima kmam foba gniyaké!''}\\
	\gll okay ausi za\stem{kor}a ŋame nzürna ŋare=f wan\stem{mri}nzr (.) ausi=f sa\stem{kor}a ane=ma fof gu\stem{ko}nzé nima kma=m foba gni\stem{yak}é\\
	okay {old.woman} \Stsg:\Sbj>\Tsg.\F:\Obj:\Pst:\Pfv/speak mother nzürna woman=\Erg{} \Stsg:\Sbj>\Fsg:\Obj:\Nonpast:\Ipfv:\Venit/chase (.) {old.woman=\Erg} \Stsg:\Sbj>\Tsg.\Masc:\Obj:\Pst:\Pfv/speak \Dem=\Char{} \Emph{} \Fsg:\Sbj>\Ssg:\Obj:\Rpst:\Ipfv/speak \Quot{} \Pot=\Appr{} \Dist:\Abl{} \Ssg:\Sbj:\Imp:\Ipfv/walk\\
	\trans `Okay, he said to the old woman: ``Mother, the \emph{Nzürna} woman is chasing after me!'' The old woman told him: ``That is why I told you: Don't go there!'''\Corpus{tci20120827-03}{KUT \#114-115}
	\label{ex463}
\end{exe} 

Examples (\ref{ex2}) and (\ref{ex3}) show the \isi{ergative} \isi{non-singular} formative. This is \emph{=é} when the word is consonant final and \emph{=yé} when it is vowel final. Example (\ref{ex2}) is taken from a procedural text about a little whistle made from a coconut leaf. In example (\ref{ex3}), the speaker complains about some families whose children seem to be shifting from Komnzo to \ili{Wära}. 

\begin{exe}
	\ex \emph{rusa räkumgsir zane äfiyokwrth ... \textbf{sraké}.}\\
	\gll rusa räkumg-si=r zane ä\stem{fiyok}wrth (.) srak=é\\
	deer attract-\Nmlz{}=\Purp{} \Dem:\Prox{} \Stpl:\Sbj>\Stpl:\Obj:\Nonpast:\Ipfv/make (.) boy=\Erg.\Nsg{}\\
	\trans `They make this one for attracting deer ... the boys (make it).'\\\Corpus{tci20120914}{RNA \#61}
	\label{ex2}
\end{exe}
\begin{exe}
	\ex \emph{... a \textbf{ŋameyé} nafanme zokwasimöwä thwasäminzrmth}\\
	\gll (.) a ŋame=é nafanme zokwasi=me=wä thwa\stem{sämi}nzrmth\\
	(.) and mother=\Erg.\Nsg{} \Tnsg{}.\Poss{} speech=\Ins{}=\Emph{} \Stpl:\Sbj>\Stpl:\Obj:\Pst:\Dur/teach\\
	\trans `... and the mothers were teaching their own language (to the children).'\\\Corpus{tci20120924-02}{ABM \#37-38} 
	\label{ex3}
\end{exe}

The \isi{ergative} \isi{case} can be used to encode \isi{inanimate} actors who for some reason are attributed an actor-like behaviour. Example (\ref{ex464}) comes from a hunting story where the speaker reaches the camp of his family in the night and sees a gaslamp hanging on the bamboos. Here, the wind (\emph{füsfüs}) is marked with the \isi{ergative}.

\begin{exe}
	\ex \emph{nabi tutin fä fof zumirwanzrm \textbf{füsfüsf}.}\\
	\gll nabi tuti=n fä fof zu\stem{mirwa}nzrm füsfüs=f\\
	bamboo branch=\Loc{} \Dist{} \Emph{} \Stsg:\Sbj>\Tsg.\F:\Obj:\Pst:\Dur/swing wind=\Erg{}\\
	\trans `The wind was swinging (the lamp) on the bamboo branch.'\Corpus{tci20111119-03}{ABB \#117}
	\label{ex464}
\end{exe}

Example (\ref{ex465}) is taken from an origin myth in which the island of New Guinea and the continent of Australia were still connected. The myth describes the rising see levels and how the people had to take refuge on both sides. The \isi{inanimate} referent \emph{no} `water' is flagged with the \isi{ergative} \isi{case}.

\begin{exe}
	\ex \emph{\textbf{nof} nä nima thärkothmako ... nä nima thänkothma nzezawe.}\\
	\gll no=f nä nima thär\stem{kothm}ako (.) nä nima thän\stem{kothm}a nze-zawe\\
	water=\Erg{} some {like.this} \Sg:\Sbj>\Stpl:\Obj:\Pst:\Pfv:\Andat/chase (.) some {like.this} \Stsg:\Sbj>\Stpl:\Obj:\Pst:\Pfv:\Venit/chase \Fnsg.\Poss-right\\
	\trans `The water chased some away like this ... and it chased some here to our side like this.'\\\Corpus{tci20131013-01}{ABB \#125-126}
	\label{ex465}
\end{exe}

Experiencer-object constructions are quite common in Komnzo, whereby the \isi{stimulus} receives the \isi{ergative} \isi{case} and the \isi{experiencer} the ablative. Constructions of this type have been described for \ili{Kalam} by Pawley et al. (\citeyear{Pawley:2000vp}) and for \ili{Nen} by Evans (\citeyear{Evans:2015wy}). As in \ili{Kalam}, \isi{experiencer-object} constructions are often used to express bodily and mental processes. Example (\ref{ex468}) comes from a story about a man who was angry and tried to shock people at a dance. The fact that he was infuriated is expressed literally as `anger finished him'. Likewise, in example (\ref{ex467}) the speaker announces that she wants to go to bed because `fear has grabbed her'. 

\begin{exe}
	\ex \emph{\textbf{nokuyé} fthé sabtha.}\\
	\gll noku=yé fthé sa\stem{bth}a\\
	anger=\Erg.\Nsg{} when \Stsg:\Sbj>\Tsg.\Masc:\Pst:\Pfv/finish\\
	\trans `That is when he got really angry.' (Lit. `That is when anger finished him.')\\\Corpus{tci20120909-06}{KAB \#39}
	\label{ex468}
\end{exe}
\begin{exe}
	\ex \emph{\textbf{wtrif} z zwefaf.}\\
	\gll wtri=f z zwe\stem{faf}\\
	fear=\Erg{} \Iam{} \Stsg>\Fsg:\Rpst:\Pfv/hold\\
	\trans `I am already scared.' (Lit. `Fear already holds me.')\Corpus{tci20130901-04}{RNA \#164}
	\label{ex467}
\end{exe}

The \isi{ergative} \isi{case} can also be attached to a nominalised verb as in example (\ref{ex469}). This example is about a \ili{Marind} headhunter who was trying to distract the people by imitating the sound that dogs make when they chew on bones. The poor guy was so busy making this noise that he did not hear how the village people were approaching him. Hence, it is the \isi{infinitive} of `crack' which receives the \isi{ergative} \isi{case} in (\ref{ex469}).

\begin{exe}
	\ex \emph{bäne thuwänzrm fof ... zarfa surmänwrm \textbf{ane wäsifnzo}.}\\
	\gll bäne thu\stem{wä}nzrm fof (.) zarfa su\stem{rmän}wrm ane wä-si=f=nzo\\
	\Dem:\Med{} \Stsg:\Sbj>\Stpl:\Obj:\Pst:\Dur/crack fof (.) ear \Stsg:\Sbj>\Tsg.\Masc:\Obj:\Pst:\Dur/close \Dem{} crack-\Nmlz=\Erg=\Only\\
	\trans `He was cracking those (coconut shells). This cracking was blocking his ears.'\Corpus{tci20120818}{ABB \#67-68}
	\label{ex469}
\end{exe}

Thus, the \isi{ergative} \isi{case} can also function at the cross-clausal level. Example (\ref{ex470}) shows that the infinitive to which the \isi{ergative} is attached may also take an \isi{object}. In the example a malevolent spirit, who lives in a tree, is about to be burned by an angry mob. She does not notice the fire at first because she is too concentrated on weaving a mat. The `mat weaving' receives the \isi{ergative}.

\begin{exe}
	\ex \emph{mni wthomonwath a zräföfth ... fi \textbf{yame yrsifnzo} zukonzrm boba wämne yfön fof.}
	\gll mni w\stem{thomon}wath a zrä\stem{föf}th (.) fi yame yr-si=f=nzo zu\stem{ko}nzrm boba wämne yfö=n fof\\
	fire \Stpl:\Sbj>\Tsg.\F:\Obj:\Pst:\Ipfv/pile.firewood and \Stpl:\Sbj>\Tsg.\F:\Obj:\Irr:\Pfv/burn (.) but mat weave-\Nmlz=\Erg=\Only{} \Stsg:\Sbj>\Tsg.\F:\Pst:\Dur/do \Med.\Abl{} tree hole=\Loc{} \Emph{}\\
	\trans `They piled up the firewood and started to burn it ... but she was concentrated on weaving the mat there in the tree hole.' (Lit. `The mat weaving did her.')\Corpus{tci20120901-01}{MAK \#155-156}
	\label{ex470}
\end{exe}

Contructions showing the \isi{ergative} at the cross-clausal level are infrequent in the corpus. Note that in both examples above, the \isi{exclusive} \isi{clitic} \emph{=nzo} is attached to the ergative-marked infinitive in order to highlight that it was ``only that event'' which acted on a person.

\section{Dative \emph{=n}, \emph{=nm}} \label{datcase}

The \isi{dative} \isi{case} marker is \emph{=n} (\Sg) or \emph{=nm} (\Nsg). It operates at the clausal level and encodes the semantic role of (\isi{animate}) \isi{recipient} or \isi{goal}. If it is attached to a \isi{place name}, as in example (\ref{ex460}) below, the people of that place are meant, not the place. The \isi{dative} is categorised as a core \isi{case} because a \isi{dative} marked noun phrase is indexed in the verb, as in the verb form \emph{thägathinza} in (\ref{ex460}). Unlike in other \ili{Tonda} languages, for example in \ili{Ngkolmpu} (\citealt{Carroll:Ngkolmpu}), the \isi{dative} \isi{case} cannot be used adnominally to mark a \isi{possessor}.\\
 
In example (\ref{ex460}), the speaker talks about the different places where he used to own a garden plot. Example (\ref{ex462}) comes from a set of stimulus videos.

\begin{exe}
	\ex \emph{nzone daw bä mane rera safsen ... \textbf{nafanm} thägathinza ... \textbf{safs karnm}.}\\
	\gll nzone daw bä mane \stem{rä}ra safs=en (.) nafanm thä\stem{gathi}nza (.) safs kar=nm\\
	\Fsg.\Poss{} garden \Med{} which \Tsg.\F:\Pst:\Ipfv/be safs=\Loc{} (.) \Tnsg.\Poss{} \Sg:\Sbj>\Stpl:\Io:\Pst:\Ipfv/leave (.) safs village=\Dat.\Nsg{}\\
	\trans `As for my garden there in Safs, I left it for them ... for the Safs people.'\Corpus{tci20120805-01}{ABB \#652-653}
	\label{ex460}
\end{exe}
\begin{exe}
	\ex \emph{emoth a srak markai no ŋarinth ... emothf yarithr \textbf{srakn}.}\\
	\gll emoth a srak markai no ŋa\stem{ri}nth (.) emoth=f ya\stem{ri}thr srak=n\\
	girl and boy {white.man} water \Stdu:\Sbj:\Nonpast:\Ipfv/pour (.) girl=\Erg{} \Stsg:\Sbj>\Tsg.\Masc:\Io:\Nonpast:\Ipfv/give boy=\Dat\\ 
	\trans `The boy and the girl are pouring (each other) wine. The girl gives (it) to the boy.'\Corpus{tci20111028-01}{RNA \#27-28}
	\label{ex462}
\end{exe}
	
The formatives in \tabref{caseformatives} might suggest a syncretism between the \isi{dative} \isi{case} and the \isi{locative} \isi{case}. The \isi{singular} marker of the \isi{dative} is \emph{=n}, and the \isi{locative} marker is also \emph{=n} when it attaches to a vowel final word (for consonant final words, it is \emph{=en}). However, no syncretism takes place because (i) inanimates do not take \isi{dative} \emph{=(e)n}, and (ii) \isi{animate} referents receive a special formative of the \isi{locative} \isi{case} (\emph{=dben}). Moreover, there is some variation for the \isi{dative} when it is attached to a vowel final word. For example, the word \emph{ŋafe} `father' with the \isi{dative} \emph{=n} can be pronounced as [ŋaβen], [ŋaβeʔə̆n] or [ŋaɸjə̆n] (as in \ref{ex461} below).%\\

In terms of meaning, there is some overlap between the \isi{allative} and the \isi{dative} \isi{case}. Example (\ref{ex461}) concludes an origin myth, and the speaker points out how the story was passed through the generations. The noun phrase \emph{ŋafynm} `for/to the fathers' marks a \isi{goal}. This could be equally expressed with an \isi{allative} \isi{case} marker \emph{ŋafemedbo} `to the fathers'.\footnote{Note that the verb \emph{ŋanritakwa} `it (was) passed' does not index the dative noun phrase \emph{ŋafynm} `for/to the fathers'. This occurs in (\ref{ex461}) because the noun phrase is separated by a pause, by a moment of hesitation.}

\begin{exe}
	\ex \emph{trikasi mane nŋatrikwé fof ... \textbf{ŋafynm} ... badafa ane fof ŋanritakwa fof.}\\
	\gll trik-si mane n=ŋa\stem{trik}wé fof (.) ŋafe=nm (.) bada=fa ane fof ŋan\stem{ritak}wa fof\\
	tell-\Nmlz{} which \Immpst=\Fsg:\Sbj:\Nonpast:\Ipfv/tell \Emph{} (.) father=\Dat.\Nsg{} (.) ancestor=\Abl{} \Dem{} \Emph{} \Stsg:\Obj:\Pst:\Ipfv:\Venit/pass \Emph{}\\
	\trans `The story which I have just told ... was really passed to the fathers from the ancestor(s).'\Corpus{tci20131013-01}{ABB \#405}
	\label{ex461}
\end{exe}

\section{Possessive marking}\label{possessivemarking}
	
\subsection{Possessive \emph{=ane}, \emph{=aneme}} \label{posscase}

The \isi{possessive} \isi{case} is \emph{=ane} (\Sg) or \emph{=aneme} (\Nsg). It marks the semantic role of \isi{possessor}, and the noun or \isi{noun phrase} to which it attaches always functions adnominally. Examples (\ref{ex446}) and (\ref{ex447}) show \isi{animate} possessors. Example (\ref{ex446}) is taken from a story about marriage customs and (\ref{ex447}) is from a procedural about traditional fishing baskets. Note that all occurrences of the \isi{possessive} \isi{case} in (\ref{ex447}) are within noun phrases whose \isi{nominal} head is omitted because it can be recovered from the context.

\begin{exe}
	\ex \emph{\textbf{bafane mezü} rera ... \textbf{masenane mezü}.}\\
	\gll bafane mezü \stem{rä}ra (.) masen=ane mezü\\
	\Recog.\Poss.\Sg{} widow \Tsg.\F:\Pst.\Ipfv/be (.) masen=\Poss.\Sg{} widow\\
	\trans `She was this one's widow ... Masen's widow.'\Corpus{tci20120814}{ABB \#18-20}
	\label{ex446}
\end{exe}
\begin{exe}
	\ex \emph{wati, net ane mane erä \textbf{markaianeme} erä ane ... zane zf ... \textbf{kar kambeaneme} zfrärm ... \textbf{nzenme}.}\\
	\gll wati net ane mane e\stem{rä} markai=aneme e\stem{rä} ane (.) zane zf (.) kar kambe=aneme zf\stem{rä}rm (.) nzenme\\
	then net \Dem{} which \Stpl:\Sbj:\Nonpast:\Ipfv/be {white.man=\Poss.\Nsg} \Stpl:\Sbj:\Nonpast:\Ipfv/be \Dem{} (.) \Dem:\Prox{} \Imm{} (.) village man=\Poss.\Nsg{} \Tsg.\F:\Sbj:\Pst:\Dur/be (.) \Fnsg.\Poss{}\\ 
	\trans `As for those nets, they are the white man's (nets). This right here, this was the village people's (fishbasket) ... ours.'\Corpus{tci20120906}{SKK \#53-56}
	\label{ex447}
\end{exe}

Examples (\ref{ex444}) and (\ref{ex445}) show the \isi{possessive} \isi{case} with inanimate possessors. When the host word is vowel final, there are different variants. In careful pronunciation, a glottal stop is inserted, for example [ɸiraʔane] in (\ref{ex444}). In fast speech, this does not occur. Either the vowel is lengthened (if the word ends in /a/) or a glide is inserted, for example [ɸira.ne] in (\ref{ex444}) and [ðarisijane] in (\ref{ex445}). However, sometimes a velar nasal is inserted, and example (\ref{ex444}) could be realised as [ɸiraŋane]. 

\begin{exe}
	\ex \emph{faw wbrigwath ... \textbf{firraane zanma} fof.}\\
	\gll faw w\stem{brig}wath (.) firra=ane zan=ma fof\\
	payback \Stpl:\Sbj>\Tsg.\F:\Obj:\Pst:\Ipfv/return (.) firra=\Poss.\Sg{} killing=\Char{} \Emph\\
	\trans `They brought the payback ... because of the killing of Firra.'\\\Corpus{tci20111119-01}{ABB \#5-6}
	\label{ex444}
\end{exe}
\begin{exe}
	\ex \emph{wati, nima fof kwafiyokwrme ... tharisi taemen ... \textbf{tharisiane efoth} fthé zfrärm.}\\
	\gll wati nima fof kwa\stem{fiyok}wrme (.) thari-si taem=en (.) thari-si=ane efoth fthé zf\stem{rä}rm\\
	then {like.this} \Emph{} \Fpl:\Sbj:\Pst:\Dur/make (.) dig-\Nmlz{} time=\Loc{} (.) dig-\Nmlz=\Poss.\Sg{} day when \Tsg.\F:\Sbj:\Pst:\Dur{}\\
	\trans `Well, this is what we were doing ... in the harvesting time ... when it was the day of harvesting.'\Corpus{tci20120805-01}{ABB \#354-356}
	\label{ex445}
\end{exe}

\subsection{Close possession} \label{closeposs}

There is a second \isi{possessive} construction in Komnzo, which involves a prefix. The formatives are given below in \tabref{posspref}. Formally, these prefixes seem to be reductions of personal pronouns. Surprisingly, they originate not from the \isi{possessive} but the \isi{dative} pronouns (\S{}\ref{personalpron}). This is evident from the vowel quality which signals the \isi{number} distinction. For example, the first \isi{person} \isi{singular} \isi{possessive} \isi{pronoun} is \emph{nzone} `my', and the first \isi{singular} \isi{dative} \isi{pronoun} is \emph{nzun} `for me'. The \isi{possessive} prefixes of the first and second \isi{singular} show the /u/ vowel of the latter, not the /o/ vowel of the \isi{possessive} series. Note that the \isi{number} distinction is lost in third \isi{person}. This is caused by the fact that in the third \isi{person} pronouns (\isi{possessive} as well as \isi{dative}) there is no change in the vowel quality. The close \isi{possessive} construction can also occur with other \isi{nominal}s, which are then treated like prefixes. I will discuss this at the end of this section.

\begin{table}
\caption{Possessive prefixes} 
\label{posspref}
	\begin{tabularx}{.5\textwidth}{XXl}
		\lsptoprule
		{person}	&\Sg&\Nsg\\ \midrule
		\First{}&\emph{nzu-}&\emph{nze-}\\
		\Second{}&\emph{bu-}&\emph{be-}\\
		\Third&\multicolumn{2}{c}{~~~\emph{nafa-}}\\
		\lspbottomrule
	\end{tabularx}	
\end{table}%Possessive prefixes

I label this type of \isi{possessive marking} ``close possession'' rather than ``inalienable possession''. Although close \isi{possessive marking} is used for entities which are characterised as being inalienable, for example kinterms or the origin of a person, close \isi{possessive marking} is not obligatory for these concepts, but merely one of two options. Furthermore, some of the concepts which fall under the rubric of inalienability, for example body-part terms, rarely occur in the close \isi{possessive} construction in Komnzo. Finally, for those lexical items which can be used in both \isi{possessive} constructions, there is a semantic difference between normal versus close possession.%\\

From a historical perspective, \isi{frequency} can help to explain the emergence of the close \isi{possessive} construction (see Bybee (\citeyear[142]{Bybee:2010wb}) for a discussion of \isi{frequency} and language change). Given that some lexical items occur frequently in a \isi{possessive} construction, we can assume that, in the course of time, the preceding \isi{pronoun} reduced in form and turned into a prefix. Frequency is only one explanation and the inherently relational nature of some lexical items, such as kinterms, can also provide a pathway for the emergence of the close \isi{possessive} construction. It is important to point out that the prefix pattern was not extended to all other \isi{nominal}s. On the contrary, the two marking patterns were associated with a semantic distinction between (normal) possession and close possession. Synchronically, this means that there is no clear-cut categorisation as there is with alienable/inalienable systems. Some lexical items are judged ungrammatical by my informants in a close \isi{possessive} construction. For example, I was told that \textsuperscript{$\ast$}\emph{nzumnz} `my house' is ungrammatical, and \emph{nzone mnz} should be used instead. However, I am cautious about these judgements, because I have overheard the very construction in conversation. On the other hand, informants agree that there are many lexical items which can alternate between the two \isi{possessive} constructions depending on how a speaker wants to frame the connection between \isi{possessed} and \isi{possessor}, for example \emph{nzone kar} `my village' (normal possession) or \emph{nzukar} `my village' (close possession). Finally, there is no class of words for which close possession is obligatory.%\\

Example (\ref{ex448}) shows the \isi{possessive} prefix on the word \emph{kar} `village/place'. The example is taken from a myth, where the two protagonists are withholding a particular food source from each other.

\begin{exe}
	\ex \emph{``be nzun fof kwathungr! \textbf{bukaren} ane fof bä safak emgthkwa.''}\\
	\gll be nzun fof kwa\stem{thung}r bu-kar=en ane fof bä safak e\stem{mgth}kwa\\
	\Ssg.\Erg{} \Fsg.\Dat{} \Emph{} \Stsg:\Sbj>\Fsg:\Io:\Rpst:\Ipfv/trick \Ssg.\Poss-village=\Loc{} \Dem{} \Emph{} \Med{} saratoga \Stsg:\Sbj>\Stpl:\Obj:\Pst:\Ipfv/feed\\
	\trans ```You have played a trick on me! In your place there, you have been feeding these saratogas.'''\Corpus{tci20110802}{ABB \#121-122}
	\label{ex448}
\end{exe}

Example (\ref{ex449}) shows a double \isi{possessive} construction `their father's story' involving both types of \isi{possessive marking}. Note that (\ref{ex449}) could also be expressed using a \isi{possessive} \isi{pronoun} as \emph{nafane ŋafeane trikasi}.
	
\begin{exe}
	\ex \emph{\textbf{nafaŋafeane trikasi} ŋariznth.}\\
	\gll nafa-ŋafe=ane trik-si ŋa\stem{ri}znth\\
	\Third.\Poss-father=\Poss.\Sg{} tell-\Nmlz{} \Stdu:\Sbj:\Nonpast:\Ipfv/hear\\
	\trans `They are listening to their father's story.'\Corpus{tci20111004}{RMA \#164}
	\label{ex449}
\end{exe}

Close possession is also possible with personal names as possessors. In this case, the personal name is treated like a prefix, i.e. it is syllabified together with the \isi{possessed}. This can be seen in example (\ref{ex450}). The \isi{possessor} is the personal name \emph{Bäi} [\super{m}bˈæ͡ı], and the \isi{possessed} is \emph{fzenz} [ɸˈə̆tse\super{n}ts] `wife'. A normal \isi{possessive} construction would add the \isi{possessive} \isi{case} to the \isi{possessor}: \emph{Bäiane fzenz} [\super{m}bˈæjane ɸˈə̆tʃe\super{n}ts] `Bäi's wife'. Both words receive \isi{stress} separately, and both are syllabified independently. In the close \isi{possessive} construction, the two words are syllabified as one word: \emph{Bäyfzenz} [\super{m}bˈæjə̆ɸtʃe\super{n}ts]. Note that the initial consonant of \emph{fzenz} is resyllabified as a coda, the epenthentic vowel [ə̆] occurs between the two words, and \emph{fzenz} does not receive separate \isi{stress}. All this is evidence that the \isi{possessor} (the personal name) is treated like the prefixes described above.

\begin{exe}
	\ex \textit{wati, \textbf{bäyfzenzf} zwäkor ``bone dagon fof erä!''}\\
	\gll wati bäi-fzenz=f zwä\stem{kor} bone dagon fof e\stem{rä}\\
	then bäi-wife=\Erg{} \Stsg:\Sbj>\Fsg{}:\Obj:\Rpst:\Pfv/speak \Ssg{}.\Poss{} food \Emph{} \Stpl:\Sbj:\Nonpast:\Ipfv/be:\\ 
	\trans `Then, Bäi's wife said to me ``Your food is here!'' '\Corpus{tci20120922-24}{MAA \#81}
	\label{ex450}
\end{exe}

Note that in this construction there is no morph signalling the \isi{possessive} relation, i.e. there is no \isi{possessive} \isi{case} marker. Only the fact that the \isi{possessor} and \isi{possessed} are syllabified as one word shows the presence of \isi{possessive} semantics. Consequently, there is no ``\isi{possessive}'' in the \isi{gloss}, and only the hyphen between the two words shows that they are in a (close) \isi{possessive} relationship.%\\

For some items in a close \isi{possessive} construction, there is an /a/ element between \isi{possessor} and \isi{possessed}. As in example (\ref{ex489}) below: \emph{kowi-a-fis} `Kowi's husband'. Thus, in these cases there is an overt marker of the close \isi{possessive} construction. The /a/ element seems to be a reduction of the \isi{possessive} \isi{case} marker \emph{=ane}. The example is taken from a conversation about food taboos. The speaker is joking about his sister \textendash{} a young unmarried woman.\footnote{The fact that in example (\ref{ex489}) the possessive case \emph{=ane} is encliticised to \emph{kowiafis} `Kowi's husband' is not relevant for the point here. This always occurs when the characteristic case is attached to an animate referent (\S{}\ref{charcase}).}

\begin{exe}
	\ex \emph{fi \textbf{kowiafisanemanzo} fthé z änathre ... kowiane kabe fthé srarä.}\\
	\gll fi kowi-a-fis=ane=ma=nzo fthé z ä\stem{na}thre (.) kowi=ane kabe fthé sra\stem{rä}\\
	but kowi-\Poss-husband=\Poss.\Sg=\Char=\Only{} when \Iam{} \Fpl:\Sbj>\Stpl:\Obj:\Nonpast:\Ipfv/eat (.) kowi=\Poss.\Sg{} man when \Tsg.\Masc:\Io:\Irr.\Ipfv/be\\
	\trans `Only from Kowi's husband we will eat (food) ... If Kowi had a husband.'\\\Corpus{tci20120922-26}{DAK \#137-138}
	\label{ex489}
\end{exe}

\section{Spatial cases} \label{spatialcase}

There are three spatial cases in Komnzo: the \isi{locative} (\emph{=en}), \isi{allative} (\emph{=fo}) and \isi{ablative} (\emph{=fo}). All three cases have special formatives for \isi{animate} referents with a \isi{number} distinction (\Sg{}, \Nsg{}): \isi{locative} (\emph{=dben}, \emph{=medben}), \isi{allative} (\emph{=dbo}, \emph{=medbo}) and \isi{ablative} (\emph{=dba}, \emph{=medba}). They function at the clausal level and are categorised as semantic cases. Unlike neighboring languages, for example \ili{Nama} and \ili{Nen}, there is no perlative \isi{case} in Komnzo. All three spatial cases have various semantic extensions. For example, they can be used in a \isi{temporal} sense, even though there is a set of dedicated \isi{temporal} \isi{case} markers (\S{}\ref{temporalcase}).%\\

All three \isi{animate} \isi{non-singular} \isi{case} markers show some variation in their pronunciation. For example, \emph{kabe=nmedben} and \emph{kabe=medben} `with/at the people' are both grammatical. The former contains an /n/, whereas the latter does not. I will offer an explanation for this in \S{}\ref{casenotes}.

\subsection{Locative \emph{=en}} \label{locativecase}

The \isi{locative} \isi{case} marker is \emph{=en}, for example \emph{mnz=en} `in the house'. If the host word ends in a vowel, the formative is \emph{=n}, for example \emph{mni=n} `in the fire'. There are designated formatives for \isi{animate} referents, which make a \isi{singular} versus \isi{non-singular} contrast. Example (\ref{ex431}) shows the \isi{locative} \isi{case} in its basic use. Example (\ref{ex430}) comes from a text about a young boy who drowned in the Morehead river after he got stuck underwater in the mud. The example is a detailed description of how the body was recovered from the river.

\begin{exe}
	\ex \emph{nzone fäms byé \textbf{safsen}}\\
	\gll nzone fäms b=\stem{yé} safs=en\\
	\Fsg.\Poss{} {exchange.man} \Med=\Tsg.\Masc:\Sbj:\Nonpast:\Ipfv/be safs=\Loc{}\\
	\trans `My exchange man is there in Safs.'\Corpus{tci20120805-01}{ABB \#269}
	\label{ex431}
\end{exe}
\begin{exe}
	\ex \emph{zä thabr thentharfa ... ŋakarkwa gwargwarfa ... srefzath ... neba thabr nima sfrärm \textbf{nagayedben} ... neba ... nebame kwansogwrm \textbf{nabin}.}\\
	\gll zä thabr then\stem{tharf}a (.) ŋa\stem{kark}wa gwargwar=fa (.) sre\stem{fzath} (.) neba thabr nima sf\stem{rä}rm nagaye=dben (.) neba (.) neba=me kwan\stem{sog}wrm nabi=n\\
	\Prox{} arm \Stsg:\Sbj>\Stpl:\Obj:\Pst:\Pfv:\Venit/put.under (.) \Stsg:\Sbj:\Pst:\Ipfv/pull mud=\Abl{} (.) \Stsg:\Sbj>\Tsg.\Masc:\Obj:\Irr:\Pfv/pull.out (.) opposite arm {like.this} \Tsg.\Masc:\Sbj:\Pst:\Dur{} child=\Loc.\Anim.\Sg{} (.) opposite (.) opposite=\Ins{} \Stsg:\Sbj:\Pst:\Dur/ascend bamboo=\Loc{}\\
	\trans `He put the arm underneath ... he pulled him from the mud ... he pulled him out ... one arm was like this on the boy ... the other ... with the other he climbed up on the bamboo.'\Corpus{tci20120904-02}{MAB \#189-193}
	\label{ex430}
\end{exe}

The \isi{locative} can be translated to \ili{English} with the prepositions `in', `on' or `at'. In order to express that some entity is inside something else, one can use the \isi{locational} \isi{nominal} \emph{mrmr} `inside' (\ref{ex432}). See \S{}\ref{locationals} for locationals. Note that example (\ref{ex432}) shows that the \isi{locative} marker attaches to the last item \emph{mrmr} `inside' of the phrase \emph{firra kar mrmr} `inside the village of Firra'.
	
\begin{exe}
	\ex \emph{\textbf{firra kar mrmren} kabe thwamnzrm fobo.}\\
	\gll firra kar mrmr=en kabe thwa\stem{m}nzrm fobo\\
	firra village inside=\Loc{} man \Stpl:\Sbj:\Pst:\Dur/dwell \Dist:\All{}\\
	\trans `The people lived inside the village of Firra.'\Corpus{tci20120901-01}{MAK \#27}
	\label{ex432}
\end{exe}

The \isi{locative} \isi{case} can be extended to cover various abstract, non-spatial domains. In example (\ref{ex434}) it is used temporally: `on that day' and `in the afternoon'. Example (\ref{ex435}) shows a metaphorical use of the \isi{locative} \isi{case}: \emph{zokwasi=n} `in words'. This sentence was a description of a man who got infuriated at the demand of some of his relatives to give them his daughter as an exchange sister.

\begin{exe}
	\ex \emph{\textbf{ane efothen} ... \textbf{ane zizin} ... Kukufia we sathora fof.}\\
	\gll ane efoth=en (.) ane zizi=n (.) Kukufia we sa\stem{thor}a fof\\
	\Dem{} sun=\Loc{} (.) \Dem{} afternoon=\Loc{} (.) kukufia also \Tsg.\Masc:\Sbj:\Pst:\Pfv/arrive \Emph\\
	\trans `On that day ... in the afternoon, Kukufia arrived again.'\Corpus{tci20100905}{ABB \#105-107}
	\label{ex434}
\end{exe}
\begin{exe}
	\ex \emph{fi \textbf{zokwasin} kwanänzüthzr.}\\
	\gll fi zokwasi=n kwa\stem{nänzüth}zr\\
	\Third.\Abs{} speech=\Loc{} \Stsg:\Sbj:\Rpst:\Ipfv/bury\\
	\trans `He got into a fuss.' (Lit. `He buried himself in words.')\Corpus{overheard}{}
	\label{ex435}
\end{exe}

The above functions of the \isi{locative} were all at the clausal level. At the cross-clausal level, the \isi{locative} can also be used with a \isi{nominalisation} as the counterpart of an adverbial subordinator where it encodes an event that occurs simultaneously with that of the main \isi{clause}. Example (\ref{ex490}) is taken from a story about a malevolent spirit who had killed a man. In the example, she realises that others have discovered the truth.

\begin{exe}
	\ex \emph{wtri we z zära nima ``z zwemarth ... \textbf{ane yam fiyoksin}.''}\\
	\gll wtri we z zä\stem{r}a nima z zwe\stem{mar}th (.) ane yam fiyok-si=n\\
	fear also \Iam{} \Stsg:\Sbj:\Pst:\Pfv/do \Quot{} \Iam{} \Stpl:\Sbj>\Fsg:\Obj:\Rpst:\Pfv/see (.) \Dem{} event make-\Nmlz=\Loc{}\\
	\trans `She was already afraid and said: ``They have already seen me doing that thing.'''\Corpus{tci20120901-01}{MAK \#150-152}
	\label{ex490}
\end{exe}

\subsection{Allative \emph{=fo}} \label{allativecase}

The \isi{allative} \isi{case} marker is \emph{=fo} for \isi{inanimate} referents and \emph{=dbo} (\Sg) or \emph{=medbo} (\Nsg) for \isi{animate} referents. It encodes a spatial \isi{goal}. Example (\ref{ex436}) describes how the speaker and his family received the news that a widow from the neighbouring village should get remarried (to one of his friends). 

\begin{exe}
	\ex \emph{wati \textbf{nzedbo} zanrifthath \textbf{mayawanmedbo} rouku \textbf{bänefo} ... \textbf{masufo}.}\\
	\gll wati nzedbo zan\stem{rifth}ath mayawa=medbo rouku bäne=fo (.) masu=fo\\
	then \Fnsg.\All{} \Stpl:\Sbj>\Tsg.\F:\Obj:\Pst:\Pfv/send mayawa=\All.\Anim.\Nsg{} rouku \Recog=\All{} (.) masu=\All{}\\
	\trans `Then they sent the word to us ... to the Mayawas in Rouku ... to there ... to Masu.'\Corpus{tci20120814}{ABB \#34-35}
	\label{ex436}
\end{exe}
	 
The \isi{allative} can be translated as movement `to' or `towards' some entity (\ref{ex436}), but also as movement `inside' some entity (\ref{ex437}).

\begin{exe}
	\ex \emph{\textbf{zbo} n zräthbé \textbf{yare kwosifo}.}\\
	\gll zbo n zrä\stem{thb}é yare kwosi=fo\\
	\Prox.\All{} \Imn{} \Fsg:\Sbj>\Tsg:\F:\Obj:\Irr:\Pfv/put.inside bag old=\All{}\\
	\trans `I will try and put it here ... in the old bag.'\Corpus{tci20130907-02}{RNA \#261}
	\label{ex437}
\end{exe}

The \isi{allative} can also be used metaphorically as in example (\ref{ex438}), which is taken from a public speech.
	
\begin{exe}
	\ex \emph{\textbf{zokwasifo} buthorakwr.}\\
	\gll zokwasi=fo b=wo\stem{thorak}wr\\
	speech=\All{} \Med=\Fsg:\Sbj:\Nonpast:\Ipfv/arrive\\
	\trans `I get to the point now!' (Lit. `I arrive there to the words.')\\\Corpus{tci20121019-04}{ABB \#135}
	\label{ex438}
\end{exe}

The \isi{animate}/\isi{inanimate} distinction mentioned in \S{}\ref{spatialcase} can be used to mark definiteness of \isi{animate} referents, for example animals. In (\ref{ex439}), the speaker points out that sorcerers usually do not attack a person directly, but they put a deadly spell on a person's dog or some other animal. Later, when the animal suffers and dies, the human victim will also die. Thus, in (\ref{ex439}) `the dog' and `the wallaby' are generic, and therefore marked with the (\isi{inanimate}) \isi{allative} \isi{case} marker. In contrast, example (\ref{ex440}) is taken from a story about a dog and a crow. Both have been introduced previously and are known to the speaker as individual characters. Consequently, the dog in (\ref{ex440}) is marked with the \isi{animate} \isi{allative}.\footnote{Unfortunately, there is no corpus example of a referent which undergoes a change from inanimate allative to animate allative when tracked through a discourse.}

\begin{exe}
	\ex \emph{\textbf{taurifo} tmatm zrafiyokwr o \textbf{ŋathafo}.}\\
	\gll tauri=fo tmatm zra\stem{fiyok}wr o ŋatha=fo\\
	wallaby=\All{} event \Stsg:\Sbj:\Irr:\Ipfv/make or dog=\All\\
	\trans `(The sorcerer) would do this thing to a wallaby or to a dog.'\\\Corpus{tci20130903-04}{RNA \#114-115}
	\label{ex439}
\end{exe}
\begin{exe}
	\ex \emph{kofä ane zätr ... ymdane zr yföfa \textbf{ŋathadbo}.}\\
	\gll kofä ane zä\stem{tr} (.) ymd=ane zr yfö=fa ŋatha=dbo\\
	fish \Dem{} \Stsg:\Sbj:\Rpst:\Pfv/fall (.) bird=\Poss{} tooth hole=\Abl{} dog=\All.\Anim\\
	\trans `That fish fell down ... from the bird's mouth to the dog.'\\\Corpus{tci20111107-03}{RNA \#68-69}
	\label{ex440}
\end{exe}
	
Although it is possible to attach the \isi{allative} to \isi{temporal} nouns like \emph{efoth} `day', there are no corpus examples of this, and it is generally quite rare. The reason for this is the existence of a \isi{temporal} \isi{purposive} \isi{case} marker \emph{=thamar} (see \S{}\ref{temporalpurposivecase}).

\subsection{Ablative \emph{=fa}} \label{ablativecase}

The \isi{ablative} \isi{case} marker is \emph{=fa} for \isi{inanimate} referents and \emph{=dba} (\Sg) or \emph{=medba} (\Nsg) for \isi{animate} referents. Example (\ref{ex441}) shows the (\isi{inanimate}) \isi{ablative} \isi{case} marker, and example (\ref{ex442}) shows the \isi{animate} \isi{ablative} \isi{case} marker.

\begin{exe}
	\ex \emph{\textbf{torres strait islandfa} thunrärm ... ane masis.}\\
	\gll torres strait island=fa thun\stem{rä}rm (.) ane masis\\
	torres strait island=\Abl{} \Stpl:\Sbj:\Pst:\Dur:\Venit/be (.) \Dem{} matches\\
	\trans `They came from the Torres Strait Islands ... those matchboxes.'\\\Corpus{tci20120909-06}{KAB \#10}
	\label{ex441}
\end{exe}
\begin{exe}
	\ex \emph{trikasi zane mane wnrä ... nzä mane ŋatrikwé ... \textbf{badabadamedba} wnrä.}\\
	\gll trik-si zane mane wn\stem{rä} (.) nzä mane ŋa\stem{trik}wé (.) badabada=medba wn\stem{rä}\\
	tell-\Nmlz{} \Dem:\Prox{} which \Tsg.\F:\Sbj:\Nonpast:\Ipfv:\Venit/be (.) \Fsg.\Abs{} which \Fsg:\Sbj:\Nonpast:\Ipfv/tell (.) ancestor=\Abl.\Anim.\Nsg{} \Tsg.\F:\Sbj:\Nonpast:\Ipfv:\Venit/be\\
	\trans `As for this story ... which I am telling ... it comes from the ancestors.'\\\Corpus{tci20110802}{ABB \#15-17}
	\label{ex442}
\end{exe}

The \isi{ablative} can be used with a \isi{temporal} meaning. There is only one corpus example of the \isi{case} marker \emph{=fa} (\ref{ex443}), but the \isi{deictic} demonstratives are frequently used with \isi{temporal} semantics. Example (\ref{ex443}) concludes a headhunting story in which the speaker points out that the population has increased after this had ceased. The word \emph{zenafa} (`from now') is best translated as `nowadays'.
	
\begin{exe}
	\ex \emph{wati, \textbf{zenafa} ... ni tüfr nagayé kwakonzre.}\\
	\gll wati zena=fa (.) ni tüfr nagayé kwa\stem{ko}nzre\\
	then today=\Abl{} (.) \Fnsg{} plenty children \Fpl:\Sbj:\Rpst:\Ipfv/become\\
	\trans `Nowadays, we have got plenty of children.'\Corpus{tci20111107-01}{MAK \#150-151}
	\label{ex443}
\end{exe}

Example (\ref{ex711}) shows the use of the \isi{deictic} \isi{demonstrative} \emph{foba} `from over there' with a \isi{temporal} meaning, i.e. it expresses a starting point. In the example, the speaker states why he does not know what happened to his family's rain magic stones, and \emph{foba} means `from that time onwards'.

\begin{exe}
	\ex \emph{nzenme ŋafe fthémäsü kwosi yara ... watik \textbf{foba} ni miyamr nrä mafadben zena ethn.}\\
	\gll nzenme ŋafe fthémäsü kwosi ya\stem{r}a (.) watik foba ni miyamr n\stem{rä} mafa=dben zena e\stem{thn}\\
	\Fnsg.\Poss{} father meanwhile dead \Tsg.\Masc:\Sbj:\Pst:\Ipfv/be (.) then \Dist.\Abl{} \Fnsg{} ignorance \Fpl:\Sbj:\Nonpast:\Ipfv/be who=\Loc.\Anim.\Nsg{} today \Stpl:\Sbj:\Nonpast:\Ipfv/lie.down\\
	\trans `In the meantime our father died ... and from then one we don't know with whom (the rain stones) are today.'\Corpus{tci20131013-01}{ABB \#399}
	\label{ex711}
\end{exe}

The \isi{allative} can also be used metaphorically as in example (\ref{ex767}), which is taken from a picture task. In the picture story, a man refuses to drink with his mates, because his alcoholism had brought him to jail. Thus, the \isi{allative} on the word \emph{zrin} `problem' means `from this problem'.

\begin{exe}
	\ex \emph{ane \textbf{zrinfa} watik ziyara.}\\
	\gll ane zrin=fa watik z=ya\stem{r}a\\
	\Dem{} problem=\Abl{} enough \Prox=\Tsg.\Masc:\Pst:\Ipfv/be\\
	\trans `He had enough of this problem here.'\Corpus{tci20111004}{MAE \#2}
	\label{ex767}
\end{exe}

\section{Temporal cases}\label{temporalcase}

Komnzo has a set of \isi{temporal} \isi{case} markers: the \isi{temporal} \isi{locative}, \isi{purposive} and \isi{possessive}. All three \isi{temporal} cases only attach to \isi{temporal} \isi{nominal}s (\S{}\ref{temporals}), like \emph{ezi} `morning' or the \isi{interrogative} \emph{fthé} `when'. I adopt the labels \isi{locative}, \isi{purposive} and \isi{possessive} because of the formal and semantic similarities with the respective cases. Formally, the \isi{temporal} \isi{case} markers consist of \emph{=tham(a)} plus the \isi{case} marker after which they are named. For example, the \isi{temporal} \isi{locative} is \emph{=thamen}. At the present time, there is no etymological explanation for the \emph{=tham(a)} element.
   
\subsection{Temporal locative \emph{=thamen}} \label{temporallocativecase}  

The \isi{temporal} \isi{locative} indicates that something took place in a particular time frame. It is the time frame, usually a \isi{temporal} \isi{nominal}, to which the \isi{temporal} \isi{locative} attaches. Hence, it overlaps with the \isi{locative} \isi{case}, which can also be used on \isi{temporal} \isi{nominal}s. Expressions like \emph{ane efoththamen} `in that day' (with a \isi{temporal} \isi{locative}) and \emph{ane efothen} (with a \isi{locative}) are equivalent. There is only a handful of corpus examples of the \isi{temporal} \isi{locative}. Example (\ref{ex420}) comes from a narrative in which a young boy was attacked by a sorcerer at night in his garden. The young man shot the sorcerer with an arrow, and the sorcerer ran away. The next day a trail of blood could be seen as far as until the garden entrance. In the example, the speaker points out that he was bleeding only at the beginning and the \isi{temporal} \isi{locative} attaches to \emph{zöftha} `first'. Thus, it locates the predicate `bleed' into that time frame. In this case, the resulting form is not \emph{zöfthathamen} as would be expected, but it is reduced to \emph{zöfthamen}.

\begin{exe}
	\ex \emph{\textbf{zöfthamen} zamatho frk komnzo zä wtnägwrmo ...}\\
	\glll zöftha=thamen z-a-math-o-\Zero{} frk komnzo zä~~~~~~~~~~~ w-tnäg-wr-m-o-\Zero\\
	first=\Temp.\Loc{} \M.\Gam-\Ndu-run.\Rs-\Andat-\Stsg{} blood only \Prox{} \Tsg.\F.\Alph-lose.\Ext-\Ndu-\Dur-\Andat{}\\
	~ ~ {\Stsg:\Sbj:\Rpst:\Pfv:\Andat/run} ~ ~ ~ {\Stsg:\Sbj>\Tsg.\F:\Rpst:\Dur:\Andat/lose}\\
	\trans `At first, when he started to run and he was just losing blood here ...'\\\Corpus{tci20130901-04}{YUK \#40}
	\label{ex420}
\end{exe}

\subsection{Temporal purposive \emph{=thamar}} \label{temporalpurposivecase}

The \isi{temporal} \isi{purposive} \isi{case} indicates that something is meant for a particular point in time. The case marker attaches to a \isi{temporal} \isi{nominal}, which specifies that point in time. Example (\ref{ex421}) comes from a procedural text about poison-root fishing. While the speaker explains all the steps, others in the background are busy preparing and cooking the fish. At the end of the recording, he points out how some of the food is `for the afternoon' and the leftovers are `for tomorrow'.

\begin{exe}
	\ex \emph{okay, \textbf{zizithamar} kwa ane fof erä ... nä thzé \textbf{kaythamar} thrägathinze.}\\
	\gll okay zizi=thamar kwa ane fof e\stem{rä} (.) nä thzé kayé=thamar thrä\stem{gathinz}e\\
	okay afternoon=\Temp.\Purp{} \Fut{} \Dem{} \Emph{} \Stpl:\Sbj:\Nonpast:\Ipfv/be (.) some ever tomorrow=\Temp.\Purp{} \Fpl:\Sbj>\Stpl:\Obj:\Irr:\Pfv/leave\\
	\trans `Okay, those are for the afternoon ... whatever (is there), we will leave it for tomorrow.'\Corpus{tci20110813-09}{DAK \#60}
	\label{ex421}
\end{exe}

\subsection{Temporal possessive \emph{=thamane}} \label{temporalpossessivecase}

The \isi{temporal} \isi{possessive} \isi{case} indicates that something is from a particular point in time. It attaches to a \isi{temporal} \isi{nominal}, which specifies that point in time. Example (\ref{ex422}) comes from a story, in which the two protagonists are withholding a particular food source from each other. In (\ref{ex423}), one of them sees a ground oven in the other's camp and asks him about it. The other one responds in (\ref{ex424}) by saying that it is `yesterday's oven'. Here the \isi{temporal} \isi{possessive} inherits the possibility of functioning adnominally from the \isi{possessive} \isi{case}.  

\begin{exe}
\ex \label{ex422}
\begin{xlist}
	\ex \emph{``nzungath, rar karo zane erä?''}\\
	\gll nzun-gath ra=r {karo} zane e\stem{rä}\\
	\Fsg.\Poss-friend what=\Purp{} {earth.oven} \Dem:\Prox{} \Stpl:\Sbj:\Nonpast:\Ipfv/be\\
	\trans ``My friend, what is this earth oven for?''
	\label{ex423}

	\ex \emph{``keke ... kadakada sutränwé ... kayé. \textbf{kaythamane karo} rä!''}\\
	\gll keke (.) {kadakada} su\stem{trän}wé (.) kayé kayé=thamane karo \stem{rä}\\
	\Neg{} (.) {yamcake} \Fsg:\Sbj>\Tsg.\Masc:\Obj:\Rpst:\Ipfv/slice (.) yesterday yesterday=\Temp.\Poss{} {ground oven} \Tsg.\F:\Sbj:\Nonpast:\Ipfv/be\\
	\trans ``No, I cut the yam cake ... yesterday. This is yesterday's oven.''\\\Corpus{tci20110802}{ABB \#90-94}
	\label{ex424}
\end{xlist}
\end{exe}
	
In example (\ref{ex425}) below, the \isi{temporal} \isi{possessive} \isi{case} at the \isi{clause} level, not within a \isi{noun phrase}. The example is from a stimulus picture task. The last part of the task is to retell a story from a first-person perspective. In the example, one of the participants instructs the other one to retell the story `from today onward'.

\begin{exe}
	\ex \emph{nima befe we zakwther! \textbf{zenathamane} be katrikwé!}\\
	\gll nima befe we za\stem{kwther} zena=thamane be ka\stem{trik}wé\\
	{like.this} \Ssg.\Erg.\Emph{} also \Ssg:\Sbj>\Tsg.\F:\Obj:\Imp:\Pfv/change today=\Temp.\Poss{} \Ssg.\Erg{} \Ssg:\Sbj:\Imp:\Ipfv/tell\\
	\trans `You change it like this! You tell it from today.'\Corpus{tci20111004}{MAE \#5}
	\label{ex425}
\end{exe}

\section{Instrumental \emph{=me}} \label{inscase}

The \isi{instrumental} \isi{case} is used for material and immaterial instruments. It usually operates at the clausal level. Example (\ref{ex407}) is taken from a conversation about a sorcerer who \textendash{} after being shot \textendash{} received help from his friend. The friend closed the wound `with mud'. Example (\ref{ex405}) comes from the same story and shows an immaterial instrument. The origin of sorcerer could be identified because he spoke \ili{Wära} or `with Safis language'. Example (\ref{ex408}) comes from a public speech, where the speaker announces that he speaks on behalf of two older men (`speak with their mouths').

\begin{exe}
	\ex \emph{naf we \textbf{gwargwarme} ane yfö yanrmänwa.}\\
	\gll naf we {gwargwar=me} ane yfö yan\stem{rmän}wa\\
	\Tsg.\Erg{} also {mud=\Ins} \Dem{} hole \Stsg:\Sbj>\Tsg.\Masc:\Io:\Nonpast:\Ipfv/close\\
	\trans `He also closed that hole on him with mud.'\Corpus{tci20130901-04}{RNA \#123}
	\label{ex407}
\end{exe}
\begin{exe}
	\ex \emph{\textbf{safs zokwasime} zenafthma.}\\
	\gll safs zokwasi=me ze\stem{nafth}ma\\
	safs language=\Ins{} \Stsg:\Sbj:\Pst:\Pfv/talk\\
	\trans `He talked in Wära.' (Lit. `He talked with Safs language.')\\\Corpus{tci20130901-04}{RNA \#57}
	\label{ex405}
\end{exe}
\begin{exe}
	\ex \emph{\textbf{nafanme zr yföme} ŋanafé ... sowai a karbu ... zena zbär.}\\
	\gll nafanme zr yfö=me ŋa\stem{na}fé (.) sowai a karbu (.) zena zbär\\
	\Tnsg.\Poss{} tooth hole=\Ins{} \Fsg:\Sbj:\Nonpast:\Ipfv/speak (.) sowai and karbu (.) today night\\
	\trans `Tonight, I am talking for them, for Sowai and Karbu.' (Lit. `I talk with their mouths.')\Corpus{tci20121019-04}{ABB \#91-92}
	\label{ex408}
\end{exe}
	
At the cross-clausal level the \isi{instrumental} \isi{case} is used for resultative constructions (\ref{ex404}). Resultative constructions typically employ the copula and a nominalised verb form: \emph{rfithzsime} translates literally as `with hiding'.

\begin{exe}
	\ex \emph{nge kwa erifthznth ... nafaŋamayé ... \textbf{rifthzsime} kwa enrn.}\\
	\gll nge kwa e\stem{rifth}znth (.) nafa-ŋame=é (.) rifthz-si=me kwa en\stem{r}n\\
	child \Fut{} \Stpl:\Sbj>\Stdu:\Obj:\Nonpast:\Ipfv/hide (.) \Third.\Poss-mother=\Erg.\Nsg{} (.) hide-\Nmlz=\Ins{} \Fut{} \Stdu:\Sbj:\Nonpast:\Ipfv:\Venit/be\\
	\trans `The mothers will hide the two children ... They will be hidden.'\\\Corpus{tci20110817-02}{ABB \#72-73}
	\label{ex404}
\end{exe}

The \isi{instrumental} \isi{case} is frequently used on property nouns (\ref{ex406}) and adjectives (\ref{ex409}) with an adverbial function. In example (\ref{ex406}) the speaker talks about customs surrounding the yam harvest, and in (\ref{ex409}) he explains why he is not planting big gardens anymore. In both examples, the \isi{instrumental} \isi{case} derives a \isi{manner adverb}.  

\begin{exe}
	\ex \emph{\textbf{zünzme} befe fthé zanathé bonemäwä keke tüfr thrarä.}\\
	\gll zünz=me befe fthé za\stem{na}thé bone=ma=wä keke tüfr thra\stem{rä}\\
	greed=\Ins{} \Ssg.\Erg.\Emph{} when \Ssg:\Sbj:\Imp:\Pfv/eat \Ssg.\Poss=\Char=\Emph{} \Neg{} plenty \Stpl:\Sbj:\Irr:\Ipfv/be\\
	\trans `If you eat greedily, your own (yams) will not be plenty.'\\\Corpus{tci20120805-01}{ABB \#760}
	\label{ex406}
\end{exe}
\begin{exe}
	\ex \emph{watik, nzone tmä we \textbf{katanme} ŋarsörm.}\\
	\gll watik nzone tmä we katan=me ŋa\stem{rsö}rm\\
	then \Fsg.\Poss{} strength also small=\Ins{} \Stsg:\Sbj:\Rpst:\Dur/recede\\
	\trans `Well, my strength has gone down a little.'\Corpus{tci20120805-01}{ABB \#664}
	\label{ex409}
\end{exe}
	
The \isi{instrumental} \isi{case} can also be attached to demonstratives as in (\ref{ex410}), where the speaker explains to me how to protects one's bamboo bow against insects. In (\ref{ex411}), the \isi{instrumental} is attached to \emph{mane} `which' and used as a relative \isi{pronoun} `with which'.

\begin{exe}
	\ex \emph{\textbf{ngazime} o zaru ... nzanzama ... watik \textbf{aneme} zminzakwé zabth.}\\
	\gll ngazi=me o zaru (.) nzanza=ma (.) watik ane=me z\stem{minzak}wé za\stem{bth}\\
	coconut=\Ins{} or candlenut (.) woodworm=\Char{} (.) then \Dem=Ins{} \Ssg:\Sbj>\Tsg.\F:\Imp:\Ipfv/paint \Stsg:\Sbj:\Rpst:\Pfv/finish\\
	\trans `With coconut or candlenut, because of the woodworm. Finally, you paint (the bow) with that one and it is finished.'\\\Corpus{tci20120922-23}{MAA \#81-83}
	\label{ex410}
\end{exe}
\begin{exe}
	\ex \emph{kitr zane erä ... yame yrsima ... amaf \textbf{maneme} yame wrwr.}\\
	\gll kitr zane e\stem{rä} (.) yame yr-si=ma (.) ama=f mane=me yame w\stem{r}wr\\
	river.pandanus \Dem:\Prox{} \Stpl:\Sbj:\Nonpast:\Ipfv/be (.) mat weave-\Nmlz=\Char{} (.) mother=\Erg{} which=\Ins{} mat \Stsg:\Sbj>\Tsg.\F:\Obj:\Nonpast:\Ipfv/weave\\ 
	\trans `This is \emph{Kitr}, for weaving mats ... with which mother weaves the mat.'\Corpus{tci20130907-02}{JAA \#235-236}
	\label{ex411}
\end{exe}

The \isi{instrumental} attaches productively to several \isi{interrogative} pronouns: \emph{ra=me} `with what' or \emph{mane=me} `with which'. The \isi{interrogative} \emph{mon} `how' can occur with or without the \isi{instrumental} \isi{case}; both \emph{mon} and \emph{monme} can be used interchangeably.

\section{Purposive \emph{=r}} \label{purpcase}

The \isi{purposive} \isi{case} is used at the clausal and cross-clausal level. It expresses someone's intention (\ref{ex414} and \ref{ex413}) or the inherent purpose of some entity (\ref{ex415}). In example (\ref{ex414}), a man informs his younger brothers about his plans for the night. Example (\ref{ex413}) comes from a procedural about gardening and the speaker explains the purpose of the different steps involved.

\begin{exe}
	\ex \emph{naf ni nzräkor ``ngthé ... nima nyak ŋarsfo etfthmöwä \textbf{kofär} ... zbär kwa \textbf{zuzir} ŋarzre.''}\\
	\gll naf ni nzrä\stem{kor} ngthé (.) nima n\stem{yak} ŋars=fo etfth=me=wä kofä=r (.) zbär kwa zuzi=r ŋa\stem{r}zre\\
	\Tsg.\Erg{} \Fnsg{} younger.sibling \Stsg:\Sbj>\Fpl:\Obj:\Irr:\Pfv{} (.) {like.this} \Fpl:\Sbj:\Nonpast:\Ipfv/go river=\All{} sleep=\Ins=\Emph{} fish=\Purp{} (.) night \Fut{} fishing=\Purp{} \Fpl:\Sbj:\Nonpast:\Ipfv/throw\\ 
	\trans `He said to us: ``Hey small brothers! We will go to the river ... overnight ... for fish. We will throw the fishing line in the night.'''\Corpus{tci120904-02}{MAB \#26-29}
	\label{ex414}
\end{exe}
\begin{exe}
	\ex \emph{efäefä krarzrth \textbf{ŋaraker} ... \textbf{wotu wotu räzsir}.}\\
	\gll {efäefä} kra\stem{r}zrth ŋarake=r (.) wotu-wotu räz-si=r\\
	aisle \Stpl:\Sbj:\Irr:\Ipfv/throw fence=\Purp{} (.) \Redup-stick erect-\Nmlz=\Purp\\
	\trans `They cut an aisle for the fence ... for erecting the sticks.'\\\Corpus{tci20120805-01}{ABB \#51-52}
	\label{ex413}
\end{exe}	
	
The last noun phrase in (\ref{ex413}) and in (\ref{ex415}) below show the \isi{purposive} \isi{case} operating at the cross-clausal level. In both cases the \isi{purposive} \isi{case} marker is attached to an infinitival adjunct. In (\ref{ex415}), the speaker talks about scorcerers who visit a deceased man's grave after the burial to extract certain body parts. In both examples, the clause marked with the \isi{purposive} contains the infinitive as well as the \isi{object} of the event in the ablative, for example \emph{tmä yarisi} `strength giving' in (\ref{ex415}).
	
\begin{exe}
	\ex \emph{fi fenz ane \textbf{bänemrnzo} rä ... \textbf{tmä yarisir}.}\\
	\gll fi fenz ane bänemr=nzo \stem{rä} (.) tmä yari-si=r\\
	but body.liquid \Dem{} \Recog.\Purp=\Only{} \Tsg.\F:\Nonpast:\Ipfv/be (.) strength give-\Nmlz=\Purp{}\\
	\trans `but that body liquid is only for this ... for giving strength.'\\\Corpus{tci20130903-04}{RNA \#139-140}
	\label{ex415}
\end{exe}

The noun phrase or the infinitival adjunct marked with \emph{=r} ascribes a specific purpose, and in this ascriptive function, the \isi{purposive} overlaps with the \isi{characteristic} \isi{case}. Hence, in (\ref{ex415}) both \emph{tmä yarisir} and \emph{tmä yarisima} would be grammatical and identical in meaning. I described the nature of this overlap in \S{}\ref{charcase}.%\\

There is a set of \isi{purposive} personal pronouns in Komnzo. All the pronouns share a \emph{-nar} element, for example \emph{nzunar} `for me', \emph{nzenar} `for us'.\footnote{See \tabref{perspron} on page \pageref{perspron}.} However, these pronouns are rarely used, in fact so rarely that I came accross them only very late in my fieldwork. Moreover, there is not a single token in the text corpus. As one would predict from the semantics of the \isi{purposive} \isi{case}, these pronouns encode a \isi{beneficiary} or a \isi{goal}. But this function is already covered by the \isi{dative} \isi{case}. I will offer a hypothetical semantic shift scenario at the end of this chapter which partly explains why the \isi{purposive} pronouns are so rarely used.

\section{Characteristic \emph{=ma}} \label{charcase}

The \isi{characteristic} \isi{case} covers a number of semantic roles which are source, reason and purpose. The \isi{characteristic} operates at all three levels: adnominal (\ref{ex419}), clausal (\ref{ex390}) and cross-clausal (\ref{ex391}). In example (\ref{ex419}), \emph{karma} `from the village' functions within a matrix \isi{noun phrase}. In this example, the \isi{characteristic} could be left out, and \emph{ane karma kabe} or \emph{ane kar kabe} are both grammatical.\footnote{In \emph{zane kar kabe}, the phrasal head consists of a compound \emph{kar kabe}. In \emph{zane karma kabe}, the noun phrase \emph{zane kar} is embedded in a matrix noun phrase. Thus, the reference of the demonstrative \emph{zane} is different between the two examples. In the former case \emph{zane} refers to the complex head, but in the latter case \emph{zane} refers only to the head of the embedded noun phrase. See \S{}\ref{headslot} for a discussion of noun phrases.} In example (\ref{ex390}), the \isi{characteristic} \isi{case} attaches to a separate \isi{noun phrase} and functions at the clause level. In example (\ref{ex391}), the speaker comments on the exhausting work of dragging a sago palm trunk. The \isi{characteristic} \isi{case} attaches to an infinitival adjunct (`dragging') and, thus, operates at a cross-clausal level.

\begin{exe}
	\ex \emph{keke thufnzrm ... \textbf{ane karma kabe}}\\
	\gll keke thu\stem{fn}zrm (.) ane kar=ma kabe\\
	\Neg{} \Stsg:\Sbj>\Stpl:\Obj:\Pst:\Dur/hit (.) \Dem{} village=\Char{} man\\
	\trans `She was not killing them ... the people from this village.'\\\Corpus{tci20120901-01}{MAK \#50}
	\label{ex419}
\end{exe}
\begin{exe}
	\ex \emph{\textbf{zane karma} minzü fefe nafa dagon swafiyokwrmth bänema z zbo ŋabrüza.}\\
	\gll zane kar=ma minzü fefe nafa dagon swa\stem{fiyok}wrmth bäne=ma z zbo ŋa\stem{brü}za\\
	\Dem:\Prox{} village=\Char{} very real \Tnsg.\Erg{} food \Stpl:\Sbj>\Tsg.\Masc:\Io:\Pst:\Dur/make \Recog=\Char{} \Iam{} \Prox.\All{} \Sg:\Sbj:\Pst:\Ipfv/drown\\
	\trans `From this village the people made a lot of food for him because he drowned here.'\Corpus{tci20150906-10}{ABB \#296-297}
	\label{ex390}
\end{exe}
\begin{exe}
	\ex \emph{festh tayo tayo nrä ... \textbf{bäne thärkusima}.}\\
	\gll festh tayo tayo n\stem{rä} (.) bäne thärku-si=ma\\
	body weak weak \Fpl:\Sbj:\Nonpast:\Ipfv/be (.) \Dem:\Med{} drag-\Nmlz=\Char{}\\
	\trans `Our bodies are weak from that dragging (of the sago palm).'\\\Corpus{tci20120929-02}{SIK \#66-67}
	\label{ex391}
\end{exe}

In example (\ref{ex391}), the semantic role of spatial origin or source is extended to non-spatial origin, that is reason or cause. Note that the source of motion cannot be expressed using the \isi{characteristic} \isi{case}. Instead the \isi{ablative} \emph{=fa} has to be used. Non-spatial origin is also found at the clausal level, for example in (\ref{ex392}) where the speaker explains why she was hesitant at first about working for the anthropologist Mary Ayres.

\begin{exe}
	\ex \emph{nzä \textbf{ane markai zokwasima} wtri kwarärm.}\\
	\gll nzä ane markai zokwasi=ma wtri kwa\stem{rä}rm\\
	\Fsg.\Abs{} \Dem{} outsider language=\Char{} fear \Fsg.\Sbj:\Pst:\Dur/be\\
	\trans `I was afraid of that white man's language.'\Corpus{tci20130911-03}{MAA \#15}
	\label{ex392}
\end{exe}

Example (\ref{ex395}) concludes a recording taken inside a yam house where the speaker has talked about the different types of yams and the sorting principle in the storage house. He launches a whole battery of noun phrases marked with the \isi{characteristic} \isi{case} to express what the story `was about', and thus the \isi{case} marker can also be used to express the topic of a conversation. In the example, the noun phrases are marked by angled brackets.

\begin{exe}
	\ex \emph{watik zane zizin [\textbf{wawama}] [\textbf{trikasi tharisima}] [\textbf{tafoma}] [\textbf{sagusaguma}] ... mon eworthre ... mane [\textbf{dagonma}] erä ... mane tafo erä ... zbo zf zbthe brä trikasi ... eso kafar [\textbf{bone namä yarizsima}].}\\
	\gll watik zane zizi=n wawa=ma trik-si thari-si=ma tafo=ma sagu-sagu=ma (.) mon e\stem{wor}thre (.) mane dagon=ma e\stem{rä} (.) mane tafo e\stem{rä} (.) zbo zf z\stem{bth}e b=\stem{rä} trik-si (.) eso kafar bone namä yariz-si=ma\\
	then \Dem:\Prox{} afternoon=\Loc{} yam=\Char{} tell-\Nmlz{} harvest-\Nmlz=\Char{} yam.type=\Char{} \Redup-yam.type=\Char{} (.) how \Fpl:\Sbj>\Stpl:\Obj:\Nonpast:\Ipfv/plant (.) which food=\Char{} \Stpl:\Sbj:\Nonpast:\Ipfv/be (.) which yam.type \Stpl:\Sbj:\Nonpast:\Ipfv/be (.) \Prox.\All{} \Imm{} \Fdu:\Sbj>\Tsg.\F:\Rpst:\Pfv/finish \Med=\Tsg.\F:\Sbj:\Nonpast:\Ipfv/be tell-\Nmlz{} (.) thanks big \Ssg.\Poss{} good listen-\Nmlz=\Char{}\\
	\trans `Well, in this afternoon ... (we talked) about yams, the story about harvesting, about Tafo yams and Sagu Sagu yams ... how we plant them ... which ones are for eating ... which ones are for Tafo (storing). We have finished it now there. Thank you for listening.'\Corpus{tci20121001}{ABB \#215-221}
	\label{ex395}
\end{exe}

Note that the last two tokens of \emph{=ma} in example (\ref{ex395}) are different in their semantics. The \isi{noun phrase} \emph{dagonma} does not translate as `about the food', but as `for eating'. The last token of \emph{=ma} can be translated as both reason or purpose: \emph{eso kafar [bone namä yarizsima]} `thanks because of your listening' or `thanks for your listening'. Without examples like these the labels `source' and `cause' would be sufficient descriptions for this \isi{case} marker. However, quite frequently \emph{=ma} encodes a purpose and, therefore, I choose the cover term `\isi{characteristic}'.%\\

Consider example (\ref{ex393}) below which comes from a walk through the forest. Along the path, the speaker shows me a particular grass. The leaf of this grass can be placed between the lips, and one can produce a high cheeping sound by blowing through it. She explains that this can be used `for attracting snakes', thus, the \isi{characteristic} is marking a purpose in (\ref{ex394}). After demonstrating how to produce the sound, she repeats in (\ref{ex396}) why the snake is coming (\emph{kwanma} `because of the sound') and concludes that she would not usually blow this grass (\emph{anema} `therefore'). Here the \isi{characteristic} \isi{case} marks a reason.

\begin{exe}
\ex
\label{ex393}
\begin{xlist}
 	\ex \emph{\textbf{kaboth räkumgsima} yé.}\\
 	\gll kaboth räkumg-si=ma \stem{yé}\\
 	snake attract-\Nmlz=\Char{} \Tsg.\Masc:\Sbj:\Nonpast:\Ipfv/be\\ 
 	\trans `It is for attracting snakes.'\Corpus{tci20130907-02}{RNA \#612}
 	\label{ex394}

 	\ex \emph{kaboth kwa ŋankwir \textbf{ane kwanma} ... \textbf{anema} fof keke efsgwre.}\\
 	\gll kaboth kwa ŋan\stem{kwi}r ane kwan=ma (.) ane=ma fof keke e\stem{fsg}wre\\
 	snake \Fut{} \Stsg:\Sbj:\Nonpast:\Ipfv:\Venit/run \Dem{} noise=\Char{} (.) \Dem=\Char{} \Emph{} \Neg{} \Fpl:\Sbj>\Stpl:\Obj:\Nonpast:\Ipfv/blow\\
 	\trans `The snake will run here because of that sound ... therefore we do not blow them.'\Corpus{tci20130907-02}{RNA \#615-616}
 	\label{ex396}
\end{xlist}
\end{exe}

In her analysis of Ancient Greek, Luraghi suggests that ``the notion of Reason, which, as remarked by Croft (\citeyear{Croft:cogn}), mediates between Cause and Purpose, really constitutes a kind of undifferentiated area, in which the reason that motivates an agent to act is cognitively equivalent to the purpose of the action, so that the two notions overlap completely'' (\citeyear[46]{Luraghi:2003vi}). See also Luraghi (\citeyear{Luraghi:2001bv}) for a cross-linguistic study of semantic roles. In Komnzo, this overlap does not play out as a diachronic process, but as coexisting uses of the \isi{characteristic} \isi{case}. Example (\ref{ex397}) below supports the point made by Luraghi. The noun \emph{yasema} can be translated to \ili{English} as cause/motivation (`because of meat') as well as purpose (`for meat'). The reason for the action and the purpose of the action are expressed by \emph{=ma}.

\begin{exe}
	\ex \emph{nabimäre fthé gnräré bone nagayé kwa änor ... \textbf{yasema}.}\\
	\gll nabi=märe fthé gn\stem{rär}é bone nagayé kwa ä\stem{nor} (.) yase=ma\\
	bow=\Priv{} when \Ssg:\Sbj:\Imp:\Ipfv/be \Ssg.\Poss{} children \Fut{} \Stpl:\Sbj:\Nonpast:\Ipfv/shout{} (.) game=\Char{}\\
	\trans `When you are without a bow, your children will cry for meat / because of meat.'\Corpus{tci20120922-23}{MAA \#89-91}
	\label{ex397}
\end{exe}%397

The \isi{characteristic} \isi{case} competes with the \isi{purposive} \isi{case} in marking the semantic role of purpose. In many utterances, they can be used interchangeably. Consider examples (\ref{ex399}) and (\ref{ex398}) below, where both can be used to express an inherent purpose of some entity (`the leaf is for rolling cigarettes'). Likewise, in (\ref{ex394}) above, the \isi{purposive} could be used (\emph{kaboth räkumgsir yé} `it is for attracting snakes'). An intentional purpose of some individual (e.g. `he goes for hunting') is most frequently expressed by the \isi{purposive} \isi{case}, not by the \isi{characteristic}.

\begin{exe}
	\ex \emph{zane mane yé ... bänemr yrärth ... \textbf{sukufa knsir}.}\\
	\gll zane mane \stem{yé} (.) bänemr y\stem{rä}rth (.) sukufa kn-si=r\\
	\Dem:\Prox{} which \Tsg.\Masc.\Nonpast:\Ipfv/be (.) \Recog.\Purp{} \Stpl:\Sbj>\Tsg.\Masc:\Nonpast:\Ipfv/do (.) tobacco roll-\Nmlz=\Purp{}\\
	\trans `As for this one ... they use is for that ... for rolling cigarettes.'\\\Corpus{tci20130907-02}{RNA \#506-508}
	\label{ex399}
\end{exe}
\begin{exe}
	\ex \emph{ane taga mane erä \textbf{sukufa knsima} we erä.}\\
	\gll ane taga mane e\stem{rä} sukufa kn-si=ma we e\stem{rä}\\
	\Dem{} leaf which \Stpl:\Sbj:\Nonpast:\Ipfv/be tobacco roll-\Nmlz=\Char{} also \Stpl:\Sbj:\Nonpast:\Ipfv/be\\
	\trans `As for those leaves, they are also used for rolling cigarettes.'\\\Corpus{tci20130907-02}{RNA \#567}
	\label{ex398}
\end{exe}

With \isi{animate} referents, the \isi{dative} is used to mark a \isi{goal} or \isi{beneficiary}. The \isi{purposive} \isi{case} can be used for more abstract \isi{animate} referents, for example \emph{fäms ŋare=r} `for/as exchange woman'.\footnote{Example (\ref{ex173}) on page \pageref{ex173} provides a textual example of \emph{fäms ŋarer}.} The \isi{characteristic} \isi{case} cannot serve for marking purpose in this sense. Instead, with \isi{animate} referents it always marks a reason, origin or cause. Additionally, \isi{animate} referents must take the \isi{possessive} \isi{case} first, and then the \isi{characteristic} \emph{=ma} attaches to the \isi{possessive}. In example (\ref{ex400}), a young man explains how the food will be shared during an upcoming feast. The \isi{characteristic} is attached to the \isi{possessive} pronouns. Example (\ref{ex401}) comes from a story in which the wife of a man had been killed, and at the end of the story he cries bitterly because of her. In both examples, the \isi{characteristic} \isi{case} attaches to a \isi{possessive}: \emph{nzenmema} and \emph{nafaŋareanema}. It is ungrammatical to use the unmarked (\isi{absolutive}) forms: \textsuperscript{$\ast$}\emph{nima} and \textsuperscript{$\ast$}\emph{nafaŋarema}.

\begin{exe}
	\ex \emph{we nafa \textbf{nzenmema} sräthoroth ... ni \textbf{nafanmema} fof sränthore.}\\
	\gll we nafa nzenme=ma srä\stem{thor}oth (.) ni nafanme=ma fof srän\stem{thor}e\\
	also \Tnsg.\Erg{} \Fnsg.\Poss=\Char{} \Stpl:\Sbj>\Tsg.\Masc:\Obj:\Irr:\Pfv:\Andat/carry (.) \Fnsg{} \Tnsg.\Poss=\Char{} \Emph{} \Fpl:\Sbj>\Tsg.\Masc:\Obj:\Irr:\Pfv:\Venit/carry\\
	\trans `They will take it from us and we will take it from them.'\\\Corpus{tci20120929-02}{SIK \#97-98}
	\label{ex400}
\end{exe}
\begin{exe}
	\ex \emph{yanzo bobo yanora \textbf{nafaŋareanema}.}
	\gll ya=nzo bobo ya\stem{nor}a nafa-ŋare=ane=ma\\
	cry=\Only{} \Med.\All{} \Tsg.\Masc:\Sbj:\Pst:\Ipfv/cry \Third.\Poss-woman=\Poss=\Char\\
	\trans `He cried badly there because of his wife.'\Corpus{tci20120901-01}{MAK \#208-209}
	\label{ex401}
\end{exe}

The \isi{characteristic} suffix is used to derive cardinal numerals: \emph{eda} `two' $\rightarrow$ \emph{edama} `second' (see \S\ref{numerals}). In example (\ref{ex402}), the speaker explains what I have to do during an upcoming namesake ceremony.

\begin{exe}
	\ex \emph{chrisf yathugwr keke kwa srefaf yakme ... \textbf{ethama} mane yé ... kwa fthé fof yfathwr.}\\
	\gll chris=f ya\stem{thug}wr keke kwa sre\stem{faf} yak=me (.) etha=ma mane \stem{yé} (.) kwa fthé fof y\stem{fath}wr\\
	chris=\Erg.\Sg{} \Stsg:\Sbj>\Tsg.\Masc:\Obj:\Nonpast:\Ipfv/trick \Neg{} \Fut{} \Stsg:\Sbj>\Tsg.\Masc:\Obj:\Irr:\Pfv/hold run=\Ins{} (.) three=\Char{} which \Tsg.\Masc:\Nonpast:\Ipfv/be (.) \Fut{} when \Emph{} \Stsg:\Sbj>\Tsg.\Masc:\Nonpast:\Ipfv/hold\\
	\trans `Chris will trick him, he will not hold him quickly ... Only at the third (time) ... (that is) when he will really hold him.'\Corpus{tci20110817-02}{ABB \#89-91}
	\label{ex402}
\end{exe}

The \isi{characteristic} \isi{case} is frequently used on \isi{demonstrative} pronouns as in (\ref{ex396}) meaning `therefore'. In some words, the \isi{characteristic} \isi{case} has become lexicalised, for example: \emph{rma} `why' from \emph{ra} `what' plus \emph{=ma} or \emph{karama wath} `karama dance' from \emph{kara} which is a place in the West. Other lexical items show a \emph{ma} element, but the connection to the \isi{characteristic} \isi{case} is hypothetical at the moment, for example \emph{nzagoma} `for later, in advance' and \emph{madma} `female'.

\section{Proprietive \emph{=karä}} \label{propcase}

The \isi{proprietive} is used at the clausal and cross-clausal level. It expresses the semantic role of association (`with something' or `with someone') or property (`having some quality').\footnote{It follows that the labels proprietive and associative are equally well justified. I choose proprietive because it contrasts with the private case.} In expressing the role of association, the \isi{proprietive} overlaps with the \isi{associative} \isi{case} (see \S{}\ref{comcase}). The role of property (assignment) employs an existential construction as in (\ref{ex347}) and (\ref{ex346}).%\\

Although the \isi{proprietive} \emph{=karä} attaches to one noun phrase relating it semantically to another noun phrase, the two NPs do not form a syntactic constituent, i.e. the \isi{proprietive} does not function adnominally. In example (\ref{ex344}), the speaker is boasting about his big yam garden: `I am the one with the biggest garden'. In example (\ref{ex345}), a woman describes a namesake ceremony, where the mother `with her child' are hidden behind a curtain of coconut leaves waiting to be officially presented to their relatives. In both examples, the noun phrase marked with the \isi{proprietive} is printed in bold, and the noun phrase to which it associates some entity is underlined.

\begin{exe}
	\ex \emph{\underline{nzänzo} zä zf worä \textbf{kafarwä dawkarä} fof.}\\
	\gll nzä=nzo zä zf wo\stem{rä} kafar=wä daw=karä fof\\
	\Fsg.\Abs{}=\Only{} \Prox{} \Imm{} \Fsg:\Sbj:\Nonpast:\Ipfv/be big=\Emph{} garden=\Prop{} \Emph{}\\  
	\trans `I am the only one here with a really big garden.'\Corpus{tci20120805-01}{ABB \#655}
	\label{ex344}
\end{exe}
\begin{exe}
	\ex \emph{\underline{nzä} zweyafürath \textbf{ngekarä}  ... samtherath warfo ``nge zyé!''}\\
	\gll nzä zwe\stem{yafür}ath nge=karä (.) sa\stem{mther}ath warfo nge z=\stem{yé}\\
	\Fsg.\Abs{} \Stpl:\Sbj>\Fsg:\Io:\Pst:\Pfv/open child=\Prop{} (.) \Stpl:\Sbj>\Tsg.\Masc:\Obj:\Pst:\Pfv/lift.up above child \Prox=\Tsg.\Masc:\Sbj:\Nonpast:\Ipfv/be\\
	\trans `They opened it for me with the child. They lifted him up high (and said) ``Here is the boy!'''\Corpus{tci20130823-08}{WAM \#43}
	\label{ex345}
\end{exe}
	
The \isi{proprietive} is frequently used with the copula to express a property or quality of something: `with dust' in (\ref{ex347}), or someone: `with facial hair' in (\ref{ex346}). The kinds of properties assigned are usually portrayed as being of temporary nature.

\begin{exe}
	\ex \emph{\textbf{gwrmgkarä} \underline{zane kar} rä.}\\
	\gll gwrmg=karä zane kar \stem{rä}\\
	dust=\Prop{} \Dem:\Prox{} place \Tsg.\F:\Sbj:\Nonpast:\Ipfv/be\\
	\trans `This is a dusty place.'\Corpus{tci20121019-04}{ABB \#7}
	\label{ex347}
\end{exe}
\begin{exe}
	\ex \emph{kabe yé ... \textbf{fäk thäbukarä} yé.}\\
	\gll kabe yé (.) fäk thäbu=karä \stem{yé}\\
	man \Tsg.\Masc:\Sbj:\Nonpast:\Ipfv/be (.) jaw hair=\Prop{} \Tsg.\Masc:\Sbj:\Nonpast:\Ipfv/be\\
	\trans `This is a man. He has a beard.'\Corpus{tci20111004}{RMA \#90}
	\label{ex346}
\end{exe}

Examples (\ref{ex351}) and (\ref{ex352}) contrast the \isi{proprietive} \isi{case} with the instrumental \isi{case}. In example (\ref{ex351}), the speaker talks about local medicine and how one has to mix the liquid of a particular plant with water. Hence, \emph{nokarä} has to be translated as addition: `(together) with the water'. In example (\ref{ex352}), the shallow water on the riverbank acts as an instrument making it easier to roll a heavy sago stem. Consequently, \emph{nome} has to be translated as: `with (the help of) the water'.

\begin{exe}
	\ex \emph{\textbf{nokarä} swathknwé! ... ane käznob!}\\
	\gll no=karä s\stem{wathkn}wé (.) ane käz\stem{nob}\\
	water=\Prop{} \Ssg:\Sbj>\Tsg.\Masc:\Obj:\Imp:\Ipfv/stir (.) \Dem{} \Ssg:\Sbj:\Imp:\Pfv/drink\\
	\trans `You stir it with water and drink that!'\Corpus{tci20130907-02}{RNA \#189}
	\label{ex351}
\end{exe}
\begin{exe}
	\ex \emph{sathkäfake bi frezsi thenzgsi ... \textbf{anemöwä} töna sakorake ... zane \textbf{nome}.}\\
	\gll sa\stem{thkäf}ake bi frez-si thenzg-si (.) ane=me=wä töna sa\stem{kor}ake (.) zane no=me\\
	\Fpl:\Sbj>\Tsg.\Masc:\Obj:\Pst:\Pfv/start sago bring.up.from.river-\Nmlz{} roll-\Nmlz{} (.) \Dem=\Ins=\Emph{} high.ground \Fpl:\Sbj>\Tsg.\Masc:\Obj:\Pst:\Pfv/become (.) \Dem:\Prox{} water=\Ins{}\\
	\trans `We started bringing up the sago from the river by rolling it ... with that we brought it to the high ground ... with the water.'\\\Corpus{tci20120929-02}{SIK \#57-58}
	\label{ex352}
\end{exe}

The \isi{proprietive} \isi{case} operates at the cross-clausal level when it is attached to nominalised verb (\ref{ex350}). Unlike the instrumental \isi{case}, the \isi{proprietive} does not form a resultative construction. In (\ref{ex350}), the relationship between \emph{borsi} `laugh' and the predicate `he looks' is one of association or \isi{simultaneity}. It can also be translated as a manner adverbial (`He stands laughingly.'). In example (\ref{ex428}) the father comes while telling a story. In contrast, in resultative constructions, the result of some previous event is emphasised. For example, in (\ref{ex427}) the speaker points to a stack of yams in his storage house stressing the fact that he has piled up different types of yam tubers. This can be analysed as a pseudo-\isi{passive} construction (\S{}\ref{passiveclause}).
	
\begin{exe}
	\ex \emph{gon z zefaf ... \textbf{borsikarä} efoth ymarwr.}\\
	\gll gon z ze\stem{faf} (.) bor-si=karä efoth y\stem{mar}wr\\
	hip \Iam{} \Fsg:\Sbj:\Rpst:\Pfv/hold (.) laugh-\Nmlz=\Prop{} sun \Stsg:\Sbj>\Tsg.\Masc:\Obj:\Nonpast:\Ipfv/see\\ 
	\trans `He has his hands on his hips. As he looks up at the sun. he laughs.'\\\Corpus{tci20111004}{RMA \#502-503}
	\label{ex350}
\end{exe}
\begin{exe}
	\ex \emph{nafaŋafe \textbf{trikasikarä} yanyak.}\\
	\gll nafa-ŋafe trik-si=karä yan\stem{yak}\\
	\Third.\Poss-father tell-\Nmlz=\Prop{} \Tsg.\Masc:\Sbj:\Nonpast:\Ipfv:\Venit/walk\\
	\trans `The father walks here while he is telling a story.'\Corpus{tci20111004}{RMA \#329}
	\label{ex428}
\end{exe}
\begin{exe}
	\ex \emph{zane \textbf{fukthksime} erä.}\\
	\gll zane fukthk-si=me e\stem{rä}\\
	\Dem:\Prox{} mix-\Nmlz=\Ins{} \Stpl:\Sbj:\Nonpast:\Ipfv/be\\
	\trans `These ones have been mixed.'\Corpus{tci20121001}{ABB \#178}
	\label{ex427}
\end{exe}

At the clausal level, the \isi{proprietive} can also attach to a nominalised verb. Example (\ref{ex348}) is the description of a picture card which depicts a prisoner sitting in his cell. Example (\ref{ex349}) comes from the same recording, when the prisoner is set free and handed back his belongings. These two examples presuppose some kind of result \textendash{} `has been tied' and `has been opened' respectively \textendash{} but the previous event remains implicit. For example, in (\ref{ex349}) the speaker draws attention to the fact that the door is open with the help of a \isi{demonstrative} \isi{identifier} \emph{brä}. If the instrumental \isi{case} was used instead (\emph{yafüsime}), the result of the opening event would be emphasised.

\begin{exe}
	\ex \emph{wati ane fóf yamnzr ... fam ngarär ... fafen \textbf{wäthsikarä} yé.}\\
	\gll wati ane fof ya\stem{m}nzr (.) fam nga\stem{rär} (.) fafen wäth-si=karä \stem{yé}\\
	Then \Dem{} \Emph{} \Tsg.\Masc:\Sbj:\Nonpast:\Ipfv/sit (.) thoughts \Stsg:\Sbj:\Nonpast:\Ipfv/do (.) during tie-\Nmlz=\Prop{} \Tsg.\Masc:\Sbj:\Nonpast:\Ipfv/be\\
	\trans `Well, that one is sitting ... he is thinking ... with his hands tied.'\\\Corpus{tci20111004}{RMA \#133-134}
	\label{ex348}
\end{exe}
\begin{exe}
	\ex \emph{zrfö bana z seyafürth ... zrfö \textbf{yafüsikarä} brä.}\\
	\gll zrfö bana z se\stem{yafür}th (.) zrfö yafü-si=karä b=\stem{rä}\\
	door poor \Iam{} \Stpl:\Sbj>\Tsg.\Masc:\Io:\Rpst:\Pfv/open (.) door open-\Nmlz=\Prop{} \Med=\Tsg.\Masc:\Sbj:\Nonpast:\Ipfv/be\\
	\trans `They have already opened the door for the poor guy. (See) there, the door is open!'\Corpus{tci20111004}{RMA \#432-433}
	\label{ex349}
\end{exe}
	
There is a second variant of the \isi{proprietive} marker, which is \emph{=kaf}. In terms of \isi{frequency}, the distribution of the two formatives is rather skewed: \emph{=kaf} is attested 22 times in the corpus compared to 194 occurences of \emph{=karä}. The distribution patterns neither with age or language portfolio of individual speakers. In the close varieties \ili{Wära} and \ili{Anta} both formatives are also attested.

\section{Privative \emph{=märe}} \label{privcase}

The \isi{privative} \isi{case} \emph{=mär} or \emph{=märe} is the opposite of the \isi{proprietive}. It is used to indicate that some entity lacks something (\ref{ex354}), someone (\ref{ex353}) or some quality (\ref{ex355}). The \isi{privative} operates usally at the clausal level. Like the \isi{proprietive} \isi{case}, it can establish a semantic link between two noun phrases, but the two noun phrases do not form a syntactic constituent. In example (\ref{ex353}), the speaker talks about older lineages of his clan. The example contrasts the \isi{proprietive} and the \isi{privative} \isi{case}. The absence (\emph{ngemär}) or existence (\emph{ngekarä}) marked on \emph{nge} `child' relates those noun phrases to \emph{fi} `they' and \emph{bäi} respectively. In the following examples (\ref{ex354} and \ref{ex355}), the noun phrase to which the privative-marked noun phrase links is omitted.

\begin{exe}
	\ex \emph{\underline{sitau} \underline{bagi} \underline{fi} zabthath \textbf{ngemär} ... \underline{bäinzo} \textbf{ngekarä} yara fof.}\\
	\gll sitau bagi fi za\stem{bth}ath nge=mär (.) bäi=nzo nge=karä ya\stem{r}a fof\\
	sitau bagi \Third.\Abs{} \Stpl:\Sbj:\Pst:\Pfv/finish child=\Priv{} (.) bäi=\Only{} child=\Prop{} \Tsg.\Masc:\Sbj:\Pst:\Ipfv/be \Emph\\ 
	\trans `Sitau and Bagi, they died without children ... only Bäi had children.'\\\Corpus{tci20120814}{ABB \#508}
	\label{ex353}
\end{exe}
\begin{exe}
	\ex \emph{frasi kwa nrä \textbf{ŋanzmäre} fthé gnräré}\\
	\gll frasi kwa n\stem{rä} ŋanz=märe fthé gn\stem{rä}ré\\
	hunger \Fut{} \Ssg:\Sbj:\Nonpast:\Ipfv/be row=\Priv{} when \Ssg:\Sbj:\Imp:\Ipfv/be\\
	\trans `You will be hungry, if you don't have a row (of yams in the garden).'\\\Corpus{tci20130822-08}{JAA \#54}
	\label{ex354}
\end{exe}
\begin{exe}
	\ex \emph{\textbf{miyomäre} worä ... mrn ŋarake \textbf{miyomäre}.}\\
	\gll miyo=märe wo\stem{rä} (.) mrn ŋarake miyo=märe\\
	desire=\Priv{} \Fsg:\Sbj:\Nonpast:\Ipfv/be (.) clan garden desire=\Priv{}\\
	\trans `I don't want to make a family/clan garden (anymore).'\\\Corpus{tci20130823-06}{STK \#77}
	\label{ex355}
\end{exe}

There is one lexeme where the \isi{privative} \isi{case} has fused with a lexical item. The word \emph{miyatha} `knowledge' or `knowledgeable' is used in constructions expressing a positive epistemic state; usually of the structure \emph{miyatha worä} `I know' (Lit. `with knowledge I am' or `knowledgeable I am'). In addition to the \isi{negator} \emph{keke}, one can negate this (`I do not know') by using the word \emph{miyamr} `ignorance' or `ignorant', which contains \emph{miya} and an element \emph{mr}. The latter is a reduced and lexicalised form of the \isi{privative} \isi{case} marker \emph{=mär}. We can see this in example (\ref{ex356}), which comes from a myth where two brothers are trying to kill a creature by shooting an arrow into its heart.
	 
\begin{exe}
	\ex \emph{naf nima ``keke fi \textbf{miyamr} erä fofosa mä rä.''}\\
	\gll naf nima keke fi miyamr e\stem{rä} fofosa mä \stem{rä}\\
	\Tsg.\Erg{} \Quot{} \Neg{} \Third.\Abs{} ignorant \Stpl:\Sbj:\Nonpast:\Ipfv/be heart where \Tsg.\F:\Sbj:\Nonpast:\Ipfv/be\\
	\trans `He said ``No, they do not know where the heart is.'''\\\Corpus{tci20131013-01}{ABB \#104-105}
	\label{ex356}
\end{exe}%356

\section{Associative \emph{=ä}} \label{comcase}

The \isi{associative} \isi{case} is used to express accompaniment at the clausal level or \isi{simultaneity} of another event at the cross-clausal level. In both cases, the formative is \emph{=ä}. With animates, there are two formatives \emph{=r} and \emph{=ä}, and a set of pronominals (\tabref{comcase-table} in \S{}\ref{inclusorycontruction}). These are used for a special construction for which I adopt the term ``\isi{inclusory} construction'' based on (\citealt{Lichtenberk:2000hr}) and (\citealt{Singer:inclu}). I describe the \isi{inclusory} construction in the context of the syntax of the \isi{noun phrase} (see \S{}\ref{inclusorycontruction}).\\  

The \isi{associative} \isi{case} on \isi{inanimate} referents is a minor pattern, because it overlaps in its semantics with the \isi{proprietive} \isi{case} \emph{=karä} (\S{}\ref{propcase}). It may operates at the clausal level (\ref{ex337}) or at the cross-clausal level (\ref{ex336}). Example (\ref{ex336}) is taken from a storyboard picture task where the speaker describes one of the pictures as part of a narration. The \isi{associative} is attached to the nominalised verb \emph{thweksi} `rejoice' which acts as an infinitival adjunct. 

\begin{exe}
	\ex \emph{kfänrsöfth \textbf{thweksiä}.}\\
	\gll kfän\stem{rsöfth} thwek-si=ä\\
	\Stsg:\Sbj:\Pst:\Iter:\Venit/descend rejoice-\Nmlz=\Assoc{}\\
	\trans `She always came down (the stairs) and was happy.'\Corpus{tci20120925}{MKA \#369}
	\label{ex336}
\end{exe}

Example (\ref{ex337}) is taken from a story about a boy who drowned in the Morehead river. A group of policemen were on guard to deter crocodiles, while another man was trying to recover the body from the river. The phrase \emph{markai nabiä} `with shotguns' (literally: `with white man bows') can also be marked with the \isi{proprietive} \isi{case} like the preceding phrase \emph{gardakarä} `with canoes' (\ref{ex336}).

\begin{exe}
	\ex \emph{fath wäfiyokwath neba wazi neba wazi ... frisman fi gardakarä \textbf{markai nabiä} ... bara kwarafinzrmth ... nümgarma}\\
	\gll fath wä\stem{fiyok}wath neba wazi neba wazi (.) frisman fi garda=karä markai nabi=ä (.) bara kwa\stem{rafi}nzrmth (.) nümgar=ma\\
	clearing \Stpl:\Sbj>\Tsg.\F:\Obj:\Nonpast:\Ipfv/make opposite side opposite side (.) policeman \Third.\Abs{} canoe=\Prop{} {white.man} bow=\Assoc{} (.) paddle \Stpl:\Sbj:\Pst:\Dur/paddle (.) crocodile=\Char{}\\
	\trans `The cleared the place along both sides ... the policemen with canoes and shotguns ... they were paddling because of crocodiles.'\\\Corpus{tci20120904-02}{MAB \#162-165}
	\label{ex337}
\end{exe}

The third example (\ref{ex341}) comes from visiting one of the many waterholes around Rouku, where people catch fish with poison-root during the dry season. The speaker points out how thoughtfully (`with thoughts') the ancestors looked after this place.

\begin{exe}
	\ex \emph{kofä kwot kwarkonzrmth namä yamme ... nä kafar zra zane zf \textbf{famä} zumarwrmth nafa zf ... kafar kwarké.}\\
	\gll kofä kwot kwa\stem{rko}nzrmth namä yam=me (.) nä kafar zra zane zf fam=ä zu\stem{mar}wrmth nafa zf (.) kafar kwark=é\\
	fish properly \Stpl:\Sbj:\Pst:\Dur/distribute good custom=\Ins{} (.) some big swamp \Dem:\Prox{} \Imm{} thought=\Assoc{} \Stpl:\Sbj>\Tsg.\F:\Obj:\Pst:\Dur/see \Tnsg.\Erg{} \Imm{} (.) big deceased=\Erg.\Nsg{}\\
	\trans `They shared the fish in a good way. They looked after this swamp here thoughtfully ... the late elders.'\Corpus{tci20120922-21}{DAK \#37-38}
	\label{ex341}
\end{exe}

\section{Similative \emph{=thatha}} \label{similcase}

The \isi{similative} \isi{case} functions at the clause level, and its semantics are quite compatible to the \ili{English} expressions `like X' or `similar to X'. In example (\ref{ex417})\footnote{The word \emph{pike} [pıke] comes from Wrigley's PK\textsuperscript{\textregistered} chewing gum which has the initials of Philip Knight Wrigley printed in big letters on the package.}, the speaker shows me a plant called \emph{ŋaziŋazi} `Exocarpus sp' and comments that its fruits taste a bit like a chewing gum and that it is similar to \emph{ŋazi} `coconut'.

\begin{exe}
	\ex \emph{ŋaziŋazi ... \textbf{pikethatha} yé ... \textbf{ŋazithatha} ... nafane yawi.}\\
	\gll {ŋaziŋazi} (.) pike=thatha \stem{yé} (.) ŋazi=thatha (.) nafane yawi\\
	ŋaziŋazi (.) chewing.gum=\Simil{} \Tsg.\Masc:\Sbj:\Nonpast:\Ipfv/be (.) coconut=\Simil{} (.) \Tsg.\Poss{} fruit\\ 
	\trans `\emph{ŋaziŋazi} ... its fruit is like a chewing gum ... like a coconut.'\\\Corpus{tci20130907-02}{RNA \#308-309}
	\label{ex417}
\end{exe}

Hence, the element marked with \emph{=thatha} is portrayed as being similar to another element. Often enough that second element is established from context and the respective noun phrase is omitted as in example (\ref{ex418}) where the speaker describes an man hanging upside down from the branch of a tree.

\begin{exe}
	\ex \emph{\textbf{bidrthatha} zbo sumithgrm ... wämnen.}\\
	\gll bidr=thatha zbo su\stem{mi}thgrm (.) wämne=n\\
	{{flying.fox}=\Simil{}} {\Prox.\All{}} \Tsg.\Masc:\Sbj:\Pst:\Dur:\Stat/be.hanging (.) tree=\Loc\\
	\trans `He was hanging like a flying fox ... on the tree.'\Corpus{tci20130901-04}{RNA \#48}
	\label{ex418}
\end{exe}

There are a few cases where the \isi{similative} \isi{case} is attached to \isi{recognitional} \isi{pronoun} \emph{bänethatha} `like that one' or to the manner \isi{demonstrative} \emph{nimathatha} `like in this way' as in example (\ref{ex416}), where the speaker comments that some plants along the way look as if they had been planted by someone.

\begin{exe}
	\ex \emph{\textbf{nimathatha} erä ... kma thuworthrth.}\\
	\gll nima=thatha e\stem{rä} (.) kma thu\stem{wor}thrth\\
	{like.this}=\Simil{} \Stpl:\Sbj:\Nonpast:\Ipfv/be (.) \Pot{} \Stpl:\Sbj>\Stpl:\Obj:\Rpst:\Ipfv/plant\\ 
	\trans `These (plants) look a bit like ... as if they have planted them.'\\\Corpus{tci20130907-02}{JAA \#281}
	\label{ex416}
\end{exe}

\section{Further nominal morphology}\label{furthernommorph}

This section describes a number of \isi{nominal} enclitics or suffixes that do not mark a semantic role.

\subsection{Emphatic \emph{=wä}} \label{emphathicwae}

The \isi{emphatic} \isi{enclitic} \emph{=wä} is used to intensify its host. For example, attached to a \isi{temporal} adjective \emph{zafe} `old', it means `really long ago' (\ref{ex471}). If it is attached to a \isi{possessive} \isi{pronoun}, it is often translated as `my own' instead of `my' (\ref{ex472}). As Komnzo has no dedicated marker for comparatives, the \isi{emphatic} \isi{enclitic} can be used for this (\ref{ex473}). 

\begin{exe}
	\ex \emph{nze kwa natrikwé bun ... no kzima ... \textbf{zaföwä} ni monme no kzi thwafiyokwrme.}\\
	\gll nze kwa na\stem{trik}wé bun (.) no kzi=ma (.) zafe=wä ni mon=me no kzi thwa\stem{fiyok}wrme\\
	\Fsg.\Erg{} \Fut{} \Fsg:\Sbj>\Ssg:\Io:\Nonpast:\Ipfv/tell \Ssg.\Dat{} (.) rain barktray=\Char{} (.) old=\Emph{} \Fnsg{} how=\Ins{} rain barktray \Fnsg:\Sbj>\Stpl:\Obj:\Pst:\Dur/make\\
	\trans `I will tell you ... about the rain-making barktray ... a really long time ago ... how we were making the rain-making barktray.'\Corpus{tci20110810-01}{MAB \#1-3}
	\label{ex471}
\end{exe}
\begin{exe}
	\ex \emph{\textbf{nzonewä} zane zf erä!}\\
	\gll nzone=wä zane zf e\stem{rä}\\
	\Fsg.\Poss=\Emph{} \Dem:\Prox{} \Imm{} \Stpl:\Sbj:\Nonpast:\Ipfv/be\\
	\trans `These ones right here are my own!'\Corpus{tci20121001}{ABB \#129}
	\label{ex472}
\end{exe}
\begin{exe}
	\ex \emph{\textbf{katakatanwä} thfrä. nzenme kafar erä.}\\
	\gll kata-katan=wä thf\stem{rä} nzenme kafar e\stem{rä}\\
	\Redup-small=\Emph{} \Stpl:\Sbj:\Rpst:\Ipfv/be \Fnsg.\Poss{} big \Stpl:\Sbj:\Nonpast:\Ipfv/be\\
	\trans `Those (yams) were smaller. Ours are big.'\Corpus{tci20120805-01}{ABB \#403}
	\label{ex473}
\end{exe}

Some words seem to have lexicalised the \isi{emphatic} \isi{enclitic}, i.e. they do not occur without \emph{=wä}. One example is \emph{nzüthamöwä} `time' (in the sense `instance of something happening'). This word can take the \emph{=nzo} `only' \isi{enclitic}, for example \emph{näbi nzüthamöwänzo} `only one time'. Elsewhere, the \isi{emphatic} \isi{enclitic} \emph{=wä} and the \isi{exclusive} \isi{enclitic} \emph{=nzo} may not co-occur. Other examples are \emph{bramöwä} `all' and \emph{gadmöwä} `good fortune'. Note that all three contain a /mö/ element. I suspect that this is a lexicalised version of the instrumental \isi{case} marker \emph{=me}. The vowel of the instrumental \emph{=me} is regularly rounded in the presence of \emph{=wä}. However, removing these putative lexicalised enclitics from these words results in three non-words: \textsuperscript{$\ast$}\emph{nzütha}, \textsuperscript{$\ast$}\emph{bra} and \textsuperscript{$\ast$}\emph{gad}.\\
	
The \isi{emphatic} \isi{enclitic} can attach to lexical items preceding the \isi{case} marker. Example (\ref{ex474}) is from a story about two characters who each have a ford in the river where they place a fishing basket. In \emph{edawäneme}, the \isi{enclitic} has scope over the numeral \emph{eda} `two'. Thus, it is emphasizing the fact that there are two, which suggests a distributive reading: `each one had a trapping place'. If the \isi{enclitic} was attached after the \isi{case} marker (\emph{edaanemöwä}), the possession would be emphasised `two of their own'. Example (\ref{ex474}) is the only instance in the corpus where the \isi{emphatic} \isi{enclitic} occurs between a lexical item and a \isi{case} marker. Hence, it is a possible yet very rare construction. 
 
\begin{exe}
	\ex \emph{krsi zn we fä thwarnm ... \textbf{edawäneme}.}\\
	\gll kr-si zn we fä thwa\stem{rn}m (.) eda=wä=aneme\\
	block-\Nmlz{} place also \Dist{} \Stdu:\Io:\Pst:\Dur/be (.) two=\Emph=\Poss.\Nsg{}\\
	\trans `They also had a fishing place there ... each had one.'\\\Corpus{tci20110802}{ABB \#58-59}
	\label{ex474}
\end{exe}

\subsection{Exclusive \emph{=nzo}} \label{exclusivenzo}

The \isi{exclusive} \isi{enclitic} \emph{=nzo} has been described in \S{}\ref{clitics}. It forms the \isi{nominal} counterpart to the discourse particle \emph{komnzo} `only' (\S{}\ref{discourse-particles}) from which the language gets its name. The \isi{exclusive} \isi{enclitic} can attach to all \isi{nominal}s including pronouns, thus it occurs with a high frequency in the corpus. It usually attaches to the last element of the noun phrase over which it has scope. It is glossed as \Only{} in the examples.%\\

In example (\ref{ex475}) the \isi{exclusive} \isi{clitic} attaches to a noun phrase with an adverbial function, \emph{frme} `straight'. In (\ref{ex476}), it is attached to an adjective.

\begin{exe}
	\ex \emph{zokwasi mane rera komnzo \textbf{frmenzo} wyaka nzudbo.}\\
	\gll zokwasi mane re\stem{r}a komnzo fr=me=nzo w\stem{yak}a nzudbo\\
	speech which \Tsg.\F:\Sbj:\Pst:\Ipfv/be only line=\Ins=\Only{} \Tsg.\F:\Sbj:\Pst:\Ipfv/walk \Fsg.\All{}\\
	\trans `As for the message, it just came straight to me.'\Corpus{tci20120814}{ABB \#50-51}
	\label{ex475}
\end{exe}
\begin{exe}
	\ex \emph{zasath ``bä \textbf{namänzo} nrä?'' ``keke nzä nimäwä worä.''}\\
	\gll za\stem{s}ath bä namä=nzo n\stem{rä} keke nzä nima=wä wo\stem{rä}\\
	\Stdu:\Sbj:\Pst:\Pfv/ask \Second.\Abs{} good=\Only{} \Ssg:\Sbj:\Nonpast:\Ipfv/be \Neg{} \Fsg.\Abs{} {like.this}=\Emph{} \Fsg:\Sbj:\Nonpast:\Ipfv/be\\
	\trans `They asked each other: ``Are you alright?'' ``No, I am like this.'''\\\Corpus{tci20120827-03}{KUT \#159}
	\label{ex476}
\end{exe}

\subsection{Etcetera \emph{=sü}} \label{etceterasue}

The \isi{enclitic} \emph{=sü} only attaches to either the \isi{associative} or the \isi{proprietive} \isi{case} marker. It is often translated as ``and all'' by my informants. Consider example (\ref{ex477}), in which a speaker reports how he and some of his brothers transported a heavy sago stem with a couple of canoes. The \emph{=sü} \isi{enclitic} expresses that there are more items than just the sago. Therefore, I label \emph{=sü} as \isi{etcetera} marker, and I \isi{gloss} it with \Etc.

\begin{exe}
	\ex \emph{masenf fä fof nzräs ``kwa känthfe \textbf{bikaräsü} zbo!'' ... watik \textbf{bikaräsü} ŋarafinzake.}\\
	\gll masen=f fä fof nzrä\stem{s} kwa kän\stem{thf}e bi=karä=sü zbo (.) watik bi=karä=sü ŋa\stem{rafi}nzake\\
	masen=\Erg{} \Dist{} \Emph{} \Stsg:\Sbj>\Fpl:\Obj:\Irr:\Pfv/call \Fut{} \Spl:\Sbj:\Imp:\Pfv:\Venit/walk sago=\Prop=\Etc{} \Prox.\All{} (.) then sago=\Prop=\Etc{} \Fpl:\Sbj:\Pst:\Ipfv/paddle\\ 
	\trans `Masen called out to us: ``Come over here with the sago and all!'' ... Then, we paddled with the sago and everything.'\Corpus{tci20120929-02}{SIK \#41-42}
	\label{ex477}
\end{exe}

Example (\ref{ex478}) show the \isi{etcetera} \isi{enclitic} attached to the \isi{associative} \isi{case} in an inclusory construction. The speaker describes how his friends slept in a camp where his father and other relatives were staying.

\begin{exe}
	\ex \emph{ni \textbf{ŋafyäsü} fä fof nrugra.}\\
	\gll ni ŋafe=ä=sü fä fof n\stem{rugr}a\\
	\Fnsg{} father=\Assoc.\Pl=\Etc{} \Dist{} \Emph{} \Fpl:\Sbj:\Pst:\Ipfv/sleep\\
	\trans `We slept there with father and all the others.'\Corpus{tci20110810-02}{MAB \#11}
	\label{ex478}
\end{exe}
	
Example (\ref{ex479}) is taken from an origin myth in which two brothers are fighting with a creature. One of them warns his brother that he will shoot the creature now and he should be prepared. Hence, the second clause literally translates as ``you must be with thoughts and all''.

\begin{exe}
	\ex \emph{watik ngth biruthé! \textbf{famkaräsü} gnräré!}\\
	\gll watik ngth b=y\stem{ru}thé fam=karä=sü gn\stem{rä}ré\\
	then {younger sibling} \Med=\Fsg:\Sbj>\Tsg.\Masc:\Obj:\Nonpast:\Ipfv/shoot thought=\Prop=\Etc{} \Ssg:\Sbj:\Imp:\Ipfv/be\\
	\trans ``Okay brother, I will shoot him now. You have to think and be prepared!''\\\Corpus{tci20131013-01}{ABB \#108-109}
	\label{ex479}
\end{exe}

\subsection{Distributive \emph{-kak}} \label{distributivekak}

I analyze the \isi{distributive} marker \emph{-kak} as a suffix rather than an \isi{enclitic} because it does not operate on the level of the phrase. It can only be suffixed to numerals and some quantifiers. Its meaning can be translated to \ili{English} with `each' or `individually'. The \isi{distributive} is often followed by the instrumental as in (\ref{ex481}). In this example, the speaker had lost his dogs during hunting. The \isi{distributive} highlights that the dogs came back individually.

\begin{exe}
	\ex \emph{ŋatha katakatan thunthorakwrm \textbf{näbikakme}.}\\
	\gll ŋatha kata-katan thun\stem{thorak}wrm näbi-kak=me\\
	dog \Redup-small \Stpl:\Sbj:\Pst:\Dur:\Venit/arrive one-\Distr=\Ins\\
	\trans `The small ones were arriving one by one.'\Corpus{tci20111119-03}{ABB \#69}
	\label{ex481}
\end{exe}

In example (\ref{ex480}), a woman has finished presenting to me what she has caught during the day. This includes different fish, a goanna and a turtle. She concludes with the words ``There is plenty of meat''. This could be translated as \emph{faso tüfr erä} without the \isi{distributive}. The \isi{distributive} in (\ref{ex480}) expresses that she has caught different kinds of meat.

\begin{exe}
	\ex \emph{watik, faso \textbf{tüfrkak} erä.}\\
	\gll watik faso tüfr-kak e\stem{rä}\\
	then meat plenty-\Distr{} \Stpl:\Sbj:\Nonpast:\Ipfv/be\\
	\trans `Well, there is plenty of different meat.'\Corpus{tci20120821-01}{LNA \#68}
	\label{ex480}
\end{exe}
	 
In example (\ref{ex482}), the speaker tells me about different types of bows. He concludes by pointing out that different people like different types.
	
\begin{exe}
	\ex \emph{zawe \textbf{ffrükakmenzo} erä}\\
	\gll zawe f-frü-kak=me=nzo e\stem{rä}\\
	talent \Redup-alone-\Distr=\Ins=\Only{} \Stpl:\Sbj:\Nonpast:\Ipfv/be\\
	\trans `People have different preferences.' (Lit. `There are different individual talents.')\Corpus{tci20120922-23}{MAA \#104}
	\label{ex482}
\end{exe}

\subsection{Diminuitive \emph{fäth}}\label{diminuitivefaeth}

I take the \isi{diminuitive} \emph{fäth} `small one' as an example to describe a small group of lexemes which behave similar to the enclitics described above. However, I do not analyse them as enclitics but rather as lexemes on the verge to becoming grammaticalised. The two main reasons are: (i) they often occur by themselves without a host, and (ii) they have a more lexical meaning. Out of the four lexemes, two have to do with location: \emph{zn} and \emph{faf}, both mean `place', and two have to do with smallness or compactness: \emph{fäth} `small one' (glossed as \Dim) and \emph{fur} `bundle'.%\\

Example (\ref{ex491}) illustrates that \emph{fäth} can occur as a free lexeme. However, \emph{fäth} frequently occurs after a noun, as in (\ref{ex492}) and (\ref{ex493}). We could analyse \emph{fäth} in (\ref{ex492}) either as a compound of two nouns (`story' + `small one'), or as a \isi{diminuitive} \isi{enclitic} which has scope over a preceding host. The latter analysis is supported by the fact that the two elements form an intonational unit.

\begin{exe}
	\ex \emph{nzone ŋafe fthé fof \textbf{katan fäth} sfrärm.}\\
	\gll nzone ŋafe fthé fof katan fäth sf\stem{rä}rm\\
	\Fsg.\Poss{} father when \Emph{} small \Dim{} \Tsg.\Masc:\Sbj:\Pst:\Dur/be\\
	\trans `My father was a small boy at that time.'\Corpus{tci20111107-01}{MAK \#34}
	\label{ex491}
\end{exe}
\begin{exe}
	\ex \emph{\textbf{trikasi fäth} fobo fof zwaythik fof.}\\
	\gll trik-si fäth fobo fof zwa\stem{ythik} fof\\
	tell-\Nmlz{} \Dim{} \Dist.\All{} \Emph{} \Tsg.\F:\Sbj:\Rpst:\Ipfv/come.to.end \Emph\\
	\trans `There, the small story comes to an end.'\Corpus{tci20111119-03}{ABB \#197}
	\label{ex492}
\end{exe}

If there is a case marker present, it will attach to \emph{fäth} (\ref{ex493}); a fact which supports both analyses.

\begin{exe}
	\ex \emph{\textbf{emoth fäthnm} thrätrif.}\\
	\gll emoth fäth=nm thrä\stem{trif}\\
	girl \Dim=\Dat.\Nsg{} \Stsg:\Sbj>\Stpl:\Io:\Rpst:\Pfv/tell\\
	\trans `He told the small girls.'\Corpus{tci20120901-01}{MAK \#181}
	\label{ex493}
\end{exe}

On the basis of the arguments above, I decide to treat \emph{fäth} as an independent lexeme. The same applies to \emph{zn}, \emph{faf} (both `place') and \emph{fur} (`bundle'). I analyse them as lexemes which are on the verge of becoming grammaticalised. Note that only for \emph{fäth} I employ the \isi{gloss} \Dim{} instead of a more lexical one (`small one').

\section{A few historical notes}\label{casenotes}

The \isi{case} markers presented in this chapter show some semantic and formal overlaps which invite speculations at to their emergence. I want to lay out some hypotheses here. My main point is that the \isi{dative} and the \isi{possessive} are historically related, and that the original form played some role in marking animacy.%\\

In \tabref{casediscussionss}, we can see a subset of the personal pronouns for different cases and the respective \isi{case} enclitics for \isi{animate} referents. Note that only the first and second \isi{person} is shown. The third \isi{person} forms are not relevant for the argument advanced here. For reasons of comparison, the table includes the \isi{possessive} prefixes, even though they are not \isi{case} markers.

\begin{table}
\begin{center}
\caption{Case marking with animate referents} 
\label{casediscussionss} 
	\begin{tabularx}{\textwidth}{Xllllll}	
		\lsptoprule
		&\multicolumn{4}{c}{{personal pronouns}}&\multicolumn{2}{c}{{case enclitics}}\\
		&\Fsg{}&\Fnsg{}&\Ssg{}&\Snsg{}&\Sg{}&\Nsg{}\\ \midrule
		\Char{}&\emph{nzonema}&\emph{nzenmema}&\emph{bonema}&\emph{benmema}&\emph{=anema}&\emph{=anemema}\\
		\Poss{}&\emph{nzone}&\emph{nzenme}&\emph{bone}&\emph{benme}&\emph{=ane}&\emph{=aneme}\\ 
		\Poss-&\emph{nzu-}&\emph{nze-}&\emph{bu-}&\emph{be-}&n/a&n/a\\ 
		\Dat{}&\emph{nzun}&\emph{nzenm}&\emph{bun}&\emph{benm}&\emph{=n}&\emph{=nm}\\ 
		\Loc{}&\emph{nzudben}&\emph{nzedben}&\emph{budben}&\emph{bedben}&\emph{=dben}&=(\emph{n})\emph{medben}\\ 
		\All{}&\emph{nzudbo}&\emph{nzedbo}&\emph{budbo}&\emph{bedbo}&\emph{=dbo}&=(\emph{n})\emph{medbo}\\ 
		\Abl{}&\emph{nzudba}&\emph{nzedba}&\emph{budba}&\emph{bedba}&\emph{=dba}&=(\emph{n})\emph{medba}\\ 
		\lspbottomrule
	\end{tabularx}
\end{center}
\end{table}%Case marking with \isi{animate} referents

One observation from the table is that the \isi{characteristic} pronouns are built from the \isi{possessive} pronouns, for example the first singular \isi{possessive} \emph{nzone} `my' is used to express the meaning `because of me' by simply attaching the \isi{characteristic} \isi{case} marker \emph{=ma}. In fact, the pattern is so transparent that instead of analysing a form like \emph{nzonema} as \Fsg.\Char{} an alternative analysis would be to analyse it in a more compositional way: \emph{nzone=ma} \Fsg.\Poss=\Char{}. This also holds true for nouns. Note that the use of the \isi{possessive} is only required for \isi{animate} referents. For example, \emph{no=ma} `because of the rain' can do without the \isi{possessive}, but \textsuperscript{$\ast$}\emph{kabe=ma} `because of the man' is ungrammatical, and it has to be \emph{kabe=ane=ma} (man=\Poss.\Sg=\Char). Hence, the \isi{possessive} functions as a marker of animacy. I want to argue that in the other \isi{case} formatives, we find frozen morphology that points to a similar strategy.%\\

A second observation from the table lies in the formal similarity of the \isi{possessive} and the \isi{dative} \isi{case} enclitics. The \isi{dative} formatives resemble the \isi{possessive} ones, but they lack the vowels: \emph{=ane} (\Poss) vs. \emph{=n} (\Dat), and \emph{=aneme} (\Poss.\Nsg) vs. \emph{=nm} (\Dat.\Nsg). Furthermore, the table shows that all \isi{case} enclitics share an element marking \isi{non-singular} \isi{number}. This is /m/ for the \isi{dative} and /me/ for all other cases. Again, we may analyse this element as a separate morpheme, for example \emph{=ane=me} (=\Poss=\Nsg{}) and \emph{=n=m} (=\Dat=\Nsg{}). In the remainder of this section, I want to argue for three points: (i) that these the \isi{possessive} and the \isi{dative} have developed from the same source, (ii) that the function of that source was to mark animacy, and (iii) that the source itself was segmentable into one morpheme marking animacy (\emph{=n} or \emph{=ane}) and a second morpheme marking \isi{non-singular} \isi{number} (\emph{=m} or \emph{=me}).%\\

The main point of evidence comes from a variant of the \isi{non-singular} formatives of the spatial cases. For example, the \isi{locative} can be \emph{=medben}, but there is a variant \emph{=nmedben}. The latter includes an /n/ which is also found in the \isi{possessive} and the \isi{dative} enclitics. Note that for the \isi{possessive} and the \isi{dative}, /n/ is found in the \isi{singular} and the \isi{non-singular} formatives. For \isi{non-singular} formatives of the three spatial cases, I want to argue that the /n/-variant is the older one. Note that the /n/ element is also present in the \isi{singular} formatives of the three spatial cases, but it is difficult to recognize it as a segment, because all three \isi{case} enclitics begin with a prenasalised alveolar plosive [\textsuperscript{n}d]. Therefore, I want to suggest a more transparent analysis:

\begin{table}
\begin{center}
\caption{Revised analysis of case markers for animate referents} 
\label{casediscussionshist} 
	\begin{tabularx}{\textwidth}{XXll}	
		\lsptoprule
		&=\Anim={case}&=\Anim=\Nsg{}={case}\\ \midrule
		\Poss{}&\emph{=ane}&\emph{=ane=me}\\ 
		\Dat{}&\emph{=n}&\emph{=n=m}\\ 
		\Loc{}&\emph{=n=dben}&(\emph{=n})\emph{=me=dben}\\ 
		\All{}&\emph{=n=dbo}&(\emph{=n})\emph{=me=dbo}\\ 
		\Abl{}&\emph{=n=dba}&(\emph{=n})\emph{=me=dba}\\ 
		\lspbottomrule
	\end{tabularx}
\end{center}
\end{table}%Revised analysis of case markers for \isi{animate} referents

The revised analysis in \tabref{casediscussionshist} suggests that the spatial case enclitics attached to an /n/  formative which I suggest is a marker of animacy. Moreover, there is the /me/ formative for marking \isi{non-singular} \isi{number}.%\\

This analysis rests on the assumption that the \isi{dative} and the \isi{possessive} are historically related. I want to draw on four points of evidence to support this claim. First, the enclitics of the two cases are similar, if we assume that the \isi{dative} formatives once had vowels: \emph{=ane} > \emph{=n} and \emph{=aneme} > \emph{=nm}. Secondly, the close \isi{possessive} prefixes in \tabref{casediscussionss} show that the vowel in the singular prefixes groups them with the \isi{dative}, not with the \isi{possessive}. The first \isi{person} close \isi{possessive} prefix is \emph{nzu-} like the first \isi{person} \isi{dative} \isi{pronoun} \emph{nzun}, whereas the first \isi{person} \isi{possessive} \isi{pronoun} is \emph{nzone}. Thirdly, the argumentation in the preceding paragraph shows that the element \emph{=nme}, which precedes the spatial \isi{case} markers, is historically related to both the \isi{possessive} and the \isi{dative}. The fourth piece of evidence comes from a comparison with \ili{Ngkolmpu}, a related \ili{Tonda} language spoken in Indonesia. In \ili{Ngkolmpu}, the \isi{dative} marks the \isi{possessor} role in its adnominal function (\citealt{Carroll:Ngkolmpu}).%\\

This leaves us wondering about the p\ili{Tonda} or pYam system and the path of grammaticalisation in Komnzo. The scenario sketched out above suggests that the original system was more like \ili{Ngkolmpu} where one \isi{case} marker serves both functions, \isi{dative} and \isi{possessive}. Alternatively, the predecessor could have had a much more general function. I have argued above that this functions was to mark animacy. Although speculative at present, I want to point out that a possible source of the animacy marker could be the \isi{anaphoric} \isi{demonstrative} \emph{ane}, which can occur in postposed position. For the moment, we can only speculate on the path of grammaticalisation. More data from the other \ili{Yam languages} is needed to settle this question.