%!TEX root = ../main.tex

\chapter{Clausal syntax} \label{cha:clausalsyntax}

\section{Introduction}

This chapter addresses the syntax within simple clauses. In Komnzo, a large part of the argument structure is encoded in the \isi{verb} morphology. This is described in \S{}\ref{alignmtemplates}, and summarised in Table \ref{argalignverbs}. Therefore, the following description of \isi{clause} types is brief for those types which have been addressed before, but more detailed for other types where the \isi{verb} morphology plays a smaller role.

\section{Constituent order}\label{constitorder}

The dominant word order in Komnzo is AUV (actor \isi{undergoer} verb). Recipients of ditransitives also precede the verb and follow the actor noun phrase, but there is no clear position with respect to the \isi{theme} argument. Evidence for basic word order comes from the use of the \isi{recognitional} \isi{demonstrative} (see \S{}\ref{recognitional-pronoun}). In example (\ref{ex686}), the \isi{object} argument is expressed first by the \isi{recognitional} \emph{bäne} `those' and then by the noun \emph{züm} `centipedes'. The speaker uses the \isi{recognitional} in absolutive case in the position where the constituent normally occurs. This is a tip-of-the-tongue situation, and therefore the speaker fills in the appropriate referent after the verb. Note that there is usually a break in the intonation contour if any constituent occurs after the verb.

\begin{exe}
	\ex \emph{nzürna ŋaref \textbf{bäne} sasryoftha \textbf{züm}.}\\
	\gll nzürna ŋare=f bäne sa\stem{sryofth}a züm\\
	nzürna woman=\Erg{} \Recog.\Abs{} \Sg:\Sbj>\Tsg.\Masc:\Io:\Pst:\Pfv/send centipede\\
	\trans `The \emph{nzürna} woman sent those ones after him ... the centipedes.'\\\Corpus{tci20120827-03}{KUT \#138}
	\label{ex686}
\end{exe}

Experiencer-object constructions (\S{}\ref{expobjconstr}) deviate from the basic word order. The \isi{experiencer} is placed almost always before the \isi{stimulus}, i.e. the \isi{undergoer} comes first and the actor follows (\ref{ex687}). This can be explained by the relative salience of the \isi{experiencer} in such constructions and the fact that it almost always ranks higher in terms of animacy.

\begin{exe}
	\ex \emph{ŋatha kawakawaf bthefaf.}\\
	\gll ŋatha kawakawa=f b=the\stem{faf}\\
	dog madness=\Erg{} \Med=\Stsg:\Sbj>\Stpl:\Obj:\Rpst:\Pfv/hold\\
	\trans `The dogs went crazy there.' (Lit. `Madness has grabbed the dogs.')\\\Corpus{tci20130907-02}{JAA \#488}
	\label{ex687}
\end{exe}

AUV word order is only a tendency in Komnzo. In fact, most clauses lack overt \isi{noun phrase}s for the respective constituents. The flagging of \isi{noun phrase}s with \isi{case} allows for some flexibility in the arrangement of constituents. However, deviations from the basic word order are often pragmatically motivated. In example (\ref{ex716})\footnote{Note that the stem \emph{fath-} means `hold', but in a suppressed-object construction it means `marry' (see \S{}\ref{suppressedobjectclause}).}, the speaker replies to a question whether a particular individual is his brother in-law. He says `really my brother in-law' and then gives an explanation in the following clause, where the \isi{undergoer} appears before the actor. The reversal of constituents can be explained as a strategy to focus the \isi{undergoer} argument, that is \emph{mayawa emoth} `Mayawa sister' is focussed by fronting.

\begin{exe}
	\ex \emph{nzone ngom fof ... \textbf{mayawa emoth} naf zefafa fof.}\\
	\gll nzone ngom fof (.) mayawa emoth naf ze\stem{faf}a fof\\
	\Fsg.\Poss{} brother.in.law \Emph{} (.) mayawa girl \Tsg.\Erg{} \Sg:\Sbj:\Pst.\Pfv/marry \Emph\\
	\trans `My brother in-law ... He married a Mayawa sister.'\\\Corpus{tci20120814}{ABB \#391-392}
	\label{ex716}
\end{exe}

In example (\ref{ex688}), both constituents follow the verb. The \isi{undergoer} comes first and after a short pause the actor follows. Examples like these are rare, but frequently one of the constituents follows the verb. This can occur because the speaker wants to clarify the state of affairs or because she wants to put emphasis on the referent. There is usually a break in the intonation contour after the verb form.

\begin{exe}
	\ex \emph{keke thufnzrm ane karma kabe ... naf.}\\
	\gll keke thu\stem{fn}nzrm ane kar=ma kabe (.) naf\\
	\Neg{} \Sg:\Sbj>\Stpl:\Obj:\Pst:\Dur/kill \Dem{} village=\Char{} man (.) \Tsg.\Erg\\
	\trans `She did not attack those village people.'\Corpus{tci20120901-01}{MAK \#50}
	\label{ex688}
\end{exe}

While the order of constituents is flexible to some extent, it rarely occurs that other elements like adverbs, TAM particles or the \isi{negator} follow the verb. Komnzo supports a number of cross-linguistic generalisations found in verb final languages (\citealt{Dryer:2007wordorder}), for example that the \isi{possessor} precedes the \isi{possessed}. A second generalisation is that verb-final languages tend to have postpositions rather than prepositions. Komnzo does not have a category of adpositions, but \isi{locational} nouns like \emph{tharthar} `side' or \emph{mrmr} `inside' always follow the noun whose location they specify (see \S{}\ref{locationals}).

\section{Clause types}\label{clause types}

\subsection{Non-verbal clauses}\label{nonverbalclauses}

Non-verbal clauses are a marginal phenomenon in Komnzo. This section describes the few types of verbless clauses. These are usually short one or two word utterances including an element which has some verb-like semantics, for example TAM particles or property nouns.

The TAM particles \emph{kwa} \Fut{} and \emph{kma} \Pot{} can stand alone, when they are used as commands. For example, \emph{kma} can mean `You have to!', and with the \isi{apprehensive} clitic \emph{m} attached, it can mean the opposite: \emph{kmam} `You must not!'. In example (\ref{ex560}), the \isi{future} particle \emph{kwa} is used in the sense of `Wait!'. The speaker describes poison-root fishing and how they have to hold back the children from jumping into the water too early.

\begin{exe}
 	\ex \emph{katakatan kwa zöbthé thrängathinzth nima ``\textbf{kwa}! komnzo \textbf{kwa}!''}\\
 	\gll kata-katan kwa zöbthé thrän\stem{gathinz}th nima kwa komnzo kwa\\
	\Redup-small \Fut{} first \Stpl:\Sbj>\Stpl:\Obj:\Irr:\Pfv:\Venit/stop \Quot{} \Fut{} only \Fut{}\\
 	\trans `First, they will hold back the small ones and say: ``Wait! Just Wait!'''\\\Corpus{tci20110813-09}{DAK \#25}
 	\label{ex560}
\end{exe}

Another possible type of verbless \isi{clause} is with the property nouns \emph{miyo} `desire' and \emph{miyatha} `knowledge' and their antonyms \emph{miyomär} `aversion, dislike' and \emph{miyamr} `ignorance'. These words are usually used as \isi{nominal} predicates with light verbs or with the \isi{copula}. As a consequence, we find examples like ({\ref{ex561}}), where the last \isi{clause} \emph{nzä miyamr} does not contain a verb. It is possible to insert the \isi{copula} in the appropriate inflection (\emph{worera} \Fsg:\Sbj:\Pst:\Ipfv/be), but often it is left out. Apart from examples like these, there are no verbless clauses in Komnzo.

\begin{exe}
	\ex \emph{fi kafar mane erera näbi ane ofe ŋarerath. mobo erera? ... \textbf{nzä miyamr}}\\
	\gll fi kafar mane e\stem{rä}ra näbi ane ofe ŋa\stem{rä}rath. mobo e\stem{rä}ra (.) nzä miyamr\\
	but big which \Stpl:\Sbj:\Pst:\Ipfv/be one \Dem{} disappearance \Stpl:\Sbj:\Pst:\Ipfv/do where.\All{} (.)  \Stpl:\Sbj:\Pst:\Ipfv/be \Fsg.\Abs{} ignorance\\
	\trans `As for the big dogs, they disappeared for good. Where did they go? ... I (do) not know.'\Corpus{tci20111119-03}{ABB \#70-72}
	\label{ex561}
\end{exe}

\subsection{Copula clauses}\label{copclause}

Copula clauses are a subtype of \isi{non-verbal} \isi{predication}. They are described here in a separate subsection because the \isi{copula} shows a number of idiosyncrasies. First, the \isi{copula} has no \isi{restricted stem}. Note that this can be predicted because the main function of the \isi{restricted stem} the \isi{perfective} \isi{aspect}. Secondly, the stem of the \isi{copula} is sensitive to duality: the non-dual stem is \emph{rä}, while the dual stem is \emph{rn}. Thirdly, the third \isi{person} singular inflections are irregular (in non-past): \isi{masculine} \emph{yé}; \isi{feminine} \emph{rä}. Table \ref{copulanonpast} shows the \isi{copula} forms in \isi{non-past}, \isi{recent past} and \isi{past} \isi{tense}. Finally, the \isi{copula} stem \emph{rä} can be used in an ambifixing template with the meaning `do'. This last point is discussed as part of the description of light verbs in \S\ref{lightverb}.

\begin{table}[H]
\caption{Copula inflection}
\label{copulanonpast}
	\begin{tabular}{llllll}
		\lsptoprule
		\textsc{gloss}&\Nonpast&\Rpst &\Rpst:\Dur& \Pst&\Pst:\Dur\\\hline
		\Fsg&\emph{worä}&\emph{kwofrä}&\emph{worärm}&\emph{worera}&\emph{kwofräm}\\
		\Fdu&\emph{nrn}&\emph{nzfrn}&\emph{nrnm}&\emph{nrna}&\emph{nzfrm}\\
		\Fpl&\emph{nrä}&\emph{nzfrä}&\emph{nrärm}&\emph{nrera}&\emph{nzfrärm}\\
		\Ssg&\emph{nrä}&\emph{nzfrä}&\emph{nrärm}&\emph{nrera}&\emph{nzfrärm}\\
		\Tsg.\F&\emph{rä}&\emph{zfrä}&\emph{rärm}&\emph{rera}&\emph{zfrärm}\\
		\Tsg.\Masc&\emph{yé}&\emph{sfrä}&\emph{yrärm}&\emph{yara}&\emph{sfrärm}\\
		\Stdu&\emph{ern}&\emph{thfrn}&\emph{ernm}&\emph{erna}&\emph{thfrnm}\\
		\Stpl&\emph{erä}&\emph{thfrä}&\emph{erärm}&\emph{erera}&\emph{thfrärm}\\
		\lspbottomrule
	\end{tabular}
\end{table}%Copula inflection

The \isi{copula} takes a \isi{copula} \isi{subject} and a \isi{copula} \isi{complement}. Copula clauses may express identity between two NPs (\ref{ex611}). They are used in presentational constructions; usually with a \isi{clitic} \isi{demonstrative} (\ref{ex612}).

\begin{exe}
	\ex \emph{ni fthé miyatha zäkorake ``babai zane bthan kabe yé.''}\\
	\gll ni fthé miyatha zä\stem{kor}ake babai zane bthan kabe \stem{yé}\\
	\Fnsg{} when knowledge \Fpl:\Sbj:\Pst:\Pfv/become uncle \Dem:\Prox{} black.magic man \Tsg.\Masc:\Sbj:\Nonpast:\Ipfv:\Cop\\
	\trans `That was when we realised ``The uncle is this sorcerer.'''\\\Corpus{tci20130901-04}{RNA \#45}
	\label{ex611}
\end{exe}
\begin{exe}
	\ex \emph{yorär ziyé ... zikogr.}\\
	\gll yorär z=\stem{yé} (.) z=y\stem{kogr}\\
	yorär \Prox=\Tsg.\Masc:\Sbj:\Nonpast:\Ipfv/be (.) \Prox=\Tsg.\Masc:\Sbj:\Nonpast:\Stat/stand\\
	\trans `Yorär (Syzygium sp) is here. It stands here.'\Corpus{tci20130907-02}{JAA \#450-451}
	\label{ex612}
\end{exe}

The \isi{complement} may be marked with the \isi{proprietive} case (see \S\ref{propcase}) or the \isi{privative} case (see \S\ref{privcase}) to express the existence or non-existence of some entity in relation to the \isi{copula} \isi{subject}. The former is shown in (\ref{ex744}), where they speaker literally says `the village is with a name' to express that it has some reputation. The latter is shown in (\ref{ex745}), where the speaker tells how he was looking for a creek that carried water.

\begin{exe}
	\ex \emph{zane kar mane rä yfkarä rä.}\\
	\gll zane kar mane \stem{rä} yf=karä \stem{rä}\\
	\Dem:\Prox{} village which \Tsg:\F:\Sbj:\Nonpast:\Ipfv:\Cop{} name=\Prop{} \Tsg:\F:\Sbj:\Nonpast:\Ipfv:\Cop{}\\
	\trans `As for this village, it has a (good) reputation.'\Corpus{tci20120805-01}{ABB 447-448}
	\label{ex744}
\end{exe}
\begin{exe}
	\ex \emph{buyak we ttfö ane zräbrmé nimame ... keke ... nomär rä.}\\
	\gll b=wi\stem{yak} we ttfö ane zrä\stem{brm}é nima=me (.) keke (.) no=mär \stem{rä}\\
	\Med=\Fsg:\Sbj:\Nonpast:\Ipfv/walk also creek \Dem{} \Fsg:\Sbj:\Irr:\Pfv/follow like.this:\Ins{} (.) \Neg{} (.) water=\Priv{} \Tsg:\F:\Sbj:\Nonpast:\Ipfv:\Cop\\
	\trans `I walked there, I followed another creek like this ... No ... (The creek) had no water.'\Corpus{tci20130903-03}{MKW \#92-93}
	\label{ex745}
\end{exe}

Adjectives and property nouns may also be \isi{copula} complements, as shown in (\ref{ex746}) and (\ref{ex747}) respectively. In (\ref{ex746}), the speaker reports how his fathers were comparing their yam harvest. In example (\ref{ex747}), the speaker talks about how as a teenager she was afraid of the anthropologist Mary Ayres when she first visited Rouku.

\begin{exe}
	\ex \emph{katakatanwä thfrä! nzenme kafar erä!}\\
	\gll kata-katan=wä thf\stem{rä} nzenme kafar e\stem{rä}\\
	\Redup-small=\Emph{} \Tpl:\Sbj:\Rpst:\Ipfv:\Cop{} \Fnsg:\Poss{} big \Tpl:\Sbj:\Nonpast:\Ipfv:\Cop\\
	\trans `Their (yams) were a bit small! Our (yams) are big!'\Corpus{tci20120805-01}{ABB 403}
	\label{ex746}
\end{exe}
\begin{exe}
	\ex \emph{nzä wwtri kwarärm ... markaianema ... nafanema fof.}\\
	\gll nzä w-wtri kwa\stem{rä}rm (.) markai=ane=ma (.) nafane=ma fof\\
	\Fsg.\Abs{} \Redup-fear \Fsg:\Sbj:\Pst:\Dur:\Cop{} (.) outsider=\Poss.\Sg=\Char{} (.) \Tsg.\Poss=\Char{} \Emph\\
	\trans `I was a bit afraid ... of the white woman ... really (afraid) of her.'\\\Corpus{tci20130911-03}{MBR \#10-11}
	\label{ex747}
\end{exe}

\subsection{Intransitive clauses}\label{intransitiveclauses}

In terms of \isi{verb} morphology, \isi{intransitive} clauses have been described in \S{}\ref{morphologicaltemplates}. The verb inflection employs the prefixing or the \isi{middle} template. Their single argument is always in \isi{absolutive} \isi{case}. Two examples are given in (\ref{ex562}) and (\ref{ex563}).\\

The two prefixing verbs in (\ref{ex562}) have no overt \isi{subject} noun phrases, but the second \isi{clause} contains an adjunct marked with the \isi{purposive} \isi{case} \emph{karr} `for a village' (or settlement place). In example (\ref{ex563}), we see the \isi{middle} verb \emph{brigsi} `return' and the \isi{subject} \isi{pronoun} \emph{nzä} in absolutive \isi{case}.

\begin{exe}
	\ex \emph{ŋarsenzo \textbf{swanyakm} ... karr \textbf{swanrenzrm}.}\\
	\gll ŋars=en=nzo swan\stem{yak}m (.) kar=r swan\stem{re}nzrm\\
	river=\Loc=\Only{} \Tsg.\Masc:\Sbj:\Pst:\Dur:\Venit/walk (.) village=\Purp{} \Tsg.\Masc:\Sbj:\Pst:\Dur:\Venit/look.around\\
	\trans `He was coming along the river ... he was looking for a place to settle.'\\\Corpus{tci20120922-09}{DAK \#14-15}
	\label{ex562}
\end{exe}
\begin{exe}
	\ex \emph{nzä boba fthé kanathrfa \textbf{zänbrima}.}\\
	\gll nzä boba fthé kanathr=fa zän\stem{brim}a\\
	\Fsg.\Abs{} \Med.\Abl{} when kanathr=\Abl{} \Sg:\Sbj:\Pst:\Pfv:\Venit/return\\
	\trans `That was when I returned from Kanathr.'\Corpus{tci20120805-01}{ABB \#607}
	\label{ex563}
\end{exe}

\subsection{Impersonal clauses}\label{impersonalclause}

Impersonal clauses are expressed using the \isi{middle} template of the verb, in which a person-invariant \isi{middle} marker fills the prefix slot, while the suffix indexes the single argument of the predicate (see \S{}\ref{middletemplatesubsection}). The indexed noun phrase, if present at all, occurs in \isi{absolutive} \isi{case}. The salient feature of this \isi{clause} type is that the referent of the verb indexing is \isi{impersonal}, unclear or simply empty. Consider examples (\ref{ex737}) and (\ref{ex738}). In the first example, the speaker talks about rain-making magic, which involves a rotting mixture of meat and honey in bottles. These bottles or containers are opened and the rising odor is said to increase the rainfall. The third singular indexed by the verb form \emph{kfäkor} refers to the changed weather conditions, and the \ili{English} translation `it was enough' exhibits the same general or \isi{impersonal} meaning. The second example contains the noun \emph{aki} `moon', but it is unclear whether the verb really indexes this noun or whether its referent is empty. Hence, the two possible translations. During the transcription of example (\ref{ex738}), the first translation was the preferred one in this particular context.

\begin{exe}
	\ex \emph{watikthénzo fthé kfäkor ... we sgu thwäthbe woz thwärmäne.}
	\gll watik-thé=nzo fthé kfä\stem{kor} (.) we sgu thwä\stem{thb}e woz thwä\stem{rmän}e\\
	enough-\Adlzr=\Only{} when \Stsg:\Sbj:\Iter/become (.) also plug \Fpl:\Sbj>\Stpl:\Obj:\Iter/put.inside bottle \Fpl:\Sbj>\Stpl:\Obj:\Iter/close\\
	\trans `When it was enough, we put the lids back in and we closed the bottles.'\\\Corpus{tci20110810-01}{MAB \#59-62}
	\label{ex737}
\end{exe}
\begin{exe}
	\ex \emph{aki zbo kräkor.}\\
	\gll aki zbo krä\stem{kor}\\
	moon \Prox.\All{} \Stsg:\Sbj:\Irr:\Pfv/become\\
	\trans `It became moon(light) here.' or `The moon came up here.'\\\Corpus{tci20120904-02}{MAB \#47}
	\label{ex738}
\end{exe}

Example (\ref{ex557}), is a description of a picture as part of a stimulus task. The speaker takes on the role of a man in the picture and asks: `What is going on?'. Again the verb form \emph{krewär} appears in the \isi{middle} construction and indexes a third singular.

\begin{exe}
	\ex \emph{sinzo foba ynrä nima ``ra krewär bobo?''}\\
	\gll si=nzo foba yn\stem{rä} nima ra kre\stem{wär} bobo\\
	eye=\Only{} \Dist.\Abl{} \Tsg.\Masc:\Sbj:\Nonpast:\Ipfv:\Venit/be \Quot{} what(\Abs) \Stsg:\Sbj:\Irr:\Pfv/happen \Med:\All{}\\
	\trans `He was just looking from over there and wondered: ``What is going on there?'''\Corpus{tci20111004}{RMA \#353}
	\label{ex557}
\end{exe}

Impersonal constructions often involve light verbs, for example \emph{rä-} `do' and \emph{ko-} `become', which take a \isi{nominal} predicate, for example a noun or \isi{property noun}. In these cases, the \isi{nominal} predicate will be unmarked for \isi{case}, like the absolutive \isi{case}. Therefore, it may be difficult to decide whether (i) it is a \isi{nominal} predicate and the \isi{subject} is empty, or (ii) whether the noun phrase in question is the \isi{subject} indexed in the \isi{verb}. Consider example (\ref{ex559}) below, in which the speaker describes the location of the mythical place of origin \emph{Kwafar}, which is located in the Arafura sea between Papua New Guinea and Australia. The verb form \emph{ŋakonzr} `it becomes' occurs in the \isi{relative clause}, which is printed in boldface. The third singular indexed in the verb form could be \emph{mazo} `ocean' (lit: `where the ocean becomes') or it could be an empty \isi{subject} (lit: `it becomes ocean').

\begin{exe}
	\ex \emph{thden rera ... zane zena mane bad mane wythk \textbf{mazo mä ŋakonzr} a ... australiane bad mä wythk.}\\
	\gll thd=en \stem{rä}ra (.) zane zena mane bad mane w\stem{ythk} mazo mä ŋa\stem{ko}nzr a (.) australia=ane bad mä w\stem{ythk}\\
	middle=Loc{} \Tsg.\F:\Sbj:\Pst:\Ipfv/be (.) \Dem:\Prox{} today which ground which \Tsg.\F:\Sbj:\Nonpast:\Ipfv/come.to.end and (.) australia=\Poss{} ground where \Tsg.\F:\Sbj:\Nonpast:\Ipfv/come.to.end\\
	\trans `It was in the middle ... this one, where the land ends ... where it becomes ocean until where Australia's land ends.'\Corpus{tci20131013-01}{ABB \#26-30}
	\label{ex559}
\end{exe}

Weather events often have empty or \isi{impersonal} subjects. This can be shown with prefixing verbs, as well as \isi{middle} verbs. A common way to say `It is going to rain' is shown in (\ref{ex739}). It is clear that \emph{nor} `for rain' is not indexed in the verb because it is flagged with a non-core \isi{case}, the \isi{purposive} \isi{case}. Therefore, the reference of the third singular in the verb form is empty.

\begin{exe}
	\ex \emph{nor yé.}\\
	\gll no=r \stem{yé}\\
	rain=\Purp{} \Tsg.\Masc:\Sbj:\Nonpast:\Ipfv/be\\
	\trans `It will rain.' (Lit. `It is for rain')\Corpus{overheard}{}
	\label{ex739}
\end{exe}

Another example is the phrase \emph{wär kwan yanor} `it is thundering' in (\ref{ex740}). The thunder is expressed by the \isi{ideophone} \emph{wär kwan} `thundering noise', and all ideophones of this type are \isi{nominal} compounds headed by \emph{kwan} `noise, sound' (see \S{}\ref{ideophonesec}). The verb \emph{yannor} is inflected for a masculine subject, but \emph{kwan} is feminine. Hence \emph{wär kwan} is not the \isi{subject}, and a literal translation would be: `He shouts the thunder sound'. Again the reference of `he' is empty.

\begin{exe}
	\ex \emph{wär kwan yanor.}\\
	\gll {wär kwan} ya\stem{nor}\\
	thunder \Tsg.\Masc:\Sbj:\Nonpast:\Ipfv/shout\\
	\trans `It is thundering.'\Corpus{overheard}{}
	\label{ex740}
\end{exe}

Other weather or sound phenomena can be expressed by verbs in the \isi{middle} template. In example (\ref{ex558}), the verb `start' is inflected for a \Stsg{} \isi{subject}, but its referent is unclear \textendash{} partly because the verb does not index an \isi{object}. Thus, the indexed argument could be (i) the sound of the fire (`The fire sound started'), or (ii) it could be an empty \isi{subject} (`It started the fire sound').

\begin{exe}
	\ex \emph{fi mni zürnane u kwan zethkäfako.}\\
	\gll fi mni zürn=ane {u kwan} ze\stem{thkäf}ako\\
	but fire smoke=\Poss.\Sg{} {roaring.sound} \Sg:\Sbj:\Pst:\Pfv:\Andat/start\\
	\trans `but the fire smoke's sound started (rumbling).'\Corpus{tci20120827-03}{KUT \#186-187}
	\label{ex558}
\end{exe}

\subsection{`Passive' clauses}\label{passiveclause}

Passives meanings are expressed in two ways: (i) by a \isi{verb} in the \isi{middle} template which indexes a \isi{patient} role; the indexed noun phrase occurs in \isi{absolutive} \isi{case} (see \S{}\ref{middletemplatesubsection}), or (ii) by a resultative construction, in which a nominalised verb is flagged with the \isi{instrumental} \isi{case} (see \S{}\ref{inscase}). Note that both are not dedicated \isi{passive} constructions. Instead, they should be understood as constructions which can express passive-like semantics.\\

Example (\ref{ex554}) shows both constructions. The first two clauses are in a \isi{temporal} relationship to the last clause, which is signalled by \emph{fthé} `when'. This is not a subordinate relationship because \emph{fthé} can also be used in independent clauses with the meaning of `that was when'. In the first clause, the single argument of the verb is \emph{bad} `ground, earth'. This can be translated either as an \isi{reflexive}/\isi{impersonal} `the earth created (itself)' or as a \isi{passive} `the earth was created'. In the second clause, matters are clear because the verb is in a \isi{transitive} template which shows actor agreement with `father' (\Erg) and \isi{undergoer} agreement with `earth' (\Abs), thus: `the father created the earth'. The last clause, is a resultative construction. The nominalised verb \emph{rifthzsi} `hiding' takes the \isi{instrumental} case (`with hiding'), which is best translated as a \isi{passive} (`was hidden').

\begin{exe}
	\ex \emph{\textbf{bad fthé ŋafiyokwa} ... ŋafyf fthé bad wäfiyokwa ... \textbf{kidn ane rifthzsime zfrärm}.}\\
	\gll bad fthé ŋa\stem{fiyok}wa (.) ŋafe=f fthé bad wä\stem{fiyok}wa (.) kidn ane rifthz-si=me zf\stem{rä}rm\\
	earth when \Sg:\Sbj:\Pst:\Ipfv/make (.) father=\Erg{} when earth \Stsg:\Sbj>\Tsg.\F:\Obj:\Pst:\Ipfv/make (.) {eternal fire} \Dem{} hide-\Nmlz=\Ins{} \Tsg.\F:\Sbj:\Pst:\Dur/be\\
	\trans `When the earth was made ... when God made the earth ... that eternal fire was hidden.'\Corpus{tci20120909-06}{KAB \#61-63}
	\label{ex554}
\end{exe}

\subsection{Reflexive and reciprocal clauses}\label{reflrecipclause}

Formally, \isi{reflexive}/\isi{reciprocal} clauses are encoded by (i) the verb form in the \isi{middle} template and (ii) the argument noun phrase in \isi{absolutive} case. Ditransitives show exceptional grammatical behaviour in that the argument may be in absolutive or \isi{ergative} \isi{case}. There is no distinction between reflexives and reciprocals other than the fact that singulars do not allow a \isi{reciprocal} reading. Below I will describe how \isi{reflexive}/reciprocals differ from \isi{intransitive} and \isi{impersonal} \isi{clause} on the one side, and from \isi{suppressed-object} constructions on the other. This topic is also addressed in the description of the \isi{middle} template (see \S{}\ref{middletemplatesubsection}).\\

In example (\ref{ex566}) the speaker talks about a ritual which chases away evil spirits. This rather gruesome ritual involves young men shooting at each other with blunt arrows. In the last clause of the example the noun phrase \emph{kabe} `man' is in absolutive \isi{case} and the verb employs the \isi{middle} template and indexes one argument (\Stpl). The verb \emph{rusi} `shoot' has rather clear \isi{transitive} semantics and, thus, invites a \isi{reciprocal} interpretation.

\begin{exe}
	\ex \emph{kabe kwaruthrmth frkkarä.}\\
	\gll kabe kwa\stem{ru}thrmth frk=karä\\
	man(\Abs) \Stpl:\Sbj:\Pst:\Dur/shoot blood=\Prop\\
	\trans `The people were shooting at each other (until) they were bleeding.'\Corpus{tci20150906-10}{ABB \#414}
	\label{ex566}
\end{exe}

In most cases only secondary information disambiguates between \isi{intransitive}, \isi{impersonal} and \isi{reflexive}/\isi{reciprocal} interpretations. By secondary information, I mean (i) context, (ii) grammatical devices which are not used solely for \isi{reflexive}/\isi{reciprocal} constructions, (iii) statistical tendencies of individual verbs. I will address these in turn. First, context is probably the most important, and it is evident that an example like (\ref{ex566}) is usually preceded or followed by a description which disambiguates the state of affairs. Secondly, speakers may choose to repeat the absolutive noun phrase to make clear that the intended reading should be a \isi{reciprocal} one. Consider example (\ref{ex567}), which concludes a headhunting story. The \isi{pronoun} \emph{fi} occurs twice. Additionally, the utterance was accompanied by appropriate gestures to clarify the intended \isi{reciprocal} meaning. The \isi{pronoun} \emph{fi} is marked with the \isi{exclusive} \isi{enclitic} \emph{=nzo}. The repetition and the \isi{exclusive} \isi{enclitic} are secondary strategies which are not solely used to mark \isi{reflexive}/\isi{reciprocal} meanings. Note that the \isi{exclusive} \isi{enclitic} \emph{=nzo} shows cognates in other \ili{Yam languages}. In \ili{Nen}, there is a set of \isi{reflexive}/\isi{reciprocal} pronouns which all end in \emph{nzo}, for example \emph{benzo} \Ssg{} (\citealt[1072]{Evans:2015wy}). In Komnzo, the \isi{exclusive} clitic expresses the meaning of `only' without \isi{reflexive}/\isi{reciprocal} semantics.

\begin{exe}
	\ex \emph{ni woga tüfrmäre nrä ... bänema nzenme thden ane fof kwakwirm ... woga \textbf{finzo} \textbf{finzo} kwafnzrmth.}\\
	\gll ni woga tüfr=märe n\stem{rä} (.) bäne=ma nzenme thd=en ane fof kwa\stem{kwir}m (.) woga fi=nzo fi=nzo kwa\stem{fn}nzrmth\\
	\Fnsg{} man plenty=\Priv{} \Fpl:\Sbj:\Nonpast:\Ipfv/be (.) \Dem:\Med=\Char{} \Fnsg.\Poss{} \isi{middle}=\Loc{} \Dem{} \Emph{} \Stsg:\Sbj:\Pst:\Dur/run (.) man \Third.\Abs=\Only{} \Third.\Abs=\Only{} \Stpl:\Sbj:\Pst:\Dur/kill\\
	\trans `We are not many ... because this was going on in our \isi{middle} ... The people, this (group) and that (group) were killing each other.'\Corpus{tci20111107-01}{MAK \#157-158}
	\label{ex567}
\end{exe}

Although stems may alternate between different morphological templates, there is a statistical tendency to occur in a particular template for a particular stem. For example, typically \isi{transitive} meanings (\emph{rusi} `shoot', \emph{zan} `hit, kill', \emph{marasi} `see') occur most of the time in the ambifixing \isi{transitive} template. If such stems occur in a \isi{middle} template, it invites a \isi{reflexive}/\isi{reciprocal} reading rather than an \isi{impersonal} or \isi{intransitive} one. We will see in the following section that the \isi{middle} template can also be used for the \isi{suppressed-object} construction (see \S{}\ref{suppressedobjectclause}). However, in the \isi{suppressed-object} construction the noun phrase indexed in the verb form is marked for \isi{ergative} \isi{case} and not absolutive. On the other hand, stems which occur in the \isi{middle} template most of the time (\emph{maikasi} `wash', \emph{bringsi} `return') should be analysed as reflexiva tanta (\citealt{Geniusienie:1987refl}), even though they may occur in the ambifixing \isi{transitive} template (`wash someone', `bring back someone'). Hence, there is a statistical tendency for stems to occur in a particular template, which helps to disambiguate between an \isi{impersonal} or \isi{reflexive}/\isi{reciprocal} reading.\\

Next, I want to set \isi{reflexive}/reciprocals apart from what I call the \isi{suppressed-object} construction (see \S{}\ref{suppressedobjectclause}). The state of affairs in \isi{reflexive}/reciprocals is such that the actor and \isi{patient} can be exchanged. In Komnzo, both are expressed by one noun phrase which occurs in absolutive case. Herein lies the formal difference to the \isi{suppressed-object} construction. If the noun phrase \emph{kabe} `people' in example (\ref{ex566}) was in \isi{ergative} \isi{case} \textendash{} for example \emph{kabe=yé} (man=\Erg.\Nsg{}) \textendash{} the sentence would mean `they were shooting (at sth.)'. This is the \isi{suppressed-object} construction, which I describe in the following section (see \S{}\ref{suppressedobjectclause}). Note that the verb form \emph{kwaruthrmth} remains the same, only the \isi{case} marking changes.\\

For \isi{ditransitive} verbs, the \isi{case} marking is less fixed, and the argument noun phrase can appear in absolutive as well as \isi{ergative} \isi{case}, both with a \isi{reflexive}/\isi{reciprocal} meaning. In example (\ref{ex568}) the verb form \emph{ŋarinth} indexes only the \isi{subject} (\Stdu), while the prefix slot is filled with the \isi{middle} marker. The \isi{subject} argument appears in the \isi{ergative} (\emph{nafa}). A \isi{suppressed-object} reading is not possible with \isi{ditransitive} verbs. Note that the argument could also occur in absolutive \isi{case} (\emph{fi}). This would create a \isi{clause} with two absolutive noun phrases. Hence, the choice between \isi{ergative} and abolutive seems to be dependent on the kinds of referents. In (\ref{ex568}), both noun phrases are animate, and the use of the \isi{ergative} \isi{case} avoids confusion between \isi{agent} (`they') and \isi{theme} (`sisters').

\begin{exe}
 	\ex \emph{emoth nafa ŋarinth fof.}\\
 	\gll emoth nafa ŋa\stem{ri}nth fof\\
 	girl \Tnsg.\Erg{} \Stdu:\Sbj:\Nonpast:\Ipfv/give \Emph{}\\
 	\trans `They give each other sisters.'\Corpus{tci20120805-01}{ABB \#158}
 	\label{ex568}
\end{exe}

At this stage, it is impossible to investigate this topic further, because (i) \isi{noun phrase}s are frequently omitted and (ii) as I have argued in \S{}\ref{ambifixingtemp}, except for a few verbs (\emph{yarisi} `give', \emph{trikasi} `tell', \emph{fänzsi} `show') all \isi{ditransitive} verbs are derived.

\subsection{Suppressed-object clauses}\label{suppressedobjectclause}

Suppressed-\isi{object} clauses employ the \isi{middle} template of the verb. The argument indexed in the verb is treated like an actor by the case system, i.e. it is flagged with the \isi{ergative} \isi{case}. The \isi{object} may be overtly expressed with a noun phrase, but it is suppressed from indexation in the verb form.\\

I describe in \S{}\ref{middletemplatesubsection} that almost all \isi{transitive} verbs can enter into the \isi{suppressed-object} construction for semantic as well as pragmatic reasons. For example, most of the time, the referents of suppressed-objects rank low in the animacy hierarchy (\citealt{Silverstein:1976vo}). In example (\ref{ex569}) the speaker searches her shoes and complains that her friend has been wearing them. We only know about the \isi{object} of \emph{rgsi} `wear' from the previous context, since it is not expressed as a noun phrase, nor is the \isi{object} indexed in the verb form. The semantics of \emph{rgsi} renders a \isi{reflexive} reading (`she wears herself') non-sensical. Additionally, the fact that the \isi{subject} is in \isi{ergative} case (\emph{naf}) rules out the \isi{reflexive}/\isi{reciprocal} interpretation. This is important because the verb form is identical between \isi{reflexive}/reciprocals and the \isi{suppressed-object} construction.

\begin{exe}
	\ex \emph{ebar zfthnzo! naf rar ŋargwrm?}\\
	\gll ebar zfth=nzo naf ra=r ŋa\stem{rg}wrm\\
	head base=\Only{} \Tsg.\Erg{} what=\Purp{} \Sg:\Sbj:\Rpst:\Dur/wear\\
	\trans `Thickhead! Why was she wearing (the flipflops)?'\Corpus{tci20130901-04}{RNA \#173}
	\label{ex569}
\end{exe}

Objects can be suppressed for pragmatic reasons, often in addition to their low rank on the animacy hierarchy. That is because the suppression of the \isi{object} has the pragmatic effect of focussing the \isi{subject}. Example (\ref{ex570}) is taken from a text about food taboos. This topic came up while talking about a very old woman, whose old age was ascribed to her respecting all food taboos. In the example, the speaker shifts the topic from the old woman to those people who did not respect food taboos. This shift of topic is achieved by (i) a fronted \isi{relative clause} and (ii) the \isi{suppressed-object} construction. As in the previous example, we only know about the \isi{object} of \emph{rirksi} `respect, avoid' from the preceding context.

\begin{exe}
	\ex \emph{fi mafa keke kwarirkwrmth ... watik tekmär esufakwa.}\\
	\gll fi mafa keke kwarirkwrmth ... watik tekmär esufakwa\\
	but who.\Nsg.\Erg{} \Neg{} \Stpl:\Sbj:\Pst:\Dur/respect (.) then duration=\Priv{} \Stpl:\Sbj:\Pst:\Ipfv/grow.old\\
	\trans `But those who did not respect (the food taboos) ... well, they grew old quickly.'\Corpus{tci20120922-26}{DAK \#26-27}
	\label{ex570}
\end{exe}

Although the \isi{object} is suppressed from indexation in the verb form, it may occur as a noun phrase in the \isi{clause}. In example (\ref{ex571}), the speaker talks about garden magic and people who steal the soil from other people's gardens. In the \isi{relative clause}, the \isi{object} \emph{bad} `ground' is suppressed from indexation in the verb, yet it appears as a noun phrase. The \isi{subject} is indexed in the verb suffix and the corresponding noun phrase, the relative \isi{pronoun} \emph{mafa}, is in \isi{ergative} case.

\begin{exe}
	\ex \emph{nä kabenzo nnzä wawa gamokarä erä bad mafa ŋakarkwrth.}\\
	\gll nä kabe=nzo nnzä wawa gamo=karä e\stem{rä} bad mafa ŋa\stem{kark}wrth\\
	\Indf{} man=\Only{} perhaps yam spell=\Prop{} \Stpl:\Sbj:\Nonpast:\Ipfv/be ground who.\Erg.\Nsg{} \Stpl:\Sbj:\Nonpast:\Ipfv/take\\
	\trans `Perhaps only other people, who take the soil away, have yam magic.'\\\Corpus{tci20130822-08}{JAA \#42}
	\label{ex571}
\end{exe}

The \isi{suppressed-object} may also be a \isi{relative clause} as in example (\ref{ex572}), which is taken from a picture stimulus task.

\begin{exe}
	\ex \emph{emothf ŋatrikwr monme zffnzr.}\\
	\gll emoth=f ŋa\stem{trik}wr mon=me zf\stem{fn}nzr\\
	girl=\Erg{} \Stsg:\Sbj:\Nonpast:\Ipfv/tell how=\Ins{} \Stsg:\Sbj>\Tsg.\F:\Obj:\Rpst:\Ipfv/hit\\
	\trans `The girl tells (the story of) how he hit her.'\Corpus{tci20120925}{MAE \#102}
	\label{ex572}
\end{exe}

There are a few verbs which always occur in the \isi{suppressed-object} construction. A few examples are: \emph{yonasi} `drink', \emph{fathasi} `marry', \emph{frzsi} `fish/net (poison-root)', \emph{naf-} `talk, speak' and \emph{karksi} `pull'.\footnote{The stem \emph{karksi} can occur in a transitive template with the meaning `take'. If it occurs in a suppressed-object construction, it means `pull'. I analyse these as two different lexical items, because there is a difference in the semantics as well as the combinatorics of the stem.} With other verbs there is only a statistical tendency to enter this construction. For example, \emph{yarizsi} `hear' occurs 104 times in the corpus; 25 times the \isi{object} is indexed and 79 times it is suppressed. In other words, in only about a quarter of all tokens of \emph{yarizsi}, the verb means `hear X'. In the other three quarters of tokens of \emph{yarizsi}, it means `hear (sth.)'. In (\ref{ex573}), we see an example of \emph{yarizsi} and \emph{rfnaksi} `taste' in the \isi{suppressed-object} construction. The speaker explains how the news of the beginning yam harvest spread from East to West; from village to village.

\begin{exe}
	\ex \emph{watik, we masu karé kwekaristh ``oh, nafa z zärfnth!''}\\
	\gll watik, we masu karé kwe\stem{karis}th oh nafa z zä\stem{rfn}th\\
	then also masu village=\Erg.\Nsg{} \Stpl:\Sbj:\Iter/hear oh \Stnsg.\Erg{} \Iam{} \Stpl:\Sbj:\Rpst:\Pfv/taste\\
	\trans `Then the Masu people always heard (the other village): ``Oh, they have already tasted (the yams)!'''\Corpus{tci20131013-01}{ABB \#363}
	\label{ex573}
\end{exe}

\subsection{Transitive clauses}\label{transitiveclause}

This section deals with prototypical \isi{transitive} clauses, which are \isi{transitive} in their verb morphology, i.e. they are built from the ambifixing \isi{transitive} template, as well as their noun phrase syntax, i.e. the actor argument is flagged with the \isi{ergative} and the \isi{undergoer} argument is in the \isi{absolutive}. Therefore, \isi{suppressed-object} constructions (see \S{}\ref{suppressedobjectclause}) can be described as non-prototypical \isi{transitive} clauses because (i) the verb appears in the \isi{middle} template, (ii) the \isi{object} noun phrase is frequently omitted. However, noun phrases can generally be dropped in all \isi{clause} types. The ambifixing verb template is described in \S{}\ref{ambifixingtemp}. An example of a \isi{transitive} clause is given below in (\ref{ex574}).

\begin{exe}
	\ex \emph{nzürna ŋaref bäne ŋad yrtmakwa.}\\
	\gll nzürna ŋare=f bäne ŋad y\stem{rtmak}wa\\
	spirit woman=\Erg.\Sg{} \Dem:\Med{} string(\Abs) \Sg:\Sbj>\Tsg.\Masc:\Obj:\Pst:\Ipfv/cut\\
	\trans `The \emph{nzürna} woman cut that string.'\Corpus{tci20120827-03}{KUT \#142}
	\label{ex574}
\end{exe}

\subsection{Ditransitive clauses}\label{ditransitiveclause}

Di\isi{transitive} clauses employ the same template as \isi{transitive} clauses. However, the \isi{valency} changing prefix \emph{a-} shifts the reference of the \isi{verb} prefix from the direct \isi{object} to the \isi{indirect object}. The corresponding noun phrase appears in \isi{dative} case. This is described in \S{}\ref{ambifixingtemp}. Note that the \emph{a-} prefix may increase as well as decrease the \isi{valency} of a verb, hence, the label ``\isi{valency} changing prefix'' (see \S{}\ref{morphologicaltemplates}).\\

Example (\ref{ex575}) below shows the verbs \emph{trikasi} `tell' and \emph{fänzsi} `show'. The \isi{recipient} arguments are flagged for \isi{dative} case and the respective arguments are indexed in the two verbs.\\

\begin{exe}
	\ex \emph{nzone \textbf{ŋafyn} \textbf{bäin} ane trikasi \textbf{yatrikwath} ... \textbf{nzunwä} ŋafyf bäif \textbf{zwafäsa}.}\\
	\gll nzone ŋafe=n bäi=n ane trika-si ya\stem{trik}wath (.) nzun=wä ŋafe=f bäi=f zwa\stem{fäs}a\\
	\Fsg.\Poss{} father=\Dat.\Sg{} bäi=\Dat.\Sg{} \Dem{} tell-\Nmlz{} \Stpl:\Sbj>\Tsg.\Masc:\Io:\Pst:\Ipfv/tell (.) \Fsg.\Dat=\Emph{} father=\Erg.\Sg{} bäi=\Erg.\Sg{} \Stsg:\Sbj>\Fsg:\Io:\Pst:\Pfv/show\\
	\trans `They told that story to my father Bäi ... and father Bäi showed (it) to me.'\\\Corpus{tci20110802}{ABB \#18-20}
	\label{ex575}
\end{exe}

Di\isi{transitive} clauses may also contain cognate objects, as in (\ref{ex575}) \emph{trikasi yatrikwath} `they told him the story'. Another example is \emph{yathugsi} `trick (v)' which often occurs with \emph{gaso} `trick, lie'.\\

In \S{}\ref{ambifixingtemp}, I argued that \isi{ditransitive} as a category should be recognised, even though most \isi{ditransitive} verbs are derived from transitives by (i) adding the \isi{valency change} prefix \emph{a-}, which (ii) changes the reference of the verb prefix to an \isi{indirect object} (\isi{goal}, \isi{recipient}, \isi{beneficiary}) and (iii) putting the respective argument noun phrase in \isi{dative} \isi{case}. The same strategy can be used to raise possessors in the cross-referencing of the verb. In example (\ref{ex576}), it is the \isi{possessor} (\emph{nzone} `my' \Fsg), which is indexed in the verb, and not the \isi{possessed} (\emph{miyo} `desire/wish' \Tsg.\F).

\begin{exe}
	\ex \emph{\textbf{nzone miyo} kwa \textbf{wabthakwr}.}\\
	\glll nzone miyo kwa wo-a-bthak-w-r-\Zero{}\\
	\Fsg.\Poss{} desire \Fut{} \Fsg.\Alph-\Vc-finish.\Ext-\Lk-\Stsg{}\\
	{} {} {} \footnotesize{\Stsg:\Sbj>\Fsg:\Io:\Nonpast:\Ipfv/finish}\\
	\trans `You will fulfill my wish.'\Corpus{tci20130823-06}{CAM \#23}
	\label{ex576}
\end{exe}

The \isi{ditransitive} pattern is very productive and almost all \isi{transitive} verbs can enter this construction. Most verbs retain their \isi{transitive} semantics, but can index a \isi{beneficiary} of the event. For example, in (\ref{ex577}), the verb \emph{fsisi} `count' in the clause takes the \isi{object} `yam suckers'. The \isi{ditransitive} pattern only adds a \isi{beneficiary} which is indexed in the verb.

\begin{exe}
	\ex \emph{nä efothen ... \textbf{wawa tafo} \textbf{yafsinzake} ... \textbf{babuan}.}\\
	\gll nä efoth=en (.) wawa tafo ya{fsi}nzake (.) babua=n\\
	\Indf{} day=\Loc{} (.) yam sucker \Fpl:\Sbj>\Tsg.\Masc:\Io:\Pst:\Ipfv{} (.) babua=\Dat.\Sg{}\\
	\trans `Some day ... we counted yam suckers for him ... for Babua.'\\\Corpus{tci20120814}{ABB \#165-167}
	\label{ex577}
\end{exe}

As I pointed out in \S{}\ref{prefixingverbsec}, prefixing verbs (intransitives) can enter the same pattern, whereby a \isi{beneficiary} or raised \isi{possessor}, in \isi{dative} and \isi{possessive} case respectively, is indexed in the verb form. Example (\ref{ex741}) is taken from a recording where two speakers discuss the content of a picture card. The prefixing verb \emph{-thn} `be lying' in the example does not index the objects that are lying on the ground, but the \isi{possessor} instead.

\begin{exe}
	\ex \emph{ra kwa nm bäne \textbf{wäthn}? ... \textbf{nafane} nainai?}\\
	\gll ra kwa nm bäne wä\stem{thn} (.) nafane nainai\\
	what \Fut{} maybe \Dem:\Med{} \Tsg.\F:\Io:\Nonpast:\Ipfv/be.lying (.) \Tsg.\Poss{} sweet.potato\\
	\trans `What (of hers) might be lying there? ... her sweet potatoes?'\\\Corpus{tci20111004}{RMA \#108}
	\label{ex741}
\end{exe}

\subsection{Experiencer-object constructions}\label{expobjconstr}

Experiencer-object constructions express bodily, mental and emotional processes (`get sunburned', `shiver in fear', `be angry'). These are framed as \isi{transitive} clauses whereby the \isi{stimulus} acts on the \isi{experiencer}. Constructions of this type have been examined by Pawley et al. for \ili{Kalam} (\citeyear{Pawley:2000vp}) showing that experiencer-objects as well as experiencer-subjects are found in the semantic domain of bodily and mental processes.\footnote{Note that the notion of experiencer is slightly extended here to include bodily processes in addition to mental or emotional ones.} Komnzo confirms their findings. In terms of their morpho-syntax, \isi{experiencer-object} constructions are characterised by the following criteria: (i) the \isi{stimulus} argument appears in the \isi{ergative}, (ii) the \isi{stimulus} is indexed by a default \Tsg{} in the verb suffix, (iii) the \isi{experiencer} occurs in absolutive \isi{case}, and (iv) the word order is UAV (\isi{undergoer} actor verb).\\

Consider the two ways of expressing a feeling of hunger in the elicited examples in (\ref{ex5788}). In (\ref{ex578}) the \isi{experiencer} is the \isi{subject} of the copula \isi{clause}, but in (\ref{ex579}) it is the \isi{object} of the verb \emph{rmatksi} `cut'. In the latter the feeling of hunger is portrayed as somewhat stronger. Note that the choice of verb is not entirely fixed. One can replace \emph{rmatksi} `cut' with a \isi{light verb}, for example \emph{rä-} `do' (`hunger does me'), or with the phasal verb \emph{bthaksi} `finish' (`hunger finishes me'), thereby changing the degree or intensity of the experienced feeling. Thus, the \isi{experiencer-object} construction is one possibility to express mental and bodily processes.

\begin{exe}
	\ex
	\label{ex5788}
	\begin{xlist}
	\ex \emph{nzä frasi worä}\\
	\gll nzä frasi wo\stem{rä}\\
	\Fsg.\Abs{} hunger \Fsg.\Sbj:\Nonpast:\Ipfv/be\\
	\trans `I am hungry.'
	\label{ex578}
	\ex \emph{nzä frasif wortmakwr}\\
	\gll nzä frasi=f wo\stem{rtmak}wr\\
	\Fsg.\Abs{} hunger=\Erg.\Sg{} \Stsg:\Sbj>\Fsg:\Obj:\Nonpast:\Ipfv/cut\\
	\trans `I am hungry. / I am starving.' (Lit: `Hunger cuts me.')
	\label{ex579}
	\end{xlist}
\end{exe}

Examples like (\ref{ex578}) were given to me in elicitation, when asking `How do I say `I am hungry?'. I first encountered \isi{experiencer-object} constructions in more natural situations, for example in overhearing conversations or when translating recordings. Komnzo speakers explicitly regard \isi{experiencer-object} constructions as more original and creative language. Therefore, it seems natural that these were rarely offered in the context of elicitation. Experiencer-object constructions portray a situation in much more colourful terms. They often evoke some kind of emotional reaction (laughter or sympathy) from the audience, as in (\ref{ex685}), where a woman describes what happened to her as a small child when she was hiding on a tree from a pig.

\begin{exe}
	\ex \emph{nzä \textbf{wthf} warfo bä \textbf{kwräbth}.}\\
	\gll nzä wth=f warfo bä kwrä\stem{bth}\\
	\Fsg.\Abs{} faeces=\Erg.\Sg{} above \Med{} \Stsg:\Sbj>\Fsg:\Irr:\Pfv/finish\\
	\trans `I really had to take a dump there on top (of the tree).' (Lit: `Excretes finish me.')\Corpus{tci20150919-05}{LNA \#117}
	\label{ex685}
\end{exe}

Experiencer-object constructions express bodily and mental processes, and it is this internal \isi{stimulus} which `acts' on the \isi{experiencer}. Two text examples were given in the description of the \isi{ergative} \isi{case} (\S{}\ref{ergcase}) and are repeated below in (\ref{ex580}) and (\ref{ex581}).

\begin{exe}
	\ex \emph{\textbf{nokuyé} fthé \textbf{sabtha}.}\\
	\gll noku=yé fthé sa\stem{bth}a\\
	anger=\Erg.\Nsg{} when \Stsg:\Sbj>\Tsg.\Masc:\Pst:\Pfv/finish\\
	\trans `That is when he got really angry.' (Lit. `Anger finished him.')\\\Corpus{tci20120909-06}{KAB \#39}
	\label{ex580}
\end{exe}

\begin{exe}
	\ex \emph{\textbf{wtrif} z \textbf{zwefaf}.}\\
	\gll wtri=f z zwe\stem{faf}\\
	fear=\Erg.\Sg{} \Iam{} \Stsg>\Fsg:\Rpst:\Pfv/hold\\
	\trans `I am already scared.' (Lit. `Fear holds me.')\Corpus{tci20130901-04}{RNA \#164}
	\label{ex581}
\end{exe}

The \isi{stimulus} noun phrase can be modified, for example with a \isi{nominal} compound. In example (\ref{ex582}) the \isi{stimulus} \emph{miyo} `desire' is modified by two elements yielding \emph{kabe zan miyo} `desire to kill people'. This example is repeated from the discussion of complex heads in \S{}\ref{headcompounds}.

\begin{exe}
	\ex \emph{baf fthé \textbf{sräbth} nima ... \textbf{kabe zan miyof}.}\\
	\gll baf fthé srä\stem{bth} nima (.) kabe zan miyo=f\\
	\Recog.\Erg.\Sg{} when \Stsg:\Sbj>\Tsg.\Masc:\Obj:\Irr:\Pfv/finish like.this (.) man hitting desire=\Erg.\Sg\\
	\trans `That is when this overcomes him ... the bloodlust for people.' (Lit. `People killing desire finishes him.')\Corpus{tci20130903-04}{RNA \#84-85}
	\label{ex582}
\end{exe}

Experiencer-object constructions differ in their basic word order from other clauses in that the \isi{experiencer}, the \isi{object}, comes first. This can be explained by the special semantics of the \isi{experiencer-object} construction, in which the most salient element is the \isi{experiencer}. However, most of the examples in this section do not include an overt noun phrase. One example from the corpus is given in (\ref{ex583}). Note that the speaker corrects himself in this example. He first uses the \isi{absolutive} (\emph{frfr}) `shiver', but then repeats the same noun in the \isi{ergative} (\emph{frfré}).

\begin{exe}
	\ex \emph{\textbf{nge fäth} frfr a \textbf{frfré} n \textbf{safum}.}\\
	\gll nge fäth frfr a frfr=é n sa\stem{fum}\\
	child \Dim{} shiver ah shiver=\Erg.\Nsg{} \Imn{} \Stsg:\Sbj>\Tsg.\Masc:\Obj:\Rpst:\Pfv/pull\\
	\trans `The small child was almost shivering' (Lit. `The shivers were about to pull him.')\Corpus{tci20130901-04}{YUK \#26}
	\label{ex583}
\end{exe}

Note that in (\ref{ex583}), the noun phrase is marked with the \isi{non-singular} \isi{ergative} (\emph{=é}), but the verb indexes a singular actor. This also occurs in (\ref{ex580}). All other examples in the corpus employ the \isi{singular} \isi{ergative} (\emph{=f}). I take this as evidence for the limited grammatical behaviour of property nouns. All property nouns \textendash{} like \emph{noku} `anger' in (\ref{ex580}) and \emph{frfr} `shiver' in (\ref{ex583}) \textendash{} evade cross-referencing in the verb prefix slot, usually a middle is used instead. Property nouns are only indexed in \isi{experiencer-object} constructions, though not in the prefix, but with a default \Stsg{} in the suffix (see \S{}\ref{propertynouns}).\\

The second domain of \isi{experiencer-object} constructions are bodily processes, like in (\ref{ex583}) above. Below in (\ref{ex584}-\ref{ex587}) a few more examples of this type are given.

\begin{exe}
	\ex \emph{zä zf fthé \textbf{thkarf} \textbf{yafiyokwa} ziyé.}\\
	\gll zä zf fthé thkar=f ya\stem{fiyok}wa z=\stem{yé}\\
	\Prox{} \Imm{} when hardness=\Erg.\Sg{} \Stsg{}:\Sbj>\Tsg{}.\Masc{}:\Obj{}:\Pst.\Ipfv{}/make \Prox{}=\Tsg{}.\Masc{}:\Nonpast{}.be\\
	\trans `That is when it got stuck right here.' (Lit. `Hardness made it.')\\\Corpus{tci20120922-09}{DAK \#18}
	\label{ex584}
\end{exe}
\begin{exe}
	\ex \emph{\textbf{nzä} \textbf{sukufa zürnf wortmakwr}.}\\
	\gll nzä sukufa zürn=f wo\stem{rtmak}wr kwan=en\\
	\Fsg.\Abs{} tobacco smoke=\Erg.\Sg{} \Stsg:\Sbj>\Fsg:\Obj:\Nonpast:\Ipfv/cut throat=\Loc{}\\
	\trans `The tobacco is very strong.' (Lit. `Tobacco smoke cuts me.')\Corpus{overheard}{}
	\label{ex585}
\end{exe}
\begin{exe}
	\ex \emph{\textbf{nzrmf wortmakwr} kwanen.}\\
	\gll nzrm=f wo\stem{rtmak}wr kwan=en\\
	bitterness=\Erg.\Sg{} \Stsg:\Sbj>\Fsg:\Obj:\Nonpast:\Ipfv/cut throat=\Loc{}\\
	\trans `It is very sour.' (Lit. `Bitterness cuts me.')\Corpus{overheard}{}
	\label{ex586}
\end{exe}
\begin{exe}
	\ex \emph{watik nzfrä ... \textbf{efothf nfariwr}.}\\
	\gll watik nzf\stem{rä} (.) efoth=f n\stem{fari}wr\\
	enough \Fpl:\Sbj:\Rpst:\Ipfv/be (.) sun=\Erg.\Sg{} \Stsg:\Sbj>\Fpl:\Obj:\Nonpast:\Ipfv/dry\\
	\trans `We have done enough ... We are burning in the sun.' (Lit. `The sun dries us.')\Corpus{tci20111119-03}{ABB \#200}
	\label{ex587}
\end{exe}

\subsection{Cognate and pseudo-cognate object constructions}\label{pseudocognate}

Cognate objects are a common phenomenon in Komnzo. Examples (\ref{ex731}-\ref{ex730}) contain a nominalised \isi{verb} and an inflected verb. In all three examples, the \isi{nominalisation} and the inflected verb form are of the same lexeme. Hence, (\ref{ex731}) translates literally as `I tell them the telling'. The infelcted verb indexes the indirect object (\Stpl) and as other \isi{ditransitive} verbs, \emph{trikasi} is the direct \isi{object} of the verb.

\begin{exe}
	\ex \emph{nze ane \textbf{trik}asi ä\textbf{trik}wé.}\\
	\gll nze ane trik-si ä\stem{trik}wé\\
	\Fsg.\Erg{} \Dem{} tell-\Nmlz{} \Fsg:\Sbj>\Stpl:\Io:\Nonpast:\Ipfv/tell\\
	\trans `I tell them the story.' (Lit. `I tell them the telling.')\Corpus{tci20111119-03}{ABB \#161}
	\label{ex731}
\end{exe}

There is an analytical problem with verbs which occur in the \isi{middle} template. Example (\ref{ex730}) translates literally as `He laughs the laughter' or as `He laughter-laughs'. The \isi{middle} template used in (\ref{ex729}) and (\ref{ex730}) only indexes the \isi{subject} argument, not the \isi{object}. Because of this, it cannot be determined whether the nominalisations \emph{maikasi} `washing' and \emph{borsi} `laughing' function as objects or whether they function predicatively. We will see below that a predicative function is a possible analysis in some cases. From this perspective, \isi{cognate object}s and predicative \isi{nominal}s in \isi{light verb} constructions can be portrayed as contiguous phenomena. Light verb constructions are described in the following section (\S{}\ref{lightverb}).

\begin{exe}
	\ex \emph{\textbf{maik}asi bä ŋa\textbf{mayuk}wro.}\\
	\gll maik-si bä ŋa\stem{maik}wro\\
	wash-\Nmlz{} \Med{} \Sg:\Sbj:\Nonpast:\Ipfv:\Andat/wash\\
	\trans `I will wash there.' (Lit. `I washing-wash.')\Corpus{tci20130823-06}{STK \#53}
	\label{ex729}
\end{exe}
\begin{exe}
	\ex \emph{\textbf{bor}si ŋa\textbf{bor}wr.}\\
	\gll borsi ŋa\stem{bor}wr\\
	laugh-\Nmlz{} \Stsg:\Sbj:\Nonpast:\Ipfv/laugh\\
	\trans `He laughs.' (Lit. `He laughs the laughing.')\Corpus{tci20111004}{TSA \#128}
	\label{ex730}
\end{exe}

A second problem is that many verbs lack regular nominalisations, which are formed with the suffix \emph{-si}. These verbs use a common noun as in example (\ref{ex564}) below. The adjective \emph{kwosi} `dead' functions adverbially and adds the meaning of a deep sleep. The noun \emph{etfth} `sleep', however, is semantically fully included in the meaning of the verb \emph{rug-} `sleep', just like the regular \isi{nominalisation} \emph{borsi} `laugh' is included the inflected verb in (\ref{ex730}) above. As a consequence, \emph{etfth} is optional and the sentence would be grammatical without it. Note that the same is true examples (\ref{ex731}-\ref{ex730}).

\begin{exe}
	\ex \emph{fi \textbf{etfth} kwosi sfrugrm.}\\
	\gll fi etfth kwosi sf\stem{rugr}m\\
	\Third.\Abs{} sleep dead \Tsg.\Masc:\Sbj:\Pst:\Dur/sleep\\
	\trans `He was sleeping soundly.' (Lit. `He was dead sleep sleeping.')\\\Corpus{tci20120904-02}{MAB \#98}
	\label{ex564}
\end{exe}

For want of a better term, I label examples like (\ref{ex564}) `\isi{pseudo-cognate object}' constructions. They are unlike cognate objects because the verb stem and the \isi{nominal} element are formally not related. Other examples are \emph{rnzür-} `dance, sing' and \emph{wath} `dance (n), song' and \emph{-nor} `shout, emit sound' and \emph{kwan} `shout (n)'. Although the verb stem and noun are not cognate, distributional evidence shows that they stand in the same relationship as an inflected verb and the corresponding regular \isi{nominalisation} with \emph{-si}. For example, the phasal verb \emph{bthaksi} `finish' takes the noun \emph{wath} `dance (n), song' to mean `finish singing'. This is because there is no regular \isi{nominalisation} available for the verb \emph{rnzür-} `dance, sing'.

The noun in these constructions is not always redundant. For example, it can be modified as the \isi{head} of a compound, thereby modifying the predicate. In (\ref{ex565}) the noun \emph{etfth} `sleep' occurs in a compound modified by \emph{efoth} `day' indicating that the speaker was sleeping during the day.

\begin{exe}
	\ex \emph{\textbf{efoth etfth} kwofrugrm e zizi.}\\
	\gll efoth etfth kwof\stem{rugr}m e zizi\\
	day sleep \Fsg:\Sbj:\Pst:\Dur/sleep until afternoon\\
	\trans `I was sleeping during the day until the afternoon.' (Lit. `I was day-sleep sleeping.')\Corpus{tci20111119-03}{ABB \#31}
	\label{ex565}
\end{exe}

This kind of predicate modification is developed to varying degrees. The best example is the \isi{intransitive} verb \emph{nor-} `shout, emit a sound', which again lacks an infinitive and instead \emph{kwan} `shout (n), call' is used. Hence, \emph{kwan yanor} `He shouts the shout' or `He emits the shout' is a common expression. Komnzo has a long list of ideophones, which express auditory sensations (\S{}\ref{ideophonesec}). All of these enter into compounds of the type \isi{ideophone} + \emph{kwan} as in \emph{sö kwan} `sound of wallabies grunting' or \emph{nzam kwan} `the sound of smacking one's lips'. Most auditory sensations are expressed in this construction with the verb \emph{nor-}. In example (\ref{ex732}), the gurgling sound of a headhunter's victim is described.

\begin{exe}
	\ex \emph{grr kwannzo fobo zwanorm.}\\
	\gll grr kwan=nzo fobo zwa\stem{nor}m\\
	rasping.sound shout=\Only{} \Dist{}.\All{} \Tsg.\F:\Sbj:\Pst:\Dur/shout\\
	\trans `She was just gurgling.' (Lit. `She was shouting/emitting only the rasping sound.')\Corpus{tci20111119-01}{ABB \#154}
	\label{ex732}
\end{exe}

Example (\ref{ex733}) comes from a hunting trip, where I was instructed to imitate the sound of a jumping wallaby (\emph{bübü kwan}) by hitting the ground with a thick stick.

\begin{exe}
	\ex \emph{bübü kwan gnanoré!}\\
	\gll bübü kwan gna\stem{nor}é\\
	thumping.sound shout \Ssg:\Sbj:\Imp:\Ipfv/shout\\
	\trans `You must beat the ground!' (Lit. `You must shout/emit the thumping sound.')\Corpus{overheard}{}
	\label{ex733}
\end{exe}

Lastly, the verb can be modified by using a different noun. This is a marginal pattern, and I can give only two examples. Instead of \emph{kwan}, one can use the noun \emph{frk} `blood' with verb \emph{nor-} `shout' to express that someone is bleeding as in (\ref{ex734}), which comes from the description of a picture card.

\begin{exe}
	\ex \emph{ŋare frk neba komnzo wänor.}\\
	\gll ŋare frk neba komnzo wä\stem{nor}\\
	woman blood opposite only \Tsg.\F:\Sbj:\Nonpast:\Ipfv/shout\\
	\trans `The woman is only bleeding on the other side.'\Corpus{tci20111004}{RMA \#402}
	\label{ex734}
\end{exe}

The second example is the noun \emph{wanzo} `dream' which can be used with \emph{rug-} `sleep' (instead of \emph{etfth} `sleep (n)'). In example (\ref{ex735}), the speaker talks about the mythological significance of the bird of paradise, when it appears in one's dream.

\begin{exe}
	\ex \emph{... ythamama wanzo fthé nzrarugr.}\\
	\gll (.) ythama=ma wanzo fthé nzra\stem{rugr}\\
	(.) bird.of.paradise=\Char{} dream when \Stsg:\Sbj:\Irr:\Ipfv/sleep\\
	\trans `... when you are dreaming of the bird of paradise.'\Corpus{tci20120817-02}{ABB \#29}
	\label{ex735}
\end{exe}

There are a handful of (\isi{intransitive}) verbs for which pseudo-cognate constructions are possible, even though there is a regular \isi{nominalisation} with \emph{-si} available. For example, \emph{bznsi} `work (v.i.)' can occur together with \emph{znsä} `work (n)'. Another example is \emph{mthizsi} `suffer' which can occur with \emph{zi} `pain' as in example (\ref{ex736}).

\begin{exe}
	\ex \emph{zi swathizrm ... ekri zi ... kofä ysma.}\\
	\gll zi swa\stem{thi}zrm (.) ekri zi (.) kofä ys=ma\\
	pain \Tsg.\Masc:\Sbj:\Pst:\Dur/suffer (.) flesh pain (.) fish thorn=\Char\\
	\trans `He was in pain ... body pain ... from the fish spike.'\Corpus{tci20100905}{ABB \#91-93}
	\label{ex736}
\end{exe}

We have seen above that cognate and pseudo-cognate constructions are similar to \isi{light verb} constructions in that a \isi{nominal} element contributes to the meaning of the predicate. They are markedly different in the degree of modification, because light verbs are much more general in their semantics (\emph{rä-} `do', \emph{fiyoksi} `make', \emph{ko-} `become'). It might be best to view this as a cline: on one end of the spectrum we have \isi{cognate object} constructions, where the \isi{nominalisation} of the verb occurs together with the same verb as in (\ref{ex731}-\ref{ex730}). On the other end of the spectrum we have \isi{light verb} constructions, where the \isi{nominal} element not only carries most of the meaning of the predicate, but it always differs formally from the verb. Light verbs are described in the next section.

\subsection{Light verb constructions}\label{lightverb}

There are number of light verbs in Komnzo. These are \emph{rä-} `do', \emph{ko-} `become', \emph{fiyoksi} `make' and the two phasal verbs \emph{thkäfsi} `start' and \emph{bthaksi} `finish'. The first two are interesting from a lexical perspective. The \isi{light verb} \emph{rä-} is build from the same stem as the \isi{copula}. In a prefixing template this stem means `be', but in an ambifixing template it means `do'. The second stem \emph{ko-} only occurs in ambifixing templates, where it can mean `speak' or `become'. Although these are only statistical tendencies, in the middle template \emph{ko-} usually means `become', where in a \isi{transitive} template it usually means `speak'.\\

The \isi{light verb} `do' is usually used in the middle template indexing only the \isi{subject} argument. A very frequent collocation is with \emph{fam} `thought', thus, literally: `do thoughts' means `think'. Examples (\ref{ex590}) and (\ref{ex591}) are taken from a picture stimulus task. Note that \emph{fam} is not indexed in the \isi{verb} form, even if the \isi{light verb} indexes an \isi{object}. In (\ref{ex591}) \emph{fam} functions predicatively, and a literal translation of `He thinks of her' is `He thought-does her'.

\begin{exe}
	\ex \emph{wati, ane fof yamnzr \textbf{fam ŋarär}.}\\
	\gll wati ane fof ya\stem{m}nzr fam ŋa\stem{rä}r\\
	then \Dem{} \Emph{} \Tsg.\Masc:\Sbj:\Nonpast:\Ipfv/sit thought \Stsg:\Sbj:\Nonpast:\Ipfv/do\\
	\trans `Okay, this one is sitting. He is thinking.'\Corpus{tci20111004}{RMA \#133}
	\label{ex590}
\end{exe}
\begin{exe}
	\ex \emph{\textbf{zane emoth fam wrär} anema yatrikwr nafan}.\\
	\gll zane emoth fam w\stem{rä}r ane=ma ya\stem{trik}wr nafan\\
	\Dem:\Prox{} girl thought \Stsg:\Sbj>\Tsg.\F:\Obj:\Nonpast:\Ipfv/do \Dem=\Char{} \Stsg:\Sbj>\Tsg.\Masc:\Io:\Nonpast:\Ipfv/tell \Tsg.\Dat\\
	\trans `He thinks of that girl and he tells him about her.'\Corpus{tci20111004}{RMA \#52}
	\label{ex591}
\end{exe}

This is a general feature of light verbs. They require a \isi{nominal} element which functions predicatively. Hence, we find predicative \isi{nominal}s in both intransitive (\ref{ex590}) and \isi{transitive} structures (\ref{ex591}). In these examples, the predicative \isi{nominal} was the noun \emph{fam}, but very often property nouns are used for this function, especially property nouns with more event-oriented semantics. In example (\ref{ex592}), the speaker remarks that his dogs have disappeared. The meaning of disappearing is expressed by the \isi{property noun} \emph{ofe} `absent/absence'.

\begin{exe}
	\ex \emph{fi kafar mane erera näbi ane \textbf{ofe ŋarerath}.}\\
	\gll fi kafar mane e\stem{rä}ra näbi ane ofe ŋa\stem{rä}rath.\\
	but big which \Stpl:\Sbj:\Pst:\Ipfv/be one \Dem{} absent \Stpl:\Sbj:\Pst:\Ipfv/do\\
	\trans `As for the big dogs, they disappeared for good.'\Corpus{tci20111119-03}{ABB \#70}
	\label{ex592}
\end{exe}

The \isi{light verb} \emph{ko-} `become' shows a similar behaviour. It can appear with \isi{nominal}s like in (\ref{ex593}) with the adjective \emph{kafar} `big'. But often `become' occurs with property nouns which function predicatively. In (\ref{ex594}), the \isi{property noun} \emph{wefwef} `excited/excitement' contributes most of the meaning of the event.

\begin{exe}
	\ex \emph{wati fi zena ngemär ... \textbf{kafar} z \textbf{zäkor}.}\\
	\gll wati fi zena nge=mär (.) kafar z zä\stem{kor}\\
	then \Third.\Abs{} today child=\Priv{} (.) big \Iam{} \Sg:\Sbj:\Rpst:\Pfv/become\\
	\trans `Well, today she hase become already old without (having) children.'\\\Corpus{tci20120814}{ABB \#214-215}
	\label{ex593}
\end{exe}
\begin{exe}
	\ex \emph{``Daddy skri, bun ane fof yé. be ane sawob.'' watik skri ane \textbf{wefwefnzo kräkor}.}\\
	\gll daddy skri bun ane fof \stem{yé} be ane sa\stem{wob} watik skri ane wefwef=nzo krä\stem{kor}\\
	father skri \Ssg.\Dat{} \Dem{} \Emph{} \Tsg.\Masc:\Nonpast:\Ipfv/be \Ssg.\Erg{} \Dem{} \Ssg:\Sbj>\Tsg.\Masc:\Imp:\Pfv/eat then skri \Dem{} excited=\Only{} \Stsg:\Sbj:\Irr:\Pfv/become\\
	\trans ```Daddy Skri, this one is for you. You eat this one.'' Well, Skri got excited!'\\\Corpus{tci20120922-25}{ALK \#24-25}
	\label{ex594}
\end{exe}

The \isi{light verb} `become' together with the \isi{property noun} \emph{miyatha} `knowledge' is used to express coming into the state of knowing something, literally `become knowledge(able)'. In example (\ref{ex595}) a man, who fell off a coconut palm in an attempt to steal palm wine, is badly insulted. The imperative \emph{miyatha käkor} can be translated as both `you know it!' or `you feel it!'.

\begin{exe}
	\ex \emph{fof nrä! \textbf{miyatha käkor}! buŋame zakiyar!}\\
	\gll fof n\stem{rä} miyatha kä\stem{kor} bu-ŋame za\stem{kiyar}\\
	\Emph{} \Ssg:\Sbj:\Nonpast:\Ipfv/be knowledgeable \Ssg:\Sbj:\Imp:\Pfv/become \Ssg.\Poss-mother \Ssg:\Sbj>\Tsg:\F:\Imp:\Pfv/copulate\\
	\trans `It is you! You feel it now! Fuck your mother!'\Corpus{tci20120904-01}{MAB \#95}
	\label{ex595}
\end{exe}

Example (\ref{ex596}) is about the \emph{tütü} bird (Pheasant Coucal), who used to be the custodian of fire before people knew about it. In the example, they find out about the bird's secret. Note that the \isi{light verb} `become' indexes the \emph{tütü} bird (\Tsg.\F). Thus, the predicative nominal \emph{miyatha} `knowledgeable' in the \isi{light verb} construction can be used with an intransitive (\ref{ex595}) or transitive sense (\ref{ex596}).

\begin{exe}
	\ex \emph{nä kayé ... \textbf{miyatha wkonzath}. ``oh budben mni rä fof''}\\
	\gll nä kayé (.) miyatha w\stem{ko}nzath oh budben mni \stem{rä} fof\\
	\Indf{} yesterday (.) knowledgeable \Stpl:\Sbj>\Tsg.\F:\Obj:\Pst:\Ipfv/become oh \Ssg.\Loc{} fire \Tsg.\F:\Sbj:\Nonpast:\Ipfv/be \Emph{}\\
	\trans `One day ... they found out about her. ``Oh, so the fire is really with you.'''\\\Corpus{tci20131008}{KAB \#10-11}
	\label{ex596}
\end{exe}

The verb \emph{fiyoksi} `make' can occur as a proto-typical \isi{transitive} verb without the ``semantic assistance'' of a predicative nominal. However, it commonly occurs as a \isi{light verb}. In example (\ref{ex597}), we find two occurences of \emph{fiyoksi}. The first token indexes \emph{zrin} `problem, burden' (\Tsg.\F) as its object argument, and \emph{fiyoksi} can be translated as `create'. The second token of \emph{fiyoksi} indexes the subject. In the latter, the predicative nominal \emph{durua} `help' contributes most of the semantic content of the predicate.

\begin{exe}
	\ex \emph{nzä nima ``bone zrin rä bone nagayf \textbf{ane} \textbf{zrin} \textbf{zwafiyokwr} keke kwa monme \textbf{durua} \textbf{ŋafiyokwre}"}\\
	\gll nzä nima bone zrin \stem{rä} bone nagay=f ane zrin zwa\stem{fiyok}wr keke kwa mon=me durua ŋa\stem{fiyok}wre\\
	\Fsg.\Abs{} \Quot{} \Ssg.\Poss{} problem \Tsg.\F:\Sbj:\Nonpast:\Ipfv/be \Ssg.\Poss{} child=\Erg{} \Dem{} problem \Stsg:\Sbj>\Tsg.\F:\Obj:\Rpst:\Ipfv/make \Neg{} \Fut{} how=\Ins{} help \Fpl:\Sbj:\Nonpast:\Ipfv/make\\
	\trans `I said: ``This is your problem. Your child has created this problem. We will not help.'''\Corpus{tci20120922-24}{STK \#22}
	\label{ex597}
\end{exe}

Analogous to the other light verbs, \emph{fiyoksi} can be used in a \isi{transitive} structure. In example (\ref{ex598}), an infamous sorcerer is annoyed by a few other men. The main semantic contribution to the event comes from the \isi{property noun} \emph{thythy} `nuisance', while the \isi{object} indexed in the \isi{light verb} is the sorcerer (\Tsg.\Masc).

\begin{exe}
	\ex \emph{wati \textbf{thythy} zä zf \textbf{swafiyokwrmth}.}\\
	\gll wati thythy zä zf swa\stem{fiyok}wrmth\\
	then nuisance \Prox{} \Imm{} \Stpl:\Sbj>\Tsg.\Masc:\Obj:\Pst:\Dur/make\\
	\trans `Then, they were annoying him here.'\Corpus{tci20131013ß02}{ABB \#59}
	\label{ex598}
\end{exe}

The two phasal verbs usually take nominalised verbs as their complements (see \S{}\ref{phasalcomplements}), but they can also be supplemented by property nouns with more event-oriented semantics. Hence, they exhibit the same double life as full verbs and light verbs as \emph{fiyoksi}. Two examples of \emph{thkäfsi} `start' functioning as a \isi{light verb} are given below. In (\ref{ex599}) a man is trying to enter the house in which two children are hiding. The phasal verb indexes the two children, while the semantic content of the event comes solely from the \isi{property noun} \emph{zirkn} `persistence'.

\begin{exe}
	\ex \emph{wati zänfrefa yanyak \textbf{nagayé} kma n \textbf{zirkn} \textbf{thrathkäf} ... \textbf{zirkn}.}\\
	\gll wati zän\stem{fref}a yan\stem{yak} nagayé kma n zirkn thra\stem{thkäf} (.) zirkn\\
	then \Sg:\Sbj:\Pst:\Pfv/come.up \Tsg.\Masc:\Sbj:\Nonpast:\Ipfv:\Venit/walk children \Pot{} \Imn{} persistence \Stsg:\Sbj>\Stdu:\Obj:\Irr:\Pfv/start (.) persistence\\
	\trans `Then, he came up from the river, he walked. He was about to start hassling the two children ... hassling (them)'\Corpus{tci20100905}{ABB \#111}
	\label{ex599}
\end{exe}

In example (\ref{ex600}), a malevolant spirit is trying to persuade a man to stay the night in her house. Again, the \isi{property noun} \emph{garamgaram} `sweet-talk' expresses most of the semantics of the event.

\begin{exe}
	\ex \emph{\textbf{garamgaram} srethkäf.}\\
	\gll {garamgaram} sre\stem{thkäf}\\
	sweet.talk \Stsg{}:\Sbj{}>\Tsg{}.\Masc{}:\Obj{}:\Irr{}.\Pfv{}/start\\
	\trans `She started sweet-talking him.'\Corpus{tci20120901-01}{MAK \#88}
	\label{ex600}
\end{exe}

As I have shown above, that light verbs (\emph{rä-} `do', \emph{ko-} `become', \emph{fiyoksi} `make', \emph{thkäfsi} `start' and \emph{bthaksi} `finish') require semantic assistance from \isi{nominal} predicates. However, \isi{nominal} predicates can be found with other verbs, i.e. full verbs. In the following examples, the concepts of `being concentrated' (\ref{ex601}) and `being locked in' (\ref{ex602}) are expressed by the property nouns \emph{mogu} `concentration' and \emph{ttw} `inertia' respectively. Both meanings could be expressed with light verbs, for example (\ref{ex601}) could be expressed as \emph{mogu ŋaräré} `I am concentrating' (Lit. `I am concentration-doing'). The two examples below employ full verbs instead, which should be seen as more idiosyncratic way of speaking.

\begin{exe}
	\ex \emph{biskar mnifnzo \textbf{mogu kwofkämgwrm}}\\
	\gll biskar mni=f=nzo mogu kwof\stem{kämg}wrm\\
	cassava cooking=\Erg.\Sg=\Only{} concentration \Stsg:\Sbj>\Fsg:\Obj:\Pst:\Dur/block\\
	\trans `Cooking the cassava took all my attention. (Lit. `Cassava cooking concentration blocked me.')'\\\Corpus{tci20111119-03}{ABB \#79}
	\label{ex601}
\end{exe}
\begin{exe}
	\ex \emph{\textbf{ttw zwermänth}. wati fobo thufnzrmth}\\
	\gll ttw zwe\stem{rmän}th wati fobo thu\stem{fn}nzrmth\\
	inertia \Stpl:\Sbj>\Tsg.\F:\Obj:\Iter/close then \Dist.\All{} \Stpl:\Sbj>\Stpl:\Obj:\Pst:\Dur/kill\\
	\trans `They always closed off (the village). Then, they were killing them.'\\\Corpus{tci20120818}{ABB \#46-47}
	\label{ex602}
\end{exe}

I point out in \S{}\ref{verbs} that verbs are considered to be a closed \isi{word class} in Komnzo. Part of the argumentation is based on the observation that loanwords, which are verbs in the donor language, commonly end up as property nouns in a \isi{light verb} construction. One such example was shown above in (\ref{ex597}) with the \isi{property noun} \emph{durua} `help', which is a \isi{transitive} verb in Motu (\citealt[61]{Turnerlister:1935motu}). Below, two examples with \ili{English} loans are given. In example (\ref{ex603}) the verb \emph{fiyoksi} indexes the \isi{object} \emph{zokwasi} `words' (\Stpl), while the \isi{loanword} \emph{senis} `change' expresses most of the semantics (Lit. `I will not change-make the words'). In example (\ref{ex604}) the middle verb \emph{rä-} `do' is supplemented by the \ili{English} loan \emph{zek} `check' (Lit. `I check-do for water').

\begin{exe}
	\ex \emph{\textbf{zokwasi} ke kwa \textbf{senis} \textbf{thräfiyothé}.}\\
	\gll zokwasi keke kwa senis thrä\stem{fiyoth}é\\
	words \Neg{} \Fut{} change \Fsg:\Sbj>\Stpl:\Obj:\Irr:\Pfv/make\\
	\trans `I will not change my promise.'\Corpus{tci20121019-04}{ABB \#226}
	\label{ex603}
\end{exe}
\begin{exe}
	\ex \emph{kränrsöfthé mäbri ttfö ... \textbf{nor} bobo \textbf{zek} \textbf{kräré} ... keke}\\
	\gll krän\stem{rsöfth}é mäbri ttfö (.) no=r bobo zek krä\stem{r}é (.) keke\\
	\Fsg:\Sbj:\Irr:\Pfv/descend mäbri creek (.) water=\Purp{} \Med.\All{} check \Fsg:\Sbj:\Irr:\Pfv/do (.) \Neg\\
	\trans `I went down to the creek in Mäbri to check for water, but no (water).'\\\Corpus{tci20130903-03}{MKW \#146-147}
	\label{ex604}
\end{exe}

For situations of language contact, Heine and Kuteva describe how minor patterns in a language can become a major patterns (\citeyear[44]{Heine:2005wp}). It is clear that \isi{light verb} constructions are not a minor pattern in Komnzo. However, it seems evident that with more (\isi{verb}) loans entering the language, \isi{light verb} constructions will become even more widely used.

\section{Questions}\label{questions}

Content questions in Komnzo are formed by replacing the respective noun phrase with an \isi{interrogative}. Word order may or may not be changed for pragmatic purposes. As content questions are always pragmatically motivated, the element which is asked about is automatically focussed. Therefore, the \isi{interrogative} is often found in fronted position, but fronting is not part of \isi{question} formation. Example (\ref{ex701}) shows an example with the \isi{interrogative} \emph{ra} `what'.

\begin{exe}
	\ex \emph{nafafis zräs ``be ranzo kayé thwanfiyokwr?''}\\
	\gll nafa-fis zrä\stem{s} be ra=nzo kayé thwan\stem{fiyok}wr\\
	\Third.\Poss-husband \Stsg:\Sbj>\Tsg.\F:\Obj:\Irr:\Pfv/ask \Ssg.\Erg{} what=\Only{} yesterday \Stsg:\Sbj>\Stpl:\Obj:\Rpst:\Ipfv:\Venit/make\\
	\trans `Her husband asked her: ``Just what have you done to them yesterday?'''\\\Corpus{tci20120901-01}{MAK \#163}
	\label{ex701}
\end{exe}

Example (\ref{ex713}) shows an example where the \isi{interrogative} occurs inside a complex \isi{noun phrase} `whose sister'. Note that the \isi{noun phrase} which contains the \isi{interrogative} has been fronted for pragmatic reasons. This is an example of a rethorical \isi{question}, because it came up in a discussion about the type of punitive actions one would launch against one's brothers in-laws.

\begin{exe}
	\ex \emph{``mafane emoth be zufnzrm?'' nima fof skonzé}\\
	\gll maf=ane emoth be zu\stem{fn}nzrm nima fof s\stem{ko}nzé\\
	who=\Poss{} sister \Ssg.\Erg{} \Stsg:\Sbj>\Tsg.\F:\Obj:\Pst:\Dur/hit \Quot{} \Emph{} \Ssg:\Sbj>\Tsg.\Masc:\Obj:\Imp:\Ipfv/speak\\
	\trans ```Whose sister were you beating?'' that is what you must say to him.'\\\Corpus{tci20120805-01}{ABB \#219}
	\label{ex713}
\end{exe}

Polar questions are often structurally identical to indicative statement, but the have a rising intonation contour as in (\ref{ex702}) and (\ref{ex703}).  Additionally the iamitive \isi{particle} \emph{z} `already' can be used even though the verb is in the non-past (\ref{ex702}).

\begin{exe}
	\ex \emph{zbär bä zagrwä ämnzro. z wanrizrth?}\\
	\gll zbär bä zagr=wä ä\stem{m}nzro z wan\stem{riz}rth\\
	night \Second.\Abs{} far=\Emph{} \Stpl:\Sbj:\Nonpast:\Ipfv:\Andat/sit \Iam{} \Stpl:\Sbj>\Fsg.\Io:\Nonpast:\Ipfv:\Venit/hear\\
	\trans `You are sitting far away. Can you hear me?' (Lit. `You hear my (words) already?')\Corpus{tci20121019-04}{SKK \#9}
	\label{ex702}
\end{exe}
\begin{exe}
	\ex \emph{ane wri kambeyé kma n yrärth ``kwa kräznobe?'' naf ekonzr ``keke''}\\
	\gll ane wri kambe=yé kma n y\stem{rä}rth kwa krä\stem{znob}e naf e\stem{ko}nzr keke\\
	\Dem{} intoxication man=\Erg.\Nsg{} \Pot{} \Imn{} \Stpl:\Sbj>\Tsg.\Masc:\Obj:\Nonpast:\Ipfv/do \Fut{} \Fpl:\Sbj:\Irr:\Pfv/drink \Tsg.\Erg{} \Stsg:\Sbj>\Stpl:\Obj:\Nonpast:\Ipfv/speak \Neg\\
	\trans `These drunkards are trying (to convince him): ``Will we drink?'' He says to them ``No'''\Corpus{tci20111004}{RMA \#509}
	\label{ex703}
\end{exe}

Alternative questions are formed by a disjunctive \isi{coordination} with \emph{o} `or'. In (\ref{ex704}), the alternatives are expressed by two clauses, and in (\ref{ex706}) by two noun phrases.

\begin{exe}
	\ex \emph{fam kwarärmth ``kwa ywokrakwr o kwa ŋabrüzr?''}\\
	\gll fam kwa\stem{rä}rmth kwa y\stem{wokrak}wr o kwa ŋa\stem{brüz}r\\
	thought \Stpl:\Sbj:\Pst:\Dur/do \Fut{} \Tsg.\Masc:\Sbj:\Nonpast:\Ipfv/float or \Fut{} \Stsg:\Sbj:\Nonpast:\Ipfv/submerge\\
	\trans `They were thinking: ``Will it float or will it sink?'''\Corpus{tci20120929-02}{SIK \#31}
	\label{ex704}
\end{exe}
\begin{exe}
	\ex \emph{zokwasi fefeme natrikwé o markai zokwasime?}\\
	\gll zokwasi fefe=me na\stem{trik}wé o markai zokwasi=me\\
	language real=\Ins{} \Fsg:\Sbj>\Ssg:\Io:\Nonpast:\Ipfv/tell or {white man} language=\Ins\\
	\trans `Will I tell you (the story) in Komnzo or in English?' (Lit. `... in the real language or the white man's language?')\Corpus{tci20120901-01}{MAK \#1}
	\label{ex706}
\end{exe}

Question tags like \emph{o keke} `or not' can be added, which also receive a rising intonation.

\begin{exe}
	\ex \emph{kwa nm weto worär o keke?}\\
	\gll kwa nm weto wo\stem{rä}r o keke\\
	\Fut{} maybe joy \Stsg:\Sbj>\Fsg:\Obj:\Nonpast:\Ipfv/do or \Neg{}\\
	\trans `Maybe he will be happy towards me or not?'\Corpus{tci20111004}{RMA \#477}
	\label{ex705}
\end{exe}

\section{Negation}\label{negationclause}

At the \isi{clause} level, \isi{negation} is expressed periphrastically with the \isi{negator} \emph{keke} in preverbal position as in example (\ref{ex748}). See \S\ref{tamparticles} for more information on \emph{keke} and its variant \emph{kyo}.

\begin{exe}
	\ex \emph{nafanme emoth keke kränrit nzedbo.}\\
	\gll nafanme emoth keke krän\stem{rit} nzedbo\\
	\Tpl.\Poss{} girl \Neg{} \Stsg:\Sbj:\Irr:\Pfv:\Venit/cross.over \Fnsg.\All\\
	\trans `They will not exchange sisters with us.'\Corpus{tci20120814}{ABB \#319}
	\label{ex748}
\end{exe}

One exception is the prohibitive construction (see \S\ref{apprehensivem}). This construction consists of the \isi{potential} \isi{particle} \emph{kma}, the \isi{apprehensive} \isi{clitic} \emph{m=}, and the verb in the \isi{imperative}. The \isi{apprehensive} \isi{clitic} may attach either to the verb or to the \isi{potential} \isi{particle}. This construction is best translated into \ili{English} as `must not' as can be seen in example (\ref{ex749}). Note that the \isi{negator} cannot be included in this construction.

\begin{exe}
	\ex \emph{nznäbrimath ``bä kmam thiyaké! kafarnzo ni nyak!''.}\\
	\gll nznä\stem{brim}ath bä kma=m thi\stem{yak}é kafar=nzo ni n\stem{yak}\\
	\Stpl:\Sbj>\Fpl:\Obj:\Pst:\Pfv/return \Second.\Abs{} \Pot=\Appr{} \Spl:\Sbj:\Imp:\Ipfv/walk big=\Only{} \Fnsg{} \Fpl:\Sbj:\Nonpast:\Ipfv/walk\\
	\trans `They brought us back and said: ``You must not go! Only us big ones will go.'''\Corpus{tci20120904-02}{MAB \#232-233}
	\label{ex749}
\end{exe}

Negation at the level of the constituent can be expressed in a number of ways. The word \emph{matak} `nothing' is used to express non-existence, usually in a \isi{copula} \isi{clause}. This is shown in example (\ref{ex750}) where a man takes notice that he is alone in the village. \emph{Matak} can also be used in a non-verb \isi{predication}, for example \emph{nge matak} `(they were) no children'. Alternatively, any \isi{noun phrase} can be negated by using the \isi{privative} case marker \emph{=mär}. This is described in \S\ref{privcase}.

\begin{exe}
	\ex \emph{kabe matak erä nima z bramöwä kwafarkwrth.}\\
	\gll kabe matak e\stem{rä} nima z bramöwä kwa\stem{fark}wrth\\
	people nothing \Stpl:\Sbj:\Nonpast:\Ipfv:\Cop{} like.this \Iam{} all \Stpl:\Sbj:\Rpst:\Ipfv/set.off\\
	\trans `There are no people (here). All of them have already left.'\\\Corpus{tci20120901-01}{MAK \#77}
	\label{ex750}
\end{exe}

Negative indefinites expressing `none whatsoever' or `nothing at all' are constructed by adding the \isi{negator} \emph{keke} to an noun phrase that includes the \isi{indefinite} marker \emph{nä}. For example, \emph{nä kabe} means `some man' or `someone', but \emph{kabe nä keke} means `nobody at all'. Note that the \isi{indefinite} is always postposed in this construction. The same can be achieved by adding \emph{nä} to an \isi{interrogative} as in (\ref{ex765}). I describe this topic in more detail in \S\ref{indefnae}.

\begin{exe}
	\ex \emph{keke kwa ra nä zränzinth.}\\
	\gll keke kwa ra nä zrän\stem{zin}th\\
	\Neg{} \Fut{} what \Indf{} \Stpl:\Sbj>\Tsg.\F:\Io:\Irr:\Pfv:\Venit/put.down\\
	\trans `They will leave nothing for her.'\Corpus{tci20131004}{RMA \#9}
	\label{ex765}
\end{exe}