%!TEX root = ../main.tex
\chapter{Word classes} \label{cha:word classes}

In this chapter, I describe the major and minor word classes of Komnzo. I provide the necessary criteria to determine the \isi{word class} of a given lexical item based on its morphological possibilities, syntactic distribution and semantic content. This chapter contains detailed information on smaller word classes or subclasses which will not be discussed elsewhere in the grammar. For these, I list all known members for quick reference.

The eight word classes include \isi{nominal}s ({\S}\ref{nominals-sec}), verbs ({\S}\ref{verbs-sec}), adverbs ({\S}\ref{adverbs-sec}), particles ({\S}\ref{particles-sec}), clitics ({\S}\ref{clitics-sec}), connectives ({\S}\ref{connectives-sec}), ideophones (\S\ref{ideophones-sec}), and interjections ({\S}\ref{interjections-sec}). Nominals constitute a superclass comprising a variety of subclasses: nouns ({\S}\ref{nouns-sec}), property nouns ({\S}\ref{propertynouns-sec}), adjectives ({\S}\ref{adjectives-sec}), quantifiers and numerals ({\S}\ref{quantifiers-numerals-sec}), locationals ({\S}\ref{locationals-sec}), temporals ({\S}\ref{temporals-sec}), personal pronouns ({\S}\ref{personalpronouns-sec}), interrogatives ({\S}\ref{interrogatives-sec}), indefinites ({\S}\ref{indefinite-sec}), and demonstratives ({\S}\ref{demonstratives-sec}).

I categorise Komnzo word classes along a number of lines. The clearest distinction is between inflecting (\isi{nominal}s and verbs) and uninflecting word classes (all other). The distinction between open and closed word classes is more difficult to define. Only a few \isi{nominal} subclasses (nouns, property nouns, numerals) and interjections accept new members in the form of loanwords or neologisms. Although large in terms of members, verbs are not an open \isi{word class}. Major words classes are nouns, property nouns and verbs, each with more than 300 members in the current dictionary. All other word classes have less than 30 members and are considered minor classes.

\section{Nominals} \label{nominals-sec}

Nominals are the largest \isi{word class}, consisting of a number of subclasses. The largest are the open subclasses of nouns ({\S}\ref{nouns-sec}) and property nouns ({\S}\ref{propertynouns-sec}), which both readily accept borrowings from other languages, particularly \ili{English} and Motu. Adjectives ({\S}\ref{adjectives-sec}) constitute a minor, closed class. The \isi{nominal} superclass includes a number of other small, closed word classes. These are quantifiers and numerals ({\S}\ref{quantifiers-numerals-sec}), locationals ({\S}\ref{locationals-sec}), temporals ({\S}\ref{temporals-sec}), personal pronouns ({\S}\ref{personalpronouns-sec}), interrogatives ({\S}\ref{interrogatives-sec}) and demonstratives ({\S}\ref{demonstratives-sec}).

The unifying characteristic of \isi{nominal}s is their ability to serve as the host of case marking clitics. However, not all \isi{nominal} subclasses can take the full set of \isi{case} distinctions. For example, while nouns and personal pronouns are prototypical \isi{nominal}s and take all case, demonstratives, temporals, and locationals are more limited in the ability to receive case clitics.

\subsection[Overview of criteria]{Criteria for distinguishing between nouns, property nouns and adjectives} \label{criteria-nominals-sec}

Before addressing each subclass, it is necessary to give an overview of the distinction between nouns, property nouns and adjectives. The two main criteria involved are the ability to act as the \isi{head} of a \isi{noun phrase} and the ability to trigger agreement in both \isi{gender} and \isi{number}. Further criteria are the ability to enter into a \isi{possessive} construction, the possibility of taking the \isi{adjectivaliser} \emph{-thé} and the different functions of the \isi{instrumental} \isi{case} \emph{=me}. This section only lists the criteria. Examples are given in the following sections, which address each subclass in turn ({\S}\ref{nouns-sec}-\ref{adjectives-sec}).

Nouns and property nouns can act as the \isi{head} of a \isi{noun phrase}, whereas adjectives cannot. See {\S}\ref{headslot} for further discussion of headedness. An \isi{adjective} may be the only visible element of a \isi{noun phrase}, but this is possible only if the omitted \isi{head} can be established through context. This first criterion groups property nouns with nouns and singles out adjectives.

Agreement in \isi{gender} and \isi{number} is only triggered by nouns. Gender in Komnzo is covert ({\S}\ref{gender-system-sec}), and the agreement target for \isi{gender} is the 3\textsuperscript{rd} singular prefix of the verb. Number agreement is marked at various morphological sites on the \isi{verb} including the \isi{undergoer} prefix, the actor suffix, and the duality affix ({\S}\ref{number-subsec}). Adjectives fail to trigger \isi{gender} or \isi{number} agreement. Property nouns also fail to trigger \isi{gender} agreement, because they are not indexed in the prefix. However, property nouns trigger a default {\Sg} \isi{number} agreement in the suffix, for example in \isi{experiencer-object} constructions where a \isi{property noun} can be the \isi{stimulus} flagged with the \isi{ergative} \isi{case} ({\S}\ref{expobjconstr}). Nouns trigger both \isi{gender} and \isi{number} agreement. Hence, the criterion of agreement groups property nouns with adjectives and singles out nouns.

As far as the other criteria are concerned, \isi{possessive} constructions are only possible with nouns and property nouns and not with adjectives. The \isi{adjectivaliser} \emph{-thé} is common with nouns of a particular semantic field, i.e. nouns which can used to described a more general characteristic. For example, \emph{frk} `blood', \emph{nzafar} `sky' for deriving colour terms. The adjectivaliser is optional with property nouns, but ungrammatical with adjectives. The \isi{instrumental} \isi{case} marker \emph{=me} serves its prototypical function with nouns, but property nouns and adjectives function as adverbials when marked with the \isi{instrumental} \isi{case}. \tabref{nominals-overview-table} provides an overview of the criteria.

\begin{table}
\caption{Feature matrix for nominals} \label{nominals-overview-table}
	\begin{tabularx}{\textwidth}{lXXX}
		\lsptoprule  
			&{nouns}&{property nouns}&{adjectives}\\ \midrule
			gender agreement&+&--&--\\
			number agreement&+&--\super{a}&--\\
			head of NP&+&+&--\\
			possessive construction&+&+&--\\
			adjectivaliser \emph{-thé}&+&+/--&--\\
			{\Ins} case&instrument&adverbial&adverbial\\
		\lspbottomrule
			\multicolumn{4}{l}{\footnotesize \super{a}There is default number agreement ({\Sg}) in \isi{experiencer-object} constructions ({\S}\ref{expobjconstr})}
	\end{tabularx}	
\end{table}

\subsection{Nouns} \label{nouns-sec}

Nouns constitute a large, open class of lexical items which readily accepts new members by forming neologisms or adding loanwords from other languages. Nouns are typically referential and denote objects, locations, abstract notions, \isi{kinship} relations, and proper names.

Semantically nouns can be subdivided into common nouns, \isi{kinship} nouns, and proper nouns. Common nouns depict the natural world (\emph{no} `rain', \emph{ttfö} `creek', \emph{ymd} `bird') as well as artefacts (\emph{mnz} `house', \emph{nag} `grass skirt', \emph{kufraru} `bamboo flute') or abstract concepts (\emph{bthan} `magic', \emph{wath} `dance (n)', \emph{dradr} `taboo'). Common nouns are syntactically least restricted, i.e. they enter into most nominal constructions and can be marked for all cases compared to the other \isi{nominal} subclasses. Kinship nouns can intrinsically be specified for \isi{gender} (\emph{ŋafe} `father', \emph{ŋame} `mother') or be flexible as to which \isi{gender} is assigned (\emph{nane} `elder sibling', \emph{ngth} `younger sibling'). Many \isi{kinship} terms are self-\isi{reciprocal} (\emph{ŋäwi} `maternal uncle $\leftrightarrow$ sister's child', \emph{yamit} `exchange cousin $\leftrightarrow$ exchange cousin'). Kinship nouns frequently enter the close \isi{possessive} construction ({\S}\ref{closeposs}). Proper nouns consist of personal names and place names. Place names are always \isi{feminine} and they are often compounds made up of a plant name and the word \emph{zfth} `base, stem, reason' like in the \isi{place name} \emph{gani zfth} (`Endiandra brassii + base'). Proper nouns are hardly ever modified by demonstratives, quantifiers or adjectives.

Nouns are distinct from other \isi{nominal}s in being the only lexical items which trigger \isi{gender} agreement. The agreement target is the third \isi{person} singular prefix of the verb ({\S}\ref{gender-subsec}). The semantics of the \isi{gender} system is described in the following section ({\S}\ref{gender-system-sec}). Additionally, nouns trigger \isi{number} agreement, in this they resemble other \isi{nominal} subclasses such as pronouns. The agreement target for \isi{number} depends on the type of argument, but it involves three distinct verbal affix slots (the \isi{undergoer} prefix, the actor suffix, and the duality marker). The verb morphology will be laid out in chapter \ref{cha:verb morphology}, but we get a glimpse of the agreement system in examples (\ref{ex044}-\ref{ex047}).

Nominal \isi{number} marking takes place on the level of the \isi{noun phrase}, leaving aside the use of numerals. Nominal \isi{number} marking is underspecified for three reasons. First, only animates are marked for \isi{number}, especially humans. Example (\ref{ex494}) shows the \isi{allative} \isi{case} marker on several \isi{nominal}s, and only the \isi{animate} referents are marked for \isi{number}. Note that the spatial cases (\isi{locative}, \isi{allative}, \isi{ablative}) have special formatives for \isi{animate} referents ({\S}\ref{spatialcase}). Secondly, \isi{number} marking on the \isi{noun} only occurs when the respective \isi{noun phrase} is flagged with a \isi{case} marker. Thus, nouns out of syntactic context or \isi{noun phrases} in the \isi{absolutive} \isi{case}, which is \isi{zero}, have no \isi{nominal} \isi{number} marking. Thirdly, \isi{nominal} \isi{number} marking is based on a \isi{singular} versus \isi{non-singular} distinction.\footnote{The associative case is an exception. With animate referents it is used for the inclusory construction ({\S}\ref{inclusorycontruction}), and there the values are dual and plural, instead of singular and non-singular.} The full three-way distinction between \isi{singular}, \isi{dual} and \isi{plural} is encoded in the \isi{verb}. It follows that the majority of nouns or \isi{noun phrase}s are underspecified for \isi{number}, and for core \isi{case} arguments, \isi{number} is assigned morpho-syntactically via the agreement system of the verb.

\begin{exe}
	\ex \emph{wati \textbf{nzedbo} zanrifthath \textbf{mayawanmedbo} rouku \textbf{bänefo} ... \textbf{masufo}.}\\
	\gll wati nzedbo zan\stem{rifth}ath mayawa=nmedbo rouku bäne=fo (.) masu=fo\\
	then \Fnsg.{\All} \Stpl:\Sbj>\Tsg.\F:\Obj:\Pst:\Pfv/send mayawa=\All.\Anim.{\Nsg} rouku \Recog={\All} (.) masu={\All}\\
	\trans `Then they send the word to us ... to the Mayawas in Rouku ... to there ... to Masu.' \\\Corpus{tci20120814}{ABB \#34-35}
	\label{ex494}
\end{exe}

Nouns may undergo \isi{reduplication}, which signals plurality and/or non-prototypicality, as in \emph{yawiyawi} `money, coins' from \emph{yawi} `seed' or \emph{yamyam} `marks' from \emph{yam} `footprint'. An example is given in (\ref{ex503}) and (\ref{ex504}). Example (\ref{ex503}) shows the \isi{noun} \emph{znsä} `work', while the reduplicant \emph{znsäznsä} was often used for the kind of elicitation, recording and transcription work that I was doing (\ref{ex504}).

\begin{exe}
	\ex \emph{\textbf{znsä} kwabznwrme dagon fawr.}\\
	\gll znsä kwa\stem{bz}nwrme dagon faw=r\\
	work \Fpl:\Sbj:\Pst:\Dur/work food payment=\Purp\\
	\trans `We worked for food.' \Corpus{tci20120924-01}{TRK \#50}
	\label{ex503}
\end{exe}
\begin{exe}
	\ex \emph{thrma n kwot thräre bänema \textbf{znsäznsär} thwanyan.}\\
	\gll thrma n kwot thrä\stem{r}e bäne=ma znsä-znsä=r thwan\stem{yan}\\
	later {\Imn} properly \Fpl:\Sbj>\Stpl:\Obj:\Irr:\Pfv/do \Med={\Char} \Redup-work={\Purp} \Stdu:\Sbj:\Rpst:\Ipfv:\Venit/walk\\
	\trans `Later, we will get them out properly because you came for work.' \Corpus{tci20130907-02}{JAA \#251}
	\label{ex504}
\end{exe}

In order to derive adjectives, some nouns take the \isi{adjectivaliser} suffix \emph{-thé}. We can see this most clearly in the colour terms: \emph{kwayanthé} `white' from \emph{kwayan} `light' or \emph{frkthé} `red' from \emph{frk} `blood'. The productivity of \emph{-thé} is rather limited and there are a number of lexical items which show frozen morphology. For example, \emph{yfrsé} `black' from \emph{yfr} `Syzygium sp' (used for black paint) shows an irregular variant, \emph{-sé} instead of \emph{-thé}. For \emph{dbömsé} `blunt' there is no corresponding \isi{noun} without the suffix. The restrictions in terms of productivity can be explained by the presence of a class of property nouns to be discussed in {\S}\ref{propertynouns-sec}. There is an alternative strategy for deriving colour and shape adjectives. This involves the formation of a compound with the word \textit{woku} `skin' which takes the \isi{adjectivaliser} suffix. The Komnzo equivalent for \ili{English} `green' is expressed by \emph{wämne taga wokuthé} (lit. `tree leaf skin-like') or the translation of `round' is \emph{aki wokuthé} (lit. `moon skin-like'). An example of this strategy is given in (\ref{ex509}), where the speaker characterises a man as looking a bit `boyish'.

\begin{exe}
	\ex \emph{fi \textbf{sraksrak wokuthé} yara.}\\
	\gll fi srak-srak woku-thé ya\stem{r}a\\
	\Third.{\Abs} \Redup-boy skin-{\Adlzr} \Tsg.\Masc:\Sbj:\Pst:\Ipfv/be\\
	\trans `He was a bit boyish.' \Corpus{tci20131013-02}{ABB \#211}
	\label{ex509}
\end{exe}

All common nouns can serve as the host for \isi{case} clitics (\isi{ergative}, \isi{dative}, \isi{possessive}, \isi{locative}, \isi{allative}, \isi{ablative}, \isi{instrumental}, \isi{characteristic}, \isi{purposive}, \isi{associative}, \isi{proprietive}, \isi{privative}, \isi{similative}) or receive other \isi{nominal} morphology (\isi{exclusive}, \isi{emphatic}). As I describe in {\S}\ref{formfunccase}, \isi{case} markers operate at the level of the \isi{noun phrase}. Noun phrases headed by a \isi{noun} can function as arguments or adjuncts, as well as complements of the copula. This is illustrated by the \isi{ergative} and absolutive-marked arguments in example (\ref{ex044}).\footnote{The absolutive case is zero-marked in singular, and the non-singular formative \emph{-é} is rare throughout the corpus. In example (\ref{ex044}), the word \emph{garda} `canoe' is glossed with the absolutive case in brackets. Note that for most examples in this grammar, I do not gloss the absolutive if it is zero-marked. Exceptions are those examples, where the case value is important for the decription.} Example (\ref{ex045}) shows a locative-marked \isi{noun} which functions as an adjunct.

\begin{exe}
	\ex \emph{\textbf{brbrf} \textbf{garda} bifnza.}\\
	\gll brbr=f garda b=y\stem{fn}nza\\
	spirit=\Erg.{\Sg} canoe(\Abs) \Med=\Stsg:\Sbj>\Tsg.\Masc:\Obj:\Pst:\Ipfv/hit\\
	\trans `The spirit was hitting (against) the canoe there.' \Corpus{tci20120904-02}{MAB \#87}
	\label{ex044}
\end{exe}
\begin{exe}
	\ex \emph{\textbf{masun} ni fä nzwamnzrm.}\\
	\gll masu=n ni fä nzwa\stem{m}nzrm\\
	masu={\Loc} {\Fnsg} {\Dist} \Fpl:\Sbj:\Pst:\Dur/dwell\\
	\trans `We were staying in Masu over there.' \Corpus{tci20120821-02}{LNA \#100}
	\label{ex045}
\end{exe}

Nouns typically function as the \isi{head} of a \isi{noun phrase} or as the \isi{head} of a \isi{nominal} compound. Compounds are described in {\S}\ref{compounds-subsec}. Example (\ref{ex046}) shows the \isi{noun} \emph{waniwani} `picture, shadow' as the \isi{head} of the noun phrase modified by the \isi{demonstrative} \emph{zane} and the \isi{adjective} \emph{katan}. Nouns may act as modifiers within a noun phrase. In the \isi{nominal} compound in (\ref{ex047}) the two nouns act as \isi{head} (\emph{kam} `bone') and \isi{modifier} (\emph{tauri} `wallaby'). In the examples NPs are marked off by [].

\begin{exe}
	\ex \emph{fof zäbth zane katan \textbf{waniwani}.}\\
	\gll fof zä\stem{bth} [zane katan waniwani]\\
	{\Emph} \Stsg:\Sbj:\Rpst:\Pfv/finish \Dem:{\Prox} small {picture}\\
	\trans `This little movie is finished.' \Corpus{tci20120914}{RNA \#63}
	\label{ex046}
\end{exe}
\begin{exe}
	\ex \emph{ŋathayé \textbf{tauri kam} yanathrth.}\\
	\gll ŋatha=yé [tauri kam] ya\stem{na}thrth\\
	dog=\Erg.{\Nsg} wallaby bone \Stpl:\Sbj>\Tsg.\Masc:\Obj:\Nonpast:\Ipfv/eat\\
	\trans `The dogs are chewing a wallaby bone.' \Corpus{tci20120818}{ABB \#42}
	\label{ex047}
\end{exe}

\subsection{The semantics of the gender system}\label{gender-system-sec}

The \isi{gender} system is covert as there are no formal elements on a given \isi{noun} showing its \isi{gender}. Instead, the two categories, \isi{feminine} and \isi{masculine}, are shown in the verb prefix. Nouns have either fixed \isi{gender} (most nouns) or flexible \isi{gender} (kin terms, certain animals).

Animate nouns, for which sex can be determined easily, for example dogs, pigs, wallabies, and of course humans, are placed in the respective category. Words with fixed \isi{gender} allow us to set up some general semantic principles of classification. For example, elongated, big objects are usually \isi{masculine}, while small round objects are \isi{feminine}. Lexemes related to place and land are usually \isi{feminine}. Abstract concepts or nominalised verbs are usually \isi{feminine}. Most fish species are \isi{masculine}, with the exception of the numerous catfish species, which are all \isi{feminine}. Other species, like birds, are much more varied. Speakers often use the phrases \emph{srak yé} `it's a boy' or \emph{matma rä} `it's a girl', when being asked about the gender category of a particular word. \tabref{gender-table} gives an overview of the semantic characteristics and lists some examples as well as exceptions.

A number of words always occur in \isi{plural}, which means that no \isi{gender} is triggered in the agreement target. Only some of them are clear mass nouns, like \emph{kithuma} `sago pulp' and \emph{grau} `red clouds'. Others can be visually perceived as mass nouns, for example \emph{ŋarake} `fence' and \emph{nag} `grass skirt'. On the other hand, words like \emph{no} `water' are \isi{feminine} and not \isi{plural}. Interestingly, body parts that exist in pairs, like arms legs, and eye, are often used in the \isi{plural}, even though the language has a dual number category.

A few stems differ in their meaning depending on \isi{gender}. For example, \emph{mni} means `fire' when \isi{feminine}, but `firewood' when \isi{masculine}. Other examples are: \emph{ekri} (\F) `flesh' vs. \emph{ekri} (\Masc) `meat', \emph{no} (\F) `water' vs. \emph{no}  (\Masc) `rain' and \emph{efoth} (\F) `day' vs. \emph{efoth} (\Masc) `sun'.

Words with flexible \isi{gender} are mostly kin terms, for example sibling terms, which encode relative age difference, but not \isi{gender}. Thus, the word \emph{nane} can mean `older brother' or `older sister'. Many kin terms are \isi{reciprocal} and may hold between a man and a woman. For example \emph{ŋäwi} is used between a person and her/his mother's brothers. In other words, a young girl or boy calls her/his mother's brother \emph{ŋäwi}, and he uses the same term back to her/him. The same is true for a man's parents in-law. He calls both of them \emph{enat} and they call him the same. Sometimes this can be specified by adding the word for `woman' or `man', for example \emph{enat ŋare} `mother in-law' (lit. `parent-in-law woman').


\begin{table}
\caption{The semantics of the gender system}
\label{gender-table}
	\begin{tabular}{>{\raggedright}p{2cm}l>{\raggedright}p{4,4cm}>{\raggedright\arraybackslash}p{4,1cm}}
		\lsptoprule
		semantics&gender&examples&exceptions\\
		\midrule
		\multirow{6}{1,8cm}{big, elongated objects}&\multirow{6}{*}{\Masc}&\emph{naifa} `bush knife'&\emph{sifren} `grass knife'\\
		&&\emph{wämne} `tree'&\emph{waga} `leg'\\
		&&\emph{nabi} `bow'&\\
		&&\emph{turama} `python'&\\
		&&\emph{with} `banana'&\\
		&&\emph{nasi} `long yam'&\\\midrule
		\multirow{6}{1,8cm}{small, round objects}&\multirow{6}{*}{\F}&\emph{yawi} `seed, fruit'&\emph{nzagum} `fly'\\
		&&\emph{wawa} `yam'&\emph{tora} `dog whistle'\\
		&&\emph{yare} `bag'&\emph{tef} `spot'\\
		&&\emph{brnze} `lips'&\\
		&&\emph{riwariwa} `ring'&\\
		&&\emph{kwanz} `bald head'&\\\midrule
		\multirow{5}{1,8cm}{plants, trees}&\multirow{5}{*}{\Masc}&\emph{rugaruga} `tree species' (Gmelina ledermannii)&\emph{ŋazi} `coconut'\\
		&&\emph{withwith} `vine species' (Pseuduvaria sp)&\emph{gb} `palm species' (Livistona sp)\\
		&&\emph{mür} `grass species' (Cyperus sp)&\\\midrule
		\multirow{6}{1,8cm}{fish}&\multirow{6}{*}{\Masc}&\emph{find} `Giant Glassfish' (Parambassis gulliveri)&catfish species\\
		&&\emph{kwazür} `Narrow-fronted Tandan' (Neosilurus ater)&\emph{katif} `Trout Morgunde' (Mogurnda mogurnda)\\
		&&\emph{wifaza} `Seven-spot Archerfish' (Toxotes chatareus)&\\\midrule
		\multirow{4}{1,8cm}{catfish}&\multirow{4}{*}{\F}&\emph{zök} `Broad-snouted Catfish' (Potamosilurus latirostris)&\emph{ikan lele} `Walking Catfish' (Clarias batrachus)\\
		&&\emph{thrfam} `Daniel's Catfish' (Cochlefelis danielsi)&\\\midrule
		\multirow{3}{1,8cm}{events}&\multirow{3}{*}{\F}&\emph{zan} `fighting'&\emph{wath} `dance'\\
		&&\emph{borsi} `game, laughter'&\\
		&&\emph{si zübraksi} `prayer'&\\\midrule
		\multirow{5}{1,8cm}{landscape}&\multirow{5}{*}{\F}&\emph{mni} `fire'&\\
		&&\emph{kar} `place, village'&\\
		&&\emph{zra} `swamp'&\\
		&&\emph{daw} `garden'&\\
		&&\emph{ŋars} `river'&\\
		\lspbottomrule
	\end{tabular}
\end{table}

Other nouns with flexible \isi{gender} are animals for which a sex distinction is noticeable, for example \emph{tauri} `wallaby', \emph{ruga} `pig' or \emph{ŋatha} `dog'. Yet other species like fish or insects are not flexible. Birds for which there is a visible difference between male and female adults are assigned different lexemes altogether. For example, the male Eclectus Parrot (Eclectus roratus) is referred to as \emph{krara}, and the female as \emph{tiŋa}, but in Komnzo both lexemes are \isi{masculine}. Mismatches between biological \isi{gender} and grammatical \isi{gender} are quite common with birds. Two more examples are \emph{nzöyar}, the Fawn-breasted Bowerbird (Chlamydera cerviniventris) and \emph{ythama}, the Raggiana Bird-of-paradise (Paradisaea raggiana). For both species, the lexemes seem to refer only to the male birds, which can be explained by the fact that the females are less visible both in their plumage as well as in their behaviour. The Komnzo words, \emph{nzöyar} and \emph{ythama}, are assigned to the \isi{feminine} category, and they are often talked about as being female birds.

\subsection{Property nouns} \label{propertynouns-sec}

There is a class of lexical items in Komnzo which shares features of both nouns and adjectives. Henceforth, I will refer to them as ``property nouns'' because they denote either physical properties (\emph{fagwa} `width', \emph{dambe} `thickness', \emph{zrin} `heaviness') or abstract mental states (\emph{noku} `anger', \emph{miyo} `desire', \emph{miyatha} `knowledge', \emph{weto} `happiness'). A few property nouns are more event-oriented and express behavioural patterns (\emph{mogu} `concentration', \emph{ofe} `absence', \emph{müsa} `restlessness', \emph{zirkn} `persistence', \emph{waro} `theft, deception'). Note that I translate property nouns in the glosses sometimes as abstract nouns (\emph{miyamr} `ignorance', \emph{züb} `depth') and sometimes as adjectives (`ignorant' and `deep' respectively). I see no analytic gain in choosing one over the other and applying it consistently to all glosses in this grammar. The term ``\isi{property noun}'' is chosen because most members of this \isi{word class} express some physical or non-physical property, only a minority of them are event-oriented.

Property nouns can act as the \isi{head} of a \isi{noun phrase} and as such they behave as host for all \isi{case} clitics just like nouns. However, with respect to agreement, they are syntactically inert in two ways. First, property nouns do not register in the \isi{undergoer} prefix of verbs and consequently do not trigger \isi{gender} agreement. Consider the two elicited examples in (\ref{ex495}). In (\ref{ex496}), the \isi{undergoer} slot of the verb is filled by an invariant middle marker, an \emph{ŋ-} prefix.\footnote{See \S\ref{verbprelim} for an explanation of the glossing conventions of verbs in this grammar.} Only the \isi{subject} argument is indexed, which is a zero marker in suffix position. Thus, the \isi{object} is not indexed in the verb.\footnote{The middle construction has a number of functions described in {\S}\ref{middletemplatesubsection}. One of these functions is the suppressed-object function shown in (\ref{ex496}).} This especially occurs with property nouns, which creates an indeterminacy as to the argument status of \emph{twof} `heat' in (\ref{ex496}). Both translations given in (\ref{ex496}) are possible. In the first, the \isi{property noun} is the \isi{object}, in the latter it is a \isi{nominal} predicate. Example (\ref{ex497}) shows that this ambiguity is resolved, if an object argument is indexed in the undergoer slot, in this case a \emph{w-} prefix. However, the verb prefix does not index property noun like \emph{twof}. The \isi{object} argument must be a different \isi{noun}, for example \emph{bad} `ground, earth', which is put into backets in (\ref{ex497}). Note that, irrespective of whether or not the \isi{object} \isi{noun phrase} is present or omitted from the clause, the third singular feminine indexed in the verb cannot refer to the \isi{property noun} \emph{twof}.

\begin{exe}
	\ex \label{ex495}
	\begin{xlist}
	\ex \emph{efothf \textbf{twof} ŋafiyokwr.}\\
	\gll efoth=f twof ŋa\stem{fiyok}wr\\
	sun={\Erg} heat \Stsg:\Sbj:\Nonpast:\Ipfv/make\\
	\trans `The sun creates the heat.' or `The sun makes (something) hot.'
	\label{ex496}
	\ex \emph{efothf (bad) \textbf{twof} wäfiyokwr.}\\
	\gll efoth=f (bad) twof wä\stem{fiyok}wr\\
	sun={\Erg} (ground) heat \Stsg:\Sbj>\Tsg.\F:\Obj:\Nonpast:\Ipfv/make\\
	\trans `The sun makes (the ground) hot.'
	\label{ex497}
	\end{xlist}
\end{exe}

Note that with \isi{intransitive} verbs, like the copula, property nouns function as \isi{nominal} predicates. A clause like (\ref{ex031}) can only be interpreted as having an omitted \isi{subject} noun phrase which is third person singular masculine. It cannot be analysed in a way that \emph{frasi} `hunger' is the argument of the copula.

\begin{exe}
	\ex \emph{\textbf{frasi} yé.}\\
	\gll frasi \stem{yé}\\
	hungry \Tsg\Masc:\Sbj:\Nonpast:\Ipfv/be\\
	\trans `He is hungry.' \uline{not:} `It is hunger.'
	\label{ex031}
\end{exe}

Hence, we could say that property nouns escape indexation in the \isi{undergoer} prefix and as a consequence there is no \isi{gender} agreement. If informants are asked directly whether a given \isi{noun} is \isi{feminine} or \isi{masculine}, they can answer this promptly, but with property nouns, they hesitate and often answer: ``it depends''. In an example like (\ref{ex031}), it depends on the intended meaning: `she is hungry' or `he is hungry'. Thus, it depends on the \isi{gender} of the referent indexed in the copula, not on the \isi{property noun}.

Secondly, property nouns indexed in the actor suffix trigger a default \isi{singular} \isi{number} agreement. This occurs in \isi{experiencer-object} constructions (\ref{ex035}) or in the \isi{middle} template (\ref{ex040}). In (\ref{ex035}), the \isi{property noun} \emph{thkar} `hardness' is flagged with the \isi{ergative} \isi{case}, and it is indexed in the suffix of the verb \emph{fiyoksi} `make'. This example is from a myth in which a crocodile creates a large pool of water, because it got stuck, which translates literally as `hardness made it'. In (\ref{ex040}), the \isi{property noun} \emph{twof} `heat' is in the absolutive \isi{case}, and it is indexed in the suffix of the middle verb \emph{sogsi} `ascend'. In both examples, the indexed person/\isi{number} value is \Stsg. See {\S}\ref{expobjconstr} for \isi{experiencer-object} constructions and {\S}\ref{middletemplatesubsection} for a description of the middle template.

\begin{exe}
	\ex \emph{ŋanraknza zbo zf ziyé. zä zf fthé \textbf{thkarf} yafiyokwa ziyé.}\\
	\gll ŋan\stem{rak}nza zbo zf z=\stem{yé} zä zf fthé thkar=f ya\stem{fiyok}wa z=\stem{yé}\\
	\Stsg:\Sbj:\Pst:\Ipfv:\Venit/crawl \Prox.{\All} {\Imm} \Prox=\Tsg.\Masc:\Sbj:\Nonpast:\Ipfv/be {\Prox} {\Imm} when hardness={\Erg} \Stsg:\Sbj>\Tsg.\Masc:\Obj:\Pst:\Ipfv/make \Prox=\Tsg.\Masc:\Sbj:\Nonpast:\Ipfv/be\\
	\trans `It crawled here to this place. That is when it got stuck right here.' (lit. `Hardness did it.') \Corpus{tci20120922-09}{DAK \#17-18}
	\label{ex035}
\end{exe}
\begin{exe}
	\ex \emph{nafane \textbf{twof} kresöbäth nzafarfo.}\\
	\gll nafane twof kre\stem{söbäth} nzafar=fo\\
	\Tsg.{\Poss} heat \Stsg:\Sbj:\Irr:\Pfv/ascend sky={\All}\\
	\trans `Its heat rose up to the sky.' \Corpus{tci20110810-01}{MAB \#45-46}
	\label{ex040}
\end{exe}

Example (\ref{ex040}) shows that property nouns can enter into a \isi{possessive} construction. This is another characteristic they share with nouns and which sets them apart from adjectives. In this case, \emph{twof} is the \isi{possessed}. Although there are no examples attested in the corpus where a \isi{property noun} is the \isi{possessor}, this is confirmed by data from elicitation.

In both predicative and attributive constructions, property nouns take the \isi{adjectivaliser} \emph{-thé} optionally. An attributive construction in \ili{English} like `the embarrassed man' could be expressed as \emph{fäsi kabe} or \emph{fäsithé kabe}. The former could be translated as a compound `embarrassment man' and the latter `embarrassed man'. Hence, when it comes to property nouns no clear distinction can be drawn between attributive constructions and \isi{nominal} compounds in a predicative construction. Moreover, a predicative construction like \ili{English} `The man is ashamed' can also be expressed with or without the \isi{adjectivaliser} \emph{-thé} as either \emph{kabe fäsi yé} or \emph{kabe fäsithé yé}.

In addition to \isi{nominal} modification, property nouns can have a coverb function. Property nouns may occur with light verbs (\emph{rä-} `do', \emph{fiyoksi} `make', \emph{ko-} `become') or phasal verbs (\emph{thkäfsi} `start', \emph{bthaksi} `finish'). In (\ref{ex041}), a malevolent spirit is trying to lure a traveller to stay the night at her camp. In the construction, the \isi{property noun} \emph{garamgaram} `sweet talk' expresses most of the semantics of the event while the phasal verb \emph{thkäfksi} `start' takes the inflection and indexing.

\begin{exe}
	\ex \emph{\textbf{garamgaram} srethkäf. ``kwa ŋabrigwr? efoth byé!''}\\
	\gll {garamgaram} sre\stem{thkäf} kwa ŋa\stem{brig}wr efoth b=\stem{yé}\\
	sweet.talk \Stsg:\Sbj>\Tsg.\Masc:\Obj:\Irr:\Pfv/start {\Fut} \Stsg:\Sbj:\Nonpast:\Ipfv/return sun \Med{}=\Tsg.\Masc:\Nonpast:\Ipfv/be\\
	\trans `She started sweet-talking him: ``Will you go back? The sun is already setting!''' \Corpus{tci20120901-01}{MAK \#88-89}
	\label{ex041}
\end{exe}

Coverb + \isi{light verb} constructions of this kind have been described for a number of Australian languages. For example, in \ili{Jaminjung} (\citealt{SchultzeBerndt:2000wk}) or \ili{Bilinarra} (\citealt{Meakins:ul}) we find a division of labour in complex predicates whereby a distinct \isi{word class} of coverbs contributes most of the meaning of an event while a \isi{light verb} carries most of the inflectional material. In Komnzo, there are a few property nouns which seem to be more event-oriented in their semantics. However, there is insufficient morphological or distributional evidence for setting up a distinct \isi{word class} of coverbs. In addition to the coverb function in example (\ref{ex041}), property nouns can be used as secondary predicates. An example is provided in the use of \emph{wri} `intoxication' in (\ref{ex042}), where an angry man is tranquilised by giving him kava to drink.

\begin{exe}
	\ex \emph{krärme srärirfth. \textbf{wri} kwosi sfthnm.}\\
	\gll krär=me srä\stem{rirf}th wri kwosi sf\stem{thn}m\\
	kava={\Ins} \Stpl:\Sbj>\Tsg.\Masc:\Obj:\Irr:\Pfv/kill intoxicated dead \Tsg\Masc:\Sbj:\Pst:\Dur/lie.down\\
	\trans `They put him down with kava. Then he was lying down dead drunk.'\\ \Corpus{tci20120909-06}{KAB \#95-96}
	\label{ex042}
\end{exe}

Property nouns marked with the \isi{instrumental} \isi{case} have an adverbial function. In example (\ref{ex033}), the \isi{property noun} \emph{ktkt} `narrow' is the single argument of the intransitive verb. In the text, a group of headhunters prepare to attack a hamlet. The sentence is accompanied by a gesture which resembles the movement of the arms as if embracing a person. Here \emph{ktkt} is does not function as a secondary predicate and it would be incorrect to translate it as `They became narrow'. Note that the verb indexes \Stsg{} and not \Stnsg{}. Hence, a more literal translation is adequate `narrowness became/happened' or with a dummy \isi{subject} `it became narrow'. In example (\ref{ex034}), the same \isi{property noun} \emph{ktkt} takes the instrumental \isi{case} and functions adverbially. Here the speaker explains how the plant \emph{grnzari} (Chantium sp) grows.

\begin{exe}
	\ex \emph{kwot kar fthé wkrkwath wkrkwath wkrkwath a \textbf{ktkt} zäkora fof.}\\
	\gll kwot kar fthé 3x(w\stem{krk}wath) a {ktkt} zä\stem{kor}a fof\\
	properly village when 3x(\Stpl:\Sbj>\Tsg.\F:\Obj:\Pst:\Ipfv/block) and narrow \Stsg:\Sbj:\Pst:\Pfv/become {\Emph}\\
	\trans `They were blocking and blocking the village by narrowing (the circle).'\\ \Corpus{tci20111119-03}{ABB \#134}
	\label{ex033}
\end{exe}
\begin{exe}
	\ex \emph{\textbf{ktktme} erfikwr. nima fefe fof yrfikwr.}\\
	\gll {ktkt=me} e\stem{rfik}wr nima fefe fof y\stem{rfik}wr\\
	narrow={\Ins} \Stpl:\Sbj:\Nonpast:\Ipfv/grow like.this really {\Emph} \Tsg.\Masc:\Sbj:\Nonpast:\Ipfv/grow\\
	\trans `They grow closely together. This one really grows like that.'\\ \Corpus{tci20130907-02}{RNA \#705}
	\label{ex034}
\end{exe}

\subsection{Adjectives} \label{adjectives-sec}

Adjectives form a small class of lexical items in Komnzo. Semantically, adjectives denote size (\emph{kafar} `big, great', \emph{katan} `small', \emph{yabun} `fat, big', \emph{tnz} `short', \emph{zanfr} `tall'), quality (\emph{namä} `good', \emph{gathagatha} `bad'), age (\emph{zafe} `old', \emph{zöftha} `new'), physical property (\emph{kwosi} `rotten, dead', \emph{kwik} `sick', \emph{tayo} `ripe, dried', \emph{gauyé} `fresh, unripe') and human propensity (\emph{dmnzü} `silent', \emph{yoganai} `tired', \emph{zäzr} `exhausted'). Colour adjectives, as seen in {\S}\ref{nouns-sec}, are derived from nouns by suffixing \emph{-thé}. There are a few adjectives which take irregular forms of this suffix (\emph{zisé} `painful' from \emph{zi} `pain') and/or which lack a corresponding \isi{noun} or \isi{property noun} (\emph{dbömsé} `blunt'). Hence, these are treated as adjectives with frozen morphology. There are about two dozen members in the adjective word class. The low number can be explained by the presence of a class of property nouns ({\S}\ref{propertynouns-sec}).

Syntactically, adjective usually precede their head. However, this is only a tendency, as they may be follow the head too. There are three adjectives which are special in that they occur only in postposed position. Two denote human propensity: \emph{bana} `poor, pitiful, hapless' and \emph{kwark} `deceased, late' (\ref{ex049}). The third denotes quality: \emph{fefe} `true'.

Morphological evidence is provided by the \isi{adjectivaliser} \emph{-thé}, which cannot be suffixed to an \isi{adjective}: \textsuperscript{$\ast$}\emph{katanthé} `small', \textsuperscript{$\ast$}\emph{namäthé} `good' or \textsuperscript{$\ast$}\emph{tnzthé} `short'. Some nouns, for example \emph{kayanthé} `white' (from \emph{kwayan} `light'), and all property nouns can take the \isi{adjectivaliser}.

Adjectives may serve as the host for any \isi{case} enclitic if they occur in the rightmost position of the noun phrase. This occurs if (i) the \isi{head} of noun phrase has been omitted as in (\ref{ex048}) or (ii) if an \isi{adjective} has been postposed, as in (\ref{ex049}). See {\S}\ref{headslot} for further discussion of headedness and \isi{ellipsis}. Example (\ref{ex505}) shows an \isi{adjective} preceding the \isi{head} of the \isi{noun phrase}. We see from these examples, combined with the argument of \isi{ellipsis}, that adjectives cannot function as the \isi{head} of a phrase. This is supported by the observation that it is the \isi{head} of a phrase which triggers agreement in the verb prefix and not the \isi{adjective}.

\begin{exe}
	\ex \emph{wati, kofä fthé brigsir n krär, \textbf{katanf} kwa ynbrigwr zbo.}\\
	\gll wati kofä fthé \stem{brig}-si=r n krä\stem{r} katan=f kwa yn\stem{brig}wr zbo\\
	then fish when return-\Nmlz={\Purp} {\Imn} \Stsg:\Sbj:\Irr:\Pfv/do small={\Erg} {\Fut} \Stsg:\Sbj>\Tsg.\Masc:\Obj:\Nonpast:\Ipfv:\Venit/return {\Prox}.{\All}\\
	\trans `When the fish tries to get out, the small (basket) will bring them back here.' \Corpus{tci20120906}{MAB \#56-57}
	\label{ex048}
\end{exe}
\begin{exe}
	\ex \emph{nzwamnzrm fof ... oromanä fof ... oroman \textbf{kwarkä}.}\\
	\gll nzwa\stem{m}nzrm fof (.) oroman=ä fof (.) oroman kwark=ä\\
	\Fsg:\Sbj:\Pst:\Dur/dwell {\Emph} (.) {old.man=\Assoc.\Pl} {\Emph} (.) {old.man} deceased=\Assoc.{\Pl}\\
	\trans `We stayed with the old man ... with the late old man.'\Corpus{tci20130911-03}{MBR \#72-73}
	\label{ex049}
\end{exe}
\begin{exe}
	\ex \emph{bobomrwä arufe krathfänzr ... \textbf{zagr} karfo.}\\
	\gll bobomr=wä arufe kra\stem{thfä}nzr (.) zagr kar=fo\\
	until={\Emph} arufe \Stsg:\Sbj:\Irr:\Ipfv/fly (.) far village=\All\\
	\trans `He flies all the way to Arufe ... to a distant village.'\Corpus{tci20130903-04}{RNA \#144-145}
	\label{ex505}
\end{exe}

As with property nouns, adjectives with an \isi{instrumental} \isi{case} can function adverbially. In (\ref{ex050}), the \isi{adjective} \emph{gathagatha} `bad' modifies the verb. In the example, a mother is scolding her daughter because she walks carelessly through the long grass. In (\ref{ex043}), the adjective \emph{katan} `small' modifies the predicate `be rotten'. In this procedural text, the speaker demonstrates how to roll a little whistle from a coconut leaf. However, the first attempt to blow the whistle fails because the coconut leaf was not fresh.

\begin{exe}
	\ex \emph{kabothma! tayafe \textbf{gathagathamenzo} niyak! kabothma!}\\
	\gll kaboth=ma tayafe {gathagatha=me=nzo} n\stem{yak} kaboth=ma\\
	snake={\Char} tayafe bad=\Ins={\Only} \Ssg:\Sbj:\Nonpast:\Ipfv/walk snake={\Char}\\
	\trans `Tayafe, you walk in a bad way! (Watch out) for snakes!'\Corpus{tci20130907-02}{JAA \#143}
	\label{ex050}
\end{exe}
\begin{exe}
	\ex \emph{keke kwot yanor. zane \textbf{katanme} kwosi yé.}\\
	\gll keke kwot ya\stem{nor} zane katan=me kwosi \stem{yé}\\
	{\Neg} properly \Tsg.\Masc:\Sbj:\Nonpast:\Ipfv/shout \Dem:{\Prox} small={\Ins} dead \Tsg.\Masc:\Nonpast:\Ipfv/be\\
	\trans `It doesn't whistle properly. This one is a little rotten.'\Corpus{tci20120914}{RNA \#55-56}
	\label{ex043}
\end{exe}

\subsection{Quantifiers and numerals} \label{quantifiers-numerals-sec}

The \isi{quantifier} subclass typically contains lexical items that are ``modifiers of nouns that indicate quantity and scope'' (\citealt[37]{Schachter:2007vv}). Quantifiers in Komnzo fall into two subclasses: non-numerical quantifiers ({\S}\ref{quantifiers-subsec}) and numerical quantifiers ({\S}\ref{numerals-subsec}), henceforth referred to as quantifiers and numerals, respectively.

Both subclasses show similarities to adjectives. What unites them as a distinct subclass is the ability to take the \isi{distributive} suffix (\emph{-kak}). Quantifiers and numerals are the only roots that take the \isi{distributive} suffix. Like adjectives, they can be flagged for \isi{case} and may take the \isi{instrumental} \isi{case} (\emph{=me}) with an adverbial function, for example indicating how many times a particular event occurred.

\subsubsection{Quantifiers} \label{quantifiers-subsec}

There are five quantifiers in Komnzo: \emph{matak} `nothing', \emph{frü} `alone, single', \emph{etha} `few', \emph{tüfr} `many, plenty', and \emph{bramöwä} `all'.

Quantifiers may precede or follow the noun which they modify. That being said, it is much more common for a \isi{quantifier} to follow the \isi{noun}, as in (\ref{ex090}) and (\ref{ex077}). Instances of a preceding \isi{quantifier} are not attested in the corpus, but only verified through elicitation. But see (\ref{ex081}) below and footnote \ref{foot1} for a possible example.

\begin{exe}
	\ex \emph{kofä \textbf{bramöwä} fthé kränmtherth watik zzarä kwot threnthfär ... nä totkarä.}\\
	\gll kofä bramöwä fthé krän\stem{mther}th watik zzar=ä kwot thren\stem{thfär} (.) nä tot=karä\\
	fish all when \Stpl:\Sbj:\Irr:\Pfv:\Venit/come.up then net={\Assoc} properly \Stpl:\Sbj:\Irr:\Pfv:\Venit{}/jump (.) other spear={\Prop}\\
	\trans `When all the fish come up, then they jump in with the nets ... others with spears.' \\\Corpus{tci20110813-09}{DAK \#28}
	\label{ex090}
\end{exe}
\begin{exe}
	\ex \emph{sitauane ŋare mane erna minu erna ... nge \textbf{matak}.}\\
	\gll sitau=ane ŋare mane e\stem{r}na {minu} e\stem{rn}a (.) nge matak\\
	sitau=\Poss.{\Sg} woman which \Stdu:\Sbj:\Pst:\Ipfv/be {barren.woman} \Stdu:\Sbj:\Pst:\Ipfv/be (.) child nothing\\
	\trans `As for Sitau's two wives, they were barren women without children.'\\ \Corpus{tci20120814}{ABB \#469}
	\label{ex077}
\end{exe}

Quantifiers may take the \isi{distributive} suffix (\emph{-kak}) which can be translated as `each' to \ili{English}. For semantic reasons, neither \emph{matak} `nothing' nor \emph{bramöwä} `all' take this suffix. Two examples of the \isi{distributive} suffix are given in (\ref{ex742}) and (\ref{ex078}). In the first example, the speaker describes a ritual for starting the harvesting season, during which `each person' brings a tuber for cooking and tasting the first yams. In the second example, the speaker shows me her catch of the day: a lizard, several fish and a turtle. Thus, she emphasises that she caught plenty of different food.

\begin{exe}
	\ex \emph{we kwot we \textbf{näbikakme} ... we nä wawa thfrärmth katan o kafar.}\\
	\gll we kwot we näbi-kak=me (.) we nä wawa thf\stem{rä}rmth katan o kafar\\
	also properly also one-\Distr={\Ins} (.) also {\Indf} yam \Stpl:\Sbj>\Stpl:\Obj:\Pst:\Dur/do small or big\\
	\trans `Again, they took them out (of the garden plot) one by one ... small or big ones.' \Corpus{tci20131013-01}{ABB \#364}
	\label{ex742}
\end{exe}
\begin{exe}
	\ex \emph{watik, faso \textbf{tüfrkak} erä.}\\
	\gll watik, faso tüfr-kak e\stem{rä}\\
	then, meat plenty-{\Distr} \Stpl:\Sbj:\Nonpast:\Ipfv/be\\
	\trans `Okay, there is plenty of different meat.' \Corpus{tci20120821-01}{LNA \#68}
	\label{ex078}
\end{exe}

Quantifiers may take an instrumental \isi{case} (\emph{=me}) in order to derive adverbs, as is shown in example (\ref{ex079}).

\begin{exe}
	\ex \emph{kabe ane \textbf{frümenzo} tnägsi zethkäfath.}\\
	\gll kabe ane frü=me=nzo tnäg-si ze\stem{thkäf}ath\\
	man {\Dem} single=\Ins={\Only} lose-{\Nmlz} \Stpl{}:\Sbj:\Pst:\Ipfv/start\\
	\trans `The people began to scatter.' (lit. `They began losing themselves alone.')\\ \Corpus{tci20131013-01}{ABB \#54}
	\label{ex079}
\end{exe}

The \isi{distributive} and the instrumental may also be suffixed to the same \isi{quantifier}. In this case, their order is fixed: the instrumental follows the \isi{distributive}, as shown in example (\ref{ex080}). The example also shows that, like other \isi{nominal}s, quantifiers can be reduplicated to indicate plurality. Here, the speaker talks about types of bows and how different men use these according to their abilities and preferences.

\begin{exe}
	\ex \emph{zawe \textbf{ffrükakmenzo} erä.}\\
	\gll zawe f-frü-kak=me=nzo e\stem{rä}\\
	preference \Redup-single-\Distr=\Ins={\Only} \Stpl:\Sbj:\Nonpast:\Ipfv/be\\
	\trans `They each have their preferences.' \Corpus{tci20120922-23}{MAA \#104}
	\label{ex080}
\end{exe}

Example (\ref{ex081}) shows \emph{etha} meaning `few'. Note that the word \emph{etha} can also mean `three', which I describe in {\S}\ref{numerals-subsec}.

\begin{exe}
	\ex \emph{\textbf{tüfrmär} kafarkafar nrä ... komnzo \textbf{ethanzo}.}\\
	\gll tüfr=mär kafar-kafar n\stem{rä} (.) komnzo etha=nzo\\
	plenty={\Priv} \Redup-big \Fpl:\Sbj:\Nonpast:\Ipfv/be (.) only few={\Only}\\
	\trans `We are not many old people ... just a few.' \Corpus{tci20121019-04}{ABB \#187-188}
	\label{ex081}
\end{exe}

Note in passing that in (\ref{ex081})\footnote{\label{foot1}In example (\ref{ex081}) we can see that \emph{tüfr} `plenty' precedes the reduplicated adjective \emph{kafarkafar} `big'. The example is interpreted to have an elided noun \emph{kabe} `man' as its head, thus \emph{kafarkafar} means `the big ones'. This then constitutes a corpus example of a quantifier preceding its head.} the \isi{quantifier} \emph{tüfr} `plenty' is negated by using the \isi{privative} case \emph{=mär}. This is also possible with \emph{etha}.

The two quantifiers \emph{matak} `nothing' and \emph{bramöwä} `all' deviate in their behaviour from other quantifiers. As mentioned above, they do not take the \isi{distributive} suffix. Furthermore, they do not take the instrumental case \emph{=me}. At least for \emph{bramöwä} there might be an explanation as to why this is the case. The emphatic marker \emph{=wä} forces the preceding morpheme to harmonise its vowel. If the preceding morpheme is the instrumental marker, it changes from \emph{=me} to \emph{=mö}. It follows that, historically, \emph{bramöwä} could be \emph{bra=me=wä}. Since there is no corresponding lexical item \emph{bra}, we are left to speculate, and accept it as a case of frozen morphology.

\subsubsection{Numerals} \label{numerals-subsec}

The numerals of the \ili{Yam languages} have received some attention in the literature because of their unique \isi{senary} (base-6) system (cf. Donohue \citeyear{Donohue:2008bn}, Hammarström \citeyear{Hammarstrom:2009bp}, and Evans \citeyear{Evans:2009wg}). In fact, Komnzo has two numeral systems: the \isi{senary} system is unrestricted, but there is a second system with an upper limit of counting of four or five. This is similar to Donohue's description of \ili{Kanum}, where an unrestricted system coexists with a restricted system (\citealt{Donohue:2008bn}). Nowadays, one should include \ili{English} numerals which constitute a third system commonly used in Komnzo. For the remaining description, I will concentrate on the \isi{senary} system and the restricted system only.

The \isi{senary} system is predominantly employed in ritualised counting as described in {\S}\ref{yamcounting}. The number of yams counted during a feast quickly runs up to several thousands, for large feasts even tens of thousands. On the other hand, everyday counting hardly ever goes above four or five, and \ili{English} numerals are borrowed in situations where approximation of larger numbers is insufficient, for example when trading goods, charging one's mobile phone credit, or counting the eleven members of a soccer team. Hence, we find a double \isi{numeral system} in \tabref{numerals-table}.\footnote{In the table, the term for `five' shows two variants. The term for `six' also shows two variants one of which is a combination of \emph{tabuthui} `five' and \emph{nibo} `six'. Outside of ritualised yam counting, I have overheard this only a few times by younger speakers. Older speakers did not produce a term for `six' or were reluctant to do so. The combination \emph{tabuthui nibo} might be explained by the way how ritualised counting works: While two men move a set of six yams, one of them will shout out the numbers. He continues to shout the current number as long as it takes to move to the next one (e.g. `two two two three'). This means that each cycle of six ends with \emph{tabuthui nibo} `five six'. It seems that some speakers have taken this collocation and reinterpreted it to mean `six'. I take this as being indicative for the fuzzy upper limit of the restricted set.} One set of numerals is commonly used, but it is restricted to low numbers. A second set is employed only in ritualised counting, but it is unrestricted.

\begin{table}
\caption{The numeral system}
\label{numerals-table}
	\begin{tabularx}{.6\textwidth}{rp{1cm}ll}
		\lsptoprule
		\multicolumn{2}{c}{value}&{restricted}&{ritualised}\\\midrule
		1&&\emph{näbi}&\emph{näbi}\\
		2&&\emph{eda}&\emph{yda}\\
		3&&\emph{etha}&\emph{ytho}\\
		4&&\emph{asar}&\emph{asar}\\
		5&&(\emph{tabuthui, tabru})&\emph{tabuthui}\\
		6&\multicolumn{1}{c}{6}&(\emph{tabuthui nibo, nibo})&\emph{nibo}\\
		36&\multicolumn{1}{c}{6\textsuperscript{2}}&&\emph{fta}\\
		216&\multicolumn{1}{c}{6\textsuperscript{3}}&&\emph{taruba}\\
		1,296&\multicolumn{1}{c}{6\textsuperscript{4}}&&\emph{damno}\\
		7,776&\multicolumn{1}{c}{6\textsuperscript{5}}&&\emph{wärämäkä}\\
		46,656&\multicolumn{1}{c}{6\textsuperscript{6}}&&\emph{wi}\\
		\lspbottomrule
	\end{tabularx}
\end{table}%The \isi{numeral system}

Beyond the observation of cultural practices, evidence for this double system comes from the lexical items themselves. In everyday counting, the words for `two' and `three' are \emph{eda} and \emph{etha}. In ritualised counting, the words are \emph{yda} and \emph{ytho} respectively. The latter pair reflects older forms which have not undergone the loss of word-initial \emph{y}. This sound change (jə > e /\#\_) is attested in many pairs of lexical items between Komnzo and the neighbouring \ili{Tonda} varieties, e.g. \ili{Wära} \emph{ymoth} `girl' corresponds to Komnzo \emph{emoth}. Another piece of evidence comes from the fact that the numeral \emph{etha} `three' can also mean `a few' (\ref{ex081}). I take this as evidence for the fuzzy upper limit of the restricted set.

Large quantities can be constructed in the following way: a quantity of 72 is expressed as \emph{eda fta} `2 36' (or `2 6\textsuperscript{2}'). A quantity of 73 would simply add \emph{a näbi} `and one' to the expression: \emph{eda fta a näbi} `2 36 and 1'. Thus, the fact that \emph{eda} precedes \emph{fta} means `2 times 36', whereas the fact that \emph{a näbi} follows \emph{fta} means `36 plus 1'. This has the effect that values which are relatively simple in a decimal system result in a long string in Komnzo, for example \ili{English} `fifty' corresponds to Komnzo \emph{näbi fta a eda nibo a eda} (lit. `1 times 36 and 2 times 6 and 2'). A \isi{senary} system differs from a decimal system only in the location of simple and complex points in the number space, but not in its overall complexity. Consequently, there are values which require a very long string in \ili{English}, but have a short expression in Komnzo, for example `forty-six thousand and six hundred and fifty-six' corresponds to \emph{wi} in Komnzo.

Numerals can take the same morphology as quantifiers ({\S}\ref{quantifiers-subsec}). There are no corpus examples of a numeral taking either the \isi{distributive} suffix or the instrumental \isi{case} clitic, but example (\ref{ex082}) illustrates the use of both. I was taught the phrase \emph{näbikakme käznob!} `drink it one by one!' before I administered pain relief tablets to my friends and informants. I was corrected whenever I falsely used only the instrumental \emph{näbime käznob}, which means `drink it in one go!' (lit. `with one').

\begin{exe}
	\ex \emph{nä kabe \textbf{näbikakmenzo} ... finzo miyatha thfrärm fof.}\\
	\gll nä kabe näbi-kak=me=nzo (.) fi=nzo miyatha thf\stem{rä}rm fof\\
	some men one-{\Distr}={\Ins}={\Only} (.) \Third.{\Abs}={\Only} knowledge \Stpl{}:\Sbj:\Pst:\Dur{}/be {\Emph}\\
	\trans `Only some people for themselves ... only they held that knowledge.'\\\Corpus{tci20120909-06}{KAB \#13}
	\label{ex082}
\end{exe}

Ordinal numerals can be derived from cardinal numerals by attaching the \isi{characteristic} \isi{case} marker \emph{=ma}. This is shown in examples (\ref{ex083}) and (\ref{ex084}).

\begin{exe}
	\ex \emph{fi sraksrak wokuthé yara \textbf{ethama} mane yara.}\\
	\gll fi srak-srak woku-thé ya\stem{r}a etha=ma mane ya\stem{r}a\\
	\Third.{\Abs} \Redup-boy skin-{\Adlzr} \Tsg.\Masc:\Sbj:\Pst:\Ipfv/be three={\Char} which \Tsg.\Masc:\Sbj:\Pst:\Ipfv/be\\
	\trans `As for the third one, he looked a bit boyish.' \Corpus{tci20131013-02}{ABB \#211}
	\label{ex083}
\end{exe}
\begin{exe}
	\ex \emph{\textbf{ethama} bäne mane zrarä fof ... wfathwr ane fof.}\\
	\gll etha=ma bäne mane zra\stem{rä} fof (.) w\stem{fath}wr ane fof\\
	three={\Char} {\Recog} which \Tsg.\F:\Sbj:\Irr:\Ipfv/be {\Emph} (.) \Stsg:\Sbj>\Tsg{}\F{}:\Obj:\Nonpast:\Ipfv/hold {\Dem} {\Emph}\\
	\trans `At the third attempt she will really hold her up.' \Corpus{tci20110817-02}{ABB \#106-107}
	\label{ex084}
\end{exe}

The numeral \emph{näbi} `one' can be used in the sense of `one way' or `for good'. The latter meaning is exemplified in (\ref{ex085}).

\begin{exe}
	\ex \emph{wati, fi \textbf{näbi} zäbrima. zbo yamnzr ane woga oten.}\\
	\gll wati fi näbi zä\stem{brim}a zbo ya\stem{m}nzr ane woga ote=n\\
	then \Third.{\Abs} one \Sg:\Sbj:\Pst:\Pfv/return {\Prox}.{\All} \Tsg\Masc:\Sbj:\Nonpast:\Ipfv/dwell {\Dem} man ote=\Loc\\
	\trans `He returned for good. This man now lives here in Ote.'\Corpus{tci20120901-01}{MAK \#210-211}
	\label{ex085}
\end{exe}

\subsection{Locationals} \label{locationals-sec}

Komnzo has a small closed class of lexical items which I call locationals. Historically, some members of this subclass are derived from nouns. Locationals may act as hosts of \isi{case} clitics, but for spatial cases only (\isi{locative}, \isi{allative}, and \isi{ablative}). Table (\ref{locationals-table}) lists all nine members.

\begin{table}
\caption{Locationals}
\label{locationals-table}
	\begin{tabularx}{.8\textwidth}{XXl}
		\lsptoprule
		form&{gloss}&{historical derivation}\\ \midrule
		\emph{warfo}&above&\emph{war} `top layer' \emph{=fo} ({\All})\\
		\emph{banban}&underneath&-\\
		\emph{zfthen}&below&\emph{zfth} `base' \emph{=en} ({\Loc})\\
		\emph{mrmr}&inside&-\\
		\emph{zrfa}&in front&\emph{zr} `tooth' \emph{=fa} ({\Abl})\\
		\emph{tharthar}&next to&-\\
		\emph{kamfa}&behind&\emph{kam} `bone, backbone' \emph{=fa} ({\Abl})\\
		\emph{bobathm}&at the end of&-\\
		\emph{kratr}&in between&-\\
		\lspbottomrule
	\end{tabularx}
\end{table}%locationals

Locationals occur always as modifiers which follow the \isi{head} of the noun phrase. A typical example is provided in (\ref{ex051}) with \emph{banban} `underneath'. The speaker describes how people reacted when the Imperial Japanese Air Service flew attacks on Merauke in Dutch New Guinea during the Second World War.

\begin{exe}
	\ex \emph{fi fthé fof duga taga \textbf{banbanen} boba kwatharwrmth fof.}\\
	\gll fi fthé fof duga taga banban=en boba kwa\stem{thar}wrmth fof\\
	\Tsg.{\Abs} when {\Emph} taro leaf underneath={\Loc} \Med.{\Abl} \Stpl:\Sbj:\Pst:\Dur/go.underneath {\Emph}\\
	\trans `That was really when they went underneath the taro leaves.'\\ \Corpus{tci20131013-02}{ABB \#231-232}
	\label{ex051}
\end{exe}

I analyse these as \isi{locational} \isi{nominal}s rather than postpositions, because like all \isi{nominal}s, they are marked for \isi{case}. Additionally, as we can see in the third column of \tabref{locationals-table}, some of the \isi{locational} \isi{nominal}s are historically derived from nouns. For these, I propose a path of development from a \isi{nominal} compound to a lexical item of a different \isi{nominal} subclass. As an example, let us hypothesise about the origin of \emph{warfo} `above'. In the first stage, there would have been a \isi{nominal} compound \emph{mnz war} `house top' made up of two nouns \emph{mnz} `house' and \emph{war} `top'. Nominal compounds are described in {\S}\ref{compounds-subsec}. This compound can be marked with the \isi{allative} \isi{case} productively, thus, producing \emph{mnz warfo} `to the top of the house'. In the second stage, \emph{warfo} became a single lexical item `above' and lost the specific \isi{allative} semantics. As a consequence, it can now be marked for spatial cases, for example the \isi{locative} \isi{case} (\emph{=n}), producing \emph{mnz warfon} `on top of the house'. This is commonly found in Komnzo, although presently there is no example in the corpus. Lexicalisation of this kind has progressed to varying degrees with the four locationals where a \isi{nominal} origin is a possible scenario. While \emph{warfo}, \emph{kamfa} and \emph{zrfa} are commonly marked with the \isi{locative} \isi{case} clitic, this does not occur with \emph{zfthen}. Hence, \emph{zfthen} is at a transitional stage between a \isi{noun} with productive morphology (the \isi{locative} \isi{case} \emph{=en}) and a \isi{locational}. The choice depends on whether one analyses \emph{zfth} in expressions like \emph{mnz zfth} `house base' as part of a noun+\isi{noun} compound or as a \isi{noun}+\isi{locational} construction.

Two characteristics unite locationals as a \isi{word class}. Locationals always follow the \isi{head} of the noun phrase, and they take only spatial cases. As we will see in {\S}\ref{spatialcase}, spatial cases can be extended to cover \isi{temporal} semantics, as in (\ref{ex052}).

\begin{exe}
	\ex \emph{zena kwa ŋatrikwé fof ... nimame zrethkäfé zane ezi \textbf{mrmren}.}\\
	\gll zena kwa ŋa\stem{trik}wé fof (.) nima=me zre\stem{thkäf}é zane ezi mrmr=en\\
	today {\Fut} \Fsg:\Sbj:\Nonpast:\Ipfv/tell {\Emph} (.) {like.this}={\Ins} \Fsg:\Sbj:\Irr:\Pfv/start \Dem:{\Prox} morning inside={\Loc}\\
	\trans `Today, I will tell (a story) ... I will start like this in this morning.'\\ \Corpus{tci20110802}{ABB \#28-29}
	\label{ex052}
\end{exe}

\subsection{Temporals} \label{temporals-sec}

Temporals are a functional class with members from different \isi{nominal} subclasses which encode \isi{temporal} semantics. Beyond the shared reference to time, the unifying characteristic is their ability to act as hosts for a special set of \isi{temporal} \isi{case} clitics. Syntactically, these lexemes are flexible with respect to their position in the clause, but they occur most commonly in initial position.

Temporals comprise a set of lexical items which cross-cut three word classes. First, there are nouns denoting different times of the day (\emph{ezi} `morning', \emph{efoth} `day', \emph{zizi} `afternoon, dusk', \emph{zbär} `night'). Secondly, there is a group of time adverbials (\emph{zena} `now, today', \emph{kayé} `yesterday, tomorrow', \emph{nama} `two days ago, two days in the \isi{future}', \emph{nümä} `a week ago, a week ahead'). Except for \emph{zena}, these are bidirectional in their semantics. Thus, \emph{kayé} could be glossed as `$\pm$ 1 day', \emph{nama} as `$\pm$ 2 days' and \emph{nümä} as `$\pm$ a few days'. As for the latter two, the edges of the time interval are less clearly demarcated. Note that bidirectionals are also found in other Papuan languages, for example in \ili{Usan} (\citealt[70]{Reesink:1987vg}). Thirdly, there are three adjectives \emph{zöftha} `before, first', \emph{zafe} `old, long time ago', and \emph{thrma} `later, after', all unidirectional in their semantics.

The uniting characteristic of this class is its ability to inflect for \isi{temporal} \isi{case}s. There are three \isi{temporal} cases in Komnzo: the \isi{temporal} \isi{locative} (\emph{=thamen}) `at that time', the \isi{temporal} \isi{possessive} (\emph{=thamane}) `that time's', and the \isi{temporal} \isi{purposive} (\emph{=thamar}) `for that time'. Temporal cases are discussed in {\S}\ref{temporalcase}. In the following examples, the \isi{temporal} \isi{purposive} \isi{case} is used on the noun \emph{ezi} (\ref{ex057}), on the time adverbial \emph{nama} and the \ili{English} \isi{loanword} `Friday' (\ref{ex055}), and on the \isi{temporal} adjective \emph{thrma} (\ref{ex058}). In (\ref{ex057}), the speaker tells his friends to leave the work on a sago palm for the next day. In (\ref{ex055}), the speaker begins his description of a namesake ceremony which is about to be held two days later. Finally, in (\ref{ex058}), two speakers go through a set of stimulus pictures and try to sort them into a narrative.

\begin{exe}
	\ex \emph{nze thäkora ``fefe yé \textbf{ezithamar}. ezi n kwot sräfrmnze.''}\\
	\gll nze thä\stem{kor}a fefe \stem{yé} ezi=thamar ezi n kwot srä\stem{frm}nze\\
	\Fsg.{\Erg} \Fsg:\Sbj>\Stpl:\Obj:\Pst:\Pfv/speak really \Tsg.\Masc:\Sbj:\Nonpast:\Ipfv/be morning=\Temp.{\Purp} morning try properly \Fpl:\Sbj>\Tsg.\Masc:\Obj:\Irr:\Ipfv/prepare\\
	\trans `I told them: ``It is there for the morning. We will try and prepare it in the morning.''' \Corpus{tci20120929}{SIK \#65}
	\label{ex057}
\end{exe}
\begin{exe}
	\ex \emph{fam monme erä ... \textbf{namathamar} \textbf{fraidethamar} ... nge fathasi yamyam monme kwa ŋankwir.}\\
	\gll fam mon=me e\stem{rä} (.) nama=thamar fraide=thamar (.) nge fath-si yam-yam mon=me kwa ŋan\stem{kwir}\\
	thought how={\Ins} \Stpl:\Sbj:\Nonpast:\Ipfv/be (.) +|-2days=\Temp.{\Purp} friday=\Temp.{\Purp} (.) child hold-{\Nmlz} \Redup-event how={\Ins} {\Fut} \Stsg:\Sbj:\Nonpast:\Ipfv:\Venit/run\\
	\trans `(My) thoughts for the day after tomorrow, for Friday, are like this. This is how the children's ceremony will take place.' \Corpus{tci20110817-02}{ABB \#3-5}
	\label{ex055}
\end{exe}
\begin{exe}
	\ex \emph{zane mane rä \textbf{thrmathamar} zane rä.}\\
	\gll zane mane \stem{rä} thrma=thamar zane \stem{rä}\\
	\Dem:{\Prox} which \Tsg.\F:\Sbj:\Nonpast:\Ipfv/be later=\Temp.{\Purp} \Dem:{\Prox} \Tsg.\F:\Sbj:\Nonpast:\Ipfv/be\\
	\trans `As for this one, this is for later.' \Corpus{tci20111004}{RMA \#236-237}
	\label{ex058}
\end{exe}

Temporals can also take spatial cases, as in (\ref{ex053}) with the \isi{temporal} \isi{noun} \emph{ezi} `morning' and in (\ref{ex056}) with the time adverbial \emph{zena} `now'. The three adjectives of this subclass may also take spatial cases when they are in the final position of a noun phrase, as in (\ref{ex059}). In all of these cases, what is otherwise spatial marking is extended to express \isi{temporal} semantics.

\begin{exe}
	\ex \emph{frasinzo nzwamnzrm \textbf{ezifa} bobomr mor efoth.}\\
	\gll frasi=nzo nzwa\stem{m}nzrm ezi=fa bobomr mor efoth\\
	hunger={\Only} \Fpl:\Sbj:\Pst:\Dur/dwell morning={\Abl} until neck day\\
	\trans `We were staying very hungry from the morning until midday.'\\ \Corpus{tci20120924-01}{TRK \#37}
	\label{ex053}
\end{exe}
\begin{exe}
	\ex \emph{wati, \textbf{zenafa} ... ni tüfr nagayé kwakonzre.}\\
	\gll wati zena=fa (.) ni tüfr nagayé kwa\stem{ko}nzre\\
	then today={\Abl} (.) {\Fnsg} plenty children \Fpl:\Sbj:\Rpst:\Ipfv/become\\
	\trans `Nowadays, we, the children, have become plenty.' (lit. `From now on ...')\\ \Corpus{tci20111107-01}{MAK \#149-150}
	\label{ex056}
\end{exe}
\begin{exe}
	\ex \emph{twofthé fthé krafiyokwr. ane \textbf{thrmafa} zränthore.}\\
	\gll twof-thé fthé kra\stem{fiyok}wr ane thrma=fa zrän\stem{thor}e\\
	heat-{\Adlzr} when \Stsg:\Sbj:\Irr:\Ipfv/make {\Dem} after={\Abl} \Fpl:\Sbj>\Tsg.\F:\Irr:\Pfv:\Venit/carry\\
	\trans `It has dried then. After that we bring it (the drum) here.'\Corpus{tci20120824}{KAA \#78-79}
	\label{ex059}
\end{exe}

Temporal nouns may also enter into a \isi{noun}+\isi{locational} construction (\ref{ex054}), again a \isi{temporal} interpretation of the \isi{locational}.

\begin{exe}
	\ex \emph{zane namä \textbf{ezi mrmren} nzä kwa trikasi ŋatrikwé.}\\
	\gll zane namä ezi mrmr=en nzä kwa trik-si ŋa\stem{trik}wé\\
	{\Dem}:{\Prox} good morning inside={\Loc} \Fsg.{\Abs} {\Fut} tell-{\Nmlz} \Fsg{}:\Sbj:\Nonpast:\Ipfv/tell\\
	\trans `In this beautiful morning, I will tell a story.' \Corpus{tci20111119-01}{ABB \#2-3}
	\label{ex054}
\end{exe}

\subsection{Personal pronouns} \label{personalpronouns-sec}

Personal pronouns form a closed subclass of \isi{nominal}s distinguishing three persons in both \isi{singular} and \isi{non-singular} \isi{number}. Personal pronouns have distinct forms for \isi{case} (\isi{absolutive}, \isi{ergative}, \isi{dative}, \isi{possessive}, \isi{associative}, \isi{characteristic}, \isi{locative}, \isi{allative}, \isi{ablative}, and \isi{purposive}), although some cases are not found in the pronouns (\isi{proprietive}, \isi{privative}, \isi{instrumental}, and \isi{similative}). The full set of formatives is listed in \tabref{perspron-table}.

\begin{table}
	\caption{Personal pronouns} \label{perspron-table}
	\begin{tabularx}{\textwidth}{Xllllll}
	\lsptoprule
		case&\Fsg{}&{\Fnsg}&\Ssg{}&\Snsg{}&\Tsg{}&\Tnsg{}\\\hline 
		\cline{3-7}
		{\Abs}&\multicolumn{1}{l|}{\emph{nzä}}&\multirow{2}{*}{\emph{ni}}&\multicolumn{2}{|c|}{\emph{bä}\hspace{.5cm}}&\multicolumn{2}{c|}{\emph{fi}}\rule{0pt}{3.5mm}\\\cline{4-5}\cline{6-7}
		{\Erg}&\multicolumn{1}{l|}{\emph{nze}}&&\multicolumn{1}{|l}{\emph{be}}&\emph{bné}&\emph{naf}&\emph{nafa}\\\cline{3-3}
		{\Dat}&\emph{nzun}&\emph{nzenm}&\emph{bun}&\emph{benm}&\emph{nafan}&\emph{nafanm}\\
		{\Poss}&\emph{nzone}&\emph{nzenme}&\emph{bone}&\emph{benme}&\emph{nafane}&\emph{nafanme}\\
		{\Loc}&\emph{nzudben}&\emph{nzedben}&\emph{budben}&\emph{bedben}&\emph{nafadben}&\emph{nafanmedben}\\
		{\All}&\emph{nzudbo}&\emph{nzedbo}&\emph{budbo}&\emph{bedbo}&\emph{nafadbo}&\emph{nafanmedbo}\\
		{\Abl}&\emph{nzudba}&\emph{nzedba}&\emph{budba}&\emph{bedba}&\emph{nafadba}&\emph{nafanmedba}\\\cline{6-7}
		{\Purp}&\emph{nzunar}&\emph{nzenar}&\emph{bunar}&\emph{benar}&\multicolumn{2}{|c|}{\emph{nafanar}}\rule{0pt}{3.5mm}\\\cline{6-7}
		{\Char}&\emph{nzonema}&\emph{nzenmema}&\emph{bonema}&\emph{benmema}&\emph{nafanema}&\emph{nafanmema}\\
		{\Assoc}\super{a}&\emph{ninrr}&\emph{ninä}&\emph{bnrr}&\emph{bnä}&\emph{nafrr}&\emph{nafä}\\	 
	\lspbottomrule
		\multicolumn{7}{l}{\footnotesize \super{a}The associative forms encode {\Du} versus {\Pl} (\S\ref{inclusorycontruction}).}
	\end{tabularx}
\end{table}

We can see from \tabref{perspron-table} that, as with the \isi{case} markers, there is no number distinction in the \isi{absolutive}. Only the first \isi{person} is an exception here. On the other hand, in the first \isi{person} \isi{non-singular}, the \isi{absolutive} and \isi{ergative} categories are neutralised. Furthermore, \tabref{perspron-table} shows that the \isi{characteristic} pronouns are built from the \isi{possessive} forms by suffixing \emph{-ma}. The three local cases and the \isi{purposive} pronouns share formal similarity with the \isi{dative} pronouns, namely the [u] vowel in the \isi{singular} forms. Personal pronouns typically constitute a complete \isi{noun phrase} ({\S}\ref{npsyntax}). Unlike nouns, personal pronouns cannot be modified by demonstratives or quantifiers.

\subsection{Interrogatives} \label{interrogatives-sec}

Cross-cutting the division of \isi{nominal}s is the subclass of interrogatives. These are roots used to indicate that the speaker does not know the (full) identity of a referent. Interrogatives belong to the following \isi{nominal} subclasses: pronouns (\emph{ra} `what', \emph{mä} `where', \emph{mane} `who, which', \emph{rma} `why, for what'), quantifiers (\emph{rnzam} `how many'), temporals (\emph{rthé} `when') or interrogative adverbs (\emph{mon} `how'). The degree to which these can be marked for case varies. Interrogatives may constitute a full \isi{noun phrase} (\ref{ex507}) or fill the \isi{determiner} slot (\ref{ex508}) of a noun phrase. In the following examples NPs are enclosed by square brackets.

\begin{exe}
	\ex \emph{ŋafyf \textbf{ra} kwa nm enzänzr?}\\
	\gll ŋafe=f [ra] kwa nm en\stem{zä}nzr\\
	father=\Erg.{\Sg} what {\Fut} maybe \Stsg:\Sbj>\Stpl:\Obj:\Nonpast:\Ipfv:\Venit/carry\\
	\trans `What might the father be carrying?' \Corpus{tci20111004}{RMA \#79}
	\label{ex507}
\end{exe}
\begin{exe}
	\ex \emph{eh, \textbf{ra} gru zane ŋamitwanzr nabi tutin?}\\
	\gll eh [ra gru zane] ŋa\stem{mitwa}nzr nabi tuti=n\\
	eh what shooting.star \Dem:{\Prox} \Stsg:\Sbj:\Nonpast:\Ipfv/swing bamboo branch={\Loc}\\
	\trans `Hey, what shooting star is swinging here on the bamboo branch?'\\ \Corpus{tci20111119-03}{ABB \#127}
	\label{ex508}
\end{exe}%ex255

The roots which are syntactically most active are the \isi{interrogative} pronouns \emph{ra} `what, what (kind)' and \emph{mane} `who, which'. Both can host almost all case clitics as we can see in \tabref{interrogatives-table}.\footnote{Some cases are impossible on semantic grounds, for example the instrumental case with animate referents, or the associative case with inanimate referents.}

\begin{table}
\caption{Interrogative pronouns}
\label{interrogatives-table}
	\begin{tabularx}{\textwidth}{XXXX}
		\lsptoprule
		{case} & {inanimate} & {animate} \Sg & {animate} \Nsg\\
		\hline
		\Abs & \emph{ra}&\multicolumn{2}{|c|}{\emph{mane}}\\
		&\textcolor{gray}{\footnotesize what, what kind} &\multicolumn{2}{|c|}{\textcolor{gray}{\footnotesize who, which}}\\\cline{3-4}
		{\Erg} &\emph{raf}&\emph{maf} &\emph{mafa}\\
		& \textcolor{gray}{\footnotesize what, what (kind)} &\textcolor{gray}{\footnotesize who, which} &\textcolor{gray}{\footnotesize who}\\
		\Dat &\emph{rafn}&\emph{mafn} &\emph{mafnm}\\
		&\textcolor{gray}{\footnotesize to what} &\textcolor{gray}{\footnotesize to whom} &\textcolor{gray}{\footnotesize to whom}\\
		\Poss &- &\emph{mafane} &\emph{mafanme}\\
		& &\textcolor{gray}{\footnotesize whose} &\textcolor{gray}{\footnotesize whose}\\
		\Loc &\emph{rafen} &\emph{mafadben} &\emph{mafanmedben}\\
		&\textcolor{gray}{\footnotesize at what place}&\textcolor{gray}{\footnotesize at whose place} &\textcolor{gray}{\footnotesize at whose place}\\
		\All &\emph{rafo} &\emph{mafadbo} &\emph{mafanmedbo}\\
		&\textcolor{gray}{\footnotesize to what} &\textcolor{gray}{\footnotesize to whom} &\textcolor{gray}{\footnotesize to whom}\\
		\Abl &\emph{rafa} &\emph{mafadba} &\emph{mafanmedba}\\
		&\textcolor{gray}{\footnotesize from what} &\textcolor{gray}{\footnotesize from whom} &\textcolor{gray}{\footnotesize from whom}\\
		\Ins &\emph{rame}&-&-\\
		&\textcolor{gray}{\footnotesize with what}&&\\
		\Purp &\emph{rar}&\emph{mafanar} &\emph{mafanmenar}\\
		&\textcolor{gray}{\footnotesize for what} &\textcolor{gray}{\footnotesize for whom} &\textcolor{gray}{\footnotesize for whom}\\
		\Char &\emph{rma} &\emph{mafanema} &\emph{mafanemema}\\
		&\textcolor{gray}{\footnotesize for what, why}&\textcolor{gray}{\footnotesize because of whom}&\textcolor{gray}{\footnotesize because of whom}\\
		\Assoc\super{a} &- &\emph{mafrr} &\emph{mafä}\\
		& &\textcolor{gray}{\footnotesize with whom} &\textcolor{gray}{\footnotesize with whom}\\
		\lspbottomrule
		\multicolumn{4}{l}{\footnotesize \super{a}The associative forms encode {\Du} versus {\Pl} (\S\ref{inclusorycontruction}).} 
	\end{tabularx}
\end{table}%interrogative pronouns

We can make two observations from \tabref{interrogatives-table}. First, as with other \isi{nominal} morphology, only animates are marked for \isi{number}. Secondly, the root \emph{rma} `why' patterns with \emph{ra}. Thus, it reflects a reduction of an earlier more transparent form \emph{rama} consisting of \emph{ra} with the \isi{characteristic} \isi{case} marker \emph{=ma} (lit. `for what').

The interrogatives \emph{mä} `where', \emph{mobo} `whither', \emph{moba} `whence' are not shown here because these interrogatives - along with \emph{mane} `which' - are part of a paradigm of demonstratives. As I will show below, Komnzo demonstratives make a fourway distinction between \isi{proximal}, \isi{medial}, \isi{distal}, and \isi{interrogative}. Compare \tabref{demonstratives-table} in {\S}\ref{demonstratives-sec} for the full set of demonstratives. The \isi{interrogative} \emph{mane} in \tabref{interrogatives-table} can also be used for inanimates, as in \emph{mane kar} `which village'.

Other interrogatives show a behaviour that aligns them with their respective \isi{nominal} subclass. The \isi{temporal} \isi{interrogative} \emph{rthé} `when' may be marked for \isi{temporal} \isi{case}, for example \emph{rthéthamane} `of what time' in (\ref{ex086}), where the speaker explains that he will move his garden plot closer to the road each year.

\begin{exe}
	\ex \emph{highway kwa wthayfakwé fi \textbf{rthéthamane}? ... ysokwren?}\\
	\gll highway kwa w\stem{thayfak}wé fi rthé=thamane (.) ysokwr=en\\
	road {\Fut} \Fsg:\Sbj>\Tsg.\F:\Obj:\Nonpast:\Ipfv/bring.out but when=\Temp{}.{\Poss} (.) rainy.season={\Loc}\\
	\trans `I will bring (the garden) up to the road, but when ... in which year (will I get there)?' \Corpus{tci20130823-06}{STK \#164-165}
	\label{ex086}
\end{exe}%ex086

The \isi{interrogative} \isi{adverb} \emph{mon} `how' frequently occurs with an instrumental \isi{case} (\emph{=me}). This is entirely optional and does not change its meaning (\ref{ex087}). \emph{Mon} or \emph{monme} are the interrogative counterpart to the manner demonstrative \emph{nima} `this way' (\S\ref{manner-demonstrative-subsec}).

\begin{exe}
	\ex \emph{bä \textbf{monme} miyatha zäkor komnzo fi nimäwä miyatha zfrärm ... komnzo zokwasi.}\\
	\gll bä mon=me miyatha zä\stem{kor} komnzo fi nima=wä miyatha zf\stem{rä}rm (.) komnzo zokwasi\\
	\Second.{\Abs} how={\Ins} knowledge \Stsg:\Sbj:\Rpst:\Pfv/become komnzo \Third.{\Abs} like={\Emph} knowledge \Tsg\F{}:\Sbj:\Pst:\Dur{}/be (.) komnzo language\\
	\trans `How you have learned Komnzo, she also knew it ... the Komnzo language.' \Corpus{tci20130911-03}{MBR \#18}
	\label{ex087}
\end{exe}%ex087

The \isi{interrogative} \isi{quantifier} \emph{rnzam} `how many, how much' occurs with a \isi{nominal} \isi{head}. It is possible for \emph{rnzam} to be marked for case if it follows its \isi{head}. However, there are no occurrences of this in the corpus. (\ref{ex088}) shows an example where the \isi{nominal} \isi{head} (\emph{kabe} `man') has been elided and consequently \emph{rnzam} is flagged with the \isi{ergative} \isi{case}. In the example, the speaker explains how a piece of wallaby skin is glued onto a kundu drum.

\begin{exe}
	\ex \emph{\textbf{rnzamé} thzé krekarth ... asar kabe o tabuthui kabe? ... neba thrakogr krekarth bäne ... tauri woku.}\\
	\gll rnzam=é thzé kre\stem{kar}th (.) asar kabe o tabuthui kabe (.) neba thra\stem{kogr} kre\stem{kar}th bäne (.) tauri woku\\
	how.many=\Erg.{\Nsg} ever \Stpl:\Sbj:\Irr:\Pfv/pull (.) four man or five man (.) opposite \Stpl:\Sbj:\Irr:\Stat/stand \Stpl:\Sbj:\Irr:\Pfv/pull \Recog.{\Abs} (.) wallaby skin\\
	\trans `However many will pull ... four or five people? They will stand opposite and pull that one ... the wallaby skin.' \Corpus{tci20120824}{KAA \#89-92}
	\label{ex088}
\end{exe}%ex088

\subsection{Indefinites}\label{indefinite-sec}

The \isi{indefinite} \isi{determiner} in Komnzo is \emph{nä}, and it covers the meaning of `some, other, another'. It behaves syntactically like a \isi{demonstrative}, i.e. occurs in the same slots of a noun phrase (\S\ref{npstructure}). Note that the numeral \emph{näbi} `one' is etymologically related to the \isi{indefinite}. Historically, this analysis is supported by other \ili{Yam languages}, for example \ili{Nen} where \emph{ämb} means `some' and \emph{ämbs} means `one' (\citealt{Evans:quant}). \emph{Nä} is used to form the \isi{indefinite} \isi{pronoun} \emph{nä bun} `someone, some other'. In example (\ref{ex069}), there are two occurrences of \emph{nä bun} in the dative \isi{case} and in the \isi{characteristic} \isi{case}. The speaker explains the way how people used to exchange the yams from the first harvest.

\begin{exe}
	\ex \emph{fi \textbf{nä bunn} saro! \textbf{nä bunanema} be zawob!}\\
	\gll fi {nä.bun=n} sa\stem{r}o {nä.bun=ane=ma} be za\stem{wob}\\
	but {someone=\Dat.\Sg} \Sg:\Sbj>\Tsg.\Masc:\Io:\Imp:\Pfv:\Andat/give {someone=\Poss.\Sg={\Char}} \Ssg{}.{\Erg} \Stsg:\Sbj:\Imp:\Pfv/eat\\
	\trans `But you give it (the yam) to someone else! You eat from someone else's!'\\ \Corpus{tci20120805-01}{ABB \#763-764}
	\label{ex069}
\end{exe}%ex069

Historically, \emph{nä bun} seems to derive from a combination of \emph{nä} and the second person singular \isi{dative} \isi{pronoun} \emph{bun} (see \tabref{perspron-table}), but it is unclear how this has happened. Synchronically, speakers no longer parse the two components as separate items.\footnote{Hence, it might also be written as one word, \emph{näbun} instead of \emph{nä bun}.} This is reflected in its grammatical behaviour: \emph{nä bun} can be marked for the same range of cases as personal pronouns, and like personal pronouns it may constitute a complete \isi{noun phrase}. \tabref{indefpron-table} lists all the \isi{case} forms of \emph{nä bun}.

\begin{table}
\caption{The indefinite pronoun}
\label{indefpron-table}
	\begin{tabularx}{.6\textwidth}{XXl}
		\lsptoprule
			case&{\Sg}&{\Nsg}\\ 
			\hline
			{\Abs}&\multicolumn{2}{|c|}{\emph{nä bun}}\\\cline{2-3}
			{\Erg}&\emph{nä bunf}&\emph{nä buné}\\
			{\Dat}&\emph{nä bunn}&\emph{nä bunnm}\\
			{\Poss}&\emph{nä bunane}&\emph{nä bunaneme}\\
			{\Loc}&\emph{nä bundben}& \emph{nä bunmedben}\\
			{\All}&\emph{nä bundbo}&\emph{nä bunmedbo}\\
			{\Abl}&\emph{nä bundba}&\emph{nä bunmedba}\\
			{\Purp}&\emph{nä bunar}&\emph{nä bunmenar}\\
			{\Char}&\emph{nä bunanema}&\emph{nä bunanemema}\\
			{\Assoc}\super{a}&\emph{nä bunrr}&\emph{nä bunä}\\		
		\lspbottomrule
			\multicolumn{3}{l}{\footnotesize \super{a}The associative forms encode {\Du} versus {\Pl} (\S\ref{inclusorycontruction}).}
	\end{tabularx}
\end{table}%indefinite \isi{pronoun}

Like the demonstratives ({\S}\ref{demonstratives-sec}), the \isi{indefinite} \emph{nä} can stand alone and take a subset of \isi{case} clitics. These are the \isi{instrumental} (\emph{näme} `with some other'), \isi{characteristic} (\emph{näma} `because of some other'), \isi{purposive} (\emph{nämr} `for some other'), \isi{proprietive} (\emph{näkarä} `with some other'). More commonly \emph{nä} functions as an \isi{indefinite} \isi{determiner}, as in \emph{nä kar} `some, other place' $\rightarrow$ `somewhere' or \emph{nä rokar} `some, other stuff' $\rightarrow$ `something' or \emph{nä kayé} `some yesterday\textbar{}tomorrow' $\rightarrow$ `sometime'. This can be extended to \emph{nä kabe} `some, another man' $\rightarrow$ `someone'. Two examples of the determiner use are given in (\ref{ex720}) and (\ref{ex500}). In the first example, the speaker just explained that it is possible to `borrow a sister' for exchange marriage from a clan, with whom one shares a land boundary. In the second example, he talks about tall posts, which were used to show off a clan's success in competitive yam cultivation.

\begin{exe}
	\ex \emph{wati ane \textbf{nä kayé} thräkorth ``ft kabe.''}\\
	\gll wati ane nä kayé thrä\stem{kor}th ft kabe\\
	well {\Dem} {\Indf} yesterday \Stpl:\Sbj>\Stpl:\Obj:\Irr:\Pfv/say ft people\\
	\trans `Sometimes they call those ones ``\emph{ft people}''.' \Corpus{tci20120814}{ABB \#322}
	\label{ex720}
\end{exe}
\begin{exe}
	\ex \emph{masu mane rera \textbf{nä far} fä yrästhgra.}\\
	\gll masu mane \stem{rä}ra nä far fä y\stem{räs}thgra\\
	masu which \Tsg.\F:\Sbj:\Pst:\Ipfv/be {\Indf} post {\Dist} \Tsg.\Masc:\Sbj:\Pst:\Stat/be.erected\\
	\trans `As for Masu, there was another post planted over there.'\Corpus{tci20120805-01}{ABB \#472}
	\label{ex500}
\end{exe}

Negative indefinites are expressed by adding the \isi{negator} \emph{keke}. Thus, \emph{nä zokwasi} means `some words', but if negated by \emph{keke} it expresses `no words whatsoever'. This is the way how the speaker describes the shameful reaction of one of the characters in (\ref{ex501}).

\begin{exe}
	\ex \emph{zokwasimär ŋafiyokwa ... \textbf{keke nä zokwasi}.}\\
	\gll zokwasi=mär ŋa\stem{fiyok}wa (.) keke nä zokwasi\\
	word={\Priv} \Stsg:\Sbj:\Pst:\Ipfv/make (.) {\Neg} {\Indf} words\\
	\trans `He was speechless ... no words whatsoever' \Corpus{tci20110802}{ABB \#115-116}
	\label{ex501}
\end{exe}

Negative indefinites can also be constructed with interrogatives. This is a strategy attested in many languages (\citealt{Haspelmath:1997indefinite}, \citealt{wals-115}). Thus, the concept of `nobody' can be expressed by \emph{kabe nä keke} (lit. `people some not') or with an \isi{interrogative}, for example \emph{mane nä keke} (lit. `who some not'). The order of elements is somewhat fixed in that the \isi{indefinite} always follows the \isi{interrogative}. In example (\ref{ex697}), the speaker describes a ritual, whereby an arrow is shot into a tree trunk to mark a particular woman for marriage.

\begin{exe}
	\ex \emph{\textbf{keke mane nä} yanyaka keräfi fumaksir fof.}\\
	\gll keke mane nä yan\stem{yak}a keräfi fumak-si=r fof\\
	{\Neg} who {\Indf} \Tsg.\Masc:\Sbj:\Pst:\Ipfv:\Venit/walk arrow pull.out-\Nmlz={\Purp} \Emph\\
	\trans `Nobody came to pull out that arrow.' \Corpus{tci20120814}{ABB \#144}
	\label{ex697}
\end{exe}

In example (\ref{ex719}), the speaker talks about \emph{tütü} `Pheasant Coucal', who was the guardian of fire before people knew about its existence. The first token of \emph{nä} has scope over \emph{kabe miyatha} (`people knowledge') and literally means `no people's knowledge whatsoever'. The second token of \emph{nä} is with the interrogative \emph{ra} (what.{\Abs}) and literally means `she made them knowledgeable about nothing'.

\begin{exe}
	\ex \emph{zwärifthmo ... \textbf{kabe miyatha keke nä} ... \textbf{keke ra nä} miyatha thfkonzrm. finzo miyatha zfrärm.}\\
	\gll zwä\stem{rifthm}o (.) kabe miyatha keke nä (.) keke ra nä miyatha thf\stem{ko}nzrm fi=nzo miyatha zf\stem{rä}rm\\
	\Sg:\Sbj>\Tsg.\F:\Obj:\Rpst:\Pfv:\Andat/hide (.) people knowledge {\Neg} {\Indf} (.) {\Neg} what {\Indf} kowledgeable \Stsg:\Sbj>\Stpl:\Obj:\Pst:\Dur/become \Third.\Abs={\Only} knowledge \Tsg.\F:\Sbj:\Pst:\Dur/be\\
	\trans `She hid away (the fire) ... no one knew ... she told them nothing. Only she knew.' \Corpus{tci20131008-01}{KAB \#27-29}
	\label{ex719}
\end{exe}

Positive indefinites are expressed without the use of \emph{nä}. Instead, the \isi{particle} \emph{thzé} `ever' is postposed to an \isi{interrogative}, resulting in \emph{ra thzé} `whatever', \emph{mane thzé} `whoever, whichever'. An example with \emph{rnzam} `how many' was shown in (\ref{ex088}). An example with \emph{maf} `who' is given in (\ref{ex089}), where the speaker has just shown be a particular, but then leaves it on the path.

\begin{exe}
	\ex \emph{zbo kwa sräzine \textbf{maf thzé} srewakuth.}\\
	\gll zbo kwa srä\stem{zin}e maf thzé sre\stem{wakuth}\\
	{\Prox}.{\All} {\Fut} \Fpl{}:\Sbj>\Tsg.\Masc:\Obj:\Irr:\Pfv/put.down who.{\Erg} ever \Stsg:\Sbj>\Tsg.\Masc:\Obj:\Irr:\Pfv/pick.up\\
	\trans `We will put it down here (for) whoever will pick it up.'\Corpus{tci20130907-02}{RNA \#479}
	\label{ex089}
\end{exe}%ex089

\subsection{Demonstratives} \label{demonstratives-sec}

Komnzo has a rich set of demonstratives. These form a functional class comprised of pronouns, determiners, adverbials, and verbal (pro-)clitics. They are treated as a subclass of \isi{nominal}s because all can be marked for a subset of the cases. Only the verb clitics and the \isi{immediate} \isi{demonstrative} cannot be marked for \isi{case}.

Dixon defines a \isi{demonstrative} as ``any item, other than 1st and 2nd pronouns, which can have pointing (or deictic) reference'' (\citeyear[61-62]{Dixon:2003dj}). We can see in \tabref{demonstratives-table} that among the more typical functions of demonstratives, i.e. spatial functions, there are some which border the notion of `\isi{deictic} reference'. These functions are \isi{recognitional} (`shared knowledge'), \isi{anaphoric} (`tracking'), \isi{immediate} (`attention'), \isi{interrogative} (`lack of knowledge'), and \isi{apprehensive} (`warning'). In spite of this diversity of functions, the main formatives constitute a neat paradigm with a four-way distinction between \isi{proximal}, \isi{medial}, \isi{distal} and \isi{interrogative}. This quadripartite structure builds formally on the initial consonants: \emph{z}, \emph{b}, \emph{f} and \emph{m} respectively. The structure of the system is quite similar to \ili{Japanese} demonstratives, as described by Coulmas (\citeyear{Coulmas:1982wl}).

\begin{table}
\caption{Demonstratives}
\label{demonstratives-table}
	\begin{tabularx}{\textwidth}{lXXXXX}
		\lsptoprule
		&pronoun&adverbial&adv.{\All}&adv.{\Abl}&verb clitic\\ \midrule
		{\Prox}&\emph{zane}&\emph{zä}&\emph{zbo}&\emph{zba}&\emph{z=}\\
		&\textcolor{gray}{\footnotesize{\Dem:{\Prox} `this'}}&\textcolor{gray}{\footnotesize{{\Prox} `here'}}&\textcolor{gray}{\footnotesize{\Prox.{\All} `hither'}}&\textcolor{gray}{\footnotesize{\Prox.{\Abl} `hence'}}&\textcolor{gray}{\footnotesize{\Prox= `here'}}\\
		\Med{}&{\cellcolor[gray]{.90}}\emph{bäne}&\emph{bä}&\emph{bobo}&\emph{boba}&\emph{b=}\\
		&{\cellcolor[gray]{.90}}\textcolor{gray}{\footnotesize{\Dem:\Med{} `that'}}&\textcolor{gray}{\footnotesize{\Med{} `there'}}&\textcolor{gray}{\footnotesize{\Med.{\All} `thither'}}&\textcolor{gray}{\footnotesize{\Med.{\Abl} `thence'}}&\textcolor{gray}{\footnotesize{\Med= `there'}}\\
		{\Dist}&{\cellcolor[gray]{.90}}\emph{ane}&\emph{fä}&\emph{fobo}&\emph{foba}&\emph{f=}\\
		&\textcolor{gray}{{\cellcolor[gray]{.90}}\footnotesize{\Dem~}}&\textcolor{gray}{\footnotesize{{\Dist} `yonder'}}&\textcolor{gray}{\footnotesize{\Dist.{\All} `to over there'}}&\textcolor{gray}{\footnotesize{\Dist.{\Abl} `from over there'}}&\textcolor{gray}{\footnotesize{\Dist= `yonder'}}\\
		\textsc{interrog}&\emph{mane}&\emph{mä}&\emph{mobo}&\emph{moba}&{\cellcolor[gray]{.90}}\emph{m=}\\
		&\textcolor{gray}{\footnotesize{`which'}}&\textcolor{gray}{\footnotesize{`where'}}&\textcolor{gray}{\footnotesize{where.{\All} `whither'}}&\textcolor{gray}{\footnotesize{where.{\Abl} `whence'}}&\textcolor{gray}{{\cellcolor[gray]{.90}}\footnotesize{where=, \Appr=}}\\ \midrule
		{\Imm}&&{\cellcolor[gray]{.90}}\emph{zf}&&&\\
		&&{\cellcolor[gray]{.90}}\textcolor{gray}{\footnotesize{{\Imm} `right here'}}&&&\\
		{\Recog}&{\cellcolor[gray]{.90}}\emph{baf}&&&&\\		
		&{\cellcolor[gray]{.90}}\textcolor{gray}{\footnotesize{{\Recog} `that one'}}&&&&\\
		\lspbottomrule
	\end{tabularx}
\end{table}

Following Diessel (\citeyear{Diessel:2009tg}), I outline the syntactic distribution of demonstratives first. In \tabref{demonstratives-table}, a number of demonstratives appear in shaded cells. These have additional functions and to some extent different syntactic distributions. They will be discussed in separate sections to follow.

Diessel (\citeyear{Diessel:2009tg}) defines four syntactic contexts in which demonstratives occur: as independent pronouns that occupy an adpositional or verbal argument position (``pronominal''); with nouns in noun phrases (``adnominal''); as verb modifiers (``adverbial''); and in \isi{copula} and \isi{non-verbal} clauses (``identificational''). Some languages have distinct lexical categories for each function. Thus, Diessel calls the four categories: \isi{demonstrative} pronominals, \isi{demonstrative} determiners, \isi{demonstrative} adverbs, and \isi{demonstrative} identifiers (\citeyear[3]{Diessel:2009tg}). See Himmelmann (\citeyear{Himmelmann:1996tp}) who makes similar distinctions. Demonstratives in Komnzo occur in all four syntactic contexts. Below, I use the \isi{proximal} in order to illustrate the different syntactic contexts.

\subsubsection{Pronominal and adnominal demonstratives} \label{pronominal-demonstratives-subsec}

Demonstratives can be used pronominally (\ref{ex061}) or adnominally (\ref{ex060}).

\begin{exe}
	\ex \emph{moba \textbf{zane} nm nzyaniyak?}\\
	\gll moba zane nm nz=yan\stem{yak}\\
	where.{\Abl} \Dem:{\Prox} maybe \Immpst=\Tsg.\Masc:\Sbj:\Nonpast:\Ipfv:\Venit/walk\\
	\trans `Where might this (man) have come from?' \Corpus{tci20120901-01}{MAK \#87}
	\label{ex061}
\end{exe}
\begin{exe}
	\ex \emph{\textbf{zane} namä ezi mrmren nzä kwa trikasi ŋatrikwé.}\\
	\gll zane namä ezi mrmr=en nzä kwa trik-si ŋa\stem{trik}wé\\
	\Dem:{\Prox} good morning inside={\Loc} \Fsg.{\Abs} {\Fut} tell-{\Nmlz} \Fsg:\Sbj:\Nonpast:\Ipfv/tell\\
	\trans `In this beautiful morning, I will tell a story.' \Corpus{tci20111119-01}{ABB \#2-3}
	\label{ex060}
\end{exe}

When used pronominally, demonstratives serve as the host for a subset of the case clitics. The examples below show case marking with the \isi{instrumental} (\ref{ex064}), \isi{purposive} (\ref{ex062}), and \isi{characteristic} case (\ref{ex063}). Rarely, they occur with the \isi{proprietive} (\ref{ex096}), and there are no corpus examples with the \isi{privative} case. Demonstratives are not marked for other cases, but they can take other \isi{nominal} morphology like the \isi{exclusive} \isi{clitic} \emph{=nzo} or the \isi{emphatic} \isi{clitic} \emph{=wä}.

\begin{exe}
	\ex \emph{arammba yare \textbf{zaneme} zf äfiyokwre.}\\
	\gll arammba yare zane=me zf ä\stem{fiyok}wre\\
	arammba bag \Dem:\Prox={\Ins} {\Imm} \Fpl:\Sbj>\Stpl:\Obj:\Nonpast:\Ipfv/make\\
	\trans `We make the \ili{Arammba} bags with this one right here.'\Corpus{tci20130907-02}{JAA \#410}
	\label{ex064}
\end{exe}
\begin{exe}
	\ex \emph{ebar fobo fof zäbtha. \textbf{zanemr} zena znrä.}\\
	\gll ebar fobo fof zä\stem{bth}a zane=mr zena z=n\stem{rä}\\
	head \Dist.{\All} {\Emph} \Stsg:\Sbj:\Pst:\Pfv/finish \Dem:\Prox={\Purp} today \Prox=\Fpl:\Sbj:\Nonpast:\Ipfv/be\\
	\trans `From this time onwards, the head-hunting finished. For this (reason), we are here today.' \Corpus{tci20111107-01}{MAK \#148-149}
	\label{ex062}
\end{exe}
\begin{exe}
	\ex \emph{nafanmedben keke znsä rä. \textbf{zanemanzo} ŋathwekwrth ... yusi fathasimanzo.}\\
	\gll nafanmedben keke znsä \stem{rä} zane=ma=nzo ŋa\stem{thwek}wrth (.) yusi fath-si=ma=nzo\\
	\Tnsg.\Anim.{\Loc} {\Neg} work \Tsg.\F:\Sbj:\Nonpast:\Ipfv/be \Dem:\Prox=\Char={\Only} \Stpl:\Sbj:\Nonpast:\Ipfv/be.happy (.) grass hold-\Nmlz=\Char={\Only}\\
	\trans `The (hard) work is not theirs (but ours). They are happy with doing just this ... just the weeding.' \Corpus{tci20130823-06}{STK \#109-111}
	\label{ex063}
\end{exe}
\begin{exe}
	\ex \emph{zane fthé keke srarä ziyarä keke kwa sräthorth moneyme. \textbf{zanekaräsü} ane srarä kwot.}\\
	\gll zane fthé keke sra\stem{rä} z=ya\stem{rä} keke kwa srä\stem{thor}th money=me zane=karä=sü ane sra\stem{rä} kwot\\
	\Dem:{\Prox} when {\Neg} \Tsg.\Masc:\Irr:\Ipfv/be \Prox=\Tsg.\Masc:\Io:\Nonpast:\Ipfv/be {\Neg} {\Fut} \Stpl:\Sbj>\Tsg.\Masc:\Obj:\Irr:\Pfv/carry money={\Ins} \Dem:\Prox=\Prop=\Etc{} {\Dem} \Tsg.\Masc:\Sbj:\Irr:\Ipfv/be properly\\
	\trans `If this (root) is not here, they won't buy it. Only with all of this will, they buy it.' \Corpus{tci20130907-02}{RNA \#471-473}
	\label{ex096}
\end{exe}

Case marked demonstratives are frequently used as conjunctions to connect the following clause, especially demonstratives marked for the \isi{characteristic} (\emph{zanema, bänema, anema} `therefore, because'), the \isi{instrumental} (\emph{zaneme, bäneme, aneme} `with this/that, thereby') and the \isi{purposive} (\emph{zanemr, bänemr, anemr} `therefore'). See (\ref{ex097}) for an example with \emph{bänema}.

\begin{exe}
	\ex \emph{naf nima ``samg! \textbf{bänema} nä buné fof yruthrth byé ... keke kwosi yathizr.''}\\
	\gll naf nima sa\stem{mg} bäne=ma {nä bun=é} fof y\stem{ru}thrth b=\stem{yé} (.) keke kwosi ya\stem{thi}zr\\
	\Tsg.{\Erg} {\Quot} \Ssg:\Sbj>\Tsg.\Masc:\Obj:\Imp:\Pfv/shoot \Dem:\Med={\Char} {\Indf=\Erg.\Nsg} {\Emph} \Stpl:\Sbj>\Tsg.\Masc:\Obj:\Nonpast:\Ipfv/shoot \Med=\Tsg.\Masc:\Sbj:\Nonpast:\Ipfv/be (.) {\Neg} dead \Tsg.\Masc:\Sbj:\Nonpast:\Ipfv/die\\
	\trans `He said: ``Shoot it! Because others are shooting hard and it is not dying.'''\\ \Corpus{tci20131013-01}{ABB \#101-103}
	\label{ex097}
\end{exe}

What has been mentioned above about case marked demonstratives also holds for the \isi{interrogative} \emph{mane} `who, which' in \tabref{demonstratives-table}. Like other interrogatives, it can be used as a relative \isi{pronoun}, and it can be marked for a subset of the \isi{case} clitics: \isi{absolutive} \emph{mane} `who, which', \isi{characteristic} \emph{manema} `because of which', \isi{instrumental} \emph{maneme} `with which', and \isi{purposive} \emph{manemr} `for which'.\footnote{The animate referents for cases other than the absolutive are expressed by the interrogatives in \tabref{interrogatives-table}.} An example with \emph{maneme} is given in (\ref{ex502}).

\begin{exe}
	\ex \emph{ane fathnzo zfrärm. ... wämne keke ... dödönzo ... dödö \textbf{maneme} ŋarenwre fath.}
	\gll ane fath=nzo zf\stem{rä}rm (.) wämne keke (.) dödö=nzo (.) dödö mane=me ŋa\stem{ren}wre fath\\
	{\Dem} clearing={\Only} \Tsg.\F:\Sbj:\Pst:\Dur/be (.) tree {\Neg} (.) dödö={\Only} (.) dödö which={\Ins} \Fpl:\Sbj:\Nonpast:\Ipfv/sweep clearing.\\
	\trans `It was a clear place ... no trees ... only \emph{dödö} ... that \emph{dödö} with which we sweep the place.' \Corpus{tci20120821-02}{LNA \#25-27}
	\label{ex502}
\end{exe}

The description of demonstratives leaves us with an analytic problem. Is there justification for setting up two separate subcategories: \isi{demonstrative} pronouns and \isi{demonstrative} determiners? The fact that they can stand for a whole \isi{noun phrase} is not sufficient evidence for setting up an independent subcategory of \isi{demonstrative} pronouns because the \isi{head} of a \isi{noun phrase} can be omitted and leave only a modifier including a \isi{demonstrative} \isi{determiner}. The demonstratives described here do not take the full range of cases as other pronouns, for example the personal pronouns (\ref{personalpronouns-sec}), the \isi{indefinite} (\ref{indefinite-sec}) and \isi{recognitional} \isi{pronoun} (\ref{recognitional-pronoun-subsec}). Therefore, I describe them simply as demonstratives with a \isi{pronominal} and adnominal function.

\subsubsection{Adverbial demonstratives} \label{adverbial-demonstratives-subsec}

\tabref{demonstratives-table} includes a column of adverbial demonstratives (e.g. \emph{zä} `here') with a dedicated form for the \isi{allative} (\emph{zbo} `hither') and the \isi{ablative} \isi{case} (\emph{zba} `from here'). These are used for verbal modification, as in example (\ref{ex066}) with \emph{zä} `here' and in example (\ref{ex065}) with \emph{foba} `from there' and \emph{zbo} `hither'.

\begin{exe}
	\ex \emph{taurianeme moth \textbf{zä} wnthn.}\\
	\gll tauri=aneme moth zä wn\stem{thn}\\
	wallaby=\Poss.{\Nsg} path {\Prox} \Tsg.\F:\Sbj:\Nonpast:\Ipfv:\Venit/lie.down\\
	\trans `The wallabies' path lies here.' \Corpus{tci20130903-01}{MKW \#35}
	\label{ex066}
\end{exe}
\begin{exe}
	\ex \emph{wati, ane \textbf{foba} ŋanmonziknwr. \textbf{zbo} wänyak. zane mnz zf wrwr.}\\
	\gll wati ane foba ŋan\stem{monzikn}wr zbo wän\stem{yak} zane mnz zf w\stem{r}wr\\
	then {\Dem} \Dist.{\Abl} \Stsg:\Sbj:\Nonpast:\Ipfv:\Venit/prepare \Prox.{\All} \Tsg.\F:\Sbj:\Nonpast:\Ipfv:\Venit/walk \Dem:{\Prox} house {\Imm} \Stsg:\Sbj>\Tsg.\F:\Obj:\Nonpast:\Ipfv/build\\
	\trans `Then, this (bird) prepares over there and she comes here to build her nest right here.' \Corpus{tci20120815}{ABB \#48}
	\label{ex065}
\end{exe}

The allative adverbials are often found with an /mr/ element attached to them: \emph{zbomr}, \emph{bobomr} and \emph{fobomr}. I take this as frozen morphology of the purposive case marker \emph{=r}. These forms are often used as connectives to mean `until' (\S\ref{connectives-sec}).

\subsubsection{Clitic demonstratives} \label{demonstrative-identifiers-subsec}

Diessel (\citeyear{Diessel:2009tg}) includes the syntactic context of identification (identificational demonstratives) and finds a distinct class (\isi{demonstrative} identifiers) in a number of languages. We find both the syntactic context as well as the distinct class in the language.

Komnzo possesses a set of \isi{deictic} verbal proclitics which I call \isi{clitic} demonstratives (\tabref{demonstratives-table}). These clitics are used for identification and can attach to any inflected verb. In example (\ref{ex067}), two brothers are trying to kill a creature by shooting an arrow into its heart.

\begin{exe}
	\ex \emph{naf nima ``keke fi miyamr erä fofosa mä rä. nze komnzo \textbf{zimarwé} fof.''}\\
	\gll naf nima keke fi miyamr e\stem{rä} fofosa mä \stem{rä} nze komnzo z=y\stem{mar}wé fof\\
	\Tsg.{\Erg} {\Quot} {\Neg} \Third.{\Abs} ignorance \Stpl:\Sbj:\Nonpast:\Ipfv/be heart where \Tsg.\F:\Sbj:\Nonpast:\Ipfv/be \Fsg.{\Erg} only \Prox=\Fsg:\Sbj>\Tsg.\Masc:\Obj:\Nonpast:\Ipfv/see {\Emph}\\
	\trans `He said: ``They don't know where its heart is. I can see it here.'''\\ \Corpus{tci20131013-01}{ABB \#104-105}
	\label{ex067}
\end{exe}

While they can attach to any verb, \isi{clitic} demonstratives are found with the \isi{copula} in 90\% of the tokens. Usually, the copula follows the main verb, as in example (\ref{ex068}) and (\ref{ex070}). The \isi{clitic} \isi{demonstrative} plus copula stands in apposition to the main clause, but they often form one intonational unit.

\begin{exe}
	\ex \emph{fi zena zane zf dö sakwré \textbf{zyé}.}\\
	\gll fi zena zane zf dö sa\stem{kwr}é z=\stem{yé}\\
	but today \Dem:{\Prox} {\Imm} goanna \Fsg:\Sbj>\Tsg.\Masc:\Obj:\Rpst:\Pfv/hit \Prox=\Tsg.\Masc:\Sbj:\Nonpast:\Ipfv/be\\
	\trans `But today I have killed this goanna here.' \Corpus{tci20120821-01}{LNA \#67}
	\label{ex068}
\end{exe}
\begin{exe}
	\ex \emph{yasifa foba fof ni zane zewärake zena \textbf{znrä}.}\\
	\gll yasi=fa foba fof ni zane ze\stem{wär}ake zena z=n\stem{rä}\\
	yasi={\Abl} \Dist.{\Abl} {\Emph} {\Fnsg} \Dem:{\Prox} \Fpl:\Sbj:\Pst:\Ipfv/crack today \Prox=\Fpl:\Sbj:\Nonpast:\Ipfv/be\\
	\trans `From Yasi, we originate from him and (therefore) we are here today.'\\ \Corpus{tci20111107-01}{MAK \#86}
	\label{ex070}
\end{exe}

The \isi{clitic} \isi{demonstrative} plus \isi{copula} is the primary strategy to make an identificational reference much like \ili{English} `there it is' or `here you go'. This is usually accompanied by a pointing gesture. Diessel points out that in other languages ``\isi{demonstrative} identifiers are often functionally equivalent to a \isi{demonstrative} plus copula'' (\citeyear[10]{Diessel:2009tg}). Komnzo confirms this pattern and, therefore, I analyse the \isi{clitic} \isi{demonstrative} plus copula as one unit. I adopt the label \isi{demonstrative} \isi{identifier} from Diessel. I address this topic in the description of verb morphology (\S\ref{deicticcliticssection}).

The \isi{demonstrative} \isi{identifier} always agrees with some element in the main clause. Hence, if the argument in the clause is modified by a \isi{medial} \isi{demonstrative}, that same \isi{medial} category will be used in the \isi{demonstrative} \isi{identifier}. An example with the \isi{proximal} is given in (\ref{ex071}). Note that the \isi{medial} \isi{demonstrative} \isi{identifier} \emph{byé} instead of the \isi{proximal} \emph{ziyé} would render the sentence ungrammatical.

\begin{exe}
	\ex \emph{zane kabe zf yé \textbf{zyé}.}\\
	\gll zane kabe zf \stem{yé} z=\stem{yé}\\
	\Dem:{\Prox} man {\Imm} \Tsg.\Masc:\Sbj:\Nonpast:\Ipfv/be \Prox=\Tsg.\Masc:\Sbj:\Nonpast:\Ipfv/be\\
	\trans `It is this man right here.' \Corpus{tci20111004}{RMA \#51}
	\label{ex071}
\end{exe}

The verbal \isi{clitic} \emph{m=} is a special case. It can be attached to a \isi{copula}, which will produce a \isi{question}. In example (\ref{ex094}), the speaker looks around for a particular tree species to show to me. Then she suddenly finds it.

\begin{exe}
	\ex \emph{\textbf{myé} yorär? yorär zyé ... zikogr.}\\
	\gll m=\stem{yé} yorär yorär z=\stem{yé} (.) z=y\stem{kogr}\\
	where=\Tsg.\Masc:\Sbj:\Nonpast:\Ipfv/be yorär yorär \Prox=\Tsg.\Masc:\Sbj:\Nonpast:\Ipfv/be (.) \Prox=\Tsg.\Masc:\Sbj:\Nonpast:\Stat/stand\\
	\trans `Where is \emph{yorär}? \emph{Yorär} is here ... It stands here.'\Corpus{tci20130907-02}{JAA \#449-451}
	\label{ex094}
\end{exe}

The same \emph{m=} \isi{clitic}, when attached to verb forms in \isi{imperative} or irrealis mood, receives an \isi{apprehensive} interpretation: `don't do X' or `you might X'. An example is given in (\ref{ex095}). The \emph{m=} \isi{clitic} is discussed in {\S}\ref{verbal-proclitics-subsec} and again in {\S}\ref{apprehensivem} as part of the description of the TAM system.

\begin{exe}
	\ex \emph{aya msar \textbf{mkrätrth}!}\\
	\gll aya msar m=krä\stem{tr}th\\
	oh ant {\Appr}=\Stpl:\Sbj:\Irr:\Pfv/fall\\
	\trans `Oh, the ants might fall down!' \Corpus{tci20130907-02}{RNA \#678}
	\label{ex095}
\end{exe}

\subsubsection{Anaphoric \emph{ane}} \label{anaphoric-demonstrative-subsec}

In \tabref{demonstratives-table} \emph{ane} has been glossed as a general \isi{demonstrative} ({\Dem}), even though it is placed in the paradigm position where one would expect the \isi{distal} \isi{demonstrative}. However, \emph{ane} has no spatial reference, but it is used for \isi{anaphoric} reference. It marks a referent which has been established in the preceding context. Consequently, \emph{ane} marks definiteness and is the opposite of the indefinite \emph{nä} ({\S}\ref{indefinite-sec}). Both cannot occur in the same noun phrase.

There is evidence from several sources that \emph{ane} is the result of phonological reduction and semantic bleaching. Recordings from the 1980s by Mary Ayres contain a number of occurrences of a \isi{demonstrative} \emph{fane}, and older speakers today identify this as `the way, how old people used to speak'. Indeed, the position in the paradigm would suggest an initial consonant \emph{f}. This is attested in other \ili{Tonda} varieties, e.g. \ili{Wartha} Thuntai \emph{fana}. We can conclude that this \isi{demonstrative} has undergone phonological reduction from \emph{fane} to \emph{ane} over the last two generations of speakers. Moreover, we can infer semantic bleaching from spatial (\isi{distal}) to \isi{anaphoric} (tracking) from its position in the paradigm. However, we cannot put a time frame to the process of semantic bleaching, because it is unclear whether or not \emph{fane} had a spatial meaning in the old recordings in addition to its \isi{anaphoric} use.

The \isi{anaphoric} \isi{demonstrative} behaves in other respects like the \isi{demonstrative} pronouns and determiners ({\S}\ref{pronominal-demonstratives-subsec}). One exception is the agreement described in {\S}\ref{demonstrative-identifiers-subsec} between the \isi{demonstrative} in the main clause and the \isi{demonstrative} \isi{identifier}. Since \emph{ane} has no spatial reference, it may combine with the \isi{proximal} and the \isi{medial} \isi{demonstrative} \isi{identifier} as can be seen in example (\ref{ex092}) and (\ref{ex091}), respectively.

\begin{exe}
	\ex \emph{fintäth \textbf{ane} \textbf{ziyé} ... yemaneme dagon.}\\
	\gll fintäth ane z=\stem{yé} (.) yem=aneme dagon\\
	fintäth {\Dem} \Prox=\Tsg.\Masc:\Sbj:\Nonpast:\Ipfv/be (.) cassowary={\Poss}.{\Nsg} food\\
	\trans `This \emph{fintäth} (fruit) here is the cassowaries' food.'\Corpus{tci20130907-02}{RNA \#316}
	\label{ex092}
\end{exe}
\begin{exe}
	\ex \emph{watik, nge \textbf{ane} zefar \textbf{byé} ruga monegsir.}\\
	\gll watik nge ane ze\stem{far} b=\stem{yé} ruga moneg-si=r\\
	then child {\Dem} \Stsg:\Sbj:\Rpst:\Pfv/set.off \Med{}=\Tsg.\Masc:\Sbj:\Nonpast:\Ipfv/be pig wait-{\Nmlz}={\Purp}\\
	\trans `Then the boy there set off to take care of the pig.' \Corpus{tci20130901-04}{YUK \#7}
	\label{ex091}
\end{exe}

\subsubsection{Immediate \emph{zf}} \label{immediate-demonstrative-subsec}

The \isi{immediate} \isi{demonstrative} \emph{zf} is related to the proximate series on the basis of it sharing the first consonant. The \isi{immediate} adds a pragmatic component to the spatial function of demonstratives, in that it draws the addressee's attention to someone or something in close proximity. It is often accompanied by a pointing gesture. Therefore I translate \emph{zf} as `right here' to \ili{English}. We have seen \emph{zf} already in examples (\ref{ex064}), (\ref{ex065}) and (\ref{ex071}).

\emph{Zf} is syntactically inert as it cannot be marked for \isi{case}. It occurs in preverbal position and only the TAM particles or the \isi{negator} may occur between the \isi{immediate} \isi{demonstrative} and the verb, as in example (\ref{ex072}).

\begin{exe}
	\ex \emph{zane \textbf{zf} kwa esinzre zöbthé.}\\
	\gll zane zf kwa e\stem{si}nzre zöbthé\\
	{\Dem}:{\Prox} {\Imm} {\Fut} \Fpl:\Sbj>\Stpl:\Obj:\Nonpast:\Ipfv/cook first\\
	\trans `We will cook these (yams) here first.' \Corpus{tci20121001}{ABB \#62}
	\label{ex072}
\end{exe}

\subsubsection{Recognitional \emph{baf}} \label{recognitional-pronoun-subsec}

Following Himmelmann (\citeyear{Himmelmann:1996tp}), I use the term ``\isi{recognitional} \isi{demonstrative}'' for \textit{baf}. Himmelmann describes a distinct \isi{recognitional} use of demonstratives, which has become grammaticalised in some languages. Amongst these are a number of Australian languages, for example \ili{Nunggubuyu} (\citealt{Heath:1984uk}) and \ili{Yankunytjatjara} (\citealt{Goddard:1985tw}). See Himmelmann (\citeyear[231ff.]{Himmelmann:1996tp}) for further discussion. Komnzo \emph{baf} counts as another example for this grammaticalisation. I analyse \emph{baf} as a \isi{pronoun} because it can be marked for all cases. In contrast to other demonstratives, there are both \isi{animate} and \isi{inanimate} forms (\tabref{recogpron-table}).

Garde characterises the \isi{recognitional} \isi{demonstrative} in \ili{Bininj Gunwok} as reflecting ``a belief on the part of the speaker that sufficient common ground exists for hearers to make the necessary inferences'' (\citeyear[250]{Garde:2013ut}). In Komnzo \textit{baf} has a number of uses which all echo the notion of common ground. A speaker may use \emph{baf} to introduce a referent which he believes the hearer to know about. This can be a first mention of a referent which is not topical or in \isi{focus} (i.e. from an earlier part of a narrative). Moreover, the \isi{recognitional} is often used as a filler in tip-of-the-tongue situations like `whatchamacallit' in \ili{English}. The \isi{recognitional} can be described as an invitation to the addressee to ask for the referent or, more commonly, to fill in herself the appropriate word. Hence, the \isi{recognitional} can be used pragmatically to keep a conversation going and assure the addressee's attention. Often the \isi{recognitional} is employed as a strategy of circumspection, for example if the speaker is in a taboo relationship with a specific person and, therefore, has to avoid using her proper name.

Example (\ref{ex093}) is a first mention of a particular person in a narrative. Although not required, it is quite common for the speaker to fill in the `missing' referent after a short lapse. Thus, the phrase \emph{masenane mezü} `Masen's widow' refers back to \emph{bafane mezü} `that one's widow'.

\begin{exe}
	\ex \emph{mabata fi mezü zwamnzrm. \textbf{bafane} mezü rera ... masenane mezü.}\\
	\gll mabata fi mezü zwa\stem{m}nzrm baf=ane mezü \stem{rä}ra (.) masen=ane mezü\\
	mabata \Third.{\Abs} widow \Tsg\F{}:\Sbj:\Pst:\Dur{}/dwell {\Recog}=\Poss.{\Sg} widow \Tsg\F{}:\Sbj:\Pst:\Ipfv/be (.) masen=\Poss.{\Sg} widow\\
	\trans `Mabata stayed as a widow. She was that one's widow ... Masen's widow.' \Corpus{tci20120814}{ABB \#18-20}
	\label{ex093}
\end{exe}

The \isi{recognitional} \isi{demonstrative} is built on the \isi{medial} \isi{demonstrative}, as we can tell by the initial consonant \emph{b}. It follows that the \isi{recognitional} must have emerged through semantic extension from the \isi{medial} \isi{demonstrative}, and only later developed distinct forms for all the cases. We find that a number of forms serve a double function. For example, \emph{bäne} can function as \isi{demonstrative} \isi{pronoun} (`that') and as \isi{recognitional} \isi{pronoun} (`the one I presume that you know about'). But the two differ in their combinatorics. While the \isi{demonstrative} can modify as well as replace a \isi{nominal} \isi{head} of a phrase, the \isi{recognitional} operates only pronominally. I have already shown in example (\ref{ex093}) that it is quite common for a speaker to fill in the intended referent of a \isi{recognitional} herself, sometimes after the clause, sometimes after a short pause. This leaves us with the problem of distinguishing the \isi{medial} \isi{demonstrative} from the \isi{recognitional} in a phrase like \emph{bäne kabe}. However, prosody signals which of the two it is. If both words belong to the same intonation contour, it is the \isi{medial} \isi{demonstrative}: `that man'. If there is short break in the intonation or a longer pause, it is the \isi{recognitional}: `that one ... the man'. The other \isi{case} forms which are formally identical are impossible to distinguish in a clear way. For example, \emph{bänema} `therefore, because' is often used to connect another clause ({\S}\ref{pronominal-demonstratives-subsec}). In this case we always find a break in the intonation. It is best to interpret the formal identity as a signal of the semantic extension of the \isi{medial} \isi{demonstrative}. That being said, it would be wrong to conclude that the \isi{recognitional} is merely a function of the \isi{medial} \isi{demonstrative}.

\begin{table}
\caption{The recognitional pronoun}
\label{recogpron-table}
	\begin{tabularx}{\textwidth}{XXXX}
		\lsptoprule
			{case} & {inanimate} & {animate} \Sg & {animate} \Nsg\\
			\hline
			\Abs &\multicolumn{3}{|c|}{\emph{bäne}}\\\cline{2-4}
			{\Erg} &\multicolumn{2}{|c|}{\emph{baf}}&\emph{bafa}\\\cline{2-3}
			\Dat &-&\emph{bafn} &\emph{bafnm}\\
			\Poss &- &\emph{bafane} &\emph{bafanme}\\
			\Loc &\emph{bafen} &\emph{bafadben} &\emph{bafanmedben}\\
			\All &\emph{bänefo} &\emph{bafadbo} &\emph{bafanmedbo}\\
			\Abl &\emph{bänefa} &\emph{bafadba} &\emph{bafanmedba}\\
			\Ins &\emph{bäneme}&-&-\\
			\Purp &\emph{bänemr}&-&-\\
			\Char &\emph{bänema} &\emph{bafanema} &\emph{bafanemema}\\
			\Prop&\emph{bänekarä}&-&-\\
			\Priv&\emph{bänemär}&-&-\\
			\Assoc\super{a} &- &\emph{bafrr} &\emph{bafä}\\		
		\lspbottomrule
			\multicolumn{4}{l}{\footnotesize \super{a}The associative forms encode {\Du} versus {\Pl} (\S\ref{inclusorycontruction}).}
	\end{tabularx}
\end{table}

As we can see in Table (\ref{recogpron-table}), the \isi{recognitional} can be marked for all cases. In this respect, the \isi{recognitional} surpasses even personal pronouns in the richness of its distinctions because there are animate and inanimate \isi{case} forms.

\subsubsection{Manner demonstrative \emph{nima}} \label{manner-demonstrative-subsec}

Komnzo has a manner \isi{demonstrative} \emph{nima} which is best translated as `like this' or `do this way'. In some languages this demonstrative is assigned to the class of verbs, for example in \ili{Boumaa Fijian} and \ili{Dyirbal} (\citealt[72]{Dixon:2003dj}). In other languages it is a \isi{nominal}, for example in \ili{Kayardild} (\citealt[214]{Evans:1995uf}). \emph{Nima} falls in the latter category. It is a \isi{nominal} which can be marked for a subset of cases (\isi{instrumental}, \isi{characteristic}, \isi{purposive}, \isi{proprietive}, and \isi{privative}). It shares no morpho-syntactic characteristics with verbs, but may either modify a verb (\ref{ex099}) or express a whole event (\ref{ex100}). Example (\ref{ex099}) is from a pig hunting story and \emph{nima} is accompanied by the appropriate gesture describing how and where the person was standing. In (\ref{ex100}) it expresses the whole following clause (`that I was walking towards them').

\begin{exe}
	\ex \emph{ruga ŋankwira \textbf{nima} sankuka bä byé.}\\
	\gll ruga ŋan\stem{kwir}a nima san\stem{kuk}a bä b=\stem{yé}\\
	pig \Stsg:\Sbj:\Pst:\Ipfv:\Venit/run {like.this} \Tsg.\Masc:\Sbj:\Pst:\Pfv:\Venit/stand \Med{} \Med{}=\Tsg.\Masc:\Sbj:\Nonpast:\Ipfv/be\\
	\trans `The pig came running, and he stood like this over there.'\Corpus{tci20110810-02}{MAB \#34}
	\label{ex099}
\end{exe}
\begin{exe}
	\ex \emph{fi miyamr thfrärm \textbf{nima} ... nzä we ane fof kwofiyakmo nafanmedbo ... we nzä miyamr kwofrärm.}\\
	\gll fi miyamr thf\stem{rä}rm nima (.) nzä we ane fof kwof\stem{yak}mo nafanmedbo (.) we nzä miyamr kwof\stem{rä}rm\\
	\Third.{\Abs} ignorant \Stpl:\Sbj:\Pst:\Dur/be {like.this} (.) \Fsg.{\Abs} also {\Dem} {\Emph} \Fsg:\Sbj:\Pst:\Dur:\Andat/walk \Third{}{\Nsg}.{\All} (.) also \Fsg.{\Abs} ignorant \Fsg:\Sbj:\Pst:\Dur/be\\
	\trans `They did not know about this ... (that) I was walking towards them ... and I did not know either.' \Corpus{tci20111119-03}{ABB \#136-137}
	\label{ex100}
\end{exe}

\emph{Nima} is used for three functions: \isi{deictic} reference (actual or mimicked), anaphora, or introducing \isi{direct speech}. When introducing \isi{direct speech} \emph{nima} may occur with a speaking verb (\ref{ex102}) or just by itself (\ref{ex097}). In these instances, it is glossed as a \isi{quotative} marker (\Quot). This function is further described in {\S}\ref{directspeechthought}.

\begin{exe}
	\ex \emph{nzä \textbf{nima} zukorth ``be fafä zane nagayé fäth zä thamonegwé!''}\\
	\gll nzä nima zu\stem{kor}th be fafä zane nagayé fäth zä tha\stem{moneg}wé\\
	\Fsg.{\Abs} {\Quot} \Stdu:\Sbj>\Fsg:\Obj:\Pst:\Pfv/speak \Ssg{}.{\Erg} after.this {\Dem}:{\Prox} children \Dim{} {\Prox} \Ssg:\Sbj>\Stpl:\Obj:\Imp:\Ipfv/take.care\\
	\trans `The two told me: ``You take care of these small children here!'''\\ \Corpus{tci20121019-04}{ABB \#91-92}
	\label{ex102}
\end{exe}

When marked with the instrumental case \emph{=me}, \emph{nima} is often used as an emphatic affirmative, as \ili{English} `Just like this!'. In (\ref{ex101}), the speaker explains how his grandmother grew very old because she followed all the food taboos.

\begin{exe}
	\ex \emph{nafaŋamane zokwasi nafaŋafane zokwasi naf mon zekarisa. \textbf{nimame} fof!}\\
	\gll nafa-ŋame=ane zokwasi nafa-ŋafe-ane zokwasi naf mon ze\stem{karis}a nima=me fof\\
	\Third.{\Poss}-mother=\Poss.{\Sg} language \Third.{\Poss}-father=\Poss.{\Sg} language \Tsg.{\Erg} how \Stsg:\Sbj:\Pst:\Pfv/hear {like.this}={\Ins} {\Emph}\\
	\trans `She listened to her mother's words and to her father's words. Just like this!' \Corpus{tci20120922-26}{DAK \#60}
	\label{ex101}
\end{exe}

\section{Verbs} \label{verbs-sec}

Verbs are by far the most complex lexical items in Komnzo with respect to morphology. Here, only a brief overview and some of the definitional criteria for identifying a particular item as a \isi{verb} are given. For a full discussion of verbal morphology in Komnzo the reader is referred to chapters \ref{cha:verb morphology} and \ref{TAMpalooza}.

With around 380 members, verbs are the second largest \isi{word class} after nouns. In spite of its inventory size, verbs constitute a closed \isi{word class}. There are no observed cases of loanwords or neologisms. Evidence for the closed status comes from two observations. First, the lack of derivational morphology (and shared roots) within the \isi{word class}, but also between verbs and other word classes. Secondly, the fact that loanwords which are verbs in the donor language never end up in the \isi{verb} class in Komnzo.

Within the \isi{word class} of verbs there is no productive derivational morphology. Only a few non-productive patterns can be discerned, but the interpretation of these remains highly speculative. One such example is the pair of verbs \emph{knsi} `roll' and \emph{myuknsi} `roll, twist'. The former is often used for rolling cigarettes, while the latter is used for rolling up a tape measure. Hence, we could translate them as \emph{knsi} `roll lengthwise' and \emph{myuknsi} `roll widthwise', ignoring the second sense of \emph{myuknsi} `twist'. Without the \isi{nominaliser}, the stems are \emph{kn} and \emph{myukn}, and a possible hypothesis is that the \emph{myu} says something about the orientation of the object that is rolled up. However, \emph{myu} is not a word in Komnzo, nor is the pattern attested elsewhere in the \isi{verb} lexicon. Another example is the pair \emph{misoksi} `look up' and \emph{risoksi} `look down'. The formal difference lies only in the first consonant. I analyse these as idiosyncrasies of particular stems which might reflect frozen derivational morphology.

The same observation can be made for the relation between the \isi{verb} class and other word classes. There are currently only four examples where a \isi{verb} stem is identical or similar to a nominal element and a semantic bridge can be established. The first is the verb \emph{rmrsi} `rub, grind' and the \isi{property noun} \emph{rmr} `roughness'. The second is the verb \emph{miyogsi} `beg, ask for' and \emph{miyo}, which can be either a \isi{property noun} `desire' or a noun `wish, taste'. The third is the verb \emph{wasisi} `shine light on' and the word for the masked owl \emph{wasi}.\footnote{The Masked Owl (Tyto novaehollandiae), like most owls, has large eyes.} The last example is the verb \emph{fokusi} `miss out on sth.' and the word \emph{fokufoku} which describes a patch of bush that was not burned or a patch of grass that was not cut down. There is a clear semantic overlap in the nominal and verbal semantics, but we cannot determine the direction of derivation. However, the scarcity of such examples is striking.

One wonders then how new \isi{verb} meanings enter the language. The clearest answer to this question comes from loanwords. Komnzo speakers were exposed to \ili{Hiri Motu} during a short period in the 1950s when the local Mission school was run by Motu-speaking teachers. Since the 1960s the dominant educational as well as administrative language has been \ili{English}. All loanwords which are verbs in \ili{Hiri Motu} or \ili{English} end up in the \isi{nominal} subclass of property nouns, not in the \isi{verb} class. Some Komnzo examples are \emph{durua} `help' and \emph{tarawat} `law, rightfulness' from Motu, \emph{senis} `change' and \emph{boil} `boil' from \ili{English}. It is the complex \isi{verb} morphology, for example stem types sensitive to aspectual distinctions, which prevents new material from being incorporated into the \isi{verb} class. Instead, these loan verbs are property nouns in Komnzo, and they are employed in a \isi{light verb} construction ({\S}\ref{lightverb}). Cross-linguistically, this is a common strategy to integrate loan verbs (\citealt{Wichmann:2008loanverbs}).\footnote{From observation it is clear that younger speakers have already begun to replace some Komnzo verbs with \ili{English} loans using a light verb construction with `do'. For example, \emph{thofiksi} `disturb' is commonly expressed as \emph{disturb ŋarär}, whereby \emph{ŋarär} is the inflected verb `do', and the expression can be literally translated as `he does the distraction/disturbing'. One may predict that this pattern will become more dominant in the future. The shift from minor to major patterns in contact situations has been described by Heine and Kuteva (\citeyear[44]{Heine:2005wp}).}

Morpho-syntactically, we can define verbs as those lexemes which inflect for \isi{gender}, \isi{person}, \isi{number}, \isi{tense}, \isi{aspect}, \isi{mood}, \isi{valency}, and \isi{directionality}, as can be seen in examples (\ref{ex006}) and (\ref{ex007}). With the exception of \isi{person} and \isi{number}, these are only found in verbs. The glossing of these grammatical categories, however, cannot be done straightforwardly, because a number of them can only be understood after unifying values from different morphological slots. For example, the aspectual value \Pst:\Dur{} in (\ref{ex006}) is encoded simultaneously in the verb stem, the prefix and the durative suffix. Prior to this unification, each morpheme taken by itself is underspecified with respect to any particular grammatical category. The only exceptions are the two \isi{directional} affixes. In this subsection, I will employ a double glossing style as in the chapters on verb morphology (chapters \ref{cha:verb morphology} and \ref{TAMpalooza}). A segmented, itemised glossing line is given first, while a second line shows the unified \isi{gloss} in smaller print. Morphological complexity in verbs is discussed in {\S}\ref{verbprelim}, where the reader also finds a more detailed justification for the double-lined glossing convention.

\begin{exe}
	\ex \textit{nafane nagayé \textbf{thfrärm}. naf \textbf{thwamonegwrm}.}\\
	\glll nafane nagayé thf-rä-rm naf\\
	\Tsg{\Poss} children \Stnsg.\Betatwo-\Cop.\Ndu-\Dur{} \Tsg{\Erg}\\
	~ ~  {\footnotesize \Stpl:\Sbj:\Pst:\Dur/be} ~\\
	\sn
	\glll thu-a-moneg-wr-m-\Zero{}\\
	\Stnsg.\Betaone-\Vc-take.care.\Ext-\Ndu-\Dur\\
	{\footnotesize \Stsg:\Sbj>\Stpl:\Obj:\Pst:\Dur/take.care}\\
	\trans `They were her children. She took care of them.' \Corpus{tci20120901-01}{MAK \#47}
	\label{ex006}
\end{exe}
\begin{exe}
	\ex \emph{fi fthé \textbf{enthorakwa} ... mnz kabe fof. nima \textbf{thäzigrthma} ``nä tmatm fefe \textbf{nzŋawänzr}. manema kabe zä naf \textbf{nziyanathr}?''}\\
	\glll fi fthé e-n-thorak-w-a-\Zero{} (.) mnz kabe fof nima th-ä-zingrthm-a nä tmatm fefe\\
	\Third.{\Abs} when \Stnsg.\Alph-\Venit-arrive.\Ext-\Ndu-\Pst-\Stsg{} (.) house people {\Emph} {\Quot} \Stnsg.\Gam-\Vc.\Ndu-look.around.\Rs-\Pst{} some event real\\
	~ ~ {\footnotesize \Stpl:\Sbj:\Pst:\Ipfv:\Venit/arrive} ~ ~ ~ ~ ~ {\footnotesize \Stpl:\Sbj:\Pst:\Pfv/look.around} ~ ~ ~\\
	\sn
	\glll nz=ŋ-a-wä-nzr-\Zero{} mane=ma kabe zä naf\\
	\Immpst{}=\M.\Alph-\Vc-break.\Ext-\Ndu-\Stsg{} which={\Char} man {\Prox} \Tsg{\Erg}\\
	{\footnotesize \Immpst=\Stsg:\Sbj:\Nonpast:\Ipfv/break} ~ ~ ~ ~\\
	\sn
	\glll nz=y-a-na-thr-\Zero{}\\
	\Immpst{}=\Tsg.\Masc.\Alph-\Vc-eat.\Ext-\Ndu-\Stsg{}\\
	 {\footnotesize \Immpst=\Stsg:\Sbj>\Tsg.\Masc:\Obj:\Nonpast:\Ipfv/eat}\\
	\trans `At that time the house owners returned to the village. They looked around and said, ``Something terrible has happened. From which village was the man who she ate here?''' \Corpus{tci20120901-01 MAK}{\#106-111}
	\label{ex007}
\end{exe}

Examples (\ref{ex006}) and (\ref{ex007}) show the intricate architecture of Komnzo verbs. The verb forms in both examples are inflected for various grammatical categories. The agreement target for \isi{gender} is the third \isi{person} singular prefix on the verb, as can be seen in the last verb `eat' in example (\ref{ex007}). Person and \isi{number} are encoded in the \isi{undergoer} prefix as well as the actor suffix. However, these slots are underspecified: the second and third \isi{person} in the \isi{non-singular} are neutralised in both slots. The first \isi{non-singular} and second \isi{singular} are neutralised in the prefixes. These can be disambiguated by the free pronouns. In both slots, \isi{dual} and \isi{plural} are neutralised. The system of \isi{number} marking combines a \isi{singular} vs. \isi{non-singular} opposition in the prefix and suffix with a \isi{dual} vs. \isi{non-dual} opposition in the duality affix. Thereby, one arrives at the three \isi{number} values (\Sg, \Du, \Pl). For about half a dozen high frequency verbs, such as the \isi{copula} (\ref{ex006}), the stem itself is sensitive to duality. For all other verbs, duality is either encoded by a prefix, as in the second verb `look around' in (\ref{ex007}) or by a suffix as in all other verbs in (\ref{ex006}) and (\ref{ex007}). The morphological site of duality marking depends on the stem type. Almost all verbs in Komnzo have two stems from which aspectual distinctions can be build. I label the two stem types `restricted' ({\Rs}) and `extended stem' (\Ext). It follows that \isi{tense}, \isi{aspect} and \isi{mood} are expressed by a combination of verb stem, prefixes, and further suffixal material. As for the prefixes, there are five different \isi{prefix series} labelled \Alph{}, \Bet{}, \Betaone{}, \Betatwo{}, and \Gam{} and an \isi{immediate past} \isi{proclitic} (for example in the last two verbs of \ref{ex007}). Beyond TAM, the prefixes encode information about \isi{person}, \isi{number}, and \isi{gender}. Examples for the suffixal material are the durative suffix (\Dur{}) in both verb forms in (\ref{ex006}) and the \isi{past} suffix (\Pst{}) in the first two verb forms in (\ref{ex007}). The TAM value is calculated by unifying these different exponents. As the final category to mention here, the first verb `arrive' in (\ref{ex007}) is inflected for \isi{directionality}. The two values of \isi{direction} are \isi{venitive} `towards' (\Venit) and \isi{andative} `away' (\Andat).

Verbs are the only lexical items which can take the nominalising suffix (\emph{-si}). Nominalisations or infinitives are used as a citation form in the dictionary. Frequently, nominalisations were frequently given to me as \emph{zokwasi ebar} `head words' for an inflected verb form. Nominalisations are non-finite forms without inflectional material. Nominalisations can be treated like underived nouns. They can function as complements of phasal verbs (\textit{finish, start, become}) (\ref{ex008}) or infinitival adjuncts (\ref{ex009}). Example (\ref{ex008}) is taken from a story in which two birds have a competition on how long each one can hold its breath under water. Thus, \emph{fsisi zäbthath} can be translated as `the counting finished'. Example (\ref{ex009}) can be translated as `in the planting (season)'.

\begin{exe}
	\ex \textit{ane zwafsinzrm kwot e boböwä bäne zefafath ... \textbf{fsisi} zäbthath.}\\
	\glll ane zu-a-fsi-nzr-m-\Zero{} kwot e bobo=wä\\
	{\Dem} \Tsg.\F.\Betatwo-\Vc-count.\Ext-\Ndu-\Dur-\Stsg{} properly until \Med{}.{\All}={\Emph}\\
	~ {\footnotesize \Stsg:\Sbj>\Tsg.\F:\Io:\Pst:\Dur/count} ~ ~ ~ ~\\
	\sn
	\glll bäne z-ä-faf-a-th (.) fsi-si\\
	\Recog.{\Abs} \M.\Gam-\Vc.\Ndu-hold.\Rs-\Pst-\Stnsg{} (.) count-{\Nmlz}\\
	~ {\footnotesize \Stpl:\Sbj:\Pst:\Pfv/hold} ~ ~\\
	\sn
	\glll z-ä-bth-a-th\\
	\M.\Gam-\Vc.\Ndu-finish.\Rs-\Pst-\Stnsg{}\\
	{\footnotesize \Stpl:\Sbj:\Pst:\Pfv/finish}\\
	\trans `He counted for her until he reached that number. Then the counting was finished.' \Corpus{tci20130923-01}{ALA \#28-30}
	\label{ex008}
\end{exe}
\begin{exe}
	\ex \textit{fä fof sfrugrm ... nima eftharen zf ... nima \textbf{worsin} zf.}\\
	\glll fä fof sf-rug-rm (.) nima efthar=en zf (.) nima\\
	{\Dist} {\Emph} \Tsg.\M.\Betatwo-sleep.\Ext.\Ndu-\Dur{} (.) {like.this} {dry.season=\Loc} {\Imm} (.) {like.this}\\
	~ ~ {\footnotesize \Tsg.\M:\Sbj:\Pst:\Dur/sleep} {} {} ~ ~ ~ ~ ~ ~\\
	\sn
	\gll wor-si=n zf\\
	plant-\Nmlz={\Loc} \Imm\\
	\trans `He slept over there ... like this in the dry season ... like this in the planting season.' \Corpus{tci20131013-02}{ABB \#140-142}
	\label{ex009}
\end{exe}

In other respects, nominalised verbs can be treated like any other \isi{noun}. They can take \isi{case}, for example the \isi{ergative} (\ref{ex012}) or the instrumental in a resultative construction (\ref{ex103}). They can be reduplicated, as in (\ref{ex010}). They can enter into \isi{possessive} constructions either as \isi{possessed} (\ref{ex010}) or as \isi{possessor} (\ref{ex011}).

\begin{exe}
	\ex \textit{zarfa surmänwrm ane \textbf{wäsifnzo}.}\\
	\glll zarfa su-rmän-wr-m-\Zero{} ane wä-si=f=nzo\\
	ear \Tsg.\Masc.\Betatwo-close.\Ext-\Ndu-\Dur-\Stnsg{} {\Dem} break-{\Nmlz}={\Erg}={\Only}\\
	~ {\footnotesize \Stsg:\Sbj>\Tsg.\Masc:\Obj:\Pst:\Dur/close} ~ ~\\
	\trans `That breaking noise was blocking his ears.' \Corpus{tci20120818}{ABB \#68}
	\label{ex012}
\end{exe}
\begin{exe}
	\ex \emph{ŋafyf frthé bant wäfiyokwa, kidn ane rifthzsime zfrärm.}\\
	\glll ŋafe-f fthé bant w-a-fiyok-w-a-\Zero{} kidn ane~~~~~~~~  \textbf{rifthz-si=me} zf-rä-rm\\
	father-\Erg.{\Sg} when ground \Tsg.\F.\Alph-\Vc-make.\Ext-\Ndu-\Pst-\Stsg{} ancient.fire {\Dem} hide-{\Nmlz}={\Ins} \Tsg.\F.\Betatwo-\Cop.\Ndu-\Dur\\
	~ ~ ~ {\footnotesize \Stsg:\Sbj>\Tsg.\F:\Obj:\Pst:\Ipfv/make} ~ ~ ~ {\footnotesize \Tsg.\F:\Sbj:\Pst:\Dur/be}\\
	\trans `When God made the Earth, the ancient fire was hidden.'\Corpus{tci20120909-06}{KAB \#62-63}
	\label{ex103}
\end{exe}
\begin{exe}
	\ex \textit{fi miyomär yé. wri kabeaneme \textbf{ttrikasi} naf krarizr.}\\
	\glll fi miyo=mär \stem{yé} wri kabe=aneme\\
	\Third.{\Abs} desire={\Priv} \Tsg.\Masc.\Alph.\Cop.{\Ndu} drunk man={\Poss}.{\Nsg}\\
	~ ~ {\footnotesize \Tsg.\Masc:\Sbj:\Nonpast:\Ipfv/be} ~ ~\\
	\sn
	\glll t-trik-si naf k-ra-ri-zr-\Zero{}\\
	\Redup-tell-{\Nmlz} \Tsg{\Erg} \M.\Bet-\Irr.\Vc-hear.\Ext-\Ndu-\Stsg{}\\
	~ ~ {\footnotesize \Stsg:\Sbj:\Irr:\Ipfv/hear}\\
	\trans `He doesn't want to listen to those drunk people's stories.'\Corpus{tci20111004}{RMA \#140}
	\label{ex010}
\end{exe}
\begin{exe}
	\ex \textit{... \textbf{tharisiane} efoth fthé zfrärm.}\\
	\glll (.) thari-si=ane efoth fthé zf-rä-rm\\
	(.) dig-{\Nmlz}=\Poss.{\Sg} day when \Tsg.\F.\Betatwo-\Cop.\Ndu-\Dur\\
	~ ~ ~ ~ {\footnotesize \Tsg.\F:\Sbj:\Pst:\Dur/be}\\
	\trans `... when it was harvesting season.' \Corpus{tci20120805-01}{ABB \#356}
	\label{ex011}
\end{exe}

Almost all verbs have an \isi{infinitive} derived by means of the \isi{nominaliser} (\emph{-si}). However, there are a few exceptions where either an underived \isi{noun} is used or an \isi{nominal} form is lacking altogether. For the most part, these are verbs of high frequency. In the following three examples, the \isi{noun} meaning is given first and the verb meaning second: \emph{zan} `fight, war (n); hit, kill (v)', \emph{wath} `dance, song (n); dance, sing (v)', \emph{zrin} `heaviness, burden (n); carry (v)'.

There are two options to analyse nominalisations. While I stress their verbal character, one could argue that they should be analysed as (deverbal) nouns. I believe that this is an analytic decision and that there are good arguments for both sides. I address this question here because the decision impacts several other parts of the grammar, for example the description of the interclausal function of the \isi{case} markers (\S\ref{formfunccase}) and subordinate clauses (chapter \ref{cha:interclausalsyntax}), both of which involve infinitives. As shown above, nominalised verbs behave like nouns in terms of morphology, that is they can form reduplications and \isi{nominal} compounds. Moreover, they can serve as hosts for the \isi{case} enclitics. This supports the analysis of nominalisations as nouns. However, nominalised verbs retain particular verbal features, for example their argument structure. The \isi{agent} (or most agent-like argument) of the finite verb can be expressed with the non-finite verb by means of a \isi{possessive} construction. In \emph{nafane tharisi} `her digging', the third singular \isi{possessor} refers to the \isi{agent} argument. The \isi{patient} (or most patient-like argument) can be expressed by the modifying element of a \isi{nominal} compound. In \emph{wawa tharisi} `yam digging', the word for `yam' is the \isi{patient} of the event. Noun phrases of this type can be captured by the notion of an action \isi{nominal}, which Comrie \& Thompson describe as ``a noun phrase that contains, in addition to a noun derived from a verb, one or more reflexes of a proposition or predicate'' (\citeyear[343]{Comrie:2007nom}).

The verbal character of nominalisations in Komnzo is clearest in \isi{raising} constructions. In example (\ref{ex610}), the speaker demonstrates how to produce a children's toy from a coconut leaf. She uses a \isi{raising} construction (`start rolling') with a nominalised form of `roll'. This is followed by the finite form of `roll'. We find that argument indexing of the finite `roll' (\Fsg:\Sbj>\Tsg.\Masc:\Obj) has been raised to the phasal verb `start'. In conclusion, I acknowledge that nominalised verbs can be analysed as either (deverbal) nouns or infinitives. I have made explicit why I choose the latter option.

\begin{exe}
	\ex \emph{myuknsi srethkäfe ... zane zf ymyuknwé.}\\
	\glll myukn-si s-rä-thkäf-é (.) zane zf\\
	roll-{\Nmlz} \Tsg.\Masc.\Gam-\Irr.\Ndu-start.\Rs-\Fsg{} (.) \Dem:{\Prox} {\Imm} \\
	~ {\footnotesize \Fsg:\Sbj>\Tsg.\Masc:\Obj:\Irr:\Pfv/start} ~ ~ ~ \\
	\sn
	\glll y-myukn-w-é\\
	\Tsg.\Masc.\Alph{}-roll.\Ext-\Ndu-\Fsg{}\\
	{\footnotesize \Fsg:\Sbj>\Tsg.\Masc:\Obj:\Nonpast:\Ipfv/roll}\\
	\trans `I (usually) start rolling (the leaf). I roll this one right here.'\Corpus{tci20120914}{RNA \#45}
	\label{ex610}
\end{exe}

Word order in Komnzo is predominantly \isi{SOV}, or more accurately AUV (agent \isi{undergoer} verb). For pragmatic reasons, elements may follow the verb, but they are usually part of a separate intonation group. The only exceptions are the \isi{emphatic} particle \textit{fof} ({\S}\ref{discourse-particles}) and the \isi{demonstrative} \isi{identifier} ({\S}\ref{demonstrative-identifiers-subsec}).

Verbs can be subcategorised along both grammatical and semantic lines. As for the latter, we find a class of \isi{positional} verbs, which take a special \isi{stative} suffix and encode postural or \isi{positional} semantics, for example \emph{migsi} `hang', \emph{thorsi} `be inside', \emph{rngthksi} `be in a tree fork' ({\S}\ref{positionalverbs}). Morphologically, one interesting fact is that only a small part of \isi{intransitive} verbs are purely prefixing. Most \isi{intransitive} verbs employ both the prefix and the suffix. In this case, an invariant \isi{middle} prefix is used and the single argument is indexed in the suffix ({\S}\ref{middletemplatesubsection}). Transitive verbs index their \isi{subject} in the suffix and the \isi{object} in the prefix ({\S}\ref{ambifixingtemp}). Most stems can be applicativised by adding the \emph{a-} prefix. In this case, the reference of the \isi{person} prefix changes from the \isi{object} (or \isi{subject} of a prefixing verb) to an \isi{indirect object} (usually a \isi{recipient}, \isi{beneficiary}, or raised \isi{possessor}). I label the \emph{a-} prefix {\Vc} for `\isi{valency change}'. This is because \emph{a-} is used to increase as well as to decrease the \isi{valency} of a verb. For example, the \isi{middle} template, which can be used to form reflexives from transitive verbs, always takes the \emph{a-} prefix (\S\ref{morphologicaltemplates}). A general feature of Komnzo verbs is a high degree of flexibility, whereby most stems may enter various morphological templates and a handful of stems can be cycled through all. This is discussed in detail in {\S}\ref{alignmtemplates}.

\section{Adverbs} \label{adverbs-sec}

Adverbs make up a small closed class of about a dozen lexical items. A number of \isi{nominal}s, such as temporals and demonstratives have an adverbial function. Moreover, the instrumental \isi{case} (\emph{=me}) on adjectives and property nouns marks an adverbial function. Some of the adverbs show remnants of frozen morphology. For example, \emph{watmame} `for a daytrip' shows a \emph{=me} element, but the corresponding form \textsuperscript{$\ast$}\emph{watma} is missing.

Temporals have been discussed in {\S}\ref{temporals-sec}. They are a functional subclass of \isi{nominal}s, which can have an adverbial function. Spatial adverbials are expressed by the rich set of demonstratives discussed in {\S}\ref{adverbial-demonstratives-subsec}. Hence, only manner adverbs comprise a word class in their own right. These are uninflecting words which are fairly free with respect to their position in the clause. Most commonly, they occur in preverbal position. \tabref{manner-adverbs-table} lists the currently attested \isi{manner adverb}s.

\begin{table}
\caption{Manner adverbs}
\label{manner-adverbs-table}
	\begin{tabularx}{.5\textwidth}{Xl}
		\lsptoprule
		Komnzo&gloss\\\midrule
		\emph{eräme}&`together'\\
		\emph{kwot}&`properly'\\
		\emph{matar}&`quietly'\\
		\emph{minzü}&`very, too much'\\
		\emph{nezä}&`in return'\\
		\emph{nm, nnzä}&`perhaps, maybe'\\
		\emph{ŋarde}&`for the first time'\\
		\emph{gaso}&`badly'\\
		\emph{gräme}&`slowly'\\
		\emph{dmnzü}&`silently'\\
		\emph{rürä}&`alone, lonely'\\
		\emph{watmame}&`for a daytrip'\\
		\emph{yakme}&`fast, quickly'\\
		\emph{nzagoma}&`in advance'\\
		\emph{ŋwä}&`instead (of)'\\
		\lspbottomrule
	\end{tabularx}
\end{table}%Manner adverbs

\section{Particles} \label{particles-sec}

There are two types of particles; TAM particles and discourse particles. Both are morphologically invariant, but differ slightly in their syntactic distribution. The TAM particles are discussed in more detail in {\S}\ref{tam-particles-sec}.

\subsection{TAM particles} \label{tam-particles-subsec}
\largerpage
There are five particles which are part of the tense-aspect-\isi{mood} system. Most frequently, they occur in preverbal position, but other elements may intervene. These are important for TAM because even though Komnzo has a rich set of TAM related inflections on the verb, some categories can only be expressed by means of the particles, for example \emph{kwa} for futurity and \emph{z} for completion. The five particles are shown in \tabref{tam-particles-table}. Note that there are the proclitics \emph{n=} and \emph{m=}, which play a role in TAM marking as well. Depending on their morpho-syntactic context they can be analysed as clitics or as particles. This point is discussed in {\S}\ref{verbal-proclitics-subsec}.

\begin{table}
\caption{TAM particles}
\label{tam-particles-table}
	\begin{tabularx}{.8\textwidth}{XXXX}
		\lsptoprule
		Komnzo&gloss&function &translation\\
		\midrule
		\emph{kwa}& {\Fut} &future &`will'\\
		\emph{z}& {\Iam} &iamitive &`already'\\
		\emph{nomai}&\Hab{} &habitual &`often', `always'\\
		\emph{kma}&{\Pot} &potential &`might', `could'\\
		\emph{keke} or \emph{kyo}&{\Neg} &negator &`not'\\
		\lspbottomrule
	\end{tabularx}
\end{table}%TAM particles

The \isi{future} marker \emph{kwa}, sometimes just \emph{ka}, is the only way of expressing the futurity of an event. It occurs with the \isi{non-past} \isi{tense} and the \isi{irrealis} mood (\ref{ex104}), both of which are insufficient for indicating that a particular event will take place in the \isi{future}. The \isi{particle} may occur just by itself, in which case it is an \isi{imperative} that means `wait!' (\ref{ex104}). The \isi{future} \isi{particle} \emph{kwa} is discussed in {\S}\ref{futurekwa}.

\begin{exe}
	\ex \emph{katakatan \textbf{kwa} zöbthé thrängathinzth nima: ``\textbf{kwa}! komnzo \textbf{kwa}!''}\\
	\gll kata-katan kwa zöbthé thran\stem{gathi}nzth nima kwa komnzo kwa\\
	\Redup-small {\Fut} first \Stpl:\Sbj>\Stpl:\Obj:\Irr:\Pfv:\Venit/stop {\Quot} wait only wait\\
	\trans `First, they will stop the small children (from jumping in). They will say: ``Wait! Just wait!''' \Corpus{tci20110813-09}{DAK \#25}
	\label{ex104}
\end{exe}

The \isi{iamitive} marker \emph{z} functions as a completive marker. I adopt the term ``iamitive'' from Olsson (\citeyear{Olsson:2013vn}), who has coined it based on Latin \emph{iam} `already'. I use the gloss label {\Iam}. The iamitive combines with all tense-aspect-mood categories, except for the \isi{imperative}. The TAM system and the distinction between \isi{imperfective} and \isi{perfective} does not focus on completion, rather it draws a distinction between durative versus inceptive/punctual. The \isi{iamitive} \isi{particle} is the only way to indicate completion. It may be used in declarative sentences (\ref{ex130}) or with a rising intonation in polar questions (\ref{ex106}). The \isi{particle} \emph{z} is discussed in more detail in {\S}\ref{iamitivez}.

\begin{exe}
	\ex \emph{foba yakkarä enrera ``oh, firran \textbf{z} thäkwrth.''}\\
	\gll foba yak=karä en\stem{rä}ra oh firra=n z thä\stem{kwr}th\\
	{\Dist}.{\Abl} walk={\Prop} \Stpl:\Sbj:\Pst:\Ipfv:\Venit/be oh firra={\Loc} {\Iam} \Stpl:\Sbj>\Stpl:\Obj:\Rpst:\Pfv/hit\\
	\trans `They came fast from there (and said:) ``Oh, they already killed them in Firra.''' \Corpus{tci20131013-02}{ABB \#80}
	\label{ex130}
\end{exe}
\begin{exe}
	\ex \emph{\textbf{z} safäs?}\\
	\gll z sa\stem{fäs}\\
	{\Iam} \Stsg:\Sbj>\Tsg.\Masc:\Io:\Rpst:\Pfv/present\\
	\trans `Did you show him already?' \Corpus{tci20130907-02}{RNA \#540}
	\label{ex106}
\end{exe}

\largerpage
The \isi{habitual} marker \emph{nomai} either indicates that an event happened regularly or that it took place for an extended time (\ref{ex107}). There is a variant \emph{nomair}, which expresses `forever' or `for a very long time' (\ref{ex108}). The final /r/ element might be related to the \isi{purposive} \isi{case}. Its origin is still unclear, as particles cannot host \isi{case} clitics. The \isi{habitual} \isi{particle} \emph{nomai} is discussed in {\S}\ref{habitualnomai}.

\begin{exe}
	\ex \emph{fi swathugwrm gaso. nimanzo \textbf{nomai} swafiyokwrm e \textbf{nomai} \textbf{nomai} \textbf{nomai}.}\\
	\gll fi swa\stem{thug}wrm gaso nima=nzo nomai swa\stem{fiyok}wrm e 3x(nomai)\\
	\Tsg.{\Abs} \Stsg:\Sbj>\Tsg.\Masc:\Obj:\Pst:\Dur/trick badly {like.this}={\Only} \Hab{} \Stsg:\Sbj>\Tsg.\Masc:\Obj:\Pst:\Dur/make until 3x(\Hab)\\
	\trans `He tricked him badly. He kept on doing this to him for a long, long time.'\\ \Corpus{tci20110802}{ABB \#95-96}
	\label{ex107}
\end{exe}
\begin{exe}
	\ex \emph{\textbf{nomair} kwa namnzr kwot kwot kwot kwot e namä kakafar kwot käkorm.}\\
	\gll nomair kwa na\stem{m}nzr 4x(kwot) e namä k-kafar kwot kä\stem{kor}m\\
	\Hab{} {\Fut} \Ssg:\Sbj:\Nonpast:\Ipfv/dwell 4x(properly) until good \Redup-big properly \Ssg:\Sbj:\Futimp:\Pfv/become\\
	\trans `You will live forever ... all the time until you really grow old.'\\ \Corpus{tci20120922-26}{DAK \#16}
	\label{ex108}
\end{exe}

The \isi{potential} marker \emph{kma} occurs with verbs of different \isi{aspect} values. It marks counterfactuality with deontic or epistemic interpretation, for example potentiality of an event (`could' or `could have') or obligation (`should' or `should have'). In example (\ref{ex109}), the speaker blames his wife for not telling him about a bushfire. In example (\ref{ex110}), the speaker describes how he fought a bushfire in his garden. The \isi{particle} \emph{kma} is discussed in {\S}\ref{potentialkma}.

\begin{exe}
	\ex \emph{nzä tosaiaŋama \textbf{kma} kwräkor ``käthf!'' nzä nima fefe kwamnzrm kifa sfrwrmé.}\\
	\gll nzä tosai-a-ŋame kma kwrä\stem{kor} kä\stem{thf} nzä nima fefe kwa\stem{m}nzrm kifa sf\stem{r}wrmé\\
	\Fsg.{\Abs} baby-\Poss-mother {\Pot} \Stsg:\Sbj>\Fsg:\Obj:\Irr:\Pfv/speak \Ssg:\Sbj:\Imp:\Pfv/walk \Fsg.{\Abs} {like.this} really \Fsg:\Sbj:\Pst:\Dur/sit rattan.wall \Fsg:\Sbj>\Tsg.\Masc:\Obj:\Pst:\Dur/weave\\
	\trans `The baby's mother could have told me ``You go!'' but I was just sitting like this and weaving the rattan wall.' \Corpus{tci20120922-24}{STK \#8-10}
	\label{ex109}
\end{exe}
\begin{exe}
	\ex \emph{\textbf{kma} wämne ane fof kwakarkwé ane fof ... wämnef mane thänarfa ... keke ... watikthémäre.}\\
	\gll kma wämne ane fof kwa\stem{kark}wé ane fof (.) wämne=f mane thä\stem{narf}a (.) keke (.) watik-thé=märe\\
	{\Pot} tree {\Dem} {\Emph} \Fsg:\Sbj:\Rpst:\Ipfv/pull {\Dem} {\Emph} (.) tree={\Erg} which \Stsg:\Sbj>\Stpl:\Obj:\Pst:\Pfv/press.down (.) {\Neg} (.) enough-\Adlzr={\Priv}\\
	\trans `I should have pulled that tree off ... the one that was pushing down (the fences). No (it was) not enough'. \Corpus{tci20120922-24}{MAA \#42-43}
	\label{ex110}
\end{exe}

With verbs in \isi{imperative} or \isi{irrealis} mood, \emph{kma} frequently occurs together with the \isi{clitic} \emph{m}, which is discussed in more detail ({\S}\ref{verbal-proclitics-subsec}). This combination of clitic, \isi{particle} and verb inflection expresses a prohibitive. In this case, the \isi{clitic} \emph{m} may encliticise to \emph{kma}. In fact, the resulting word \emph{kmam} can stand as an utterance by itself meaning: `Don't!' or `Don't do it!'. In (\ref{ex111}) one such example is given, which comes from a public speech during a dance. For further discussion, the reader is referred to {\S}\ref{verbal-proclitics-subsec} and {\S}\ref{apprehensivem}.

\begin{exe}
	\ex \emph{gatha fam \textbf{kmam} gnräré monwä z fam thäkuke.}\\
	\gll gatha fam kma=m gn\stem{rä}ré mon=wä z fam thä\stem{kuk}e\\
	bad thought {\Pot}={\Appr} \Ssg:\Sbj:\Imp:\Ipfv/be how={\Emph} {\Iam} thought \Fpl:\Sbj>\Stpl:\Obj:\Rpst:\Pfv/erect\\
	\trans `You must not think bad about how we made up our minds.' \Corpus{tci20121019-04}{ABB \#243-244}
	\label{ex111}
\end{exe}

The \isi{negator} \emph{keke} occurs in preverbal position (\ref{ex319}). In rapid speech it is sometimes shortened to \emph{ke}. There is a second \isi{negator} \emph{kyo} (\ref{ex365}), which is mostly used by older speakers. Both negators can stand alone in an exclamation or as the answer to a \isi{question}. Example (\ref{ex319}) comes from a story about the speaker's father's generation. Example (\ref{ex365}) is taken from a conversation about food taboos.

\begin{exe}
	\ex \emph{tüfr kabe \textbf{keke} thfrärm.}\\
	\gll tüfr kabe keke thf\stem{rä}rm\\
	plenty people {\Neg} \Stpl:\Sbj:\Pst:\Dur/be\\
	\trans `They were not many people.' \Corpus{tci20120805-01}{ABB \#517}
	\label{ex319}
\end{exe}
\begin{exe}
	\ex \emph{\textbf{kyo} kwa nr kabeyé thranathrth ... nima ivanaŋame brä.}\\
	\gll kyo kwa nr kabe=é thra\stem{na}thrth (.) nima ivan-a-ŋame b=\stem{rä}\\
	{\Neg} {\Fut} belly people=\Erg.{\Nsg} \Stpl:\Sbj>\Stpl:\Obj:\Irr:\Ipfv/eat (.) like.this ivan-\Poss-mother \Med=\Tsg.\F:\Sbj:\Nonpast:\Ipfv/be\\
	\trans `The pregnant people will not eat them ... like Ivan's mother there.'\\ \Corpus{tci20120922-26}{MAB \#38}
	\label{ex365}
\end{exe}

I was told that the teachers in the mission school during the 1960s discouraged their students from using \emph{kyo} [kə̆jo] because ``it is a bad word''. At the time, the teachers were Motu speakers and this was also the language of instruction. In Motu, the word \emph{kio} [kijo] means `vagina'. We can only hypothesise that the teachers of the mission school enacted pressure strong enough to replace the word \emph{kyo} with the word \emph{keke} whose origin is thus far unknown. Alternatively, the two negators might have existed simultaneously and the teachers' pressure only skewed their respective \isi{frequency} of use. Negation is described in \S\ref{negationclause}.

\subsection{Discourse particles} \label{discourse-particles}

There are three discourse particles in Komnzo: \emph{we} `also', the \isi{intensifier} \emph{fof} and the word from which the language name is derived, \emph{komnzo} `only, still'. These are used for different types of \isi{focus}.

The \isi{particle} \emph{we} `also' functions as an additive \isi{focus} marker. It usually has scope over a whole proposition. It is rather flexible with respect to its position, and it may occur several times in a clause. Semantically, it always presupposes some event that has been established in the previous discourse. We can see this in example (\ref{ex131}), where the speaker makes an additional comment as to why his time as a busy yam gardener has come to an end.

\begin{exe}
	\ex \emph{kafar z zäkora fof ... kafar ... watik, nzone tmä \textbf{we} katanme ŋarsörém.}\\
	\gll kafar z zä\stem{kor}a fof (.) kafar (.) watik nzone tmä we katan=me ŋa\stem{rsör}m\\
	big {\Iam} \Fsg:\Sbj:\Pst:\Pfv/become {\Emph} (.) big (.) then \Fsg{}.{\Poss} strength also small={\Ins} \Stsg:\Sbj:\Rpst:\Dur/recede\\
	\trans `I have grown old ... and my strength has also gone down a little.'\\ \Corpus{tci20120805-01}{ABB \#662-664}
	\label{ex131}
\end{exe}

The \isi{particle} \emph{fof} is the word which occurs with the highest \isi{frequency} in the corpus (around 2,000 tokens). It marks presentational \isi{focus} of quite a wide range of elements. It always follows the element over which is has scope. This may be an adjunct (\ref{ex112}), an argument (\ref{ex113}), or the whole clause if it occurs after the verb (second \emph{fof} in \ref{ex113}). In the examples below, the square brackets indicate the scope of the \isi{particle}. Both examples come from a procedural text, in which the speaker presents his yam storage house. He explains the system by which the yams are piled up and sorted.

\begin{exe}
	\ex \emph{watik zanenzo fthé \textbf{fof} krägathinzth zethn ... dagonma \textbf{fof}.}\\
	\gll watik zane=nzo [fthé fof] krä\stem{gathinz}th z=e\stem{thn} (.) dagon=ma fof\\
	then {\Dem}:{\Prox}={\Only} [when {\Emph}] \Stpl:\Sbj:\Irr:\Pfv/stop \Prox=\Stpl:\Sbj:\Nonpast:\Stat/lie.down (.) food={\Char} {\Emph}\\
	\trans `That is the time when only these ones are left. These lying here ... (are) really for eating.' \Corpus{tci20121001}{ABB \#107}
	\label{ex112}
\end{exe}
\begin{exe}
	\ex \emph{ŋazäthema wawa ane \textbf{fof} erä \textbf{fof}.}\\
	\gll [ŋazäthe=ma [wawa ane fof] e\stem{rä} fof]\\
	[ŋazäthe={\Char} [yam {\Dem} {\Emph}] \Stpl:\Sbj:\Nonpast:\Ipfv/be {\Emph}]\\
	\trans `These yams are really from \NG{}azäthe.' \Corpus{tci20121001}{ABB \#158}
	\label{ex113}
\end{exe}

The \isi{particle} \emph{komnzo} functions as a contrastive \isi{focus} marker which has scope over the predicate. The clitic \emph{=nzo} is its nominal counterpart, which is described in {\S}\ref{clitics-sec}. The formal relationship between \emph{komnzo} and \emph{=nzo} holds true for other \ili{Tonda} varieties. For example, \ili{Anta} to the north has a corresponding \isi{particle} \emph{anta} and a clitic \emph{=nta}.

In example (\ref{ex114}), we see that \emph{komnzo} has scope over the predicate, the copula in this case. I have often overheard women scolding their children by saying \emph{komnzo kämés} `Just sit down!'. In the example, a man returns to the place where the people of Firra took revenge on his wife after she had killed one of them.

\begin{exe}
	\ex	\emph{wati nagawa ŋabrigwa sir. \textbf{komnzo} rä o z kwarsir mnin?}\\
	\gll wati nagawa ŋa\stem{brig}wa si=r [komnzo \stem{rä}] o z kwa\stem{rsir} mni=n\\
	then nagawa \Stsg:\Sbj:\Pst:\Ipfv/return eye={\Purp} [only \Tsg.\F:\Sbj:\Nonpast:\Ipfv/be] or {\Iam} \Stsg:\Sbj:\Rpst:\Ipfv/burn fire={\Loc}\\
	\trans `Then Nagawa returned to check: was she still alive or did she burn in the fire?' \Corpus{tci20120901-01}{MAK \#167-170}
	\label{ex114}
\end{exe}

\section{Clitics} \label{clitics-sec}

Proclitics and enclitics are attested in Komnzo. The former are found only with verbs, whereas the latter attach to \isi{nominal}s. I follow selected criteria based on the literature on clitichood, especially Zwicky \& Pullum (\citeyear{Zwicky:1983jd}) and chapter 8 of Anderson (\citeyear{Anderson:1992uw}). The relevant criteria in Komnzo are (i) clitics operate on a phrase rather than a word level, (ii) clitics show a low degree of selectivity with respect to their hosts and (iii) clitics can attach to other clitics. A further criterion which pertains only to the verbal proclitics and the (\isi{nominal}) \isi{exclusive} \isi{enclitic} is (iv) clitics are reduced forms of independent lexical items.

\subsection{Nominal enclitics}\label{nominal-enclitics-subsec}

All the \isi{case} markers in Komnzo are analysed as clitics. Evidence for the first two criteria is given in examples (\ref{ex498}) and (\ref{ex499}), where the \isi{ergative} attaches to the rightmost element of an NP. The phrase boundaries are marked by square brackets in the examples. In (\ref{ex498}), the noun phrase is \emph{eda kwayan kabe} `two white men'. In (\ref{ex499}), the \isi{adjective} is postposed and consequently is the last element of the phrase. Although \isi{case} markers are attached only to \isi{nominal}s, they show a low degree of selectivity within this macro-\isi{word class}. For a detailed discussion of the \isi{case} markers, the reader is referred to {\S}\ref{formfunccase}.

\begin{exe}
	\ex \emph{waniwanime} [\emph{eda kwayan \textbf{kabeyé}}] \emph{yzänmth.}\\
	\gll waniwani=me eda kwayan {kabe=yé} y\stem{zä}nmth\\
	picture={\Ins} two white man=\Erg.{\Nsg} \Stdu:\Sbj>\Tsg.\Masc:\Obj:\Nonpast:\Ipfv/carry\\
	\trans `The two white people are taking a picture of it.' \Corpus{tci20120821-01}{LNA \#35}
	\label{ex498}
\end{exe}
\begin{exe}
	\ex \emph{famé wathofiyokwrmth fof ... zbomr e} [\emph{eda kabe \textbf{kafaré}}]\emph{ zukorth ``paituaf nima bänemr ŋarär.''}\\
	\gll fam=é wa\stem{thofiyok}wrmth fof (.) zbomr e eda kabe kafar=é zu\stem{kor}th paitua=f nima bänemr ŋa\stem{rä}r\\
	thought=\Erg.{\Nsg} \Stpl:\Sbj>\Fsg:\Obj:\Rpst:\Dur/disturb {\Emph} (.) until until two men big=\Erg.{\Nsg} \Stdu:\Sbj>\Fsg:\Obj:\Rpst:\Pfv/say {old.man=\Erg.\Sg} {like.this} \Recog.{\Purp} \Stsg:\Sbj:\Nonpast:\Ipfv/do\\
	\trans `These thoughts were disturbing me until the two big men told me: ``The old man thinks like this.''' \Corpus{tci20121019-04}{SKK \#22-24}
	\label{ex499}
\end{exe}

The other \isi{nominal} enclitics are no \isi{case} markers: \isi{exclusive} \emph{=nzo} (\Only), empathic \emph{=wä} (\Emph) and et cetera \emph{=sü} (\Etc). The first forms the \isi{nominal} counterpart of the \isi{particle} \emph{komnzo} ({\S}\ref{discourse-particles}). This \isi{clitic} satisfies criteria (iv) in that it is a reduced form of an independent lexical item. It functions as a contrastive \isi{focus} marker and I translate it to with \ili{English} `only'. Hence, in example (\ref{ex115}), the woman picks up the yamstick with only one thing on her mind. Note that this example shows that the \isi{clitic} \emph{=nzo} satisfies criteria (iii): the ability to attach to other clitics. The \isi{exclusive} \isi{enclitic} \emph{=nzo} will be discussed again {\S}\ref{exclusivenzo}.

\begin{exe}
	\ex \emph{yaka \textbf{zanrnzo} srewakuth.}\\
	\gll yaka zan=r=nzo sre\stem{wakuth}\\
	yamstick fight={\Purp}={\Only} \Stsg:\Sbj>\Tsg.\Masc:\Obj:\Irr:\Pfv/pick.up\\
	\trans `She picked up the yamstick to kill him.' \Corpus{tci20120901-01}{MAK \#86}
	\label{ex115}
\end{exe}

The \isi{emphatic} \isi{enclitic} \emph{=wä} shows similar behaviour. It will be addressed in {\S}\ref{emphathicwae}. The et cetera \isi{enclitic} \emph{=sü} only attaches to the \isi{associative} or \isi{proprietive} \isi{case} markers. It will be discussed in {\S}\ref{etceterasue}.

\subsection{Verbal proclitics}\label{verbal-proclitics-subsec}

Verbal clitics are exclusively proclitics. They do not fully satisfy the criteria given above. For example, they only attach to one \isi{word class} (verbs) and they have scope only over the inflected \isi{verb}. On the other hand, all but one verbal \isi{proclitic} are reduced forms of independent lexical items.

Additional evidence against analysing them as prefixes comes from \isi{phonology}. In those cases where the \isi{proclitic} creates an initial \isi{syllable} through \isi{epenthesis}, this \isi{syllable} does not receive \isi{stress}. For example, \emph{bŋasogwr} `he is climbing there' is marked with the \isi{medial} \isi{proclitic} \emph{b=}. Since all proclitics only consist of a single consonant, through \isi{syllabification} an \isi{epenthetic vowel} is inserted: [ᵐbə̆ŋˈaso{\ᵑ}gʷə̆r]. On the surface, the second \isi{syllable} is stressed. However, \isi{stress} remains word-initial, because the \isi{clitic} is not a part of the phonological word. Stress in Komnzo verbs is strictly word-initial and prefixes which create an initial \isi{syllable} (even if filled with the \isi{epenthetic vowel}) are stressed, for example \emph{ŋazi wsogwr} `he climbs the coconut' is realised as [ŋatʃi wˈə̆so{\ᵑ}gʷə̆r].

The first set of verbal proclitics are the \isi{clitic} demonstratives. These are \isi{deictic} proclitics which attach to an inflected verb form: \emph{z=} {\Prox}, \emph{b=} \Med{}, and \emph{f=} {\Dist}. They are described in {\S}\ref{demonstrative-identifiers-subsec} and {\S}\ref{deicticcliticssection}.

The second set of verbal proclitics comprises \emph{m=} and \emph{n=}. Depending on their morpho-syntactic context, they can be classified as either clitics or particles. The \emph{m=} \isi{proclitic} was briefly addressed in {\S}\ref{demonstrative-identifiers-subsec}. We saw in \tabref{demonstratives-table}, that \emph{m=} patterns with the interrogatives. Thus, it patterns with the three \isi{deictic} proclitics. However, this is a marginal function, because it is found only with the copula. More frequently, \emph{m=} occurs with verb forms in \isi{irrealis} or \isi{imperative} mood. In this case it adds the meaning of apprehension (`X might happen!'), as in (\ref{ex119}). Furthermore, with \isi{imperative} verb forms only \uline{and} with the \isi{potential} \isi{particle} \emph{kma} it expresses prohibition (`don't do X!'), as in (\ref{ex111}). In this latter function, \emph{m} is analysed as a \isi{particle} rather than a \isi{proclitic}. This is discussed in detail in {\S}\ref{apprehensivem}.

\begin{exe}
	\ex \emph{thambrnzo \textbf{mthäkwr} fafä.}\\
	\gll thambr=nzo m=thä\stem{kwr} fafä\\
	hand={\Only} {\Appr}=\Ssg:\Sbj>\Stpl:\Obj:\Imp:\Pfv/hit afterwards\\
	\trans `You might go home empty-handed afterwards.' (lit. `You might hit only your hands afterwards.') \Corpus{tci20121019-04}{ABB \#126}
	\label{ex119}
\end{exe}

The second \isi{clitic} \emph{n=} also serves a double function. If attached to a verb inflected for \isi{non-past}, it marks \isi{immediate past}.\footnote{Note that this is shown in the unified gloss: both non-past (\Nonpast) and immediate past (\Immpst) are marked on the verb. This is because the latter is expressed by a clitic, whereas the former is part of the verb morphology proper.} I \isi{gloss} it \Immpst{} and analyse it as a \isi{proclitic}. See example (\ref{ex120}), which was uttered at the end of a recording.

\begin{exe}
	\ex \emph{trikasi mane \textbf{nŋatrikwé} fof ... ngafynm ... badafa ane fof ŋanritakwa fof.}\\
	\gll trik-si mane n=ŋa\stem{trik}wé fof (.) ŋafe=nm (.) bada=fa ane fof ŋan\stem{ritak}wa fof\\
	tell-{\Nmlz} which \Immpst=\Fsg:\Sbj:\Nonpast:\Ipfv/tell {\Emph} (.) father={\Dat}.{\Nsg} (.) ancestor={\Abl} {\Dem} {\Emph} \Stsg:\Sbj:\Pst:\Ipfv:\Venit/cross {\Emph}\\
	\trans `As for the story that I have just told, it was passed on to (our) fathers from the ancestors.' \Corpus{tci20131013-01}{ABB \#403-405}
	\label{ex120}
\end{exe}

The second function of \emph{n} occurs with verbs in one of the \isi{past} tenses or in \isi{irrealis} mood. In this function, \emph{n} is analysed as a particle because it can occur in various positions. This is shown in (\ref{ex121}), where \emph{n} occurs in preverbal position, and in (\ref{ex122}), where it occurs freely in the clause. It expresses that an event was `about to occur' or that someone was `trying to do' something, and I use the gloss {\Imn} for ``\isi{ìmminent}''.  In (\ref{ex121}), the speaker reports how she saw something moving in the grass in her garden. In (\ref{ex122}), the speaker talks about trying to extinguish a fire in his garden. I refer the reader to {\S}\ref{imminentm} for further discussion of \emph{n}.

\begin{exe}
	\ex \emph{wati foba fof \textbf{n} zäbrimé ... wati nzun nima ``kaboth kma zamath.''}\\
	\gll wati foba fof n zä\stem{brim}é (.) wati nzun nima kaboth kma za\stem{math}\\
	then {\Dist}.{\Abl} {\Emph} {\Imn} \Fsg:\Sbj:\Rpst:\Pfv/return (.) then \Fsg{}.{\Dat} {\Quot} snake {\Pot} \Stsg:\Sbj:\Rpst:\Pfv/run\\
	\trans `Well, I was about to return from there ... and I thought to myself ``This must be a snake running off.''' \Corpus{tci20120821-01}{LNA \#9-10}
	\label{ex121}
\end{exe}
\begin{exe}
	\ex \emph{kwankwiré zbo \textbf{n} fam zäré damaki yföfo ... ``keke watikthémär zagr fefe rä.''}\\
	\gll kwan\stem{kwir}é zbo n fam zä\stem{r}é damaki yfö=fo (.) keke watik-thé=mär zagr fefe \stem{rä}\\
	\Fsg:\Sbj:\Nonpast:\Ipfv:\Venit/run {\Prox}.{\All} {\Imn} thoughts \Fsg:\Sbj:\Rpst:\Pfv/do dynamite.well hole={\All} (.) {\Neg} enough-{\Adlzr}={\Priv} far really \Tsg.\F:\Sbj:\Nonpast:\Ipfv/be\\
	\trans `I was running around here considering (going to) the water well, but I thought ``No, not enough, it is too far.''' \Corpus{tci20120922-24}{MAA \#49-50}
	\label{ex122}
\end{exe}

\section{Connectives} \label{connectives-sec}

There are a number of small words which I label connectives. These serve to connect various constituents: noun phrases, clauses, discourse, etc. The most common ones are \emph{a} `and', \emph{o} `or', and \emph{e} `until'. The last of the three is usually a long, stretched out vowel. See examples (\ref{ex123}), (\ref{ex124}), and (\ref{ex125}), respectively.

\begin{exe}
	\ex \emph{nagayé zbo thgathinzako ... mantma kafarwä \textbf{a} srak nge ... katanwä.}\\
	\gll nagayé zbo th\stem{gathinz}ako (.) mantma kafar=wä a srak nge (.) katan=wä\\
	children {\Prox}.{\All} \Sg:\Sbj>\Stdu:\Obj:\Pst:\Pfv:\Andat/leave (.) female big={\Emph} and boy child (.) small={\Emph}\\
	\trans `He left the two children here ... the big girl and the small boy.' \Corpus{tci20100905}{ABB \#21-23}
	\label{ex123}
\end{exe}
\begin{exe}
	\ex \emph{nafaŋamaf wnfathwr \textbf{o} ynfathwr.}\\
	\gll nafa-ŋame=f wn\stem{fath}wr o yn\stem{fath}wr\\
	\Third.{\Poss}-mother=\Erg.{\Sg} \Stsg:\Sbj>\Tsg.\F:\Obj:\Nonpast:\Venit/hold or \Stsg:\Sbj>\Tsg.\Masc:\Obj:\Nonpast:\Venit/hold\\
	\trans `(The child's) mother holds her or holds him.' \Corpus{tci20111004}{RMA \#327-328}
	\label{ex124}
\end{exe}
\begin{exe}
	\ex \emph{nzä nima waniyak \textbf{e} srn kränrsöfthé zrafo.}\\
	\gll nzä nima wa\stem{niyak} e srn krän\stem{rsöfth}é zra=fo\\
	\Fsg.{\Abs} {like.this} \Fsg:\Sbj:\Nonpast:\Ipfv/come until srn \Fsg:\Sbj:\Irr:\Pfv:\Venit/descend swamp={\All}\\
	\trans `I came like this until I walked down to the swamp in Srn.'\Corpus{tci20111119-03}{ABB \#96}
	\label{ex125}
\end{exe}

The three adverbial demonstratives in the \isi{allative} \isi{case} may also be used to express meaning `until' both in a spatial and \isi{temporal} sense. However, they have to marked for the \isi{purposive} \isi{case}, thus producing the forms \emph{zbomr} from \emph{zbo}, \emph{bobomr} from \emph{bobo}, and \emph{fobomr} from \emph{fbo}. This is not possible with the corresponding ablative forms, i.e. \emph{zbamr}, \emph{bobamr} and \emph{fobamr} are all ungrammatical. Example (\ref{ex126}) shows one occurrence of \emph{bobomr} with a \isi{temporal} meaning of `until'. Here, the speaker describes her daily routine in the high school in Daru.

\begin{exe}
	\ex \emph{frasinzo nzwamnzrm ezifa \textbf{bobomr} mor efoth.}\\
	\gll frasi=nzo nzwa\stem{m}nzrm ezi=fa bobomr mor efoth\\
	hunger={\Only} \Fpl:\Sbj:\Pst:\Dur/sit morning={\Abl} until neck day\\
	\trans `We were staying very hungry from the morning until midday.'\\ \Corpus{tci20120924-01}{TRK \#37}
	\label{ex126}
\end{exe}

The word \emph{fthé} `when' may be used to connect clauses as causal, \isi{temporal} or \isi{conditional} sequences (\S\ref{tempadverbials} and \S\ref{condiclauses}). It may also be used without reference to another clause, in which case it can be translated as `at the time when'. See example (\ref{ex127}), where the speaker talks about food taboos.

\begin{exe}
	\ex \emph{kafar ŋarr \textbf{fthé} srarä, nzmärkarä \textbf{fthé} srarä ... zöftha nagayé keke kwa sranathrth.}\\
	\gll kafar ŋarr fthé sra\stem{rä} nzmär=karä fthé sra\stem{rä} (.) zöftha nagayé keke kwa sra\stem{na}thr\\
	big bandicoot when \Tsg.\Masc:\Irr:\Ipfv/be grease={\Prop} when \Tsg.\Masc:\Irr:\Ipfv/be (.) new children {\Neg} {\Fut} \Stsg:\Sbj>\Tsg.\Masc:\Obj:\Irr:\Ipfv/eat\\
	\trans `If it is a big bandicoot, if it is one with grease, then the young children will not eat it.' \Corpus{tci20120922-26}{DAK \#82-83}
	\label{ex127}
\end{exe}

\section{Ideophones} \label{ideophones-sec}

Komnzo ideophones depict almost exclusively sounds and, thus, cover the lower spectrum of the implicational hierarchy of sensory imagery as discussed in Dingemanse (\citeyear[663]{Dingemanse:2012fc}). Komnzo ideophones cover a range of auditory phenomena: sounds from nature, animal sounds, human made noises, bodily noises, human made signals. \tabref{ideophones-table} groups them according to their semantics.

Example (\ref{ex028}) introduces the topic in the context of a rather gruesome story about an unsuccessful headhunting expedition. The \isi{ideophone} \emph{grr kwan} depicts the gurgling or rasping sound of someone breathing, in this example someone dying.

\begin{exe}
	\ex \emph{wgathiknath fobo fof. frknzo zwanorm. \textbf{grr kwannzo} fobo zwanorm.}\\
	\gll w\stem{gathik}nath fobo fof frk=nzo zwa\stem{nor}m grr.kwan=nzo fobo zwa\stem{nor}m\\
	\Stdu:\Sbj>\Tsg.\F:\Obj:\Pst:\Ipfv/leave {\Dist}.{\All} {\Emph} blood={\Only} \Tsg.\F:\Sbj:\Pst:\Dur/shout rasping.sound={\Only} {\Dist}.{\All} \Tsg.\F:\Sbj:\Pst:\Dur/shout\\
	\trans `The two left her while she was bleeding from there (the throat). She was just gurgling.' \Corpus{tci20111119-01}{ABB \#154}
	\label{ex028}
\end{exe}

Ideophones occur as a compound with the word \emph{kwan} `noise, shout, sound'. This should not be taken as evidence that speakers are merely mimicking a particular auditive phenomenon in an ad hoc way. On the contrary, ideophones are conventionalised lexical items like any other word. I will use the term \isi{ideophone} only for those lexical items which do not have a lexical meaning other than the sound they depict. We can observe a gradient from lexical items to ideophones. For example \emph{wth kwan} `fart' consists of \emph{wth} `excrete, faeces' + \emph{kwan}. It is a noun + \isi{noun} compound and it would be wrong to call \emph{wth} an \isi{ideophone}. On the other end of the spectrum we have \emph{brr kwan} `the sound of a bilabial trill' which consists of \emph{brr} + \emph{kwan}. The former refers only to the particular sound and I will therefore call \emph{brr} an \isi{ideophone}. There are some transitional cases like \emph{thmdi kwan} `sound of a sigh during sleep', which is in principle decomposable as \emph{thm} `nose' + \emph{di} `back of the head' + \emph{kwan}. However, speakers do not decompose this word anymore, but understand \emph{thmdi} as one lexical item that refers to a particular sound.

There are only two exceptions, which do not fit the above description: \emph{buay} means `someone taking off in a hurry, fleeing, running away' and \emph{bra} means `something is finished, depleted, or gone'. Both lexical items differ in their semantics from other ideophones, i.e. \emph{buay} expresses movement and \emph{bra} expresses a visual state. They also differ in that they do not occur with \emph{kwan}. However, I analyse them as ideophones following Dingemanse who defines ideophones as ``marked words that depict sensory imagery'' (\citeyear[655]{Dingemanse:2012fc}).

There are a few special phonological characteristics of ideophones. For example, I have shown in {\S}\ref{loanword-phonology} that the bilabial stop [b] is not an indigenous phoneme in Komnzo. We find [b] in a number of ideophones, for example \emph{bübü kwan} `the sound a hunter makes when hitting the ground to attract wallabies'.

{\renewcommand{\tabcolsep}{4pt}
\begin{table}
	\caption{Ideophones}
	\label{ideophones-table}
	\begin{tabularx}{\textwidth}{Xl}
		\lsptoprule
		\multicolumn{2}{l}{{sounds from nature}}\\ \midrule
		\emph{susu kwan}&sound of a running stream of water\\
		\emph{buku kwan}&sound of splashing water (fish jumping, people washing)\\
		\emph{ba kwan}&sound of something heavy falling on the ground\\
		\emph{bü kwan}&sound of a coconut falling on the ground\\
		\emph{rürü kwan}&sound of thunder (in the distance)\\
		\emph{wär kwan}&sound of thunder (close)\\
		\emph{u kwan}&sound of strong wind\\
		%&\\
		\tablevspace
		\multicolumn{2}{l}{{animal sounds}}\\ \midrule
		\emph{sö kwan}&sound of wallabies grunting\\
		\emph{gu kwan}&sound of an animal grunting (e.g. pigs, dogs)\\
		\emph{gww kwan}&sound of barking dogs\\
		%&\\
		
		\tablevspace
		\multicolumn{2}{l}{{bodily sounds}}\\ \midrule
		\emph{nzam kwan}&sound of smacking one's lips\\
		\emph{gwrr kwan}&sound of swallowing something\\
		\emph{thmss kwan}&sound of someone snuffling, snorting\\
		\emph{grr kwan}&sound of stertorous or rasping breathing\\
		\emph{thmdrr kwan}&sound of snoring\\
		\emph{thmdi kwan}&sound of a sigh during sleep\\
		\emph{brr kwan}&bilabial trill (baby babbling or someone farting)\\
		%&\\
		
		\tablevspace
		\multicolumn{2}{l}{{human made noises}}\\ \midrule
		\emph{ta kwan}&sound of something that breaks or cracks, e.g. twigs\\
		\emph{tä kwan}&sound of chopping trees\\
		\emph{yo kwan}&sound of an arrow hitting something\\
		\emph{tütü kwan}&sound of steps, someone walking\\
		\emph{rrr kwan}&sound of rustling through dried leaves\\
		\emph{suku kwan}&sound of someone walking in water\\
		%&\\
		
		\tablevspace
		\multicolumn{2}{l}{{human made signal sounds}}\\ \midrule
		\emph{bübü kwan}&sound of a hunter hitting the ground to attract wallabies\\
		\emph{ws kwan}&sound made to send the dogs after some animal\\
		\emph{äs kwan}&sound made to call the dogs\\
		\emph{knzu kwan}&sound of people shouting out for someone (usually [u:])\\
		\emph{fifiya kwan}&sound of whistling (a song)\\
		\emph{siya kwan}&sound of someone signalling by whistling\\
		\emph{ti kwan}&sound of someone singing in the distance\\
		\emph{si kwan}&hissing sound [s] in order to attract someone's attention\\
		\emph{dm kwan}&a signal of amazement produced as a series of alveolar clicks\\
		\emph{mü kwan}&a signal of approval or a backchannel marker produced as [m:]\\
		%\emph{ya kwan}&sound of someone wailing or crying\\
		\lspbottomrule
	\end{tabularx}
\end{table}}%ideophones

Ideophones can be modified by another \isi{nominal}, an \isi{adjective} or another \isi{noun}. In example (\ref{ex027}), we see the \isi{ideophone} \emph{ta kwan} `a high-pitched clicking, breaking sound' as part of a compound modified by \emph{zr} `tooth'.

\begin{exe}
	\ex \emph{mnzfa boba kwanrizrmth nzarwonaneme \textbf{zr ta kwan}.}\\
	\gll mnz=fa boba kwan\stem{ri}zrmth nzarwon=aneme zr ta.kwan\\
	house={\Abl} \Med{}.{\Abl} \Stpl:\Sbj:\Pst:\Dur:\Venit/hear barramundi={\Poss}.{\Nsg} tooth {clicking.sound}\\
	\trans `They were hearing the snapping of the barramundis from the house.'\\ \Corpus{tci20120922-21}{DAK \#8}
	\label{ex027}
\end{exe}

\section{Interjections} \label{interjections-sec}

Interjections in Komnzo are a small class of uninflecting words used to express delight, bewilderment, a negative attitude, approval or refusal, commands, greetings, or vocatives. Interjections form a separate intonation group, and they stand as an utterance by themselves. \tabref{interjections-table} gives an overview of the most common \isi{interjection}s.

\begin{table}
\caption{Interjections}
\label{interjections-table}
	\begin{tabularx}{\textwidth}{Xp{9,5cm}}
		\lsptoprule
		{form}&{translation (and context)}\\\midrule
		\emph{aiwa}&`oh no' (used to signal compassion, negative surprise, emphasizing with another person's misfortune)\\
		\emph{awe}&`come!'\\
		\emph{awkot}& (used as a sudden surprise, e.g. somebody trips over a log)\\
		\emph{awow}&`ok' (used to signal agreement)\\
		\emph{ayo}&`watch out' (used as a warning sign)\\
		\emph{kare}&`go (away)!'\\
		\emph{kiwar}&`good hunting luck' (used to wish a successful hunting either a person or ritually after setting a trap, hanging a fishnet, etc.)\\
		\emph{monzé}&`yes, of course' (used as a sign of agreement)\\
		\emph{razé}&`yeah' (used as a sign of emphatic agreement or approval)\\
		\emph{si rore rore}& (shouted out by women during poison-root fishing)\\
		\lspbottomrule
	\end{tabularx}
\end{table}%interjections