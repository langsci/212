%!TEX root = ../main.tex

\chapter{Syntax of the noun phrase}\label{cha:npsyntax}

\section{Introduction}\label{npsyntax}

The \isi{noun phrase} in Komnzo is defined as a group of \isi{nominal}s which jointly fulfil a functional role in the clause. Noun phrases may also contain a single \isi{nominal}. The \isi{case} markers which assign the specific functional role attach to the rightmost element of the \isi{noun phrase}. Noun phrases in Komnzo cannot be scrambled. Therefore, \isi{case} enclitics and the \isi{emphatic} \isi{particle} \emph{fof} \textendash{} if present \textendash{} can be used to identify the right edge of a \isi{noun phrase}. Typically one intonation contour covers a single \isi{noun phrase}.

The \isi{head} of a \isi{noun phrase} can be a \isi{noun} ({\S}\ref{nouns-sec}), a \isi{property noun} ({\S}\ref{propertynouns-sec}), a personal \isi{pronoun} ({\S}\ref{personalpronouns-sec}), the \isi{indefinite} \isi{pronoun} ({\S}\ref{indefinite-sec}), the \isi{recognitional} \isi{demonstrative} ({\S}\ref{recognitional-pronoun-subsec}) or an \isi{interrogative} ({\S}\ref{interrogatives-sec}). The \isi{head} of a \isi{noun phrase} can be omitted, leaving only a \isi{demonstrative}, \isi{adjective}, \isi{quantifier} or \isi{locational}. This is possible only if the \isi{head} of the \isi{noun phrase} can be recovered from context. Noun phrases can be dropped from the clause, in which case only the indexing in the verb provides information about the arguments. Consequently, inflected verbs can and often do stand alone as a clause.

This chapter begins with an overview of the structure of the \isi{noun phrase} in {\S}\ref{npstructure}. I describe the slots of a \isi{noun phrase} and their respective fillers in {\S}\ref{npsyntaxdeterminer} - {\S}\ref{headslot}. The chapter closes with a description of the inclusory construction in {\S}\ref{inclusorycontruction}. In this construction, two or more \isi{noun phrase}s constitute a functional unit without forming a matrix \isi{noun phrase}.

\section{The structure of the noun phrase}\label{npstructure}

I analyse noun phrases as flat structures made up by functional slots. Each slot may be filled by particular elements. The abstract structure is shown in Figure \ref{nounphrasestruc}.

\begin{figure}
	\caption{The structure of the noun phrase}
	\label{nounphrasestruc}
	\begin{tabularx}{\textwidth}{Xp{0,3cm}Xp{0,3cm}Xp{0,3cm}X}
		\cline{1-1}\cline{3-3}\cline{5-5}\cline{7-7} 
		\multicolumn{1}{|l|}{determiner slot}&&\multicolumn{1}{|l|}{premodifier slot}&&\multicolumn{1}{|l|}{head slot}&&\multicolumn{1}{|l|}{postmodifier slot}\\
		\cline{1-1}\cline{3-3}\cline{5-5}\cline{7-7}
		&&&&&&\\
		determiner&&adjective&&noun&&determiner\\
		demonstrative&&property noun&&property noun&&adjective\\
		indefinite&&numeral&&nominalised verb&&locational\\
		interrogative&&quantifier&&pers. pronoun&&numeral\\
		{\Poss} pronoun&&&&{\Recog} pronoun&&quantifier\\
		noun phrase&&&&&&\\
	\end{tabularx}
\end{figure}

I analyse the element in the \textsc{head} slot as a semantic \isi{head} which refers to the same entity as the whole phrase. This element is also the syntactic \isi{head}, in that it governs the agreement in the verb form. However, this is only visible if the \isi{noun phrase} has a core argument function. The \textsc{head} slot can be complex, for example when it is filled with a compound. All other slots serve to limit the set of possible referents in the \isi{head}. For this reason, proper nouns like personal or place names are rarely modified, and expressions like \emph{ane Naimr} `that Naimr' are only found if there are several individuals with that name and the speaker wishes to clarify which one is meant. Personal pronouns are never modified in that way.\footnote{Two exceptions are the postposed adjectives \emph{bana} `hapless, poor, pitiful', which expresses a sympathetic emotion of the speaker towards the referent, and the postposed adjective \emph{kwark} `deceased'. Both frequently occur with proper nouns (e.g. personal names) as well as personal pronouns.}

The \textsc{determiner} slot is separate from the \textsc{premodifier} slot for two reasons. First, the elements in this slot are mutually exclusive. Hence, a \isi{noun phrase} can contain either a \isi{possessive} or a \isi{demonstrative} in the \textsc{determiner} slot, but not both. This contrasts with the elements in the \textsc{premodifier} slot, of which there can be multiple instances in the same \isi{noun phrase}. Secondly, as we will see below, if the \isi{noun phrase} is not \isi{case} marked, the elements in the \textsc{determiner} slot can be postposed. If there is a \isi{case} marker, postposing the \isi{determiner} is a rare exception. Such a restriction does not apply to elements in the \textsc{premodifier} slot.

There are two \isi{modifier} slots because some word classes, for example locationals, can only occur in the \textsc{postmodifier} slot and not in the \textsc{premodifier} slot. Otherwise, almost all elements which are possible in the \textsc{premodifier} slot are also possible in the \textsc{postmodifier} slot.

Property nouns cannot be clearly assigned to the \textsc{premodifier} slot, because they can optionally take the \isi{adjectivaliser} suffix \emph{-thé}. In this case, they are derived adjectives in the \textsc{premodifier} slot, but derived adjectives show differences in their syntactic behaviour compared to non-derived adjectives. Without the \isi{adjectivaliser}, property nouns can be a \isi{modifier} element of a \isi{nominal} compound. This is discussed in {\S}\ref{compounds-subsec}.

\section{The \textsc{determiner} slot}\label{npsyntaxdeterminer}

The \textsc{determiner} slot can be filled with demonstratives (\ref{ex512}), interrogatives (\ref{ex513}), \isi{possessive} pronouns (\ref{ex514}) and whole \isi{noun phrase}s inflected for one of the adnominal cases. These include the \isi{possessive} (\ref{ex515}), \isi{temporal} \isi{possessive} (\ref{ex516}) and \isi{characteristic} \isi{case} (\ref{ex517}). In the following examples, \isi{noun phrase}s are marked by square brackets.

\begin{exe}
 	\ex {\emph{fi keke zä wrugr} [\emph{\textbf{zane} gwthen}].}\\
 	\gll fi keke zä w\stem{rugr} zane gwth=en\\
 	\Third.{\Abs} {\Neg} {\Prox} \Tsg.\F:\Sbj:\Nonpast:\Ipfv/sleep \Dem:{\Prox} nest={\Loc}\\
 	\trans `She does not sleep in this nest here.'\Corpus{tci20120815}{ABB \#19}
 	\label{ex512}
\end{exe}
\begin{exe}
 	\ex \emph{wayti erä o} [\emph{\textbf{ra} yawi}] \emph{erä?}\\
 	\gll wayti e\stem{rä} o ra yawi e\stem{rä}\\
 	watermelon \Stpl:\Sbj:\Nonpast:\Ipfv/be or what {round thing} \Stpl:\Sbj:\Nonpast:\Ipfv/be\\
 	\trans `These are watermelons or what fruits are these?'\Corpus{tci20111004}{TSA \#68}
 	\label{ex513}
\end{exe}
\begin{exe}
 	\ex {[\emph{\textbf{nzone} trikasi}] \emph{fobo fof zwaythk.}}\\
 	\gll nzone trik-si fobo fof zwa\stem{ythk}\\
 	\Fsg.{\Poss} tell-{\Nmlz} \Dist.{\All} {\Emph} \Tsg.\F:\Sbj:\Rpst:\Ipfv/come.to.end\\
 	\trans `My story has come to an end there.'\Corpus{tci20111004}{TSA \#260}
 	\label{ex514}
\end{exe}
\begin{exe}
 	\ex \emph{wth fobo fof thämira ...} [[\emph{\textbf{ane kabeane}}] \emph{wth}]\emph{.}\\
 	\gll wth fobo fof thä\stem{mir}a (.) ane kabe=ane wth\\
 	intestines \Dist.{\All} {\Emph} \Stsg:\Sbj>\Stpl:\Obj:\Pst:\Pfv/hang (.) {\Dem} man={\Poss} intestines\\
 	\trans `She hung the intestines there ... that man's intestines.'\Corpus{tci20120901-01}{MAK \#116-117}
 	\label{ex515}
\end{exe}
\begin{exe}
	\ex {[[\emph{\textbf{kaythamane}}] \emph{karo}]\emph{ rä!}}\\
	\gll kayé=thamane karo \stem{rä}\\
	yesterday=\Temp.{\Poss} {ground.oven} \Tsg.\F:\Sbj:\Nonpast:\Ipfv/be\\
	\trans `It is yesterday's oven.'\Corpus{tci20110802}{ABB \#94}
	\label{ex516}
\end{exe}
\begin{exe}
 	\ex {[[\emph{\textbf{baguma}}] \emph{kabe}] \emph{... foba ... zena mifnen zämnzr.}}\\
 	\gll bagu=ma kabe (.) foba (.) zena mifne=n z=ä\stem{m}nzr\\
 	bagu={\Char} man (.) \Dist.{\Abl} (.) now mifne={\Loc} \Prox=\Stpl:\Sbj:\Nonpast:\Ipfv/dwell\\
 	\trans `The Bagu people ... from over there ... live here in Mifne (or: Mibini) today.'\Corpus{tci20131013-01}{ABB \#175-177}
 	\label{ex517}
\end{exe}

These different fillers cannot co-occur. Consider example (\ref{ex518}), which is taken from a \emph{nzürna trikasi}, a local equivalent to European witch stories. The example contains the complex \isi{noun phrase} \emph{nä karma kabe}, in which the \isi{indefinite} \emph{nä} `some, another' and \emph{kar=ma} `from the village' are both candidates for the \textsc{determiner} slot. However, the \isi{indefinite} does not refer to \emph{kabe} `man', but to \emph{kar} `village'. In other words, the \isi{indefinite} fills the \textsc{determiner} slot of the embedded \isi{noun phrase}, and the embedded \isi{noun phrase} fills the \textsc{determiner} slot of the matrix \isi{noun phrase}. This is shown with square brackets in the example. Note that (\ref{ex515}) shows the same structure.

\begin{exe}
	\ex {[[\textbf{\emph{nä karma}}] \emph{\textbf{kabe}}] \emph{mane yanatha ... mogarkamma}}\\
	\gll nä kar=ma kabe mane ya\stem{na}tha (.) mogarkam=ma\\
	{\Indf} village={\Char} man which \Stsg:\Sbj>\Tsg.\Masc:\Obj:\Pst:\Ipfv/eat (.) mogarkam={\Char}\\
	\trans `It was a man from another village who she ate ... from Mogarkam.'\\\Corpus{tci20120901-01}{MAK \#225}
	\label{ex518}
\end{exe}

The \isi{determiner} can appear in postposed position, which I analyse as non-prototypical order. The rest of this section describes this postposed position of the \isi{determiner}. Example (\ref{ex524}) is taken from the same story as the previous example. The \isi{noun phrase} \emph{tüfr yam nä} `many other things' contains the \isi{quantifier} \emph{tüfr} in the \textsc{premodifier} slot, the noun \emph{yam} `event' in the \textsc{head} slot, and the \isi{indefinite} \emph{nä} in postposed position. This \isi{noun phrase} can be arranged in different orders, for example: \emph{nä tüfr yam}, \emph{nä yam tüfr}. However, the \textsc{determiner} and \textsc{premodifier} slots cannot be exchanged. This order of elements, for example \emph{tüfr nä yam}, would be split into two co-referential \isi{noun phrase}s, which is signalled by a break in the intonation contour and case marking on both \isi{noun phrase}s. Case markers would attach to \emph{tüfr} as well as \emph{nä yam}.

\begin{exe}
	\ex {[\emph{\textbf{tüfr yam nä}}] \emph{fefe thwafiyokwrm ... fi fathfa ane fof wäfiyokwa.}}\\
	\gll tüfr yam nä fefe thwa\stem{fiyok}wrm (.) fi fath=fa ane fof wä\stem{fiyok}wa\\
	plenty event {\Indf} really \Sg:\Sbj>\Stpl:\Obj:\Pst:\Dur/make (.) but {clear.place=\Abl} {\Dem} {\Emph} \Sg:\Sbj>\Tsg.\F:\Obj:\Pst:\Ipfv/make\\
	\trans `She really did many other things ... but she did this in public.'\\\Corpus{tci20120901-01}{MAK \#223-224}
	\label{ex524}
\end{exe}

We saw in (\ref{ex518}) that the \isi{determiner} belongs to the \isi{head} of the embedded \isi{noun phrase} and not to the \isi{head} of the matrix \isi{noun phrase}. In such cases, the embedded \isi{noun phrase} `blocks' the \textsc{determiner} slot, and postposing a \isi{determiner} is the only option for it to refer to the \isi{head} of the matrix \isi{noun phrase}. This is shown in (\ref{ex519}), where the postposed \isi{indefinite} \emph{nä} refers to the \isi{head} of the matrix \isi{noun phrase}. The embedded \isi{noun phrase} \emph{safsma} is marked with the \isi{characteristic} \isi{case} in adnominal function. It specifies the \isi{head} of the matrix \isi{noun phrase}: \emph{safsma kabe} `man from \emph{Safs}'. Note that the same could be expressed by a \isi{nominal} compound \emph{safs kabe} `\emph{Safs} man'. The semantic difference between an embedded \isi{noun phrase} marked with the \isi{characteristic} \isi{case} and a \isi{nominal} compound lies in the reference of the \isi{determiner}: \emph{ane safs kabe} `that \emph{Safs} man' versus \emph{ane safsma kabe} `man from that \emph{Safs}' (i.e. not from some other place called \emph{Safs}). It follows that the two elements in (\ref{ex519}) restrict the reference of the \isi{head} simultaneously: the embedded \isi{noun phrase} \emph{safma} and the postposed \isi{determiner} \emph{nä}. A postposed \isi{determiner} usually occurs only if the \isi{noun phrase} is not flagged with a \isi{case} marker. But there are exceptions to this. See (\ref{ex523}) discussed below.

\begin{exe}
	\ex {[[\emph{\textbf{safsma}}] \emph{\textbf{woga nä}}] \emph{fobo swamnzrm ... gfi yf}}\\
	\gll safs=ma woga nä fobo swa\stem{m}nzrm (.) gfi yf\\
	safs={\Char} man {\Indf} \Dist.{\All} \Tsg.\Masc:\Sbj:\Nonpast:\Ipfv/dwell (.) gfi name\\
	\trans `Another man from Safs lived there ... by the name of Gfi.'\Corpus{tci20111107-01}{MAK \#76}
	\label{ex519}
\end{exe}

Although very rare, both \textsc{determiner} slots \textendash{} that of the embedded \isi{noun phrase} and that of the matrix \isi{noun phrase} \textendash{} can be filled. In (\ref{ex553}), the first \isi{non-singular} \isi{possessive} \emph{nzenme} refers to \emph{mayawa}, the \isi{head} of the embedded \isi{noun phrase}, and the \isi{indefinite} \isi{determiner} \emph{nä} refers to \emph{kabe}, the \isi{head} of the matrix \isi{noun phrase}.

\begin{exe}
	\ex {[[\emph{\textbf{nzenme mayawama}}] \emph{\textbf{kabe nä}}] \emph{fä thägathizath.}}\\
	\gll nzenme mayawa=ma kabe nä fä thägathizath\\
	\Fnsg.{\Poss} mayawa={\Char} man {\Indf} {\Dist} \Stpl:\Sbj>\Stpl:\Obj:\Pst:\Ipfv/leave\\
	\trans `They left some of our Mayawa people there.'\Corpus{tci20131013-01}{ABB \#170}
	\label{ex553}
\end{exe}

It follows from the discussion above that two determiners must belong to different \isi{noun phrase}s if they occur next to each other, like \emph{zane} and \emph{nä} in example (\ref{ex521}). In this example, I analyse \emph{zane} as a \isi{noun phrase} with an omitted \isi{head}.

\begin{exe}
	\ex {[\emph{\textbf{zane}}] [\emph{\textbf{nä yawi}}] \emph{yé.}}\\
	\gll zane nä yawi \stem{yé}\\
	\Dem:{\Prox} {\Indf} {round.object} \Tsg.\Masc:\Sbj:\Nonpast:\Ipfv/be\\
	\trans `This is another fruit.'\Corpus{tci20120815}{ABB \#39}
	\label{ex521}
\end{exe}

The elements in the \textsc{determiner} slot cannot be inflected for the full range of cases. For example, demonstratives cannot be inflected for \isi{ergative}, dative, \isi{possessive} and the three spatial cases. In (\ref{ex522}), the \isi{indefinite} \emph{nä} is interpreted as referring to the \isi{object} argument, rather than the \isi{ergative} marked argument.

\begin{exe}
	\ex {[\emph{nof}] [\emph{\textbf{nä}}] \emph{nima thäkothmako.}}\\
	\gll no=f nä nima thä\stem{kothm}ako\\
	water={\Erg} {\Indf} {like.this} \Sg:\Sbj>\Stpl:\Obj:\Pst:\Pfv:\Andat/chase\\
	\trans `The flood chased away others like this.'\Corpus{tci20131013-01}{ABB \#125}
	\label{ex522}
\end{exe}

However, elicitation has shown that even this is possible, but such a structure is very rare. A textual example is shown in (\ref{ex523}), where \emph{ane} refers to the preceding noun, which is flagged with an \isi{ergative}. Note that in this example, \emph{ane} is followed by the \isi{emphatic} \isi{particle} \emph{fof}, which has always scope over the preceding phrase ({\S}\ref{discourse-particles}). Thus, \emph{fof} may `help' to mark the right edge of the \isi{noun phrase} \emph{gwamf ane}. This is, however, not the main function of \emph{fof}.

\begin{exe}
	\ex \emph{wati} [\emph{\textbf{gwamf ane}}] \emph{fof ezi ŋatha thäsa thgathgen.}\\
	\gll wati gwam=f ane fof ezi ŋatha thä\stem{s}a thgathg=en\\
	then gwam=\Erg.{\Sg} {\Dem} {\Emph} morning dog \Stsg:\Sbj>\Tsg.\Masc:\Obj:\Pst:\Pfv/call {burned.place=\Loc}\\
	\trans `Well, that Gwam called the dogs to the burned place in the morning.'\\\Corpus{tci20131013-01}{ABB \#79}
	\label{ex523}
\end{exe}

The above description shows that there are some problems with the analysis of postposing elements in the \textsc{determiner} slot. Determiners like \emph{zane} or \emph{ane} or \emph{nä}, and even \isi{possessive} phrases can stand alone if the \isi{head} of the phrase is recoverable from the context. An alternative would be to analyse the postposed elements as independent \isi{noun phrase}s which are (i) co-referential with the preceding \isi{noun phrase}, and which (ii) lack an element in the \textsc{head} slot. This is always possible and, as we will see below, it is quite common to have co-referential \isi{noun phrase}s in one clause. Sometimes intervening material, for example adverbials, allows us to make a clear decision. If there is no intervening material, only the intonation contour indicates whether a particular example should be analysed as one or two \isi{noun phrase}s.

Syntactic evidence for the possibility of postposing the \isi{determiner} comes from fronted relative clauses which are commonly used for topicalisation ({\S}\ref{info-cleft}). Fronted relative clauses of this type have the following structure: NP \emph{mane} \textsc{copula}. They only allow a full \isi{noun phrase} before the relative \isi{pronoun} \emph{mane} `which, who'. In (\ref{ex528}), the \isi{noun phrase} includes the postposed \isi{indefinite} \isi{determiner} \emph{nä} following its \isi{head} \emph{ŋatha} `dog'. The fronted \isi{relative clause} is marked by square brackets.

\begin{exe}
	\ex \emph{fi} ([\emph{\textbf{kafar ŋatha nä}}] \emph{mane erera}) \emph{fi ane bä bkwaruthrmth büdisnen mnz znen.}\\
	\gll fi kafar ŋatha nä mane e\stem{rä}ra fi ane bä b=kwa\stem{ru}thrmth büdisn=en mnz zn=en\\
	but big dog {\Indf} which \Stpl:\Sbj:\Pst:\Ipfv/be \Third.{\Abs} {\Dem} \Med{} \Med=\Stpl:\Sbj:\Pst:\Dur/bark \Pl={\Loc} house place=\Loc\\
	\trans `But, as for the other big dogs, they were barking there in Büdisn at the house.'\Corpus{tci20111119-03}{ABB \#95}
	\label{ex528}
\end{exe}

\section{The \textsc{modifier} slots}\label{npsyntaxmodifier}

The elements in the \textsc{modifier} slots are different from those in the \textsc{determiner} slot. They can all be inflected for \isi{case} if they happen to occur as the last element of the \isi{noun phrase}. This is shown in (\ref{ex525}) and (\ref{ex526}). In example (\ref{ex525}), the \isi{modifier} is an \isi{adjective} in the \textsc{premodifier} slot. In example (\ref{ex526}), the adjective follows the \isi{head} in the \textsc{postmodifier} slot, and consequently the \isi{adjective} receives the \isi{case} marker.

\begin{exe}
	\ex \emph{finzo fä fof \textbf{ane kafar emothf} thwathofiknm.}\\
	\gll fi=nzo fä fof ane kafar emoth=f thwa\stem{thofik}nm\\
	\Third.\Abs={\Only} {\Dist} {\Emph} {\Dem} big girl=\Erg.{\Sg} \Sg:\Sbj>\Stdu:\Obj:\Pst:\Dur/disturb\\
	\trans `Only they (were) there. That big girl was disturbing them.'\Corpus{tci20111119-01}{ABB \#150}
	\label{ex525}
\end{exe}
\begin{exe}
	\ex \emph{watik \textbf{yfö katanr} kwa yarenzr.}\\
	\gll watik yfö katan=r kwa ya\stem{re}nzr\\
	then hole small={\Purp} {\Fut} \Tsg.\Masc:\Sbj:\Nonpast:\Ipfv/look\\
	\trans `Then, he will look around for a small hole.'\Corpus{tci20130903-04}{RNA \#26}
	\label{ex526}
\end{exe}

There are some restrictions for specific elements, for example the locationals can only inflect for spatial cases. Furthermore, all locationals ({\S}\ref{locationals-sec}) and a few adjectives ({\S}\ref{adjectives-sec}) only occur in the \textsc{postmodifier} slot, not in the \textsc{premodifier} slot. One such adjective is \emph{kwark} `late, deceased' in (\ref{ex527}). It occurs in the \textsc{postmodifier} slot, and therefore it is flagged with the \isi{ergative} \isi{case}. Note that the proper name \emph{Wäni} is also inflected with the \isi{ergative} and forms a \isi{noun phrase} co-referential to \emph{nafaŋafe kwark} `his late father'.

\begin{exe}
	\ex \emph{wati ... \textbf{nafaŋafe kwarkf} ... wänif krekariso}\\
	\gll wati (.) nafa-ŋafe kwark=f (.) wäni=f kre\stem{karis}o\\
	then (.) \Third.\Poss-father deceased=\Erg.{\Sg} (.) wäni=\Erg.{\Sg} \Sg:\Sbj:\Irr:\Pfv:\Andat/hear\\
	\trans `Then, his late father, Wäni, heard (about it).'\Corpus{tci20120814}{ABB \#114}
	\label{ex527}
\end{exe}

Another difference between elements in the \textsc{determiner} slot and the \textsc{modifier} slots is that elements in the latter may be multiple. I can only give examples from elicitation here as there are no examples in the corpus, where (i) all slots are filled and (ii) multiple items occur in the \textsc{modifier} slot.

\begin{exe}
	\ex \label{ex530}
	\begin{xlist}
		\ex \emph{ane kafar yfrsé wämne}\\
		\gll ane kafar yfrsé wämne\\
		{\Dem} big black tree\\
		\trans `that big black tree'
		\label{ex531}
		\ex \emph{zane eda zanfr garda}\\
		\gll zane eda zanfr garda\\
		\Dem:{\Prox} two long canoe\\
		\trans `these two long canoes'
		\label{ex532}
		\ex \emph{nafane kafar mnz banbanen}\\
		\gll nafane kafar mnz banban=en\\
		\Tsg.{\Poss} big house underneath=\Loc\\
		\trans `underneath his big house'
		\label{ex533}
	\end{xlist}
\end{exe}

The lack of textual examples which display all possible fillers at once is best explained by a strong tendency to distribute information over several co-referential \isi{noun phrase}s, either in the same clause or over a series of clauses. This can be seen in examples like (\ref{ex527}), (\ref{ex528}) or (\ref{ex529}). I will address this topic in the following section.

\section{The \textsc{head} slot}\label{headslot}

As pointed out above in {\S}\ref{npstructure}, the \isi{head} of a \isi{noun phrase} is both the notional \isi{head} as well as the syntactic \isi{head}. It is the notional \isi{head} in the sense that it expresses what the whole \isi{noun phrase} is about, and all other elements in a \isi{noun phrase} serve to restrict the reference of the \isi{head}. It is the syntactic \isi{head} because it agrees in \isi{gender} and \isi{number} with the indexation in the verb. Below, I will address two points which sit at opposite ends of a spectrum: the \isi{ellipsis} of the \isi{head}, and complex heads involving compounds.

\subsection{Introduction}

However, before discussing those two points I want to make a general point about \isi{noun phrase}s in Komnzo. It is quite common to have multiple co-referential \isi{noun phrase}s. These can occur in the same clause or across a sequence of clauses. In example (\ref{ex529}), the speaker talks about an old woman who was married to three men in her lifetime, but she had children only with one of them. Several \isi{noun phrase}s are co-referential. In the example, they are indexed with subscripted numbers.

\begin{exe}
	\ex {[\emph{\textbf{ausiane nagayé}}]\textsubscript{1} \emph{...} [\emph{\textbf{anenzo}}]\textsubscript{1} \emph{fof ern} [\emph{\textbf{edanzo}}]\textsubscript{1} \emph{...} [\emph{\textbf{nä}}]\textsubscript{2} \emph{mane yarako} [\emph{\textbf{ausiane kabe}}]\textsubscript{2} [\emph{\textbf{nafafis}}]\textsubscript{2} \emph{ngemär yara ...} [\emph{\textbf{kafarkafar}}]\textsubscript{2} \emph{yara}}\\
	\gll ausi=ane nagayé (.) ane=nzo fof e\stem{rn} eda=nzo (.) nä mane ya\stem{r}ako ausi=ane kabe nafa-fis nge=mär ya\stem{r}a (.) kafar-kafar ya\stem{r}a\\
	{old woman=\Poss.\Sg} children (.) \Dem={\Only} {\Emph} \Stdu:\Sbj:\Nonpast:\Ipfv/be two={\Only} (.) {\Indf} which \Tsg.\Masc:\Sbj:\Pst:\Ipfv:\Andat/be {old woman=\Poss} man \Third.\Poss-husband child={\Priv} \Tsg.\Masc:\Sbj:\Pst:\Ipfv/be (.) \Redup-big \Tsg.\Masc:\Sbj:\Pst:\Ipfv/be\\
	\trans `The old woman has only those two children. As for the other one, the old woman's man, her husband, he was without children. He was very old (when they got married)'\Corpus{tci20131013-02}{ABB \#334-336}
	\label{ex529}
\end{exe}

At the other end of the spectrum, \isi{noun phrase}s can be wholly omitted, since the indexation in the verb is sufficient. In this way, a single verb often stands as a whole clause. Example (\ref{ex534}) describes the path which the ancestor took and what actions he did along the way. Since the protagonist is highly topical at this point in the story, the respective \isi{noun phrase} is left out. Moreover, the last two verbs \emph{zwafrmnzrm} `he was preparing it (\F)' and \emph{zurzirakwa} `he tied it (\F)' occur without any \isi{noun phrase}s. That is because the \isi{object} \isi{noun phrase} (\emph{nabi ŋatr} `bowstring') was mentioned already.

\begin{exe}
	\ex \emph{nabi ŋatr fä fof zurärm zwafrmnzrm ... zurzirakwa fof.}\\
	\gll nabi ŋatr fä fof zu\stem{rä}rm zwa\stem{frm}nzrm (.) zu\stem{rzirak}wa fof\\
	bamboo bowstring {\Dist} {\Emph} \Sg:\Sbj>\Tsg.\F:\Obj:\Pst:\Dur/do \Sg:\Sbj>\Tsg.\F:\Obj:\Pst:\Dur/prepare (.) \Sg:\Sbj>\Tsg.\F:\Obj:\Pst:\Ipfv/tie {\Emph}\\
	\trans `Over there, he made his bowstring. He prepared it. He tied it.'\\\Corpus{tci20131013-01}{ABB \#235-236}
	\label{ex534}
\end{exe}

\subsection{Ellipsis of the \textsc{head}}\label{headellipsis}

The \isi{head} of a \isi{noun phrase} is often omitted. Consider example (\ref{ex535}), where a mother tells me that she had sent two small children to dig for worms. The example starts out with the \isi{noun phrase} \emph{zane edawä kakatan} `these two small (ones)'. Ellipsis of the \isi{head} only occurs when the \isi{head} is recoverable from previous context, or if it is common ground between speaker and hearer.

\newpage 
\begin{exe}
	\ex \emph{\textbf{zane edawä kakatan} ... fosam daisy fi zarath dd etharinath}\\
	\gll zane eda=wä ka-katan (.) fosam daisy fi za\stem{r}ath dd e\stem{thari}nath\\
	\Dem:{\Prox} two={\Emph} \Redup-small (.) fosam daisy \Third.{\Abs} \Stdu:\Sbj:\Pst:\Pfv/do worm \Stdu:\Sbj>\Stpl:\Obj:\Pst:\Ipfv/dig\\
	\trans `These two small (ones), Fosam and Daisy, they did that. They dug the worms.'\\\Corpus{tci20120922-25}{ALK \#5}
	\label{ex535}
\end{exe}

Example (\ref{ex536}) shows the \isi{indefinite} \isi{demonstrative} \emph{nä} used twice without a \isi{head}. This is possible because the appropriate filler for the \textsc{head} slot \emph{zuzi} `fishing line' was already mentioned.

\begin{exe}
 	\ex \emph{zuzi thethkäfath migsi ... \textbf{nä} zba wazi ... \textbf{nä} boba wazi.}\\
 	\gll zuzi the\stem{thkäf}ath mig-si (.) nä zba wazi (.) nä boba wazi\\
	{fishing.line} \Stpl:\Sbj>\Stpl:\Obj:\Pst:\Pfv/start hang-{\Nmlz} (.) {\Indf} \Prox.{\Abl} side (.) {\Indf} \Med.{\Abl} side\\
 	\trans `They started hanging the fishing lines ... some on this side and some on the other side.'\Corpus{tci20150906-10}{ABB \#52-53}
 	\label{ex536}
\end{exe}

Example (\ref{ex537}) is a description of a fish trap. These long bamboo baskets always consist of a larger basket and a smaller basket which is placed inside the bigger one. In the example, the speaker refers to the smaller basket as \emph{nafane nge} `its child' and later only with an adjective \emph{katan} `small' which is flagged with an \isi{ergative} \isi{case} marker.

\begin{exe}
 	\ex \emph{nafane nge ... wati kofä fthé brigsir n krär ... \textbf{katanf} kwa ynbrigwr zbo ... keke kwa kränmätr.}\\
 	\gll nafane nge (.) wati kofä fthé brig-si=r n krä\stem{r} (.) katan=f kwa yn\stem{brig}wr zbo (.) keke kwa krän\stem{mätr}\\
 	\Tsg.{\Poss} child (.) then fish when return-\Nmlz={\Purp} {\Imn} \Stsg:\Sbj:\Irr:\Pfv/do (.) small={\Erg} {\Fut} \Stsg:\Sbj>\Tsg.\Masc:\Obj:\Nonpast:\Ipfv:\Venit/return \Prox.{\All} (.) {\Neg} {\Fut} \Stsg:\Sbj:\Irr:\Pfv:\Venit/exit\\
 	\trans `Its child ... well, when the fish tries to return, the small (one) will bring it back here ... it will not get out.'\Corpus{tci20120906}{MAB \#55-58}
 	\label{ex537}
\end{exe}

\subsection{Compounds}\label{compounds-subsec}

An the other end of the spectrum are complex heads. The Komnzo lexicon contains a large number of \isi{nominal} compounds. These may consist of nouns, property nouns or nominalised verbs. \tabref{nomcompounds} shows a few examples of compounds with different \isi{nominal} subclasses.

{\renewcommand{\tabcolsep}{4pt}
\begin{table}
\caption{Nominal compounds}
\label{nomcompounds}
	\begin{tabularx}{\textwidth}{lXXXl}
		\lsptoprule
		type of compound&example&\multicolumn{2}{l}{components}&{gloss}\\
		\midrule
		noun + noun &\emph{wawa mnz} &\emph{wawa} &\emph{mnz} &`yamhouse'\\
		&& \footnotesize{yam} &\footnotesize{house} &\\
		
		\tablevspace
		&\emph{wath kabe} &\emph{wath} &\emph{kabe} &`dancer(s)'\\
		&&\footnotesize{dance} &\footnotesize{man/people} &\\
		
		\tablevspace
		\isit{property noun} + noun& \emph{wri kabe} &\emph{wri} &\emph{kabe} &`drunkard'\\
		&&\footnotesize{intoxication} &\footnotesize{man/people} &\\
		
		\tablevspace
		noun + \isi{property noun}& \emph{zan miyo} &\emph{zan} &\emph{miyo} &`bloodlust'\\		
		&&\footnotesize{killing} &\footnotesize{desire} &\\
		
		\tablevspace
		nom. verb + noun &\emph{borsi zokwasi} &\emph{bor-si} &\emph{zokwasi} &`joke'\\
		&&\footnotesize{play-\Nmlz} &\footnotesize{words} &\\
		
		\tablevspace
		noun + nom. verb &\emph{si zübraksi} &\emph{si} &\emph{zübrak-si} &`prayer'\\
		&&\footnotesize{eye} &\footnotesize{close.eye-\Nmlz}&\\
		\lspbottomrule
	\end{tabularx}
\end{table}}%Nominal compounds

Compounds are always right-headed, that is, the rightmost element is not only the semantic \isi{head}, but it determines the \isi{word class}, \isi{number} and \isi{gender} of the whole compound. Although the first element in \emph{wawa mnz} `yam house' is \isi{masculine}, it is the second element \emph{mnz} `house' which determines the \isi{gender} (\F{} in this case). Likewise, although the first element in \emph{wri kabe} is a \isi{property noun} \textendash{} and property nouns do not show \isi{gender} agreement \textendash{} it is the second word \emph{kabe} `man' which enables \isi{gender} agreement for the whole compound.

Compounds can be embedded within one another, which can lead to combinations of usually up to three elements. A rare example of a compound with four elements was coined by one of my informants to describe the botanist on our team: \emph{wämne taga yf kabe} (lit. `tree leaf name man'). Embedded compounds are always left-branching, and thus we can represent long compounds in this way: [[[\emph{wämne taga}]\textsubscript{3} \emph{yf}]\textsubscript{2} \emph{kabe}]\textsubscript{1}. Two corpus examples of longer compounds are given in (\ref{ex510}) and (\ref{ex511}).

\begin{exe}
 	\ex \emph{ane \textbf{ksi kar emoth} thwanorm}\\
 	\gll ane ksi kar emoth thwa\stem{nor}m\\
 	{\Dem} bush place girl \Stpl:\Sbj:\Pst:\Dur/shout\\
 	\trans `These bush girls were shouting.'\Corpus{tci20120821-02}{LNA \#36}
 	\label{ex510}
\end{exe}
\begin{exe}
 	\ex \emph{\textbf{baf} fthé sräbth nima ... \textbf{kabe zan miyof}}\\
 	\gll baf fthé srä\stem{bth} nima (.) kabe zan miyo=f\\
 	\Recog.\Erg.{\Sg} when \Stsg:\Sbj>\Tsg.\Masc:\Obj:\Irr:\Pfv/finish like.this (.) man hitting desire=\Erg\\
 	\trans `That is when it overcomes him ... that bloodlust for people.'\\\Corpus{tci20130903-04}{RNA \#84-85}
 	\label{ex511}
\end{exe}

Complex heads are different from complex \isi{noun phrase}s, that is, compounds in the \textsc{head} slot are distinct from embedded \isi{noun phrase}s. The latter must be marked with adnominal \isi{case}. Let us take the compound from example (\ref{ex510}): \emph{ane ksi kar emoth} `those bush girls' (lit. `bush place girls'). We can embed the \isi{noun phrase} \emph{ksi kar} `bush place' into the matrix \isi{noun phrase} by adding the \isi{characteristic} \isi{case} (\emph{=ma}): \emph{ksi karma emoth} `girls from the bush'. In addition to \isi{case} marking, the reference of the \isi{demonstrative} \emph{ane} in initial position depends on whether a \isi{noun phrase} is embedded or the \isi{head} contains a compound. In the former \isi{case}, \emph{ane} refers to the \isi{head} of the embedded \isi{noun phrase}: \emph{ane ksi karma emoth} `girls from that bush place'. If the \isi{head} slot contains a compound, and no embedding takes place, the \isi{demonstrative} refers to the compound, as in (\ref{ex510}). The reference of the \textsc{determiner} slot is described above in {\S}\ref{npsyntaxdeterminer}.

Property nouns can appear in both positions of a compound (see \tabref{nomcompounds} above). If a \isi{property noun} occurs as the first element, it modifies the \isi{head} of the compound, for example \emph{wri kabe} `drunkard' in \tabref{nomcompounds}. Property nouns optionally take the \isi{adjectivaliser} \emph{-thé}. When this suffix is present, for example in \emph{writhé kabe}, it is clear that the derived \isi{adjective} appears in the \textsc{modifer-1} slot, and is not part of a compound. The semantic difference is between \emph{wri kabe} `drunkard' \textendash{} someone who is frequently drunk \textendash{} and \emph{writhé kabe} `drunk man' \textendash{} someone who is drunk. Syntactically, the derived \isi{adjective} behaves like other adjectives, for example it can appear after the \isi{head} in the \textsc{postmodifier} slot. Without the \isi{adjectivaliser}, a change in order would change the meaning of the compound, e.g. \emph{kabe wri} `people's / men's intoxication'. However, as mentioned above, the \isi{adjectivaliser} suffix is optional for property nouns. Additionally, property nouns can function predicatively (\ref{ex538}). This creates some problems for the analysis of particular examples.

\begin{exe}
	\ex \emph{kabe \textbf{wri} kwosi sfthnm.}\\
	\gll kabe wri kwosi sf\stem{thn}m\\
	man drunk dead \Tsg.\Masc:\Pst:\Dur{}/lie\\
	\trans `The man was lying down dead drunk.'{\hspace*{1pt}\hfill{\footnotesize{[overheard]}}}
	\label{ex538}
\end{exe}

Lastly, I want to address compounds which involve nominalised verbs. Consider the compounds in (\ref{ex539}) and (\ref{ex540}). In (\ref{ex539}), the speaker points out that these were \emph{mgthksi ruga} `raised pigs' as opposed to wild pigs. In (\ref{ex540}), the speaker stresses that he has raised enough pigs in his life, and that \emph{ruga mgthksi} `pig feeding' is too much work.
\largerpage

\begin{exe}
	\ex \emph{ruga tabrunzo erera nima berä ... \textbf{mgthksi ruga}}\\
	\gll ruga tabru=nzo e\stem{rä}ra nima b=e\stem{rä} (.) mgthk-si ruga\\
	pig five={\Only} \Stpl:\Sbj:\Pst:\Ipfv/be {like.this} \Med=\Stpl:\Sbj:\Nonpast:\Ipfv/be (.) feed-{\Nmlz} pig\\
	\trans `There were only five pigs like these ... raised pigs.'\Corpus{tci20120904-02}{MAB \#248-249}
	\label{ex539}
\end{exe}\largerpage
\begin{exe}
	\ex \emph{zena keke miyo worä \textbf{ruga mgthksi} ... znsä ttüfr}\\
	\gll zena keke miyo wo\stem{rä} ruga mgthk-si (.) znsä t-tüfr\\
	today {\Neg} desire \Fsg:\Sbj:\Nonpast:\Ipfv/be pig feed-{\Nmlz} (.) work \Redup-plenty\\
	\trans `Today, I do not want to feed pigs ... too much work.' (lit. `I do not desire pig feeding.')\Corpus{tci20120805-01}{ABB \#819-820}
	\label{ex540}
\end{exe}

We find that compounds which involve nominalised verbs follow the same rule as other compounds: the rightmost element acts as the \isi{head} of the compound. For example, \emph{zan kabe} (killing+man) `killer, headhunter' is a kind of man, whereas \emph{kabe zan} (man+killing) `war, fighting' is a nominalised activity.\footnote{\emph{Zan} `hit, kill' is irregular in that its infinitive is not based on the normal stem-{\Nmlz} pattern.} For the following discussion, I will refer to the first pattern as noun-headed compounds, and the latter as verb-headed compounds.

In noun-headed compounds, the argument role of the noun with respect to the verb is less determined than in verb-headed compounds. The following argument roles are found: actor (\emph{zan kabe} `killer'), \isi{patient} (\emph{mgthksi ruga} `feeding pig' in (\ref{ex539}) above), instrument (\emph{bi näbüsi wämne} `sago beating stick'), location (\emph{yonasi faf} `drinking place'), or time (\emph{tharisi efoth} `harvesting time'). This variability contrasts with verb-headed compounds, where the noun is always a \isi{patient} or \isi{theme}, as in \emph{kabe zan} `war' (lit. `people hitting'), \emph{ruga mgthksi} `pig feeding' in (\ref{ex540}) above, or \emph{wawa yarisi} `yam exchange' (lit. `yam giving'). Note that there is an implied \isi{agent} in most of these examples. It follows that (nominalised) \isi{intransitive} verbs do not participate in verb-headed compounds. For example, there can be a \emph{mthizsi kabe} `resting person' or a \emph{yathizsi kabe} `dying person'. But the reverse order is ungrammatical: \textsuperscript{$\ast$}\emph{kabe mthizsi} or \textsuperscript{$\ast$}\emph{kabe yathizsi}.

Some stems have been shown to be rather fluid in \isi{valency} depending on the morphological template ({\S}\ref{valencyalternations}), for example \emph{msaksi} `dwell, sit (v.i.), set (v.t.)'. It is no surprise that these verbs allow both types of compounds. The noun-headed compound \emph{msaksi kabe} `sitting people' can describe a group of people who stay behind, while others are attending a dance. The verb-headed compound \emph{kabe msaksi} `married life' takes on the \isi{transitive} meaning of the verb, and it means literally: `the sitting down of the man'.\footnote{From the perspective of a man, one could also use \emph{ŋare msaksi} `married life' (lit. `the sitting down of the woman').}

\section{The inclusory construction}\label{inclusorycontruction}

The \isi{inclusory} construction builds on the \isi{associative} \isi{case} ({\S}\ref{comcase}). I adopt the term ``\isi{inclusory} construction'' from Lichtenberk (\citeyear{Lichtenberk:2000hr}) and Singer (\citeyear{Singer:inclu}). Singer defines the \isi{inclusory} construction as ``an endocentric construction in which some elements of a larger group are referred to along with the larger group itself'' (\citeyear[1]{Singer:inclu}). Thus, we have a construction that involves a full set and one or more subsets. In Komnzo, the full set is always expressed in the verb form. Therefore, the \isi{inclusory} construction only involves core arguments, that is, arguments flagged with the \isi{ergative}, \isi{absolutive} or \isi{dative} \isi{case}. For the following description, I introduce the terms ``\isi{associative} phrase'' and ``core phrase''. The \isi{associative} phrase expresses the \isi{participant} who is included in the event. The core phrase expresses a subset different from the one expressed in the \isi{associative} phrase or it may express the set. We will see below why this is sometimes difficult to determine with certainty. While the reference of the core phrase does not automatically include the subset expressed in the \isi{associative} phrase, both are included in the full set which is expressed in the verb form. I choose the terms `core phrase' and `\isi{associative} phrase' over more general terms like `subset A' and `subset B' because the core phrase is flagged with the \isi{case} marker appropriate for the argument role of the set, while the \isi{associative} phrase is flagged with the \isi{associative} \isi{case}.

What is special about the \isi{inclusory} construction in Komnzo is that although both core phrase and \isi{associative} phrase refer to distinct subsets, the \isi{number} marking on each phrase has scope over the total set. Consider example (\ref{ex725}) where the total set encoded in the verb is second/third dual. The two subsets are expressed by the personal names \emph{Maureen} and \emph{Kowi}. The core phrase is flagged with a \isi{non-singular} \isi{ergative} (\emph{Maureen=é}), and the \isi{associative} phrase is flagged with an \isi{dual} \isi{associative} (\emph{Kowi=r}). The point here is that the scope of the \isi{number} value is always the total set and not the respective subsets.\footnote{Note that literal translations of the inclusory construction are rather clumsy: `Maureen with Kowi beat Sago', whereas idiomatic \ili{English} translations imply that the verb is indexing a singular, as in (\ref{ex725}).}

\begin{exe}
	\ex \emph{Maureené bi ynäbünth Kowir.}\\
	\gll maureen=é bi y\stem{näbü}nth kowi=r\\
	maureen=\Erg.{\Nsg} sago(\Abs) \Stdu:\Sbj>\Tsg.\Masc:\Obj:\Nonpast:\Ipfv/beat kowi=\Assoc.{\Du}\\
	\trans `Maureen together with Kowi beats Sago.' (lit. `Maureen with Kowi, they beat Sago.')
	\label{ex725}
\end{exe}

Example (\ref{ex725}) shows that a \isi{non-singular} attaches to a personal name. In example (\ref{ex340}), the set encoded in the verb is first \isi{plural}. Note that the core phrase is omitted here, but it could be expressed by the \isi{pronoun} \emph{ni} (\Fnsg). There are multiple \isi{associative} phrases in the example: \emph{nä srakä} `with some boy(s)', \emph{mafä thzé} `with whoever' and \emph{Mosesä} `with Moses'. Since the total set is bigger than the minimal group, i.e. bigger than two, the \isi{associative} phrase has to be marked as \isi{plural}. Therefore, the personal name \emph{Moses} is marked for \isi{plural}.

\begin{exe}
	\ex \emph{nä srakä kwa nyak ... mafä thzé ... Mosesä.}\\
	\gll nä srak=ä kwa n\stem{yak} (.) maf=ä thzé (.) moses=ä\\
	some boy=\Assoc.{\Pl} {\Fut} \Fpl:\Sbj:\Nonpast:\Ipfv/walk (.) who=\Assoc.{\Pl} ever (.) moses=\Assoc.\Pl\\
	\trans `We will go with some boy(s) ... with whoever ... with Moses.'\\\Corpus{tci20130907-02}{RNA \#749-750}
	\label{ex340}
\end{exe}

The abstract structure of the \isi{inclusory} construction is shown in Figure \ref{incluscons}. The circle represents the set, and the line in the middle cuts the total set into two subsets. The arrows on the left point to the referents expressed by each element. Note that there can be more than one \isi{associative} phrase (\ref{ex340}). Examples like (\ref{ex340}) can be further elaborated by adding \isi{associative} phrases, for example \emph{Maureenä} and \emph{Kowiä} to mean `with Moses, with Maureen, with Kowi'. These additional \isi{associative} phrases are not represented in Figure \ref{incluscons} because they would receive the same marking as the first \isi{associative} phrase.\footnote{Naturally, this is only possible if there are more than two participants in the total set.} The arrow on the right shows that the \isi{number} value encoded in each element tracks the number of the total set.

\begin{figure}[H]
\centering
	\begin{tikzpicture}
		%\path [draw=none,fill=gray, fill opacity = 0.1] (0,0) circle (2);
		\path [draw=none,fill=white, fill opacity = 1] (0,0) circle (1);
		\node [right] at (-2,5.2) {verb inflection};
		\node [right] at (1.3,5.5) {\Du};
		\node [right] at (1.3,4.9) {\Pl};
		\node [right] at (-2,3) {associative phrase};
		\node [right] at (-2,4.1) {core phrase};
		\node [right] at (1.3,3.3) {\Du};
		\node [right] at (1.3,2.7) {\Pl};
		\node [right] at (1.3,4.1) {\Nsg};
	    \draw [black,rotate=220,postaction={decorate,decoration={raise=-1ex, text effects along path, text={set}, text effects/.cd, text along path,
every character/.style={fill=white, yshift=.5ex}}}] (0,0) circle (2);
	    \draw (0,2) --(0,-2);
	    \draw (-1.8,5.5) --(-2,5.5) --(-2,4.9) --(-1.8,4.9);
		\draw (2.2,5.8) --(2.4,5.8) --(2.4,4.6) --(2.2,4.6);
		\draw (-1.8,4.4) --(-2,4.4) --(-2,3.8) --(-1.8,3.8);
		\draw (-1.8,3.3) --(-2,3.3) --(-2,2.7) --(-1.8,2.7);
		\draw (2.2,4.4) --(2.4,4.4) --(2.4,3.8) --(2.2,3.8);
		\draw (2.2,3.6) --(2.4,3.6) --(2.4,2.4) --(2.2,2.4);
		\draw [->] (-2,4.1) --(-3,4.1) --(-3,-0.7) --(-0.9,-0.7);
		\draw [->] (-2,3) --(-2.5,3) --(-2.5,0.7) --(0.2,0.7);
		\draw [->] (2.4,4.1) --(2.9,4.1) --(2.9,0) --(2,0);
		\draw (2.4,3) --(2.9,3);
		\draw (2.4,5.2) --(2.9,5.2) --(2.9,4.1);
		\draw [->] (-2,5.2) --(-3.5,5.2) --(-3.5,0) --(-2,0);
		\node[label={[label distance=0.5cm,text depth=-1ex,rotate=-90]right:\footnotesize{scope of number marking}}] at (3.2,5) {};
		\node[label={[label distance=0.5cm,text depth=-1ex,rotate=90]right:\footnotesize{expressed referent}}] at (-4,0.5) {};
		\node[label={[label distance=0.5cm,text depth=-1ex,rotate=-90]right:\footnotesize{subset B}}] at (0.5,2) {};
		\node[label={[label distance=0.5cm,text depth=-1ex,rotate=90]right:\footnotesize{subset A}}] at (-0.8,-2) {};
	\end{tikzpicture}
\caption{The inclusory construction}
\label{incluscons}
\end{figure}%\isi{inclusory} construction

\figref{incluscons} shows that the \isi{number} values differ. The core phrase is always in \isi{non-singular}. This is the expected behaviour of \isi{number} marking on \isi{nominal}s (\S\ref{formfunccase}), which makes a distinction between \isi{singular} and \isi{non-singular}, leaving the subdivision between \isi{dual} and \isi{plural} to the verb inflection. As for the \isi{associative} phrase, \isi{number} marking is more specific, showing agreement with the verb inflection, thus encoding \isi{dual} versus \isi{plural} instead of \isi{singular} versus \isi{non-singular}. Because the set in the \isi{inclusory} construction is minimally two, a \isi{singular} on the core phrase or a \isi{singular} in the verb inflection would be ungrammatical. For the \isi{associative} \isi{case}, there is no \isi{singular} \isi{number} value available. The enclitics \emph{=r} and \emph{=ä} encode \isi{dual} and \isi{plural} respectively.

\begin{table}
\caption{Associative case / pronominals}
\label{comcase-table}
	\begin{tabularx}{.8\textwidth}{XXXl}
		\lsptoprule
		& {person} & {dual} & {plural} \\\midrule
		\multirow{3}{3cm}{personal pronouns} &1 &\emph{ninrr} &\emph{ninä}\\
		&2 &\emph{bnrr} &\emph{bnä}\\
		&3 &\emph{nafrr} &\emph{nafä}\\
		\Recog&&\emph{bafrr}&\emph{bafä}\\
		\Indf&&\emph{nä bunrr}&\emph{nä bunä}\\
		interrogative&&\emph{mafrr}&\emph{mafä}\\
		case enclitic&&\emph{=r}&\emph{=ä}\\
		\lspbottomrule
	\end{tabularx}
\end{table}%Associative case / pronominals

The corresponding \isi{pronominal} forms of the \isi{associative} \isi{case} are shown in \tabref{comcase-table}.\footnote{I repeat here \tabref{comcase-table-1} in \S\ref{comcase}.} The relevant pronominals are personal pronouns, the \isi{recognitional} \isi{demonstrative}, the \isi{indefinite} \isi{pronoun} and the \isi{interrogative}. Two observations can be made from \tabref{comcase-table}. First, all forms include a /rr/ element for \isi{dual} and an /ä/ element for \isi{plural}. Secondly, most forms are built from the \isi{ergative} \isi{pronominal}. For example, the third \isi{person} absolutive is \emph{fi}, whereas the third \isi{person} \isi{ergative} is \emph{naf} (\Sg) or \emph{nafa} (\Nsg). The \isi{associative} third \isi{person} forms, \emph{nafrr} (\Du) and \emph{nafä} (\Pl) are formally closer to the \isi{ergative} than to the absolutive. Another example is the \isi{interrogative}, where the absolutive is \emph{mane} `who, which' and the \isi{ergative} is \emph{maf} (\Sg) and \emph{mafa} (\Nsg). The two exceptions are the first \isi{person} and the \isi{indefinite} \isi{pronoun}. The first \isi{person} \isi{non-singular} is \emph{ni}, and it neutralises the distinction between absolutive and \isi{ergative}. The \isi{indefinite} \isi{pronoun} is \emph{nä bun}, and it takes regular \isi{case} enclitics just like nouns. Therefore, \emph{nä bun} is analysed as being zero marked and thus absolutive.

Figure \ref{incluscons} shows that the core phrase always encodes \isi{non-singular} \isi{number}. As we have seen, this holds true for cases where there are only two participants and consequently the two subsets in the core phrase and the \isi{associative} phrase refer to a single individual respectively. The examples below show this for an ergative-marked argument, \emph{amayé nanyr} `mother with big sister' (\ref{ex605}), an absolutive-marked argument, \emph{emothé bnrr} `girl with you' (\ref{ex727}), and a dative-marked argument, \emph{sraknm nafrr} `boy with him' (\ref{ex728}). In contrasting examples without the \isi{inclusory} construction, all of these would receive a \isi{singular} marker of the respective cases. Note that the \isi{non-singular} \isi{absolutive} \emph{=é} in (\ref{ex727}) is the same as the \isi{non-singular} \isi{ergative} \emph{=é} in (\ref{ex605}). This syncretism is also found in the personal pronouns, where \emph{ni} is both first \isi{person} \isi{non-singular} \isi{absolutive} and \isi{ergative} ({\S}\ref{personalpronouns-sec}). The \isi{absolutive} \isi{singular} is always zero-marked, and the \isi{non-singular} formative \emph{=é} is optional ({\S}\ref{abscase}). In the \isi{inclusory} construction, however, \isi{non-singular} \isi{number} is obligatorily encoded on the core phrase.

\begin{exe}
	\ex \emph{mni ŋagarnth amayé nanyr.}\\
	\gll mni ŋa\stem{gar}nth ama=é nane=r\\
	firewood \Stdu:\Sbj:\Nonpast:\Ipfv/break mother=\Erg.{\Nsg} {elder.sibling}=\Assoc.\Du\\
	\trans `Mother together with big sister split firewood.' (lit. `Mother with big sister, they split firewood.')\Corpus{tci20150919-05}{LNA \#140}
	\label{ex605}
\end{exe}
\begin{exe}
	\ex \emph{kabef emothé emarn bnrr.}\\
	\gll kabe=f emoth=é e\stem{mar}n bnrr\\
	man=\Erg.{\Sg} girl=\Abs.{\Nsg} \Stsg:\Sbj>\Stdu:\Obj:\Nonpast:\Ipfv/see \Second.\Du.\Assoc\\
	\trans `The man sees the girl together with you.' (lit. `The man sees them, the girl with you.')
	\label{ex727}
\end{exe}

\newpage 
\begin{exe}
	\ex \emph{ŋafyf sraknm dunzi ärin nafrr.}\\
	\gll ŋafe=f srak=nm dunzi ä\stem{ri}n nafrr\\
	father=\Erg.{\Sg} boy=\Dat.{\Nsg} arrow \Stsg:\Sbj>\Stdu:\Io:\Nonpast:\Ipfv/give \Third.\Du.\Assoc\\
	\trans `The father gives the arrow to the boy together with him.' (lit. `Father gives them the arrow, the boy with him.')
	\label{ex728}
\end{exe}

If the total set indexed in the verb is two, then it follows that the two phrases can only refer to a single individual, even though the core phrase has to be marked for \isi{non-singular}, as in (\ref{ex725}) and (\ref{ex605}\textendash\ref{ex728}). If the total set indexed in the verb is \isi{plural}, it is unclear whether both subsets are bigger than one or whether one of them is singular and if so, which one. Example (\ref{ex340}) above is unambiguous because the \isi{associative} phrase is expressed by a personal name (\emph{Moses}=\Assoc.\Pl). If the \isi{associative} phrase it expressed by a noun or \isi{pronoun}, we are left with contextual clues. In example (\ref{ex339}), the speaker talks about marriage customs explaining that his clan will not exchange sisters with those clans with which they share a land boundary. In this example, \emph{nafä} has to be translated as a \isi{plural} `with them'.

\begin{exe}
	\ex \emph{ni nafäwä bad wkurwre ... fi neba erä ... ni neba}\\
	\gll ni nafä=wä bad w\stem{kur}wre (.) fi neba e\stem{rä} (.) ni neba\\
	{\Fnsg} \Third\Pl.\Assoc={\Emph} ground \Fpl:\Sbj>\Tsg.\F:\Nonpast:\Ipfv/split (.) \Third.{\Abs} opposite \Stpl:\Sbj:\Nonpast:\Ipfv/be (.) \First{} opposite\\
	\trans `We really share a land boundary with them. They are there and we (are) here.' (lit. `we cut the ground with them.')\Corpus{tci20120814}{ABB \#307}
	\label{ex339}
\end{exe}

In contrast, in example (\ref{ex343}) \emph{nafä} refers to a singular `with him'. This example is taken from a text about grief, and the speaker justifies a particular mourning custom by pointing out that he and his family have shared a lifetime with the deceased person.

\begin{exe}
	\ex \emph{... bänema ni nafä kwamränzrme. ni nafä nzwamnzrm.}\\
	\gll (.) bäne=ma ni nafä kwa\stem{mrä}nzrme ni nafä nzwa\stem{m}nzrm\\
	(.) \Recog={\Char} {\Fnsg} \Third\Pl.{\Assoc} \Fpl:\Sbj:\Pst:\Dur/stroll {\Fnsg} \Third\Pl.{\Assoc} \Fpl:\Sbj:\Pst:\Dur/dwell\\
	\trans `... because we walked around with him. We lived with him.'\\\Corpus{tci20120805-01}{ABB \#830-831}
	\label{ex343}
\end{exe}

It follows that out of context the \isi{pronoun} \emph{nafä} can refer to an individual or to a group of people in (\ref{ex339}) and (\ref{ex343}). This is also true for the \isi{pronoun} \emph{ni} (\Fnsg) in both examples. I pointed out above that the core phrase is always \isi{non-singular}, even if the subset expressed by the core phrase is \isi{singular}. Hence, the \isi{pronoun} \emph{ni} can refer to an individual or a group of people, and out of context example (\ref{ex339}) can be translated as `I share land with them', `We share land with him' or `We share land with them'. What it cannot mean is `I share land with him'. For this meaning, the verb would have to index a \isi{dual} and the \isi{associative} phrase would have to be marked for \isi{dual} \isi{number}.\footnote{The inclusory construction can be seen as a syntactic equivalent to distributed exponence in the verb morphology ({\S}\ref{verbprelim}).}

In the following discussion, I want to address the question whether or not the \isi{associative} phrase and the core phrase form a constituent. From a semantic perspective, we can answer this question in the affirmative, but we can also find some structural evidence that the \isi{associative} phrase and the core phrase form a functional unit. I have shown above that the \isi{associative} phrase agrees with the verb in \isi{number}. The core phrase, on the other hand, agrees with the verb in \isi{person} and \isi{number}. The \isi{number} category is very telling because it is always \isi{non-singular}. Additionally, the core phrase is assigned the appropriate \isi{case} marker by the argument structure of the \isi{verb}. I take these points as structural evidence that the \isi{associative} phrase and the core phrase form a functional unit. However, they do not constitute a phrase. In other words, the \isi{associative} case in the \isi{inclusory} construction does not function in the way that adnominal case does. For example, the characteristic case signals that one \isi{noun phrase} is embedded into a matrix \isi{noun phrase}. There is a fixed structure for embedding, and scrambling of elements which belong to the matrix phrase is not possible in Komnzo ({\S}\ref{npstructure}). There may be several instantiations of an argument in a clause, but these \isi{noun phrase}s are always marked for the same case. As we have seen above, the \isi{associative} phrase can be moved independently of the core phrase. Moreover, most corpus examples lack a core phrase altogether. In conclusion, the \isi{inclusory} construction is different from adnominal \isi{case}, like the \isi{characteristic} or \isi{possessive} case. The core phrase and the \isi{associative} phrase are not integrated into a matrix phrase.

The \isi{inclusory} construction also differs from coordinative constructions ({\S}\ref{clausecoordination}). Example (\ref{ex726}) shows the same state-of-affairs as expressed in (\ref{ex725}) above, but using a conjunctive \isi{coordination}. The main structural differences are that in \isi{coordination}: (i) a \isi{conjunction} like \emph{a} `and' is required, (ii) the coordinated \isi{noun phrase}s have to precede and follow the \isi{conjunction}, (iii) both \isi{noun phrase}s receive the same \isi{case} marker, (iv) the \isi{case} marker can be \isi{singular}. Note that in (\ref{ex725}) above the \isi{associative} phrase \emph{Kowir} could occur in all other positions. Nevertheless, the most natural positions are either after the verb or right after \emph{Maureené}.

\begin{exe}
	\ex \emph{Maureenf a Kowif bi ynäbünth.}\\
	\gll Maureen=f a Kowi=f bi y\stem{näbü}nth\\
	maureen=\Erg.{\Sg} and kowi=\Erg.{\Sg} sago(\Abs) \Stdu:\Sbj>\Tsg.\Masc:\Obj:\Nonpast:\Ipfv/beat\\
	\trans `Maureen and Kowi beat sago.'
	\label{ex726}
\end{exe}

Furthermore, the elements in an \isi{inclusory} construction can be coordinated, as in example (\ref{ex342}), where the two \isi{associative} phrases \emph{nä oromanr} `with another old man' and \emph{nä kabe} `with another man' are part of a disjunctive \isi{coordination} connected by \emph{o} `or'.

\begin{exe}
	\ex \emph{nä oromanr o nä kaber fi bämrn.}\\
	\gll nä {oroman=r} o nä kabe=r fi b=ä\stem{m}rn\\
	{\Indf} {old.man=\Assoc.\Du} or {\Indf} man=\Assoc.{\Du} \Third.{\Abs} \Med=\Stdu:\Sbj:\Nonpast:\Ipfv/sit\\
	\trans `He is sitting there with another old man or another man.' (lit. `...with some old man or with some man they two sit there.')\Corpus{tci20111004}{RMA \#343}
	\label{ex342}
\end{exe}

There is no clear semantic difference between \isi{coordination} and the \isi{inclusory} construction, but the difference seems to be pragmatic. While \isi{coordination} places the two elements on the same rank, the \isi{inclusory} construction may be used to highlight the referent expressed in the \isi{associative} phrase. This is supported by the fact that in most corpus examples, the core phrase is omitted, because its reference has been established earlier. Example (\ref{ex342}) above was uttered as the description of a set of picture cards. I reproduce the example in a longer context in (\ref{ex743}). The speaker talks about the protagonist of the story who is drinking with his friends. While describing the picture card, the speaker points out that the protagonist is sitting with another man. He then asks about the topic of their conversation. This other man is expressed in the \isi{associative} phrase. The same state of affairs could be expressed by a coordinative construction (`He and another man are sitting there'). The point is that the \isi{inclusory} construction can be used to introduce a new \isi{participant}, and thus has a pragmatic function. Note that the \isi{associative} phrase occurs in the first position of the clause.

\begin{exe}
	\ex \emph{ane fof yamnzr byé. wri kabenzo ... ane bramöwä ... fof ausi fäth nä berä ... ttrikasi ŋatrikwrth ... nä oromanr o nä kaber fi bämrn ... skiski warfo. monme fi yatrikwr ... nafan?}\\
	\gll ane fof ya\stem{m}nzr b=\stem{yé} wri kabe=nzo (.) ane bramöwä (.) fof ausi fäth nä b=e\stem{rä} (.) t-trik-si ŋa\stem{trik}wrth (.) nä {oroman=r} o nä kabe=r fi b=ä\stem{m}rn (.) skiski warfo monme fi ya\stem{trik}wr (.) nafan\\
	{\Dem} {\Emph} \Tsg.\Masc:\Sbj:\Nonpast:\Ipfv/sit \Med=\Tsg.\Masc:\Sbj:\Nonpast:\Ipfv/be drunk man={\Only} (.) {\Dem} all (.) {\Emph} old.woman \Dim{} {\Indf} \Med=\Stpl:\Sbj:\Nonpast:\Ipfv/be (.) \Redup-tell-{\Nmlz} \Stpl:\Sbj:\Nonpast:\Ipfv/tell (.) {\Indf} {old.man=\Assoc.\Du} or {\Indf} man=\Assoc.{\Du} \Third.{\Abs} \Med=\Stdu:\Sbj:\Nonpast:\Ipfv/sit (.) platform on.top how but \Stsg:\Sbj>\Tsg.\Masc:\Io:\Nonpast:\Ipfv/tell (.) \Tsg.\Dat\\
	\trans `That is the one sitting there. (They are) drunkards ... all of them. There is some woman. They are telling stories. He is sitting there with another old man or another man ... on the platform. But what is he telling him?'\\\Corpus{tci20111004}{RMA\#340-345}
	\label{ex743}
\end{exe}

Lichtenberk suggests two parameters for a typology of \isi{inclusory} pronominals: ``(i) do the \isi{inclusory} \isi{pronominal} and the included NP together form a syntactic construction, a phrase, or not?; and (ii) is there or is there not an overt marker of the relation between the \isi{inclusory} \isi{pronominal} and the included NP?'' (\citeyear[3]{Lichtenberk:2000hr}). This sets up a fourfold possibility space.\footnote{The four possibilities are: 1. +syntactic construction +overt marker, 2. +syntactic construction -overt marker, 3. -syntactic construction +overt marker, 4. -syntactic construction -overt marker.} The second parameter is clear for Komnzo: the \isi{associative} \isi{case} is an overt marker of the \isi{inclusory} construction. With respect to the first parameter, I hope to have shown above that Komnzo does not give a neat answer to these questions. In terms of agreement, we may say that the two elements agree, but they agree in their own ways. In terms of \isi{noun phrase} syntax, it would be a rather aberrant \isi{noun phrase}. Therefore, I suggest that Lichtenberk's typology should be expanded. A more fine-grained reformulation of his first parameter could help capture what constitutes a `syntactic construction', for example verb agreement and phrase structure. Singer's typology (\citeyear{Singer:inclu}) concentrates on the locus of the encoding of the whole set. She draws a distinction between Type 1, in which the set of total participants is represented by an independent \isi{pronoun}, and Type 2, in which it is represented by a verbal affix. Komnzo clearly belongs to the Type 2 category. But we can make a case for Komnzo also belonging to Type 1 because the \isi{associative} phrase, which can be a \isi{pronoun}, encodes the number of the total set.

Lichtenberk argues that the marker of \isi{inclusory} constructions is often historically related to the coordinative \isi{conjunction} or to the \isi{comitative} case, but he adds that the \isi{inclusory} construction differs from both.\footnote{``In explicit inclusory constructions, the marker of the relation between the inclusory pronominal and the included NP is typically etymologically related either to the coordinate conjunction `and' or to the comitative marker in the language.'' (\citealt[4]{Lichtenberk:2000hr}) and ``The phrasal inclusory construction is neither coordinating nor comitative; it is a construction \emph{sui generis}.'' (\citeyear[30, emphasis in original]{Lichtenberk:2000hr})} We have seen in \S\ref{comcase} that there is no \isi{inclusory} construction and no \isi{number} distinction with inanimates, and only \emph{=ä} is attached as a case marker. With inanimates, \emph{=ä} can be analysed as \isi{comitative} \isi{case}. On the other hand, the function of \emph{=r} (\Du) and \emph{=ä} (\Pl) with animates is an \isi{inclusory} function, which differs markedly from the \isi{associative} with inanimates. I follow Lichtenberk by analysing \emph{=r} and \emph{=ä} as markers of a distinct \isi{inclusory} construction, but for practical purposes I retain the label {\Assoc} in the \isi{gloss} instead of introducing a separate label for the \isi{inclusory} category.