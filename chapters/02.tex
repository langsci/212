% !TEX root =  ../main.tex

\chapter{Phonology} \label{cha:Phonology}

In this chapter, I describe the phonological system of Komnzo. The chapter begins with the segmental \isi{phonology} of consonants in {\S}\ref{consonant-segments} and vowels in {\S}\ref{vowelsegments}. Each section contains a list of \isi{minimal pair}s which establish the phonemic status of the segments. As Komnzo \isi{phonology} is characterised by widespread \isi{epenthesis}, a discussion of the non-phonemic status of \isi{schwa} is given in {\S}\ref{schwa-as-non-phoneme}. Regular phonological processes are described in {\S}\ref{regular-phon-processes}. I address phonotactics in {\S}\ref{syllable-and-phonotactics}. This section consists of a description of the \isi{syllable} structure ({\S}\ref{syllstruc}), \isi{consonant clusters} ({\S}\ref{consonantclusters}), \isi{syllabification} ({\S}\ref{syllabificationandepenthesis}), \isi{minimal word} constraints ({\S}\ref{minwordconstraints}) and \isi{stress} ({\S}\ref{stress}). Morphophonology is addressed in {\S}\ref{morphophonology}. The chapter closes with a discussion of loanwords in {\S}\ref{loanword-phonology} and an account of the development of the \isi{orthography} in {\S}\ref{orthographydev}.
\vspace{-0,3cm}

\section{Consonant phonemes} \label{consonant-segments}

\tabref{consinv} gives an overview of the consonant phonemes in Komnzo. Graphemes are given in <angle brackets>.

\begin{table}
\caption[Consonant phoneme inventory]{Consonant phoneme inventory}
\label{consinv}
\fittable{
  \begin{tabular}{lccccccc}
	\lsptoprule
	& {bilabial}& {dental} & {alveolar} & {palato-alveolar}	& {palatal} & {velar} & {labio-velar} \\ \midrule
	{stop}/{affricate}&& \multicolumn{2}{c}{t̪$\sim$t}& ts&& k & kʷ\\
	&& \multicolumn{2}{c}{\footnotesize{<t>}}& \footnotesize{<z>}&& \footnotesize{<k>} & \footnotesize{<kw>}\\
	\tablevspace
	{prenasalized } & ᵐb && ⁿd & ⁿdz && {\ᵑ}g & {\ᵑ}gʷ\\
	{stop/affricate}&\footnotesize{<b>} && \footnotesize{<d>} & \footnotesize{<nz>} && \footnotesize{<g>} & \footnotesize{<gw>}\\\\
	\tablevspace
	{fricative} & ɸ & ð & s &&&&\\
	& \footnotesize{<f>} & \footnotesize{<th>} & \footnotesize{<s>} &&&&\\
	\tablevspace
	{nasal} & m && n &&& ŋ &\\
	& \footnotesize{<m>} && \footnotesize{<n>} &&& \footnotesize{<ŋ>} &\\
	\tablevspace
	{lateral} &&& r$\sim$ɾ &&&&\\
	&&& \footnotesize{<r>} &&&&\\
	\tablevspace
	{semivowel} &&&&& j && w\\
	&&&&& \footnotesize{<y>} && \footnotesize{<w>}\\
	\lspbottomrule
  \end{tabular}
  }
\end{table} %phoneme inventory - consonants

\subsection{Obstruents} \label{obstruents}

Obstruents in Komnzo are divided into stops, affricates, and \isi{fricatives}. The stops and affricates belong to a chain of pairings of oral and prenasalised phonemes at four places of articulation: \isi{alveolar}, palato-alvealor, \isi{velar}, and labio-\isi{velar}. This symmetry is broken at the bilabial place of articulation. The bilabial oral stop is lacking from the phoneme inventory. Since it occurs only in \ili{English} loanwords and a handful of ideophones, I consider it a loan phoneme. As I will show below, the bilabial fricative /ɸ/ can be regarded as the structural counterpart of the prenasalised bilabial stop.

In the following section, I describe the oral and prenasalised stops, labialised \isi{velar} stops, affricates and \isi{fricatives}.

\subsubsection{Stops} \label{stopss}

There are two voiceless stops (/t/ and /k/) and three prenasalised stops (/{ᵐ}b/, /{ⁿ}d/, and /{\ᵑ}g/). The voiceless stops are phonetically slightly aspirated, but aspiration is not phonemic in Komnzo. The two labialised \isi{velar} stops and the two affricates follow the same pairing of voiceless and prenasalised manner of articulation, but these will be discussed in separate sections below.

All stops occur in word-initial, medial, and final positions. In only a small number of lexical items, the bilabial /{ᵐ}b/ occurs word-finally. This phoneme is also deviant as it lacks a voiceless counterpart. There is evidence from \isi{loanword} \isi{phonology} ({\S}\ref{loanword-phonology}) and from surrounding \ili{Tonda} languages that the bilabial fricative /ɸ/ occupies the same structural slot in the opposition of voiceless and prenasalised stops.

There is almost no allophonic variation within the stop series, but the prenasalised stops undergo final \isi{devoicing} ({\S}\ref{final-devoicing-section}). The /t/ phoneme varies between dental and \isi{alveolar} points of articulation. In onset clusters where C\textsubscript{2} is /r/, /t/ is always \isi{alveolar}. Elsewhere, it varies more or less freely.

\begin{figure}[H]
  $\mbox{/t/ $\rightarrow$} \left\{
    \begin{array}{l}
      \parbox{4cm}{[t] / \textsubscript{$\sigma$}[\_ɾ} \parbox{2cm}{\emph{traksi}} \parbox{3cm}{[tɾakə̆si]} \parbox{3cm}{`fall'} \\
      \parbox{4cm}{\hfill} \parbox{2cm}{\hfill} \parbox{3cm}{\hfill} \parbox{3cm}{\hfill} \\
      \parbox{4cm}{\hfill} \parbox{2cm}{\emph{tüf}} \parbox{3cm}{[tʏɸ] $\sim$ [t̪ʏɸ]} \parbox{3cm}{`soft ground'} \\
	  \parbox{4cm}{[t]$\sim$[t̪] / elsewhere} \parbox{2cm}{\emph{rata}} \parbox{3cm}{[ɾata] $\sim$ [ɾat̪a]} \parbox{3cm}{`ladder'} \\
	  \parbox{4cm}{\hfill} \parbox{2cm}{\emph{kwot}} \parbox{3cm}{[kʷɔ̆t] $\sim$ [kʷɔ̆t̪]} \parbox{3cm}{`properly'} \\
    \end{array}
  \right.$
\end{figure}%t
\begin{figure}[H]
  $\mbox{/k/ $\rightarrow$} \left\{
    \begin{array}{l}
      \parbox{3,95cm}{\hfill} \parbox{2cm}{\emph{kata}} \parbox{3cm}{[kata]} \parbox{3cm}{`bamboo knife'} \\
	  \parbox{3,95cm}{[k]} \parbox{2cm}{\emph{fokam}} \parbox{3cm}{[ɸokam]} \parbox{3cm}{`grave'} \\
	  \parbox{3,95cm}{\hfill} \parbox{2cm}{\emph{safak}} \parbox{3cm}{[saβak]} \parbox{2,5cm}{`saratoga'} \\
    \end{array}
  \right.$
\end{figure}%k
\begin{figure}[H]
  $\mbox{/{ᵐ}b/ $\rightarrow$} \left\{
    \begin{array}{l}
	  \parbox{3,75cm}{[ᵐp] / \_]\textsubscript{$\sigma$}} \parbox{2cm}{\emph{gb}} \parbox{3cm}{[{\ᵑ}gə̆ᵐp]} \parbox{3cm}{`black palm'} \\
      \parbox{4cm}{\hfill} \parbox{2cm}{\hfill} \parbox{3cm}{\hfill} \parbox{3cm}{\hfill} \\
      \parbox{3,75cm}{[ᵐb] / elsewhere} \parbox{2cm}{\emph{bone}} \parbox{3cm}{[ᵐbone]} \parbox{3cm}{\Ssg{}.{\Poss}} \\
	  \parbox{3,75cm}{\hfill} \parbox{2cm}{\emph{gaba}} \parbox{3cm}{[{\ᵑ}gaᵐba]} \parbox{3cm}{`storage yam'} \\
    \end{array}
  \right.$
\end{figure}%b
\begin{figure}[H]
  $\mbox{/{ⁿ}d/ $\rightarrow$} \left\{
    \begin{array}{l}
	  \parbox{3,76cm}{[ⁿt] / \_]\textsubscript{$\sigma$}} \parbox{2cm}{\emph{kd}} \parbox{3cm}{[kə̆ⁿt]} 	\parbox{3cm}{`star'} \\
      \parbox{3,76cm}{\hfill} \parbox{2cm}{\hfill} \parbox{3cm}{\hfill} \parbox{3cm}{\hfill} \\
      \parbox{3,76cm}{[ⁿd] / elsewhere} \parbox{2cm}{\emph{deya}} \parbox{3cm}{[ⁿdeja]} \parbox{3cm}{`tree wallaby'} \\
	  \parbox{3,76cm}{\hfill} \parbox{2cm}{\emph{rdiknsi}} \parbox{3cm}{[ɾə̆ⁿdikə̆nsi]} \parbox{3cm}{`tie around'} \\
    \end{array}
  \right.$
\end{figure}%d
\begin{figure}[H]
  $\mbox{/{\ᵑ}g/ $\rightarrow$} \left\{
    \begin{array}{l}
	  \parbox{3,76cm}{[{\ᵑ}k] / \_]\textsubscript{$\sigma$}} \parbox{2cm}{\emph{nag}} \parbox{3cm}{[na{\ᵑ}k]} \parbox{3cm}{`grass skirt'} \\
      \parbox{3,76cm}{\hfill} \parbox{2cm}{\hfill} \parbox{3cm}{\hfill} \parbox{3cm}{\hfill} \\
      \parbox{3,76cm}{[{\ᵑ}g] / elsewhere} \parbox{2cm}{\emph{gau}} \parbox{3cm}{[{\ᵑ}ga͡u]} \parbox{4cm}{`night heron'} \\
	  \parbox{3,76cm}{\hfill} \parbox{2cm}{\emph{sagara}} \parbox{3cm}{[sa{\ᵑ}gara]} \parbox{3cm}{proper name} \\
    \end{array}
  \right.$
\end{figure}%g

\subsubsection{Labialised velar stops} \label{labvelarstops}

The labialised \isi{velar} stops /kʷ/ and /{\ᵑ}gʷ/ show no allophonic variation due to their restricted distribution. Both occur only in \isi{syllable} onsets, not in the coda. Consequently, we do not find these phonemes in word-final position.\footnote{In the neighbouring language \ili{Nama} which belongs to the \ili{Nambu} subgroup, labialised velar stops may occur in coda position, as in [aukʷ] `morning'.}

\begin{figure}[H]
  $\mbox{/kʷ/ $\rightarrow$} \left\{
    \begin{array}{l}
	  \parbox{3,8cm}{[kʷ] / \textsubscript{$\sigma$}[\_} \parbox{2cm}{\emph{kwan}} \parbox{3cm}{[kʷan]} \parbox{3cm}{`shout, voice'} \\
	  \parbox{3,8cm}{\hfill} \parbox{2cm}{\emph{ysokwr}} \parbox{3cm}{[jə̆sokʷə̆ɾ]} \parbox{3cm}{`rainy season'} \\
    \end{array}
  \right.$
\end{figure}%kw
\begin{figure}[H]
  $\mbox{/{\ᵑ}gʷ/ $\rightarrow$} \left\{
    \begin{array}{l}
	  \parbox{3,7cm}{[{\ᵑ}gʷ] / \textsubscript{$\sigma$}[\_}	\parbox{2cm}{\emph{gwä}} \parbox{3cm}{[{\ᵑ}gʷæ]} \parbox{3cm}{`mosquito'} \\
	  \parbox{3,7cm}{\hfill} \parbox{2cm}{\emph{fagwa}} \parbox{3cm}{[ɸa{\ᵑ}gʷa]} \parbox{3cm}{`width'} \\
    \end{array}
  \right.$
\end{figure}%gw

I argue in favour of an analysis whereby the labialised \isi{velar} stops are complex phonemes rather than a sequence of two phonemes (\isi{velar} stop + high back vowel /u/ or \isi{velar} stop + /w/). This argument is based on two lines of evidence: onset \isi{consonant clusters} and \isi{reduplication} patterns.

Onset clusters are restricted to two consonants (C\textsubscript{1}C\textsubscript{2}V). If clusters occur, C\textsubscript{2} may only be /r/ or /w/ ({\S}\ref{syllabificationandepenthesis}). For this argument, only the /r/ is relevant. We do find words with an initial labialised \isi{velar} stop (voiceless or prenasalised) in such a cluster, for example: \emph{kwras} `Brolga' or \emph{gwra} `MacCulloch's Rainbowfish'. If /kʷ/ and /{\ᵑ}gʷ/ were to be analysed as clusters of two phonemes, a separate \isi{syllable} template (CCCV) would be required.

The second piece of evidence comes from reduplication. We find full and partial \isi{reduplication} ({\S}\ref{nomreduplication}). In full \isi{reduplication} the entire word is repeated, as in \emph{yam} `footprint, custom, event' $\rightarrow$ \emph{yamyam} `little feast'. Partial \isi{reduplication} is more frequent, where only the first consonant of the initial \isi{syllable} is copied, as in  \emph{zbär} `night' $\rightarrow$ \emph{zzbär} [tsə̆tsə̆ᵐbæɾ] `dusk, twilight'. The domain of partial \isi{reduplication} does not extend further than the first consonant. Thus, we get \emph{frasi} `hunger' $\rightarrow$ \emph{ffrasi} [ɸə̆ɸɾasi] `appetite, hunger', but not \textsuperscript{$\ast$}\emph{frfrasi} [ɸɾə̆ɸɾasi]. If the labialised \isi{velar} stops comprised two separate phonemes, we would expect that in partial \isi{reduplication} only the \isi{velar} stop is copied without the \isi{semivowel}. On the contrary, we find that the whole phoneme is copied, as in \emph{kwayan} `light' $\rightarrow$ \emph{kwkwayan} [kʷə̆kʷajan] $\sim$ [kukʷajan] `flickering light, dimmed light', but not \textsuperscript{$\ast$}\emph{kkwayan} [kə̆kʷajan].

\subsubsection{Affricates} \label{affricates}

The two consonant phonemes with the highest frequency are the affricates  /ts/ and /{ⁿ}dz/, which seem to give Komnzo its characteristic fricative sound. Both affricates occur initially, medially and finally, and show some allophonic variation. They are palatalised before front vowels, as in \emph{zi} [tʃı:] `pain' and \emph{nzikaka} [ⁿdʒıkaka] `Whistling Kite'. In all other environments they are \isi{alveolar}. There is some degree of variation between speakers. Some speakers always palatalise, while most speakers follow the allophonic rules as formalised below. The prenasalised \isi{affricate} is affected by final \isi{devoicing} ({\S}\ref{final-devoicing-section}).

\begin{figure}[H]
  $\mbox{/ts/ $\rightarrow$} \left\{
    \begin{array}{l}
      \parbox{4,2cm}{[tʃ] / \_V\textsubscript{\textsc{+front}}} \parbox{2,5cm}{\emph{zena}} \parbox{2,5cm}{[tʃena]} \parbox{3cm}{`now'} \\
      \parbox{4,2cm}{\hfill} \parbox{2,5cm}{\emph{ezi}} \parbox{2,5cm}{[ʔetʃi]} \parbox{3cm}{`morning'} \\
      \parbox{4,2cm}{\hfill} \parbox{2,5cm}{\hfill} \parbox{2,5cm}{\hfill} \parbox{3cm}{\hfill} \\
      \parbox{4,2cm}{[ts] / elsewhere} \parbox{2,5cm}{\emph{zane}} \parbox{2,5cm}{[tsane]} \parbox{3cm}{\Dem:{\Prox}} \\
      \parbox{4,2cm}{\hfill} \parbox{2,5cm}{\emph{mazo}} \parbox{2,5cm}{[matso]} \parbox{3cm}{`ocean'} \\
      \parbox{4,2cm}{\hfill} \parbox{2,5cm}{\emph{müz}} \parbox{2,5cm}{[mʏts]} \parbox{3cm}{`phallocrypt'} \\
    \end{array}
  \right.$
\end{figure}%z
\begin{figure}[H]
  $\mbox{/{ⁿ}dz/ $\rightarrow$} \left\{
    \begin{array}{l}
      \parbox{4cm}{[ⁿdʒ] / \_V\textsubscript{\textsc{+front}}} \parbox{2,5cm}{\emph{nzigfu}} \parbox{2,5cm}{[ⁿdʒi{\ᵑ}gɸu]} \parbox{2,5cm}{`rain stone'} \\
      \parbox{4cm}{\hfill} \parbox{2,5cm}{\emph{snzä}} \parbox{2,5cm}{[sə̆ⁿdʒæ]} \parbox{2,5cm}{`crayfish'} \\
      \parbox{4cm}{\hfill} \parbox{2,5cm}{\hfill} \parbox{2,5cm}{\hfill} \parbox{2,5cm}{\hfill} \\
      \parbox{4cm}{{[ⁿts] / \_]\textsubscript{$\sigma$}}} \parbox{2,5cm}{\emph{mnz}} \parbox{2,5cm}{[mə̆ⁿts]} \parbox{2,5cm}{`house'} \\
      \parbox{4cm}{\hfill} \parbox{2,5cm}{\hfill} \parbox{2,5cm}{\hfill} \parbox{2,5cm}{\hfill} \\
      \parbox{4cm}{[ⁿdz] / elsewhere} \parbox{2,5cm}{\emph{nzun}} \parbox{2,5cm}{[ⁿdzun]} \parbox{2,5cm}{\Fsg.{\Dat}} \\
      \parbox{4cm}{\hfill} \parbox{2,5cm}{\emph{rnzam}} \parbox{2,5cm}{[rə̆ⁿdzam]} \parbox{2,5cm}{`how many'} \\
    \end{array}
  \right.$
\end{figure}%nz

\subsubsection{Fricatives} \label{fricatives}

There are three \isi{fricatives} at the bilabial, dental and \isi{alveolar} places of articulation. The dental fricative is voiced, while the other two are voiceless. The bilabial fricative has a voiced allophone, which occurs intervocalically. Although voiced in most environments, the dental fricative is affected by final \isi{devoicing} ({\S}\ref{final-devoicing-section}). The \isi{alveolar} fricative is always voiceless in all environments. These rules are formalised below.

\begin{figure}[H]
  $\mbox{/ɸ/ $\rightarrow$} \left\{
    \begin{array}{l}
      \parbox{3,7cm}{[β] / V\_V} \parbox{2,5cm}{\emph{zafazafa}} \parbox{2,5cm}{[tsaβatsaβa]} \parbox{3cm}{`vine stick'} \\
      \parbox{3,7cm}{\hfill} \parbox{2,5cm}{\hfill} \parbox{2,5cm}{\hfill} \parbox{3cm}{\hfill} \\
      \parbox{3,7cm}{[ɸ] / elsewhere} \parbox{2,5cm}{\emph{fid}} \parbox{2,5cm}{[ɸıⁿt]} \parbox{3cm}{`bushrope'} \\
	  \parbox{3,7cm}{\hfill} \parbox{2,5cm}{\emph{zarfa}} \parbox{2,5cm}{[tsaɾɸa]} \parbox{3cm}{`ear'} \\
	  \parbox{3,7cm}{\hfill} \parbox{2,5cm}{\emph{karaf}} \parbox{2,5cm}{[kaɾaɸ]} \parbox{3cm}{`paddle'} \\
    \end{array}
  \right.$
\end{figure}%ɸ
\begin{figure}[H]
  $\mbox{/ð/ $\rightarrow$} \left\{
    \begin{array}{l}
	  \parbox{3,72cm}{[θ] / \_]\textsubscript{$\sigma$}} \parbox{2,5cm}{\emph{süsübäth}} \parbox{2,5cm}{[sʏsʏᵐbæθ]} \parbox{3cm}{`darkness'} \\
      \parbox{3,72cm}{\hfill} \parbox{2,5cm}{\hfill} \parbox{2,5cm}{\hfill} \parbox{3cm}{\hfill} \\
      \parbox{3,72cm}{[ð] / elsewhere} \parbox{2,5cm}{\emph{thamin}} \parbox{2,5cm}{[ðamin]} \parbox{3cm}{`tongue'} \\
	  \parbox{3,72cm}{\hfill} \parbox{2,5cm}{\emph{ŋatha}} \parbox{2,5cm}{[ŋaða]} \parbox{3cm}{`dog'} \\
    \end{array}
  \right.$
\end{figure}%ðɸ
\begin{figure}[H]
  $\mbox{/s/ $\rightarrow$} \left\{
    \begin{array}{l}
      \parbox{3,8cm}{\hfill} \parbox{2,5cm}{\emph{saisai}} \parbox{2,5cm}{[sa͡isa͡i]} \parbox{3cm}{`drizzle (n)'} \\
	  \parbox{3,8cm}{[s]} \parbox{2,5cm}{\emph{fisor}} \parbox{2,5cm}{[ɸisoɾ]} \parbox{3cm}{`turtle'} \\
	  \parbox{3,8cm}{\hfill} \parbox{2,5cm}{\emph{fis}} \parbox{2,5cm}{[ɸis]} \parbox{3cm}{`husband'} \\
    \end{array}
  \right.$
\end{figure}%s

\subsection{Nasals} \label{nasals}

There are \isi{nasal} stops at three places of articulation: bilabial, \isi{alveolar}, and \isi{velar}. These three show differences in their frequency and distribution. The \isi{velar} \isi{nasal} /ŋ/ occurs only word-initially, while bilabial /m/ and \isi{alveolar} /n/ are found initially, medially and finally. There is no allophonic variation with the \isi{nasals}.

\begin{figure}[H]
  $\mbox{/m/ $\rightarrow$} \left\{
    \begin{array}{l}
      \parbox{2,5cm}{\hfill} \parbox{2,5cm}{\emph{mifum}} \parbox{2,5cm}{[miβum]} \parbox{3cm}{`nose ornament'} \\
	  \parbox{2,5cm}{[m]} \parbox{2,5cm}{\emph{zimu}} \parbox{2,5cm}{[tʃimu]} \parbox{3cm}{`snot'} \\
	  \parbox{2,5cm}{\hfill} \parbox{2,5cm}{\emph{thm}} \parbox{2,5cm}{[ðə̆m]} \parbox{3cm}{`nose'} \\
    \end{array}
  \right.$
\end{figure}%m
\begin{figure}[H]
  $\mbox{/n/ $\rightarrow$} \left\{
    \begin{array}{l}
      \parbox{2,6cm}{\hfill} \parbox{2,5cm}{\emph{no}} \parbox{2,5cm}{[no:]} \parbox{3cm}{`water, rain'} \\
	  \parbox{2,6cm}{[n]} \parbox{2,5cm}{\emph{mane}} \parbox{2,5cm}{[mane]} \parbox{3cm}{`who' (\Abs)} \\
	  \parbox{2,6cm}{\hfill} \parbox{2,5cm}{\emph{minmin}} \parbox{2,5cm}{[minmin]} \parbox{3cm}{`Emerald Dove'} \\
    \end{array}
  \right.$
\end{figure}%n
\begin{figure}[H]
  $\mbox{/ŋ/ $\rightarrow$} \left\{
    \begin{array}{l}
	  \parbox{2,7cm}{[ŋ] / \textsubscript{\textsc{word}}[\_} \parbox{2,5cm}{\emph{ŋazi}} \parbox{2,5cm}{[ŋatʃi]} 	\parbox{3cm}{`coconut'} \\
    \end{array}
  \right.$
\end{figure}%ŋ

\subsection{Trill, tap - /r/} \label{trilltap}

The \isi{alveolar} trill /r/ is often realised as a single tap [ɾ] depending on speech rate and speaker. In onset \isi{consonant clusters} where /r/ is occupying C\textsubscript{2} position, it is always tapped. Elsewhere, the trill and the tap are in free variation. Word-finally /r/ may also become voiceless. This variation between [ɾ] and [ɾ̥] seems to be conditioned by age. Older speakers use the voiceless variant more frequently.

\begin{figure}[H]
  $\mbox{/r/ $\rightarrow$} \left\{
    \begin{array}{l}
      \parbox{3,7cm}{[ɾ]$\sim$[ɾ̥] / \_]\textsubscript{\textsc{word}}} \parbox{2cm}{\emph{msar}} \parbox{3,5cm}{[mə̯saɾ] $\sim$ [mə̯saɾ̥]} \parbox{3cm}{`green ant'} \\
      \parbox{3,7cm}{\hfill} \parbox{2cm}{\hfill} \parbox{3,5cm}{\hfill} \parbox{3cm}{\hfill} \\
	  \parbox{3,7cm}{[ɾ] / \textsubscript{$\sigma$}[C\_} \parbox{2cm}{\emph{frasi}} \parbox{3,5cm}{[ɸɾasi]} \parbox{3cm}{`hunger'} \\
	  \parbox{3,7cm}{\hfill} \parbox{2cm}{\hfill} \parbox{3,5cm}{\hfill} \parbox{3cm}{\hfill} \\
      \parbox{3,7cm}{[r]$\sim$[ɾ] / elsewhere} \parbox{2cm}{\emph{rnz}} \parbox{3,5cm}{[rə̯ⁿts] $\sim$ [ɾə̯ⁿts]} \parbox{3cm}{`ember'} \\
	  \parbox{3,7cm}{\hfill} \parbox{2cm}{\emph{ŋare}} \parbox{3,5cm}{[ŋare] $\sim$ [ŋaɾe]} \parbox{3cm}{`woman'} \\
    \end{array}
  \right.$
\end{figure}%r

\subsection{Approximants} \label{approximants}

The two approximants /w/ and /j/ occur in initial, medial and final position. In final position, they may be realised as a short offglide or become part of a \isi{diphthong}. For both approximants, but especially for the palatal /j/, we find only a handful of lexical items where they do occur word-finally.

\begin{figure}[H]
  $\mbox{/w/ $\rightarrow$} \left\{
    \begin{array}{l}
      \parbox{3,6cm}{[\_͡u]$\sim$[\_ʷ] / V\_]\textsubscript{$\sigma$}} \parbox{2cm}{\emph{daw}} \parbox{3cm}{[ⁿda͡u] $\sim$ [ⁿdaʷ]} \parbox{3cm}{`garden'} \\
      \parbox{3,6cm}{\hfill} \parbox{2cm}{\hfill} \parbox{3cm}{\hfill} \parbox{3cm}{\hfill} \\
	  \parbox{3,6cm}{[w] / elsewhere} \parbox{2cm}{\emph{wm}} \parbox{3cm}{[wə̆m]} \parbox{3cm}{`stone, gravel'} \\
	  \parbox{3,6cm}{\hfill} \parbox{2cm}{\emph{fewa}} \parbox{3cm}{[ɸewa]} \parbox{3cm}{`odour, stench'} \\
    \end{array}
  \right.$
\end{figure}%w
\begin{figure}[H]
  $\mbox{/j/ $\rightarrow$} \left\{
    \begin{array}{l}
      \parbox{3,75cm}{[\_͡ı]$\sim$[\_\super{j}] / V\_]\textsubscript{$\sigma$}} \parbox{2cm}{\emph{fäy}} \parbox{3cm}{[ɸæ͡ı] $\sim$ [ɸæ\super{j}]} \parbox{3cm}{`payment'} \\
      \parbox{3,75cm}{\hfill} \parbox{2cm}{\hfill} \parbox{3cm}{\hfill} \parbox{3cm}{\hfill} \\
	  \parbox{3,75cm}{[j] / elsewhere} \parbox{2cm}{\emph{yusi}} \parbox{3cm}{[jusi]} \parbox{3cm}{`grass'} \\
	  \parbox{3,75cm}{\hfill} \parbox{2cm}{\emph{nzöyar}} \parbox{3cm}{[ⁿdʒœjaɾ]} \parbox{3cm}{`bowerbird'} \\
    \end{array}
  \right.$
\end{figure}%y

There are a number of reasons why the two approximants are analysed as consonants rather than high vowels which alternate according to their environment. Evidence comes from case allomorphy and phonotactics. In stem final position /w/ and /j/ select the same \isi{allomorph} of the \isi{locative} case as other consonants. This can be seen in the word \emph{daw} [ⁿda͡u] $\sim$ [ⁿdaʷ] `garden' which selects \emph{=en} as its \isi{locative} case marker, thus forming \emph{dawen} [ⁿdawen] `in the garden'. Words which end in a vowel select the \emph{=n} \isi{allomorph} of the \isi{locative} case. Furthermore, the rules of \isi{syllabification} ({\S}\ref{syllabificationandepenthesis}) treat these two phonemes like consonants. Thus, we find examples like \emph{ys} [jı̆s] `thorn' and \emph{ky} [kə̆\super{j}] `yam species', where \isi{epenthesis} occurs after and before /j/, respectively.

\subsection{Minimal pairs for Komnzo consonants} \label{minimalpairsconsonants}

The following \isi{minimal pair}s and near \isi{minimal pair}s in \tabref{minpaircon} illustrate the phonemic contrast between consonants in initial, medial and final position.

\begin{longtable}{lllll}
\caption{Minimal pairs of consonant phonemes}
% \begin{tabularx}{\textwidth}{lllll}
\label{minpaircon}\\
		\lsptoprule
		segments&examples&&&\\\midrule
		\endfirsthead
		segments&examples&&&\\\midrule
		\endhead
		/kʷ/ vs. /k/		&\emph{kwafar} place name&[kʷaβaɾ]&[kaβaɾ]&\emph{kafar} `big'\\
							&\emph{sakwr} `he hit him'&[sakʷə̆ɾ]&[sakə̆ɾ]&\emph{sakr} `mustard vine'\\
							&\emph{kwath} `crow'&[kʷaθ]&[kaθ]&\emph{kath} `ankle'\\
		/{\ᵑ}gʷ/ vs. /{\ᵑ}g/&\emph{gwra} `rainbowfish'&[{\ᵑ}gʷra:]&[{\ᵑ}gra:]&\emph{gra} `tree sp'\\
		/kʷ/ vs. /w/		&\emph{kwath} `crow'&[kʷaθ]&[waθ]&\emph{wath} `dance (n)'\\
							&\emph{kwf} `stone club'&[kʷə̆ɸ]&[wə̆ɸ]&\emph{wf} `shirt, blouse'\\
		/{\ᵑ}gʷ/ vs. /w/	&\emph{gwth} `nest'&[{\ᵑ}gwə̆θ]&[wə̆θ]&\emph{wth} `faeces'\\
		/k/ vs. /w/			&\emph{kath} `ankle'&[kaθ]&[waθ]&\emph{wath} `dance (n)'\\
		/ɸ/ vs. /w/			&\emph{far} `housepost'&[ɸaɾ]&[waɾ]&\emph{war} `top layer'\\
							&\emph{kafar} `big'&[kaβaɾ]&[kawaɾ]&\emph{kawar} pers. name\\
%							&\emph{zafe} `old'&[tsaβe]&[tsawe]&\emph{zawe} `right (side)'\\
							&\emph{tfitfi} `whirlwind'&[tə̆βitə̆βi]&[tə̆witə̆wi]&\emph{twitwi} `bird sp'\\
		/s/ vs. /t/			&\emph{süfr} `tree sp'&[sʏɸə̆r]&[tʏɸə̆r]&\emph{tüfr} `many'\\
							&\emph{kisr} `lizard sp'&[kisə̆r]&[kitə̆r]&\emph{kitr} `pandanus'\\
							&\emph{wsws} `grass sp'&[wə̆swə̆s]&[wə̆twə̆t]&\emph{wtwt} `itchy'\\
		/s/ vs. /ð/			&\emph{sirsir} `glider'&[sirsir]&[ðirðir]&\emph{thirthir} `pig tusk'\\
							&\emph{bis} `bird sp'&[ᵐbi:s]&[ᵐbi:θ]&\emph{bith} `honey bee'\\
							&\emph{mus} `leech'&[mu:s]&[mu:θ]&\emph{muth} `(sago) grub'\\
		/s/ vs. /ts/		&\emph{si} `eye'&[si:]&[tʃi:]&\emph{zi} `pain'\\
							&\emph{srminz} `rainbow'&[sə̆rmints]&[tsə̆rmints]&\emph{zrminz} `roots'\\
							&\emph{ksi kar} `savannah'&[kə̆si kar]&[kə̆tʃi]&\emph{kzi} `barktray'\\
							&\emph{fs} `fish sp'&[ɸə̆s]&[ɸə̆ts]&\emph{fz} `forest'\\
		/ð/ vs. /t/			&\emph{thruthru} `bamboo sp'&[ðruðru]&[trutru]&\emph{trutru} `stream'\\
							&\emph{füth} `rotten tuber'&[ɸʏθ]&[ɸʏt]&\emph{füt} `pouch'\\
		/ð/ vs. /r/			&\emph{thusi} `fold (v.t.)'&[ðusi]&[ɾusi]&\emph{rusi} `shoot (v.t.)'\\
							&\emph{bthan} `magic'&[ᵐbə̆ðan]&[ᵐbə̆ɾan]&\emph{bran} `line'\\
							&\emph{yathizsi} `die'&[jaðitsə̆si]&[jaɾitsə̆si]&\emph{yarizsi} `hear, listen'\\
							&\emph{zithzith} `slickness'&[tʃiθtʃiθ]&[tʃiɾtʃiɾ]&\emph{zirzir} `wetness'\\
							&\emph{wath} `dance (n)'&[waθ]&[waɾ]&\emph{war} `top layer'\\
		/r/ vs. /t/			&\emph{rar} `for what'&[ɾaɾ]&[taɾ]&\emph{tar} `friend'\\
							&\emph{ŋarr} `bandicoot'&[ŋaɾə̆ɾ]&[ŋatə̆ɾ]&\emph{ŋatr} `rope'\\
							&\emph{ft} `dead tree'&[ɸə̆t]&[ɸə̆ɾ]&\emph{fr} `palm stem'\\
		/r/ vs. /ts/		&\emph{rinaksi} `pour'&[ɾinakə̆si]&[tʃinakə̆si]&\emph{zinaksi} `put down'\\
							&\emph{wari} `plant sp'&[waɾi]&[watʃi]&\emph{wazi} `side'\\
							&\emph{mür} `grass sp'&[mʏɾ]&[mʏts]&\emph{müz} `phallocrypt'\\
		/{ᵐ}b/ vs. /m/		&\emph{bith} `honey bee'&[ᵐbiθ]&[miθ]&\emph{mith} `face'\\
							&\emph{bä} \Second.{\Abs}&[ᵐbæ:]&[mæ:]&\emph{mä} `where' \\
							&\emph{züb} `depth'&[tʃʏᵐb]&[tʃʏm]&\emph{züm} `centipede'\\
		/{ⁿ}d/ vs. /n/		&\emph{dasi} `bulge'&[ⁿdasi]&[nasi]&\emph{nasi} `long yam'\\
							&\emph{badabada} `ancestor'&[ᵐbaⁿdaᵐbaⁿda]&[ᵐbana]&\emph{bana} `pitiful'\\
							&\emph{kd} `star'&[kə̆nt]&[kə̆n]&\emph{kn} `yam sp'\\
		/{\ᵑ}g/ vs. /ŋ/		&\emph{gathagatha} `bad'&[{\ᵑ}gaða{\ᵑ}gaða]&[ŋaðaŋaða]&\emph{ŋathaŋatha} `quoll'\\
							&\emph{game} `tongs'&[{\ᵑ}game]&[ŋame]&\emph{ŋame} `mother'\\
		/m/ vs. /n/			&\emph{mä} `where'&[mæ:]&[næ:]&\emph{nä} `some'\\
							&\emph{mawan} `tree sp'&[mawan]&[nawan]&\emph{nawan} `waterhole'\\
		/{ⁿ}dz/ vs. /{ⁿ}d/	&\emph{nzga} `vagina'&[ⁿdzə̆{\ᵑ}ga]&[ⁿdə̆{\ᵑ}ga]&\emph{dga} `gills'\\
							&\emph{ŋanz} `planting row'&[ŋaⁿts]&[ŋaⁿt]&\emph{ŋad} `rope'\\
							&\emph{ymnz} place name&[jə̆mə̆ⁿts]&[jə̆mə̆ⁿt]&\emph{ymd} `bird'\\
		/{ⁿ}dz/ vs. /n/		&\emph{nzä} \Fsg.{\Abs}&[ⁿdʒæ:]&[næ:]&\emph{nä} `some'\\
							&\emph{gonz} `place name'&[{\ᵑ}gɔnts]&[{\ᵑ}gɔn]&\emph{gon} `hips'\\
		/{ᵐ}b/ vs. /ɸ/		&\emph{bä} \Second.{\Abs}&[ᵐbæ:]&[ɸæ:]&\emph{fä} {\Dist}\\
							&\emph{bira} `axe'&[ᵐbiɾa]&[ɸiɾa]&\emph{fira} `betelnut'\\
							&\emph{bis} `bird sp'&[ᵐbi:s]&[ɸi:s]&\emph{fis} `husband'\\
		/{ⁿ}d/ vs. /t/		&\emph{düfr} `headdress'&[ⁿdʏɸə̆ɾ]&[tʏɸə̆ɾ]&\emph{tüfr} `plenty'\\
							&\emph{drari} `container'&[ⁿdɾaɾi]&[tɾaɾi]&\emph{trari} `strong man'\\
							&\emph{kadakada} `yamcake'&[kaⁿdakaⁿda]&[katakata]&\emph{katakata} `grass sp'\\
							&\emph{sd} `yam sp'&[sə̆ⁿt]&[sə̆t]&\emph{st} `plant sp'\\
		/{ⁿ}dz/ vs. /ts/	&\emph{nzä} \Fsg.{\Abs}&[ⁿdʒæ:]&[tʃæ:]&\emph{zä} {\Prox}\\
							&\emph{nzanza} `insect sp'&[ⁿdzaⁿdza]&[tsatsa]&\emph{zaza} `carrying'\\
							&\emph{nzr} `leftover'&[ⁿdzə̆ɾ]&[tsə̆ɾ]&\emph{zr} `tooth'\\
							&\emph{rbänzsi} `prohibit'&[ɾə̆ᵐbæⁿdzə̆si]&[ɾə̆ᵐbætsə̆si]&\emph{rbäzsi} `untie'\\
		/{\ᵑ}g/ vs. /k/		&\emph{gd} `mud'&[{\ᵑ}gə̆ⁿt]&[kə̆ⁿt]&\emph{kd} `star'\\
							&\emph{gafar} `fish sp'&[{\ᵑ}gaβaɾ]&[kaβaɾ]&\emph{kafar} `big'\\
							&\emph{gursi} `break off'&[{\ᵑ}guɾsi]&[kuɾsi]&\emph{kursi} `split'\\
							&\emph{tag} `bee sp'&[ta{\ᵑ}k]&[tak]&\emph{tak} `pandanus'\\
							&\emph{srag} pers. name&[sɾa{\ᵑ}k]&[sɾak]&\emph{srak} `boy'\\
		/w/ vs. /j/			&\emph{warsi} `chew'&[waɾsi]&[jaɾsi]&\emph{yarsi} `tired'\\
							&\emph{wf} `shirt'&[wə̆ɸ]&[jə̆ɸ]&\emph{yf} `name'\\
							&\emph{fäw} `arrow shaft'&[ɸæ͡u]&[ɸæ͡i]&\emph{fäy} `payment'\\
		\lspbottomrule
% \end{tabularx}
\end{longtable}%minimal pairs - consonant phonemes

\section{Vowel phonemes} \label{vowelsegments}

The articulatory space for vowels can be divided into four levels of height (high, mid, mid-low, and low) and three levels of backness (front, central, and back). The absence versus presence of lip rounding is phonemic for front vowels. Figure \ref{vowelinvspace} provides an overview of the vowel space, while \tabref{vowelinv} lists the segmental features and shows the graphemes with < >. Note that I include the epenthetic \isi{schwa} in parentheses. This is because there is some evidence that \isi{schwa} constitutes a marginal phoneme in word-final environment. That being said, in all other contexts it is created by \isi{epenthesis} ({\S}\ref{schwa-as-non-phoneme}).

\begin{figure}
\centering{
	\begin{vowel}[simple]
		\putvowel{i$\sim$ı}{0,3\vowelhunit}{0,4\vowelvunit}
		\putvowel{y$\sim$ʏ}{1,3\vowelhunit}{0,4\vowelvunit}
		\putvowel{e}{0,85\vowelhunit}{1,5\vowelvunit}
		\putvowel{ø$\sim$œ}{1,7\vowelhunit}{1,5\vowelvunit}
		\putvowel{æ}{1,6\vowelhunit}{2,55\vowelvunit}
		\putvowel{u}{4\vowelhunit}{0,4\vowelvunit}
		\putvowel{ɐ$\sim$a}{2,9\vowelhunit}{2,7\vowelvunit}
		\putvowel{(ə)}{2,7\vowelhunit}{1,3\vowelvunit}
		\putvowel{ɔ}{4\vowelhunit}{1,5\vowelvunit}
	\end{vowel}}%
\caption{Komnzo vowel space}
\label{vowelinvspace}
\end{figure}%Vowel space

\begin{table}
\caption{Vowel phoneme inventory}
\label{vowelinv}
	\begin{tabularx}{.6\textwidth}{XCCCC}
		\lsptoprule
		&\multicolumn{2}{c}{front}&central&back \\
		&\footnotesize{unrounded}&\footnotesize{rounded}&& \\
		\midrule
		high&i&y&&u\\
		&{\footnotesize <i>}&{\footnotesize <ü>}&{\footnotesize <u>}&\\
		mid&e&œ&(ə)&o\\
		&{\footnotesize <e>}&{\footnotesize <é>}&{\footnotesize <o>}\\
		mid-low&æ&&&\\
		&{\footnotesize <ä>}&&&\\
		low&&&a&\\
		&&&{\footnotesize <a>}&\\
		\lspbottomrule
	\end{tabularx}
\end{table}%Vowel phoneme inventory

Nasalisation is not phonemic, and nasal vowels are a marginal phenomenon. Only two words are attested, in which we find \isi{nasal} vowels. These are the \isi{conjunction}s \emph{a} [ã:] `and' and \emph{o} [ɔ̃:] `or'. Both have a second, much rarer variant with an initial \isi{velar} \isi{nasal}: \emph{ŋa} [ŋa:] and \emph{ŋo} [ŋɔ:], respectively. This suggests that \isi{nasalisation} of the vowel was caused by the loss of the preceding consonant.

There are no diphthongs in Komnzo. All diphthongs which occur on a phonetic level end in high offglides. These are analysed as allophones of the two approximants /w/ and /j/ in coda position ({\S}\ref{approximants}). In the practical \isi{orthography} these are sometimes written as diphthongs, e.g. <ai> or <au>. Two words which exemplify this are \emph{saisai} `drizzle' and \emph{kaukau} `Mouth Almighty'.

\subsection{Phonetic description and allophonic distribution of vowels} \label{phonetic-description-vowels}

\tabref{allovowel} shows that there this free variation for some most of the vowel phonemes.

\begin{table}
\caption{Vowel allophones}
\label{allovowel}
	\begin{tabular}{ll@{~~$\rightarrow$~~}l}
		\lsptoprule
		phoneme&description&allophones\\\midrule
		/i/ &high front unrounded vowel & [i]$\sim$[ı]\\
		/y/ &high front rounded vowel & [y]$\sim$[ʏ]\\
		/u/ &high back rounded vowel & [u]$\sim$[ʊ]\\
		/e/ &mid front unrounded vowel & [e]$\sim$[ɛ]\\
		/œ/ &mid front rounded vowel & [ø]$\sim$[œ]\\
		/o/ &mid back rounded vowel & [o]$\sim$[ɔ]\\
		/a/ &low central unrounded vowel & [a]$\sim$[ɐ]\\
		/æ/ &low front unrounded vowel & [æ]\\
		\lspbottomrule
	\end{tabular}
\end{table}%Vowel allophones

There is no phonemic contrast between short and long vowels. However, vowels tend to be phonetically longer in monosyllabic roots, especially if the monosyllable is light/open, as in \emph{nzä} [ⁿdʒæ:] `I' or \emph{se} [se:] `bark torch'. This process of vowel lengthening is caused by \isi{minimal word} conditions in combination with \isi{syllable} weight. I address this topic in {\S}\ref{syllstruc} and {\S}\ref{minwordconstraints}.

\subsubsection{Allophones of /o/}\label{allo-o}

There is further allophonic variation for /o/, which is related to vowel lengthening. In heavy, closed syllables, /o/ is realised as a short, centralised, rounded vowel [ɞ̆], whereas in light, open syllables it is realised as a mid back rounded vowel of normal length [ɔ]. Two words which show this allophonic variation are the language name \emph{Komnzo} [kɞ̆mⁿdzɔ] and \emph{komon} [kɔmɞ̆n] `maybe'. We find the two allophones [ɞ̆] and [ɔ] conditioned by \isi{syllable} weight in the syllables of the two words respectively. There are two rules which may override this allophonic distribution. The first is a \isi{minimal word} constraint which produces [ɔ] even in closed syllables if the root is monosyllabic, as in \emph{gon} [{\ᵑ}gɔn] `hips'. The second rule overrides \isi{syllable} weight and the impact of the \isi{minimal word} constraint. After the labio-\isi{velar} \isi{approximant} (/w/) and the two labialised-\isi{velar} stops (/kʷ/ and /{\ᵑ}gʷ/) /o/ is always realised as short, centralised, rounded vowel [ɞ̆], as in \emph{woz} [wɞ̆ts] `bottle'. Leaving the influences of the \isi{minimal word} constraint to {\S}\ref{minwordconstraints}, we can formalise these observations in the following rule:

\begin{figure}[H]
  $\mbox{/o/ $\rightarrow$} \left\{
    \begin{array}{l}
      \parbox{4cm}{[ɞ̆] / \_C]\textsubscript{$\sigma$}} \parbox{2,5cm}{\emph{emoth}} \parbox{2,5cm}{[ʔe:mɞ̆θ]} \parbox{3cm}{`girl'} \\
      \parbox{4cm}{\hfill} \parbox{2,5cm}{\emph{ymorymor}} \parbox{2,5cm}{[jə̆mɞ̆ɾjə̆mɞ̆ɾ]} \parbox{3cm}{`desire'} \\
      \parbox{4cm}{\hfill} \parbox{2,5cm}{\emph{thomgsi}} \parbox{2,5cm}{[ðɞ̆m{\ᵑ}gə̆si]} \parbox{3cm}{`help'} \\
      \parbox{4cm}{\hfill} \parbox{2,5cm}{\hfill} \parbox{3cm}{\hfill} \\
	  \parbox{4cm}{{[ɔ] / \_]\textsubscript{$\sigma$}}} \parbox{2,5cm}{\emph{nibo}} \parbox{2,5cm}{[niᵐbɔ]} \parbox{3cm}{`six'} \\
	  \parbox{4cm}{\hfill} \parbox{2,5cm}{\emph{dokre}} \parbox{2,5cm}{[ⁿdɔkɾe]} \parbox{6cm}{`frog'} \\
      \parbox{4cm}{\hfill} \parbox{2,5cm}{\hfill} \parbox{3cm}{\hfill} \\
	  \parbox{4cm}{[ɞ̆] / C\textsubscript{+labio-velar}\_}	\parbox{2,5cm}{\emph{kwosi}} \parbox{2,5cm}{[kʷɞ̆si]} \parbox{3cm}{`dead'} \\
	  \parbox{4cm}{\hfill} \parbox{2,5cm}{\emph{woku}} \parbox{2,5cm}{[wɞ̆ku]} \parbox{6cm}{`skin'} \\
    \end{array}
  \right.$
\end{figure}%/o/

There are some irregularities with these rules when it comes to other bilabial consonants, like /ɸ/. There is \emph{fofot} [ɸɔɸɞ̆t] `single child' which follows the rule, but there are a handful of words which do not follow the rule, like: \emph{fothr} [ɸɞ̆ðə̆r] `eucalyptus species' or \emph{fokufoku} [ɸɞ̆kuɸɞ̆ku] `small patch of vegetation'.

\subsubsection{Analytic problems with /œ/}\label{probl-oe}

The vowel /œ/ poses a problem because there are no \isi{minimal pair}s between /œ/ and some of its immediate neighbours (/e/, /o/, /æ/) in the corpus. There are \isi{minimal pair}s distinguishing /œ/ from /i/, /y/, /u/, /a/. The lack of \isi{minimal pair}s with the former group along with the effects of \isi{vowel harmony} ({\S}\ref{vowharmwae}) invites an analysis in which /œ/ is a variant of other phonemes, for example a rounded allophone of /e/ or a fronted allophone of /o/. However, no conditioning environment (e.g. \isi{vowel harmony} or quality of adjacent consonants) can be established. The main problem lies in the fact, that occurences of /œ/ are much rarer than all other vowels.\footnote{Among the 1700 entries in the dictionary, only 30 contain /œ/. Compare this number with 730 for /a/. This is a conservative count, in which reduplications and their respective bases, as well as simple forms and compounds have been counted once.} For the current description, /œ/ is set up as an independent vowel phoneme. Further research will have to settle this question.

\subsection{The non-phonemic status of schwa} \label{schwa-as-non-phoneme}

The most frequent vowel in Komnzo is a short \isi{schwa} [ə̆]. I will argue here that this is not a phoneme, but that it is inserted through \isi{epenthesis} in order to create a \isi{syllable} nucleus where there is none underlyingly. That being said, I will make an argument at the end of this section that \isi{schwa} can be analysed as a marginal or emerging phoneme in word-final context. The rules of \isi{epenthesis} will be laid out in {\S}\ref{syllabificationandepenthesis}.

Epenthetic vowels are known from many Papuan languages. The best documented case is certainly \ili{Kalam} \citep{Biggs:1963wk, Pawley:1966wj, Blevins:2010ee}, but epenthetic vowels have been described for other languages of the Yam family, e.g. \ili{Nen} \citep{Evans:ji}. In Komnzo, the main arguments for \isi{schwa} as an \isi{epenthetic vowel} rather than a phoneme come from syllabicity alternations, the predictability of \isi{schwa}, and its restricted distribution.

Syllabicity alternations which cause changes in the place of \isi{schwa} insertion are influenced by affixation. Two examples are the verb \emph{ttüsi} [tə̆tʏsi] `print, paint' and the noun \emph{fzenz} [ɸə̆tʃeⁿts] `wife'. In both stems \isi{schwa} occurs in the first \isi{syllable}. When we inflect the verb with an undergoer prefix, the first consonant is syllabified as a coda and \isi{schwa} needs to be inserted in a different position: \emph{yttünzr} [jə̆ttʏⁿdzə̆ɾ] `s/he paints him'. When we add a \isi{possessive} prefix to \emph{fzenz}, e.g. \emph{bufzenz} [ᵐbuɸtʃeⁿts] `your wife', again the first consonant of the stem becomes a coda. In this case \isi{schwa} disappears entirely because the \isi{possessive} prefix ends in a vowel. It follows that \isi{schwa} cannot be present in the underlying representation of these two lexemes.

\largerpage
Schwa has a very restricted distribution compared to specified vowels. It does not occur word-initially and it is very limited word-finally. I will show below that word-final schwas should be analysed as a marginal phoneme. Elsewhere, \isi{schwa} is entirely predictable and therefore not represented in the \isi{orthography} of Komnzo. The rules of \isi{schwa} insertion are discussed as part of \isi{syllabification} and possible \isi{consonant clusters} ({\S}\ref{syllabificationandepenthesis}). There are many roots which lack specified vowels altogether.\footnote{Among the 1700 entries in the dictionary, we find 105 without specified vowels. The number of entries in which the epenthetic vowel occurs together with specified vowels is much higher.} A few examples are: \emph{mnz} [mə̆ⁿts] `house', \emph{zfth} [tsə̆ɸə̆θ] `base, reason', and \emph{ggrb} [{\ᵑ}gə̆{\ᵑ}gə̆ɾə̆ᵐp] `small, unripe coconut'. The quality of the \isi{epenthetic vowel} shows only little variation. In almost all enviroments, it is realised as a mid central vowel of very short duration [ə̆]. However, there is one exception. When the \isi{epenthetic vowel} is inserted preceding the two approximants /w/ and /j/, it is realised as a high back or high front vowel. respectively. Two examples are \emph{thwak} [ðŭwak] `shoulder' and \emph{nyak} [nĭjak] `we go'.

There is one caveat to the analysis of \isi{schwa} as epenthetic: It cannot be predicted in word-final context. Although word-final \isi{schwa} is very rare in terms of types, it cannot be dismissed as the aberrant behaviour of a few lexical items. This is because it is not rare at all in terms of tokens. For example, word-final \isi{schwa} shows up in the verb morphology (\Fsg{} \emph{-é}), in the case marking (\Erg.{\Nsg} \emph{=é}) and in the \isi{adjectivaliser} suffix \emph{-thé}.\footnote{The latter could be historically related to the \isi{similative} case marker (\emph{=thatha}).} For the first singular suffix on verbs, I argue in {\S}\ref{personsuffsection} that this is the result of vowel reduction (a>ə) because neighbouring varieties have a corresponding \emph{-a} suffix. Moreover, the first \isi{person} suffix \emph{-é} disappears if other suffixal material is added to the verb. This is also found with some of the lexical items. For example, when \emph{kayé} `yesterday' is marked with a \isi{temporal} \isi{possessive} case (\emph{=thamane}), word-final \isi{schwa} disappears, as in \emph{kaythamane dagon} `yesterday's food'. This does not happen with full vowels, as in \emph{ezithamane dagon} `food from the morning' from \emph{ezi} `morning'. Thus, I analyse \isi{schwa} in word-final contexts as a marginal phoneme, which emerged or is emerging from vowel reduction. In these word-final cases \isi{schwa} is represented orthographically by <\emph{é}>.

\subsection{Minimal pairs for Komnzo vowels} \label{minimalpairsvowels}
\largerpage
The following \isi{minimal pair}s and near \isi{minimal pair}s illustrate the phonemic contrasts between vowels. Each vowel phoneme is set apart from its immediate neighbours in the vowel space. Each vowel phoneme is contrasted with the \isi{epenthetic vowel}, i.e. the absence of a specified vowel (\Zero{}). Some combinations are redundant (e.g. /i/ vs. /e/ and /e/ vs. /i/) and not repeated in the table.

\begin{longtable} {lllll}
\caption{Minimal pairs of vowel phonemes}
% \begin{tabularx}{\textwidth}
\label{minpairvow}\\
	\lsptoprule
	phonemes&examples&&&\\ \midrule
	\endfirsthead
	phonemes&examples&\\ \midrule
	\endhead

	/i/ vs. /u/	&\emph{mith} `face'&[miθ]&[muθ]&\emph{muth} `(sago) grub'\\
				&\emph{grigri} `maggots'&[{\ᵑ}gɾı{\ᵑ}gɾı]&[{\ᵑ}gɾu:]&\emph{gru} `shooting star'\\
	/i/ vs. /y/	&\emph{minzaksi} `paint (vt.)'&[miⁿdzakə̆si]&[mʏⁿdzakə̆si]&\emph{münzaksi} `allow'\\
				&\emph{di} `back of head'&[ⁿdi:]&[ⁿdʏⁿdʏ]&\emph{düdü} `in good shape'\\
	/i/ vs. /e/	&\emph{si} `eye'&[si:]&[se:]&\emph{se} `torch'\\
%				&\emph{bi} `sago'&[ᵐbi:]&[ᵐbe:]&\emph{be} \Ssg.{\Erg}\\
	/i/ vs. /œ/	&\emph{di} `back of head'&[ⁿdi:]&[ⁿdœ:]&\emph{dö} `monitor lizard'\\
	/i/ vs. \Zero{}&\emph{biribiri} `plant sp'&[ᵐbiɾiᵐbiɾi]&[ᵐbə̆ɾiᵐbə̆ɾi]&\emph{bribri} `weeding'\\
				&\emph{with} `banana'&[wiθ]&[wə̆θ]&\emph{wth} `faeces'\\
%				&\emph{fis} `husband'&[ɸis]&[ɸə̆s]&\emph{fs} `fish sp'\\
	/u/ vs. /y/	&\emph{futhfuth} `scrapes'&[ɸuθɸuθ]&[ɸʏθɸʏθ]&\emph{füthfüth} `hatched bird'\\
				&\emph{but} `kava sticks'&[ᵐbut]&[ᵐbʏt]&\emph{büt} `amputated limb'\\
				&\emph{rusi} `shoot (vt.)'&[ɾusi]&[ɾʏsi]&\emph{rüsi} `rain (v.)'\\
	/u/ vs. /o/	&\emph{muramura} `medicine'&[muɾamuɾa]&[mɔɾamɔɾa]&\emph{moramora} `tree sp'\\
				&\emph{muth} `(sago) grub'&[muθ]&[mɞ̆θ]&\emph{moth} `path'\\
				&\emph{tru} `palm sp'&[tɾu:]&[tɾɔ:]&\emph{tro} `python sp'\\
	/u/ vs. \Zero{}&\emph{kursi} `split (vt.)'&[kuɾsi]&[kə̆ɾsi]&\emph{krsi} `block (vt.)'\\
%				&\emph{kut} `trap'&[kut]&[kə̆t]&\emph{kt} `grass sp'\\
				&\emph{fuk} `in a group'&[ɸuk]&[ɸə̆k]&\emph{fk} `buttocks'\\
	/y/ vs. /e/	&\emph{fünz} `arm muscles'&[ɸʏⁿts]&[ɸeⁿts]&\emph{fenz} `puss'\\
	/y/ vs. /œ/	&\emph{nümä} `one week away'&[nʏmæ]&[nœmæ]&\emph{nömä} `yamcake'\\
				&\emph{düdü} `in good shape'&[ⁿdʏⁿdʏ]&[ⁿdœⁿdœ]&\emph{dödö} `plant sp'\\
	/y/ vs. \Zero{}&\emph{sün} `dirt, dust'&[sʏn]&[sə̆n]&\emph{sn} `yam sp'\\
				&\emph{tüfr} `plenty'&[tʏɸə̆ɾ]&[tə̆ɸə̆ɾtə̆ɸə̆ɾ]&\emph{tfrtfr} `tree sp'\\
	/e/ vs. /o/	&\emph{fethaksi} `dip in'&[ɸeðakə̆si]&[ɸɔðakə̆si]&\emph{fothaksi} `take off (bag)'\\
				&\emph{game} `tongs'&[{\ᵑ}game]&[{\ᵑ}gamɔ]&\emph{gamo} `magic spell'\\
	/e/ vs. /a/	&\emph{yem} `cassowary'&[jem]&[jam]&\emph{yam} `event'\\
				&\emph{fetr} `dangerous'&[ɸetə̆ɾ]&[ɸatə̆ɾ]&\emph{fatr} `shoulder'\\
				&\emph{gwre} `bird sp'&[{\ᵑ}gʷre:]&[{\ᵑ}gʷra:]&\emph{gwra} `fish sp'\\
	/e/ vs. /æ/	&\emph{fenz} `puss'&[ɸeⁿts]&[ɸæⁿts]&\emph{fänz} pers. name\\
%				&\emph{erbänzé} `I untie them'&[ʔeɾə̆ᵐbæⁿtsə̆]&[ʔæɾə̆ᵐbæ ⁿtsə̆]&\emph{ärbänzé} `I untie for them'\\
				&\emph{nze} \Fsg.\Erg&[ⁿdʒe:]&[ⁿdʒæ:]&\emph{nzä} \Fsg.\Abs\\
	/e/ vs. \Zero{}&\emph{menz} `story man'&[meⁿts]&[mə̆ⁿts]&\emph{mnz} `house'\\
				&\emph{fethaksi} `dip in'&[ɸeðakə̆si]&[ɸə̆ðakə̆si]&\emph{fthaksi} `take from fire'\\
				&\emph{ŋakwire} `we run'&[ŋakʷiɾe]&[ŋakʷiɾə̆]&\emph{ŋakwiré} `I run'\\
	/æ/ vs. /a/	&\emph{näbi} `one'&[næᵐbi]&[naᵐbi]&\emph{nabi} `bow, bamboo'\\
				&\emph{fätr} `left'&[ɸætə̆ɾ]&[ɸatə̆ɾ]&\emph{fatr} `shoulder'\\
				&\emph{mafä} `with whom'&[maɸæ]&[maɸa]&\emph{mafa} `who'\\
	/æ/ vs. /o/ &\emph{bärbär} `half'&[ᵐbæɾᵐbæɾ]&[ᵐbɞ̯ɾ]&\emph{bor} `rat'\\
				&\emph{nä} `some'&[næ:]&[nɔ:]&\emph{no} `water'\\
	/æ/ vs. \Zero{}&\emph{fäk} `jaw'&[ɸæk]&[ɸə̆k]&\emph{fk} `buttocks'\\
				&\emph{märmär} `slope'&[mæɾmæɾ]&[mə̆ɾmə̆ɾ]&\emph{mrmr} `inside'\\
				&\emph{bnä} `with you'&[ᵐbə̆næ]&[ᵐbə̆nə̆]&\emph{bné} \Snsg.{\Erg}\\
	/a/ vs. /œ/ &\emph{namä} `good'&[namæ]&[nœmæ]&\emph{nömä} `yamcake'\\
	/a/ vs. /o/ &\emph{zan} `fight'&[tsan]&[tsɔn]&\emph{zon} `plant sp'\\
				&\emph{karfa} `from village'&[kaɾɸa]&[kaɾɸɔ]&\emph{karfo} `to village'\\
				&\emph{far} `house post'&[ɸaɾ]&[ɸɞ̯ɾ]&\emph{for} `riverbank'\\
	/a/ vs. \Zero{} &\emph{ngath} `friend'&[nə̆{\ᵑ}gaθ]&[nə̆{\ᵑ}gə̆θ]&\emph{ngth} `young sibling'\\
				&\emph{tharthar} `next to'&[ðaɾðaɾ]&[ðə̆ɾðə̆ɾ]&\emph{thrthr} `intestines'\\
%				&\emph{mar} `pandanus sp'&[maɾ]&[mə̆ɾ]&\emph{mr} `brain'\\
				&\emph{sakwra} `I hit him' (\Pst{})&[sakʷə̆ɾa]&[sakʷə̆ɾə̆]&\emph{sakwré} `I hit him' (\Rpst{})\\
	/o/ vs. \Zero{} &\emph{borsi} `laugh'&[ᵐbɞ̯ɾsi]&[ᵐbə̆ɾsi]&\emph{brsi} `scoop water'\\
				&\emph{fothaksi} `take off'&[ɸɔðakə̆si]&[ɸə̆ðakə̆si]&\emph{fthaksi} `take from fire'\\
				&\emph{rgosi} `poke through'&[ɾə̆{\ᵑ}gɔsi]&[ɾə̆{\ᵑ}gə̆si]&\emph{rgsi} `wear clothes'\\
				&\emph{monz} `trench, ditch'&[mɔⁿts]&[mə̆ⁿts]&\emph{mnz} `house'\\
				&\emph{nzigom} `chain smoker'&[ⁿdʒi{\ᵑ}gɞ̯m]&[ⁿdʒi{\ᵑ}gə̆m]&\emph{nzigm} `stickyness'\\
	\lspbottomrule
% \end{tabularx}
\end{longtable}%minimal pairs - vowels phonemes

\section{Regular phonological processes} \label{regular-phon-processes}

\subsection{Gemination} \label{gemination-section}
\largerpage
Gemination occurs with a subset of the consonantal phonemes (/t/, /k/, /ɸ/, /ð/, /m/, /n/, and /r/). We find geminates in medial heterosyllabic \isi{consonant clusters}, where the rules of \isi{syllabification} specify that no \isi{epenthetic vowel} needs to be inserted ({\S}\ref{syllabificationandepenthesis}). Phonetically, geminates are characterised by a prolonged realisation of \isi{fricatives}, \isi{nasals}, and \isi{alveolar} trill. Geminate stops are realised with a delayed release of the airflow. Although gemination is caused by affixation in most cases, I discuss the topic here rather than as a morphophonemic rule because we also find monomorphemic roots with geminates. The list in \tabref{geminates} provides some attested examples from the corpus. In some of the examples, we find \isi{minimal pair}s based on gemination, as can be seen in the rightmost column.

\begin{table}
\caption{Geminate consonants}
\label{geminates}
	\begin{tabularx}{\textwidth}{lll}
		\lsptoprule
		{segment} & {geminate} & {non-geminate} \\ \midrule
		/t/ & \emph{yttünzr} [jə̆t:y{ⁿ}dzə̆r] `s/he paints him' & n/a \\
		&&\\
		/k/& \emph{yakkarä} [jak:aræ] `quickly' & \emph{yakarä} [jakaræ] `in tears'\\
		& {\footnotesize yak=karä} & {\footnotesize ya=karä} \\
		& {\footnotesize walk=\Prop} & {\footnotesize cry=\Prop}\\
		&&\\
		/m/& \emph{yamme} [jam:e] `through this event' & \emph{yame} [jame] `mat' \\
		& {\footnotesize yam=me} & \\
		& {\footnotesize event=\Ins} & \\
		%&&\\
		& \emph{fammäre} [ɸam:ære] `without thinking' & n/a \\
		&{\footnotesize fam=märe} & \\
		&{\footnotesize thoughts=\Priv} & \\
		&&\\
		/n/& \emph{yannor} [jan:ɞ̆r] `he shouts hither' & \emph{yanor} [janɞ̆r] `he shouts'\\
		&{\footnotesize ya-n-nor} & {\footnotesize ya-nor} \\
		&{\footnotesize \Tsg.\Masc-\Venit-shout} & {\footnotesize \Tsg.\Masc-shout}\\
		&&\\
		/ɸ/& \emph{fiyaffa} [ɸijaɸ:a] `from the hunt' & n/a \\
		&{\footnotesize fiyaf=fa} & \\
		&{\footnotesize hunt=\Abl} & \\
		&&\\
		/ð/& \emph{yththagr} [jə̆θ:a{\ᵑ}gə̆r] `it is sticking (on sth.)' 	& n/a \\
		&&\\
		/r/& \emph{firra} [ɸir:a] `place name' & \emph{fira} [ɸira] `betelnut'\\
		& \emph{kwrro} [kʷr:o] `Blue-winged Kookaburra' & n/a \\
		\lspbottomrule
	\end{tabularx}
\end{table}%Geminate consonants

Gemination is not attested for complex consonants, including the prenasalised stops (/{ᵐ}b/, /{ⁿ}d/, and /{\ᵑ}g/) as well as the two affricates (/ts/ and /{ⁿ}dz/) and /s/. Gemination is not relevant for the labialised \isi{velar} stops (/kʷ/ and /{\ᵑ}gʷ/) and the \isi{velar} \isi{nasal} (/ŋ/) because these do not occur in coda position.

\subsection{Final-devoicing} \label{final-devoicing-section}

The process of final \isi{devoicing} affects only those consonants which occur in final position, excluding non-final /kʷ/, /{\ᵑ}gʷ/, and /ŋ/. Moreover, it affects only those consonants which are voiced in all other environments, excluding voiceless /t/, /k/, /ɸ/, /s/, and /ts/. The \isi{nasal} stops and the approximants are also not affected by final \isi{devoicing}. This leaves us with the following phonemes, which are targetted by final \isi{devoicing}: /{ᵐ}b/, /{ⁿ}d/, /{\ᵑ}g/, /{ⁿ}dz/, /ð/, and /r/.

The domain of final \isi{devoicing} is the \isi{syllable}. In onset position, these phonemes are always voiced, for example /{ⁿ}dz/ in \emph{nzafar} [ⁿdzaɸaɾ] `sky' and \emph{knzun} [kə̆ⁿdzun] `parallel'. In coda position, they are voiceless, as /{ⁿ}d/ in \emph{bodkr} [{ᵐ}bɞ̆{ⁿ}tkə̆r] `stinking' and /ð/ in \emph{wathknsi} [waθkə̆nsi] `pack up'. In word-final position, they are also voiceless, for example /{ᵐ}b/ in \emph{gb} [{\ᵑ}gə̆{ᵐ}p] `pandanus species' and /{ⁿ}dz/ in \emph{mnz} [mə̆ⁿts] `house'.

We find further evidence in suffixation and encliticisation that the process is targetting the right edge of the \isi{syllable} rather than the (phonological) word. \emph{Mnz} [mə̆ⁿts] `house' may take the vowel-initial \isi{locative} \isi{enclitic} \emph{=en}, in which case /{ⁿ}dz/ occurs in onset position and is voiced: \emph{mnzen} [mə̆ⁿdzen] `in the house'. This contrasts with the consonant-initial formatives \emph{=fa} (\Abl) and \emph{=wä} (\Emph). In both cases, /{ⁿ}dz/ is syllabified in coda position and is voiceless: \emph{mnzfa} [mə̆ⁿtsɸa] `from the house' and \emph{mnzwä} [mə̆ⁿtswæ] `really the house'. We can formalise final \isi{devoicing} in the following rule:

\begin{figure}[H]
\centering
$\mbox{/{ᵐ}b/, /{ⁿ}d/, /{\ᵑ}g/, /{ⁿ}dz/, /ð/ $\rightarrow$} \left\{
\begin{array}{l}
  \parbox{3,75cm}{[-voiced] / \_]\textsubscript{$\sigma$}} \\
\end{array}
\right.$
\end{figure}%devoicing

The only excepion is /r/, where final \isi{devoicing} occurs only word-finally. However, final \isi{devoicing} of /r/ is optional and more commonly found with older speakers.

\subsection{Glottal stop insertion} \label{glottal-stop-insertion-section}

There are only few lexemes with an initial vowel. Among the 1700 entries in the dictionary, there are 54 vowel-initial lexemes: /a/ (21), /e/ (17), /o/ (8), /æ/ (4), /u/ (3), /i/ (1). Three of these are loanwords. In addition, there is a vowel-initial undergoer prefix in one of the five prefix series.\footnote{In the alpha prefixes, {\Stnsg} is \emph{e-}.} Thus, vowel-initial lexemes are a marginal phenomenon. Moreover, there are no vowel-initial syllables word-internally. A possible explanation for the occurence of vowel-initial words in Komnzo, as compared to other Tonda languages in the west, might be contact with the \ili{Nambu} languages to the east, where vowel-initial words seem to be more frequent.

For this marginal pattern we find a rule of \isi{glottal stop} insertion, as in \emph{ebar} [ʔeᵐbaɾ] `head' or \emph{ettünzr} [ʔettʏⁿdzə̆ɾ] `s/he paints them'. The glottal stop is predictable and not represented in the orthography. Its insertion is restricted to word-initial environments, because the rules of \isi{syllabification} maximise onsets in almost all cases ({\S}\ref{syllabificationandepenthesis}). There is only one exception. Word-medial \isi{glottal stop} insertion occurs with some of the vowel-initial enclitics like the associative \emph{=ä}, or the possive \emph{=ane}. When the \isi{possessive} is attached to a word which ends in a vowel, a \isi{glottal stop} is inserted at the morpheme boundary. An example is \emph{kabe} `man' $\rightarrow$ \emph{kabeane} [kaᵐbeʔane] `of the man'. However, there is a variant, whereby an approximant is inserted [kaᵐbejane].

\section{The syllable and phonotactics} \label{syllable-and-phonotactics}

The phonotactics are best described in terms of the \isi{syllable}. My description of the \isi{syllable} is influenced by Blevins \citeyearpar{Blevins:1995tt}. I begin by outlining different \isi{syllable} templates and the constraints which help to define them. In {\S}\ref{syllstruc}, I provide evidence for the internal structure of the \isi{syllable}. Consonant clusters are shown in \S\ref{consonantclusters}. I offer a step-by-step analysis of \isi{syllabification} and \isi{epenthesis} in {\S}\ref{syllabificationandepenthesis}. The section closes with a discussion of the \isi{minimal word} (\S\ref{minwordconstraints}) and \isi{stress} ({\S}\ref{stress}).

\subsection{Syllable structure} \label{syllstruc}

The template for the maximal \isi{syllable} in Komnzo is [CCVC]\textsubscript{$\sigma$}. The minimal \isi{syllable} is [CV]\textsubscript{$\sigma$} and in a more restricted environment [V]\textsubscript{$\sigma$}. Thus, a \isi{syllable} maximally consists of an onset, which may or may not be complex, a nucleus and a simple coda. Three constraints help to define the possible representations of the \isi{syllable} in Komnzo:

\begin{enumerate}
	\item Onsets are obligatory in word-medial and final position. There is a constraint against vowels in onset position: \textsuperscript{$\ast$}\textsubscript{$\sigma$}[V. The only position where we find vowels in onsets is word-initially, but this is a marginal pattern. If the process of \isi{syllabification} produces vowel-initial words, a \isi{glottal stop} fills the onset position ({\S}\ref{glottal-stop-insertion-section}).
	\item Syllables may have complex onsets with a maximal number of two adjacent consonants: \textsubscript{$\sigma$}[CC. There are constraints on the phonemes involved in CC onset clusters ({\S}\ref{tautosyllabiccc}).
	\item Syllables may only have a simple coda: C]\textsubscript{$\sigma$}. Post-vocalic consonsant clusters are always heterosyllabic VC]\textsubscript{$\sigma$}C]\textsubscript{$\sigma$}, never tautosyllabic \textsuperscript{$\ast$}VCC]\textsubscript{$\sigma$}. There are a number of constraints on the possibilities of heterosyllabic \isi{consonant clusters} ({\S}\ref{heterosyllabiccc}).
\end{enumerate}%syllable rules

From the three constraints given above, we can now derive the following possible \isi{syllable} types: CV, CVC, CCV, CCVC. Word-initially, we also find V and VC. Figure \ref{syllableinternal} presents the \isi{syllable} as a binary branching construct.

\begin{figure}
	\centering
		\Tree[.$\sigma$
		 [.onset
		   [.(C\textsubscript{1}) ]
		   [.(C\textsubscript{2}) ]
		 ]
		 [.rhyme
		   [.nucleus V ]
		   [.coda (C\textsubscript{3}) ]
		 ]
		]
	\caption{The internal structure of the syllable}\label{syllableinternal}
\end{figure}%syllable structure

A branching \isi{syllable} is chosen over a flat structure because there is evidence for the rhyme as a separate node of which nucleus and coda are subnodes. Such evidence includes the different shapes and constraints for onset and coda. Onsets may be complex. Codas can only be simple. Onsets are obligatory in almost all cases, while codas are optional. Onsets and rhyme combine freely, thus capturing the generalisation that onsets rarely influence the nucleus. All consonant phonemes may appear in a simple onset (C\textsubscript{1}). There are some restrictions, but these are internal to the onset ({\S}\ref{tautosyllabiccc}). The coda position (C\textsubscript{3}) on the other hand is more limited as to which consonant phonemes may appear. The labialised \isi{velar} stops /kʷ/ and /{\ᵑ}gʷ/ and the \isi{velar} \isi{nasal} /ŋ/ never appear in a coda.

The strongest evidence for an independent rhyme comes from \isi{syllable} weight, which impacts on vowel length of the nucleus. When there is a specified vowel in the nucleus, the vowel will be realised long in open/light syllables, and it will be realised as short in closed/heavy syllables. This affects different vowels to varying degrees. We find a good example of this in the distribution of the two allophones of /o/, which are [ɔ] and [ɞ̆]. In the language name \emph{Komnzo} [kɞ̆mⁿdzɔ] the first vowel is very short (although stressed), and the second vowel is of normal length. It follows that \isi{syllable} weight influences the length (and sometimes quality) of the vowel in the nucleus. The shortening or lengthening of nuclei may be overridden by \isi{minimal word} constraints ({\S}\ref{minwordconstraints}), but these rules hold for all polysyllabic roots. Consequently, for an adequate description, we require the rhyme as an independent subnode of the \isi{syllable}.

\subsection{Consonant clusters} \label{consonantclusters}

We find tautosyllabic and heterosyllabic \isi{consonant clusters} in Komnzo. These have very different restrictions on their combinations.

\subsubsection{Tautosyllabic clusters} \label{tautosyllabiccc}

Tautosyllabic clusters are restricted to the onset of a \isi{syllable}. No more than two consonants may occur and they only involve a subset of the phonemes. In a \textsubscript{$\sigma$}[C\textsubscript{1}C\textsubscript{2} template, C\textsubscript{2} may only be /r/ or /w/.

In a cluster with /r/ we find all consonant phonemes except for the three \isi{nasal} stops (\textsuperscript{$\ast$}\textsubscript{$\sigma$}[mr, \textsuperscript{$\ast$}\textsubscript{$\sigma$}[nr, \textsuperscript{$\ast$}\textsubscript{$\sigma$}[ŋr), the approximants (\textsuperscript{$\ast$}\textsubscript{$\sigma$}[wr and \textsuperscript{$\ast$}\textsubscript{$\sigma$}[yr), and /r/ itself (\textsuperscript{$\ast$}\textsubscript{$\sigma$}[rr). This points to an explanation in terms of a sonority hierarchy in which \isi{nasal} and approximants are more sonorous than the trill/tap. Some examples of Cr clusters are \emph{brüzi} [{ᵐ}bɾytʃi] `catfish species', \emph{frar} [ɸə̆rar] `small fishtrap', \emph{krüfr} [kɾyɸə̆ɾ] `cold', \emph{gru} [{\ᵑ}gɾu:] `shooting star', \emph{kwras} [kʷɾas] `Brolga', \emph{srima kabe} [sɾima ka{ᵐ}be] `scout, spy', \emph{thruthru} [ðɾuðɾu] `bamboo species', \emph{trisi} [tɾisi] `scratch (v)', and \emph{zra} [tsɾa:] `swamp'.

In a cluster with /w/, the restrictions on C\textsubscript{1} are more severe and roots, in which it is attested, are rare. We only find the following phonemes in C\textsubscript{1} position: /k/, /{\ᵑ}g/, /ts/, /{ⁿ}dz/, /ð/, and /s/. The first two phonemes in the list pose a problem because one has to find a distinction between a Cw cluster and the labialised \isi{velar} stops /kʷ/ and /{\ᵑ}gʷ/. This is impossible to do for roots, but we find some evidence in a morphophonemic rule in {\S}\ref{approxhighvowel}, where the vowel /u/ is realised as [w] and becomes part of a Cw cluster. Some examples of lexemes with Cw clusters are \emph{swäyé} [swæjə̆] `anchoring place', \emph{zwäf} [tswæɸ] `luke-warm', and \emph{bzwär} [ᵐbə̆zwæɾ] `place name'.

\subsubsection{Heterosyllabic clusters} \label{heterosyllabiccc}

Heterosyllabic clusters are much harder to pin down because there are syllabicity alternations, where a coda consonant may become an onset by inserting epenthetic \isi{schwa}, which breaks up the cluster ({\S}\ref{syllabificationandepenthesis}). For the following description, I label the two consonants involved C\textsubscript{a} (the coda of the first \isi{syllable}) and C\textsubscript{b} (the onset of the following \isi{syllable}).

We find that where C\textsubscript{a} and C\textsubscript{b} are identical the consonants are never broken up but always realised as geminates. The attested \isi{geminate} patterns are described as a phonological rule in {\S}\ref{gemination-section}. These patterns exclude a number of logically possible geminates: labialised \isi{velar} stops (/kʷ/ and /{\ᵑ}gʷ/), \isi{velar} \isi{nasal} (/ŋ/), and all the prenasalised phonemes (/{ᵐ}b/, /{ⁿ}d/, /{\ᵑ}g/, and /{ⁿ}dz/).\footnote{The labialised velar stop and the velar nasal may not occur as C\textsubscript{a} because these never occur in coda position.} Other heterosyllabic clusters are rather unrestricted. \tabref{heterosyllcctable} shows the possible cluster types.\footnote{The column and the row labelled ``pren. stop'' includes prenasalised stops and the prenasalised affricate.} \tabref{heterosyllcctableexamples} lists examples of these types.

\begin{table}[H]
\caption{Heterosyllabic consonant clusters}
\label{heterosyllcctable}
	\begin{tabularx}{\textwidth}{lCCCCCCCC}
		\lsptoprule
		&&{oral}&{pren.}&&&&&{labio-}\\
		& /r/ & {stop} & {stop} & {nasal} & {affr.} & {fric.} & {approx.} & {velar}\\ \midrule
		/r/ & \checkmark & \checkmark & n/a & \checkmark & \checkmark & \checkmark & \checkmark  & \checkmark\\
		{oral stop} & n/a & \checkmark & n/a & \checkmark & n/a & \checkmark & \checkmark  & \checkmark\\
		{pren. stop} & n/a & \checkmark & n/a & \checkmark & n/a & \checkmark & \checkmark  & n/a\\
		{nasal} & \checkmark & \checkmark & \checkmark & \checkmark & \checkmark & \checkmark & \checkmark  & \checkmark\\
		{affr.} & n/a & \checkmark & n/a & \checkmark & n/a & \checkmark & \checkmark & n/a\\
		{fric.} &  n/a & \checkmark & n/a & \checkmark & \checkmark & \checkmark & \checkmark  & \checkmark\\
		{approx.} &  n/a & \checkmark & n/a & \checkmark & \checkmark & \checkmark & n/a  & n/a\\
		{lab-velar} & n/a & n/a& n/a& n/a& n/a& n/a& n/a& n/a\\
		\lspbottomrule
	\end{tabularx}
\end{table}%heterosyllabic consonant clusters

\begin{longtable}{p{2cm}p{2cm}lll}
\caption{Examples of attested heterosyllabic consonant clusters}
\label{heterosyllcctableexamples}
	\\
	\lsptoprule
	C\textsubscript{a} & C\textsubscript{b} & {example} & {phonetic} & {gloss}\\\midrule
	\endfirsthead
	C\textsubscript{a} & C\textsubscript{b} & {example} & {phonetic} & {gloss}\\\midrule
	\endhead
	/r/ & [+\isi{nasal}] &\emph{ker.ma}&[ke\uline{ɾm}a] &`from tail'\\
	&&\emph{tr.nä} &[tə̆\uline{ɾn}æ] &`palm frond'\\
	/r/ &[+oral] &\emph{for.tu}&[ɸɞ̆\uline{ɾt}u] &`scar'\\
	&&\emph{ker.ko}&[ke\uline{ɾk}o] &`headdress'\\
	/r/ &[+affr.]&\emph{zr.zü}&[tsə̆\uline{ɾtʃ}ʏ] &`knee' \\
	/r/ &[+fric.] &\emph{war.fo}&[wa\uline{ɾɸ}ɔ] &`above'\\
	&&\emph{kr.si}&[kə̆\uline{ɾs}i] &`block (v)'\\
	&&\emph{tr.tha}&[tə̆\uline{ɾð}a] &`life'\\
	/r/&[+approx.]&\emph{kar.wä.si}&[ka\uline{ɾw}æsi] &`lie (v)'\\
	&&\emph{yar.yom.g.si}&[ja\uline{ɾj}ɞ̆m{\ᵑ}gə̆si] &`scream (v)'\\
	/r/&[+lab-vel]&\emph{ŋa.far.kw.re}&[ŋaɸa\uline{ɾkʷ}ə̆ɾe]&`we leave'\\
	{[+oral]}&[+oral]&\emph{wät.ku}&[wæ \uline{tk}u]&`pelican'\\
	{[+oral]} &[+\isi{nasal}]&\emph{dek.ni.ni}&[ⁿde\uline{kn}ini]&`praying mantis'\\
	&&\emph{rt.maksi}&[ɾə̆\uline{tm}akə̆si]&`cut'\\
	{[+oral]} &[+fric.]&\emph{f.rk.thé}&[ɸə̆ɾə̆\uline{kð}ə̆]&`red'\\
	&&\emph{et.fth}&[ʔe\uline{tɸ}ə̆θ]&`sleep (n)'\\
	{[+oral]} &[+approx.]&\emph{thik.ya.si}&[ði\uline{kj}asi]&`build fence'\\
	&&\emph{zok.wa.si}&[tsɞ̆\uline{kw}asi]&`speech'\\
	&&\emph{mit.wa.si}&[mi\uline{tw}asi]&`swing (v)'\\
	{[+oral]} &[+lab-vel]&\emph{tat.kwo.nam}&[ta\uline{tkʷ}ɔnam]&`tree species'\\
	{[+pren.]}&[+oral]&\emph{gb.ka.rä}&[{\ᵑ}gə̆\uline{ᵐpk}aɾæ]&`with pandanus'\\
	{[+pren.]}&[+\isi{nasal}]&\emph{ŋad.me}&[ŋaⁿtme]&`with rope'\\
	{[+pren.]}&[+fric.]&\emph{bad.fo}&[ᵐbaⁿtɸɔ]&`to the ground'\\
	{[+pren.]}&[+approx.]&\emph{mnz.wä}&[mə̆\uline{ⁿtsw}æ]&`house (\Emph)'\\
	{[+\isi{nasal}]}&/r/&\emph{nin.rr}&[ni\uline{nɾ}ə̆ɾ]&`with us'\\
	{[+\isi{nasal}]} &[+oral]&\emph{am.kf}&[ʔa\uline{mk}ə̆ɸ]&`breath'\\
	&&\emph{thun.t.nä.gwr}&[ðu\uline{nt}ə̆næ{\ᵑ}gwə̆ɾ]&`he lost them'\\
	{[+\isi{nasal}]} &[+\isi{nasal}]&\emph{kan.motha}&[ka\uline{nm}ɔða]&`river snake'\\
	{[+\isi{nasal}]} &[+pren.]&\emph{yar.yom.g.si}&[jaɾjɞ̆\uline{m{\ᵑ}g}ə̆si]&`scream (v)'\\
	&&\emph{kum.da}&[ku\uline{mⁿd}a]&`basket'\\
	&&\emph{kän.brim}&[kæ\uline{nᵐb}ɾim]&`come here!'\\
	{[+\isi{nasal}]} &[+affr.]&\emph{san.zin}&[sa\uline{ntʃ}in]&`put him down!'\\
	{[+\isi{nasal}]} &[+fric.]&\emph{zan.fr}&[tsa\uline{nɸ}ə̆ɾ]&`far'\\
	&&\emph{kam.tha.tha}&[ka\uline{mð}aða]&`like a bone'\\
	{[+\isi{nasal}]} &[+approx.]&\emph{nze.nm.wä}&[ⁿdʒenə̆\uline{mw}æ]&`for us (\Emph)'\\
	{[+\isi{nasal}]} &[+lab-vel]&\emph{ŋan.kwir}&[ŋa\uline{nkʷ}ir]&`run hither'\\
	{[+affr.]} &[+oral]&\emph{ez.kn.wr}&[ʔe\uline{tsk}ə̆nwə̆ɾ]&`he moves them'\\
	{[+affr.]} &[+\isi{nasal}]&\emph{käz.nob}&[kæ\uline{tsn}ɞ̆ᵐp]&`drink (it)!'\\
	{[+affr.]} &[+fric.]&\emph{fz.fo}&[ɸə̆\uline{tsɸ}ɔ]&`to forest'\\
	{[+affr.]} &[+approx.]&\emph{fz.wä}&[ɸə̆\uline{tsw}æ]&`forest (\Emph)'\\
	{[+fric.]} &[+oral]&\emph{mnz.wä}&[mə̆\uline{ⁿtsw}æ]&`house (\Emph)'\\
	{[+fric.]} &[+affr.]&\emph{buf.zenz}&[ᵐbu\uline{ɸtʃ}eⁿts]&`your wife'\\
	{[+fric.]} &[+fric.]&\emph{ef.thar}&[ʔe\uline{ɸð}aɾ]&`dry season'\\
	&&\emph{füs.füs}&[ɸʏ\uline{sɸ}ʏs]&`wind'\\
	{[+fric.]} &[+approx.]&\emph{nzf.wi.yak}&[ⁿtsə̆\uline{ɸw}Ijak]&`we walked'\\
	&&\emph{naf.wä}&[na\uline{ɸw}æ]&`they ({\Emph})'\\
	&&\emph{fith.wo.g.si}&[ɸi\uline{θw}ɔ{\ᵑ}gə̆si]&`take out'\\
	{[+fric.]} &[+lab-vel]&\emph{math.kwi}&[ma\uline{θkʷ}i]&`personal name'\\
	&&\emph{y.ra.kth.kwa}&[jə̆rakə̆\uline{θkʷ}a]&`he put on top'\\
	{[+approx.]} &[+oral]&\emph{faw.ka.rä}&[ɸa\uline{ʷk}aɾæ]&`with payment'\\
	{[+approx.]} &[+\isi{nasal}]&\emph{faw.ma}&[ɸa\uline{ʷm}a]&`from payment'\\
	{[+approx.]} &[+affr.]&\emph{bäw.zö}&[ᵐbæ\uline{ʷtʃ}œ]&`paperbark'\\
	{[+approx.]} &[+fric.]&\emph{wy.thk}&[wə̆\uline{\super{j}ð}ə̆k]&`comes to end'\\
	\lspbottomrule
\end{longtable}%heterosyllabic consonant clusters

We can make a number of observations from \tabref{heterosyllcctableexamples}. The prenasalised phonemes do occur in C\textsubscript{a} as well as C\textsubscript{b}. In the latter case, C\textsubscript{a} may only be another \isi{nasal}, as in \emph{kum.da} [ku\uline{mⁿd}a] `basket', \emph{kum.g.si} [ku\uline{m{\ᵑ}g}ə̆si] `smell (v)', \emph{dm.gu} [ⁿdə̆\uline{m{\ᵑ}g}u] `waterhole', \emph{tin.gwä} [ti\uline{n{\ᵑ}gʷ}æ] `tree species'. If C\textsubscript{a} is a phoneme other than a \isi{nasal}, the cluster will be broken up: \emph{ga.r.da} [{\ᵑ}ga\uline{ɾə̆ⁿd}a] `canoe', \emph{ä.th.gam} [ʔæ\uline{ðə̆{\ᵑ}g}am] `Parinari nonda', \emph{th.f.gar.w.r.mth} [ðə̆\uline{ɸə̆{\ᵑ}g}aɾwə̆\uline{ɾə̆m}ə̆θ] `they were breaking them'. There are no attested cases of a prenasalised phoneme in C\textsubscript{b} with a homorganic \isi{nasal} in C\textsubscript{a}, i.e. /m/ + /{ᵐ}b/, /n/ + /{ⁿ}dz/, /n/ + /{ⁿ}d/.

There are only few clusters which involve /r/ in the C\textsubscript{b} position. This is caused by maximizing onsets during \isi{syllabification}, which creates complex onsets clusters of the type Cr. As a consequence, the only heterosyllabic clusters with /r/ in C\textsubscript{b} position are the ones which are illegal as onset clusters (e.g. \textsuperscript{$\ast$}\textsubscript{$\sigma$}[mr, \textsuperscript{$\ast$}\textsubscript{$\sigma$}[nr, \textsuperscript{$\ast$}\textsubscript{$\sigma$}[rr). In other words, because \textsuperscript{$\ast$}\textsubscript{$\sigma$}[nr is illegal as an onset, we do find it as a heterosyllabic cluster (\emph{nin.rr} [ni\uline{nɾ}ə̆ɾ] `with us'). Likewise, because \textsubscript{$\sigma$}[fr is a legal onset cluster, we never find it as a heterosyllabic cluster.

\largerpage
We do find heterosyllabic clusters which involve /w/ in C\textsubscript{b} position and a \isi{velar} (prensalised) stop in  C\textsubscript{a} position. Evidence that these clusters are indeed heterosyllabic as opposed to an instantiation of the labialised \isi{velar} stop /kʷ/ and /{\ᵑ}gʷ/ comes from two sources. First, we find examples like \emph{zok.wa.si} [tsɞ̆kwasi] `speech' where the short, centralised allophone of /o/ shows that the first syllable is a closed \isi{syllable} ({\S}\ref{phonetic-description-vowels} and {\S}\ref{syllstruc}). Since the labialised velar stops cannot occur in coda position, we have to assume a syllable boundary between /k/ and /w/. Secondly, verb stems ending in /k/ and /{\ᵑ}g/ select the \emph{-wr} \isi{allomorph} of the non-dual suffix ({\S}\ref{allomorphdualsuffix}). In inflected verbs like \emph{ŋa.th.wek.wr} [ŋaðə̆wekwə̆r] the verb stem \emph{thwek-} and the non-dual suffix \emph{-wr} are separate morphemes and should be analysed as separated syllables. Consequently, heterosyllabic clusters /kw/ and /{\ᵑ}gw/ as well as the complex phonemes /kʷ/ and /{\ᵑ}gʷ/ are required for an adequate description of the phonological system.

\subsection{Syllabification and epenthesis} \label{syllabificationandepenthesis}

Syllable structure is generally understood not to be defined at the underlying representation (\citealt[221]{Blevins:1995tt}). Thus, we do not find \isi{minimal pair}s based on syllabicity. As was explained in {\S}\ref{schwa-as-non-phoneme}, \isi{schwa} is not a phoneme but an \isi{epenthetic vowel} inserted in order to break up \isi{consonant clusters}. There is some degree of free variation in syllabicity and \isi{schwa} insertion. An example is the word \emph{mrn} `family, clan' with the \isi{locative} suffix \emph{-en}. The resulting word \emph{mrnen} `in the family' may be realised either [mə̆ɾnen] or [mə̆ɾə̆nen]. There is no phonemic contrast and speakers find it difficult to perceive the difference in syllabicity.

The process of \isi{syllabification} will be outlined here in the form of three ordered rules, which predict \isi{epenthesis} and \isi{syllable} structure:

\begin{enumerate}
	\item Associate each specified vowel with a \isi{syllable} nucleus.
	\item Establish and maximise onsets in accordance with \isi{syllable} templates (See constraint number 2 in {\S}\ref{syllstruc} on onset clusters). A phonological rule will insert a \isi{glottal stop} if there is no consonantal onset in word-initial position ({\S}\ref{glottal-stop-insertion-section}).
	\item Break-up unsyllabified consonants with epenthetic vowels:
	\begin{enumerate}
		\item Exception: suffixes which allow no other \isi{syllabification} than inserting the \isi{epenthetic vowel} in final position. This includes the \isi{adjectivaliser} \emph{-thé}, non-singular ergative case marker \emph{-yé} and the first singular actor verb suffix \emph{-é}.
		\item Elsewhere: proceed from right to left breaking up \isi{consonant clusters}.
		\item After each \isi{schwa} insertion, establish codas in accordance with possible heterosyllabic \isi{consonant clusters}. Otherwise, maximise onsets. Exception: word-initial segments are always recognised as onsets.
		\item The \isi{epenthetic vowel} is [ŭ] and [ı̆] if followed by heterosyllabic /w/ and /j/, respectively. In all other instances it is [ə̆].
	\end{enumerate}
\end{enumerate}

The process of \isi{syllabification} attempts to map the minimal \isi{syllable} CV onto the underlying representation. The rules give preference to onsets rather than codas. Consequently, we do not find vowel-initial syllables word-medially or word-finally.

I have modelled the process of \isi{syllabification} as being divided into two steps. Syllables which contain full vowels are recognised first. In a second step epenthetic vowels are inserted to break up unsyllabified \isi{consonant clusters}. This algorithm proceeds from right to left and inserts epenthetic schwas between unsyllabified consonants to create \isi{syllable} nuclei. The insertion ensures that onsets are maximised. After each onset, the processs checks against the list of possible heterosyllabic \isi{consonant clusters} ({\S}\ref{heterosyllabiccc}) whether another insertion occurs right away or only after a coda has been recognised. In the latter case, it ``jumps'' one consonant and breaks up the next pair of unsyllabified consonants. An exception is the word-initial position, where the segment is automatically recognised as an onset. The rules ensure that no word-initial \isi{schwa} insertion occurs. The direction (right to left) explains why we never find \isi{schwa} in word-final position. There are only a handful of lexemes in which \isi{schwa} is attested word-finally, for example \emph{kayé} [kajə̆] `yesteray|tomorrow'.

The direction is important in order to explain forms like \emph{wonrsoknwr} [wɞ̆nə̆ɾsɔkə̆nwə̆ɾ]\footnote{The allophone [ɞ̆] of the phoneme /o/ occurs here not because this might be a closed syllable, but because it follows a labio-velar approximant ({\S}\ref{phonetic-description-vowels})} `s/he is bothering me' which is syllabified as \emph{wo.nr.so.kn.wr}. The algorithm is applied from right to left. This is why the cluster \emph{r.s} is first recognised as a possible heterosyllabic consonant cluster. Next, \isi{schwa} is inserted to form the syllable [nə̆ɾ]. If the process was applied from left to right, one would expect that \emph{n.r} is first recognised as a possible heterosyllabic cluster and \isi{schwa} would be inserted to form the syllable [ɾə̆], which yields the incorrect form \textsuperscript{$\ast$}\emph{won.r.so.kn.wr}. There is some degree of optionality. For example, informants accepted \isi{schwa} insertion in both places [wɞ̆nə̆ɾə̆sɔkə̆nwə̆ɾ] in elicitation.\footnote{This might be an artefact introduced by elicitation, because in fluent speech this hardly ever occurs.}

The algorithm specifies that \isi{schwa} is inserted between consonants disregarding possible onset clusters ({\S}\ref{syllstruc}), whereas syllables with specified vowels maximise their onsets and produce onset clusters. Indeed, we do not find the possible onset clusters Cr or Cw with epenthetic vowels. There are only two exceptions for Cr. The first is the verb \emph{frm.nz.si} `fix, prepare', in which the onset cluster /fr/ is never broken up even if the verb is fully inflected, as in \emph{ya.frm.nzr} `s/he prepares him'. The second exception occurs with all verbs in a specific inflection: Word-initially, the irrealis prefix \emph{ra-} becomes part of an onset cluster with the \isi{undergoer} prefix. This syllable usually contains a specified vowel, for example in \emph{thra-} ({\Stnsg}) or \emph{kwra-} ({\Fsg}). However, when the restricted verb stem is used, dual marking is encoded in the vowel of the syllable. The dual value is encoded by a zero-morpheme, as in \emph{thr.th.bth} [ðɾə̆ðə̆ᵐbə̆θ] `they (2) put them inside'.\footnote{This verb is glossed as: th-r-\Zero{}-thb-th \Stnsg-\Irr-\Ndu-put.inside.{\Rs}-\Stnsg{} It it a rare inflection because three things have to come together: irrealis mood, restricted verb stem, dual number marker (which is a zero-morpheme in this case).} In this inflection, the Cr cluster is never broken up.

In Figures \ref{syll001}-\ref{syll003}, I present four examples spelling out the algorithm step by step.

\begin{figure}
\caption{Syllabification of \emph{kwark} `deceased'}
\label{syll001}
\begin{mdframed}[linewidth=.5mm]
$\begin{array}{l}
	\parbox{3cm}{/kwark/} \parbox{5cm}{underlying representation}\\
	\parbox{3cm}{\hfill}\parbox{4cm}{\centering $\downarrow$}\\
	\parbox{3cm}{/kw\textsubscript{$\sigma$}[\uline{a}]rk/} \parbox{15cm}{Rule 1: Associate each specified vowel with a nucleus.} \\
	\parbox{3cm}{\hfill}\parbox{4cm}{\centering$\downarrow$}\\
	\parbox{3cm}{/\textsubscript{$\sigma$}[\uline{kwa}]rk/} \parbox{15cm}{Rule 2: Maximise onsets.} \\
	\parbox{3cm}{\hfill} \parbox{15cm}{$\rightarrow$ establishes the \isi{syllable} \textsubscript{$\sigma$}[kwa]} \\
	\parbox{3cm}{\hfill}\parbox{4cm}{\centering$\downarrow$}\\
	\parbox{3cm}{/\textsubscript{$\sigma$}[kwa]\textsubscript{$\sigma$}[\uline{rk}]/} \parbox{10cm}{Rule 3b: Break up consonant clusters.} \\
	\parbox{3cm}{\hfill} \parbox{10cm}{$\rightarrow$ \isi{schwa} is inserted between /r/ and /k/ and creates a} \\
	\parbox{3cm}{\hfill} \parbox{10cm}{CVC syllable} \\
	\parbox{3cm}{\hfill}\parbox{4cm}{\centering$\downarrow$}\\
	\parbox{3cm}{/kwa.rk/} 	\parbox{12cm}{syllabified form: [kʷaɾə̆k]}
\end{array}$
\end{mdframed}
\end{figure}%kwark

\begin{figure}
\caption{Syllabification of \emph{yanthugwr} `s/he tricks him here'}\label{syll002}
\begin{mdframed}[linewidth=.5mm]
$\begin{array}{l}
	\parbox{3,5cm}{/yanthugwr/} \parbox{5cm}{underlying representation}\\
	\parbox{3,5cm}{\hfill}\parbox{4cm}{\centering$\downarrow$}\\
	\parbox{3,5cm}{/y\textsubscript{$\sigma$}[\uline{a}]nth\textsubscript{$\sigma$}[\uline{u}]gwr/} 	\parbox{9cm}{Rule 1: Associate each specified vowel with a nucleus.} \\
	\parbox{3,5cm}{\hfill}\parbox{4cm}{\centering$\downarrow$}\\
	\parbox{3,5cm}{/\textsubscript{$\sigma$}[\uline{ya}]n\textsubscript{$\sigma$}[\uline{thu}]gwr/} 	\parbox{9cm}{Rule 2: Maximise onsets.} \\
	\parbox{3,5cm}{\hfill} 	\parbox{9cm}{$\rightarrow$ establishes the syllables \textsubscript{$\sigma$}[ya] and \textsubscript{$\sigma$}[thu]} \\
	\parbox{3,5cm}{\hfill}\parbox{4cm}{\centering$\downarrow$}\\
	\parbox{3,5cm}{/\textsubscript{$\sigma$}[ya]n\textsubscript{$\sigma$}[thu]g\textsubscript{$\sigma$}[\uline{wr}]/} 	\parbox{9cm}{Rule 3b: Break up consonant clusters.} \\
	\parbox{3,5cm}{\hfill} 	\parbox{9cm}{$\rightarrow$ \isi{schwa} is inserted between /w/ and /r/}\\
	\parbox{3,5cm}{\hfill}\parbox{4cm}{\centering$\downarrow$}\\
	\parbox{3,5cm}{/\textsubscript{$\sigma$}[ya]n\textsubscript{$\sigma$}[thu\uline{g}]\textsubscript{$\sigma$}[\uline{w}r]/} 	\parbox{9cm}{Rule 3c: Establish codas.} \\
	\parbox{3,5cm}{\hfill} 	\parbox{9cm}{$\rightarrow$ /g.w/ is possible} \\
	\parbox{3,5cm}{\hfill} 	\parbox{9cm}{$\rightarrow$ /{\ᵑ}g/ becomes a coda of the preceding syllable} \\
	\parbox{3,5cm}{\hfill}\parbox{4cm}{\centering$\downarrow$}\\
	\parbox{3,5cm}{/\textsubscript{$\sigma$}[ya\uline{n}]\textsubscript{$\sigma$}[\uline{th}ug]\textsubscript{$\sigma$}[wr]/} 	\parbox{9cm}{Rule 3c: Establish codas.}\\
	\parbox{3,5cm}{\hfill} 	\parbox{9cm}{$\rightarrow$ /n.th/ is possible}\\
	\parbox{3,5cm}{\hfill} 	\parbox{9cm}{$\rightarrow$ /n/ becomes coda of the preceding syllable} \\
	\parbox{3,5cm}{\hfill}\parbox{4cm}{\centering$\downarrow$}\\
	\parbox{3,5cm}{/yan.thug.wr/} 	\parbox{9cm}{syllabified form: [janðu{\ᵑ}gwə̆ɾ]}
\end{array}$
\end{mdframed}
\end{figure}%yanthugwr

\begin{figure}
\vspace{-0,7cm}
\caption{Syllabification of \emph{zwäfiyokwé} `I finished sth. for her'}\label{syll004}
\begin{mdframed}[linewidth=.5mm]
$\begin{array}{l}
	\vspace{0,1cm}
	\parbox{4,3cm}{/zwäfiyokw\textsubscript{$\sigma$}[é]/} \parbox{8,7cm}{underlying representation: final \isi{schwa} (\Fsg{}) is}\vspace{-0,1cm}\\
	\parbox{4,3cm}{\hfill} \parbox{8,7cm}{prespecified as nucleus}\vspace{-0,1cm}\\
	\vspace{0,1cm}
	\parbox{4,3cm}{\hfill}\parbox{4cm}{\centering$\downarrow$}\vspace{-0,1cm}\\
	\vspace{0,1cm}
	\parbox{4,3cm}{/zw\textsubscript{$\sigma$}[\uline{ä}]f\textsubscript{$\sigma$}[\uline{i}]y\textsubscript{$\sigma$}[\uline{o}]kw\textsubscript{$\sigma$}[é]/} 	\parbox{8,7cm}{Rule 1: Associate each specified vowel with a} \\
	\parbox{4,3cm}{\hfill} 	\parbox{8,7cm}{nucleus.} \\
	\vspace{0,1cm}
	\parbox{4,3cm}{\hfill}\parbox{4cm}{\centering$\downarrow$}\vspace{-0,1cm}\\
	\vspace{0,1cm}
	\parbox{4,3cm}{/\textsubscript{$\sigma$}[\uline{zwä}]\textsubscript{$\sigma$}[\uline{fi}]\textsubscript{$\sigma$}[\uline{yo}]k\textsubscript{$\sigma$}[\uline{wé}]/} 	\parbox{8,7cm}{Rule 2: Maximise onsets.}\vspace{-0,1cm}\\
	\parbox{4,3cm}{\hfill} 	\parbox{8,7cm}{$\rightarrow$ establishes: \textsubscript{$\sigma$}[zwä], \textsubscript{$\sigma$}[fi], \textsubscript{$\sigma$}[yo], \textsubscript{$\sigma$}[wé]}\vspace{-0,1cm}\\
	\vspace{0,1cm}
	\parbox{4,3cm}{\hfill}\parbox{4cm}{\centering$\downarrow$}\vspace{-0,1cm}\\
	\vspace{0,1cm}
	\parbox{4,3cm}{/\textsubscript{$\sigma$}[zwä]\textsubscript{$\sigma$}[fi]\textsubscript{$\sigma$}[yo\uline{k}]\textsubscript{$\sigma$}[\uline{w}é]/} 	\parbox{8,7cm}{Rule 3c: Establish codas.}\vspace{-0,1cm} \\
\parbox{4,3cm}{\hfill}\parbox{8,7cm}{$\rightarrow$ /k.w/ is possible} \\
\parbox{4,3cm}{\hfill}\parbox{8,7cm}{$\rightarrow$ /k/ becomes coda of the preceding syllable} \\
	\vspace{0,1cm}
	\parbox{4,3cm}{\hfill}\parbox{4cm}{\centering$\downarrow$}\vspace{-0,1cm}\\
	\vspace{0,1cm}
	\parbox{4,3cm}{/zwä.fi.yok.wé/} 	\parbox{8,7cm}{syllabified form: [tswæɸıjɔkwə̆]}
\end{array}$
\end{mdframed}
\end{figure}%zwäfiyokwé

\begin{figure}
\caption{Syllabification of \emph{skrifzenz} `Skri's wife'}\label{syll003}
\begin{mdframed}[linewidth=.5mm]
$\begin{array}{l}
	\vspace{0,1cm}
	\parbox{3,3cm}{/skrifzenz/} \parbox{5cm}{underlying representation}\\
	\vspace{0,1cm}
	\parbox{3,3cm}{\hfill}\parbox{4cm}{\centering$\downarrow$}\\
	\vspace{0,1cm}
	\parbox{3,3cm}{/skr\textsubscript{$\sigma$}[\uline{i}]fz\textsubscript{$\sigma$}[\uline{e}]nz/} 	\parbox{9,5cm}{Rule 1: Associate each specified vowel with a nucleus.} \\
	\vspace{0,1cm}
	\parbox{3,3cm}{\hfill}\parbox{4cm}{\centering$\downarrow$}\\
	\vspace{0,1cm}
	\parbox{3,3cm}{/s\textsubscript{$\sigma$}[\uline{kri}]f\textsubscript{$\sigma$}[\uline{ze}]nz/} 	\parbox{9,5cm}{Rule 2: Maximise onsets.} \\
	\parbox{3,3cm}{\hfill} 	\parbox{9,5cm}{$\rightarrow$ establishes: \textsubscript{$\sigma$}[\uline{kri}], \textsubscript{$\sigma$}[\uline{ze}]} \\
	\vspace{0,1cm}
	\parbox{3,3cm}{\hfill}\parbox{4cm}{\centering$\downarrow$}\\
	\vspace{0,1cm}
	\parbox{3,3cm}{/s\textsubscript{$\sigma$}[kri]f\textsubscript{$\sigma$}[ze\uline{nz}]/} 	\parbox{9,5cm}{Rule 3c: Establish codas.} \\
	\parbox{3,3cm}{\hfill} 	\parbox{9,5cm}{$\rightarrow$ no cluster with /{ⁿ}dz/} \\
	\parbox{3,3cm}{\hfill} 	\parbox{9,5cm}{$\rightarrow$ /{ⁿ}dz/ becomes the coda of the preceding syllable} \\
	\vspace{0,1cm}
	\parbox{3,3cm}{\hfill}\parbox{4cm}{\centering$\downarrow$}\\
	\vspace{0,1cm}
	\parbox{3,3cm}{/s\textsubscript{$\sigma$}[kri\uline{f}]\textsubscript{$\sigma$}[\uline{z}enz]/} 	\parbox{9,5cm}{Rule 3c: Establish codas.} \\
	\parbox{3,3cm}{\hfill} 	\parbox{9,5cm}{$\rightarrow$ /f.z/ is possible} \\
	\parbox{3,3cm}{\hfill} 	\parbox{9,5cm}{$\rightarrow$ /ɸ/ becomes the coda of the preceding \isi{syllable}.} \\
	\vspace{0,1cm}
	\parbox{3,3cm}{\hfill}\parbox{4cm}{\centering$\downarrow$}\\
	\vspace{0,1cm}
	\parbox{3,3cm}{/\textsubscript{$\sigma$}[\uline{s}]\textsubscript{$\sigma$}[\uline{k}rif]\textsubscript{$\sigma$}[zenz]/} 	\parbox{9,5cm}{Rule 3b: Break up consonant clusters.} \\
	\parbox{3,3cm}{\hfill} 	\parbox{9,5cm}{$\rightarrow$ \isi{schwa} is inserted between /s/ and /k/} \\
	\vspace{0,1cm}
	\parbox{3,3cm}{\hfill}\parbox{4cm}{\centering$\downarrow$}\\
	\vspace{0,1cm}
	\parbox{3,3cm}{/s.krif.zenz/} 	\parbox{9,5cm}{syllabified form: [sə̆kɾiɸtʃeⁿts]}
\end{array}$
\end{mdframed}
\end{figure}%skrifzenz

\subsection{Minimal word} \label{minwordconstraints}

We find some constraints on the minimal size of a word in Komnzo. I describe this here, because the \isi{minimal word} helps to explain a number of phenomena. It has an impact on allophonic variation of /o/ ({\S}\ref{phonetic-description-vowels}), vowel length in general, and \isi{epenthesis}.

Compared to polysyllables, monosyllabic roots have a slightly longer vowel if the syllable is closed, and a very long vowel if they consist of an open \isi{syllable}. This is relevant for roots with specified vowels only, not for roots with an \isi{epenthetic vowel}. Three examples are: \emph{fk} [ɸə̆k] `buttocks', \emph{fäk} [ɸæk] `jaw', and \emph{fä} [ɸæ:] `there ({\Dist})'. In moraic theory, we could rephrase the \isi{minimal word} constraint as: ``Words with specified vowels need to be at least two morae long''.

We saw in {\S}\ref{phonetic-description-vowels} that the phoneme /o/ has two allophones: a short centralised rounded vowel [ɞ̯], which occurs in closed syllables, and a rounded back vowel [ɔ], which occurs in open syllables. I employed this phenomenon in {\S}\ref{syllstruc} to justify the need of \isi{syllable} weight as a concept. As for the phoneme /o/, in monosyllabic roots the difference between these \isi{syllable} types is suspended and we do find [ɔ] in closed syllables, as in \emph{gon} [{\ᵑ}gɔn] `hips' or \emph{rot} [ɾɔt] `fence type'. Thus, the \isi{minimal word} constraint overrides these allophonic rules. The constraint applies at the root level and not the level of the inflected word. For example, we find [ɔ] instead of [ɞ̆] in the verb \emph{thorsi} [ðɔɾsi] `put inside' because \emph{thorsi} is multimorphemic (\emph{thor-} `put inside' + \emph{-si} {\Nmlz}). With polysyllabic roots, this is not the case and the two variants of /o/ follow the allophonic rule as was layed out in {\S}\ref{phonetic-description-vowels}. An example is: \emph{thomonsi} [ðɔmɞ̯nsi] `pile up firewood', which consists the stem \emph{thomon-} and the nominaliser \emph{-si}.

The \isi{minimal word} constraint impacts on \isi{syllabification} because there are two variants for monosyllabic roots of the type CrV(C). These kinds of roots may be realised with a lengthened vowel in the nucleus. Alternatively, an \isi{epenthetic vowel} may be inserted to break up the onset cluster thus creating a disyllabic form. In this case the specified vowel is of normal length and \isi{stress} does not shift to the initial \isi{epenthetic vowel} but remains with the specified vowel. Examples are: \emph{srak} [ˈsɾak] $\sim$ [sə̆ˈɾak] `boy' and \emph{zra} [ˈtsɾa:] $\sim$ [tsə̆ˈɾa] `swamp'.

\subsection{Stress} \label{stress}

Stress is a syllable-level phenomenon in Komnzo. A stressed \isi{syllable} is marked by higher intensity and sometimes higher pitch. Vowel duration is not an acoustic correlate of \isi{stress}. The \isi{epenthetic vowel} [ə̆] is frequently stressed. That being said, specified vowels usually become more centralised and shortened in word-final position, which is always unstressed.

The prosodic domain of stress asignment is the phonological word. Primary \isi{stress} (marked by preceding ˈ in the examples) is assigned to the initial \isi{syllable} of a word. There are a number of exceptions to initial \isi{stress} which I will describe below. Secondary \isi{stress} (marked by preceding ˌ in the examples) carries little function in Komnzo and it is often hard to distinguish from unstressed syllables. Secondary \isi{stress} only occurs in words with more than three syllables. Only few roots have more than three syllables and none have more than four. An example of a four-\isi{syllable} root is \emph{ngemäku} [ˈnə̆{\ᵑ}geˌmæku] `term of address between foster parent and real parent'. It follows, that all words with more than four syllables are polymorphemic. For example, inflected verbs often comprise more than four syllables, as in \emph{kwamnzokwrmth} [ˈkʷamⁿdzɞ̆kˌwə̆rə̆mə̆θ] `they were dancing.'

There are some exceptions to initial \isi{stress}. For example, in partial \isi{reduplication} ({\S}\ref{nomreduplication}) the first \isi{syllable} is unstressed, as in \emph{rrokar} [rə̆ˈrokar] `things'. In full \isi{reduplication}, we find initial \isi{stress} \emph{rokarrokar} [ˈrokarˌrokar] as with the corresponding singleton form \emph{rokar} [ˈrokar]. Another example comes from verbs with a \isi{proclitic}. In the verb form \emph{bŋatrakwr} [bə̆ˈŋatrakʷə̆r] `s/he falls there', the \isi{proclitic} \emph{b=} (\Med{}) attaches to the outer layer of the fully inflected verb. Cases like partial \isi{reduplication} and verbal proclitics should be seen as exceptions to the rule of initial \isi{stress}.

\newpage 
Stress is assigned from left to right. Words with two, three, and four syllables construct a trochee, dactyl, and ditrochee, respectively. In \tabref{stresspattern}, I present templatic \isi{stress} patterns for words between two and four syllables of length.

\begin{table}
\caption{Stress patterns of words with two to four syllables}
\label{stresspattern}
	\begin{tabularx}{\textwidth}{lXXl}
		\lsptoprule
		 {syllable} {structure}&{example}&{phonetic}&{gloss}\\
		\midrule
		ˈ$\sigma$$\sigma$& \emph{nzäthe} &[ˈⁿdʒæðe]& `namesake'\\
		&\emph{ebar}& [ˈʔeᵐbaɾ]& `head'\\
		&\emph{nzrm}& [ˈⁿdʒə̆rə̆m]& `flower'\\
		&&&\\
		ˈ$\sigma$$\sigma$$\sigma$& \emph{kafara} &[ˈkaβaɾa]& `river pandanus'\\
		&\emph{bägwrm}& [ˈbæ{\ᵑ}gʷə̆rə̆m]& `butterfly'\\
		&\emph{krbu}& [ˈkə̆rə̆ᵐbu]& `swelling'\\
		&&&\\
		ˈ$\sigma$$\sigma$ˌ$\sigma$$\sigma$& \emph{nänzüthzsi} &[ˈnæⁿdʒʏθˌtsə̆si]& `cover with soil/mud'\\
		&\emph{kukufasi}&[ˈkukuˌɸasi]&`Grey Shrike-trush'\\
		&\emph{kdewawa}&[ˈkə̆ⁿdeˌwawa]&`firefly'\\
		\lspbottomrule
	\end{tabularx}
\end{table}%Stress patterns of words with two to four syllables

Words with more than four syllables vary in their assignment of secondary \isi{stress}. Most five-\isi{syllable} words assign secondary \isi{stress} to the third \isi{syllable}, but some assign it to the fourth. Most six-\isi{syllable} and seven-\isi{syllable} words assign secondary \isi{stress} to the fourth \isi{syllable}, but there are also exceptions. The variation of stress assignment in words with more than four syllables might be explained in terms of open vs. closed syllables, or in terms of specified vs. \isi{epenthetic vowel} nucleus. The nature of secondary \isi{stress} in Komnzo remains to be investigated in more detail.

\section{Morphophonemic Processes} \label{morphophonology}

The following section addresses \isi{morphophonemic processes} which occur through affixation or cliticisation.

\subsection{Vowel harmony after \emph{=wä}} \label{vowharmwae}

Effects of vowel harmony can be found with the emphatic clitic \emph{=wä}. Encliticisation of \emph{=wä} causes a change in the quality of the vowel of the preceding \isi{syllable} regardless whether this \isi{syllable} is part of the root or another suffix or enclitic. Depending on the vowel quality its impact can be described as fronting or rounding. Some examples are given in \tabref{vowelharmwae}.

\begin{table}
\caption{Vowel harmony caused by \emph{=wä}}
\label{vowelharmwae}
	\begin{tabularx}{\textwidth}{Xll}
		\lsptoprule
		 {process}&{example}& {example with} \emph{=wä} \\ \midrule
		fronting of /o/&\emph{karfo} `to the village' & \emph{kar=fö=wä} \\
		&&village={\Abl}={\Emph}\\
		&\emph{bobo} `towards there' & \emph{bobö=wä}\\
		&&\Med{}.{\All}={\Emph}\\
		&&\\
		raising of /a/&\emph{nima} `this way' & \emph{nimä=wä}\\
		&&like.this={\Emph}\\
		&\emph{bafanema} `because of that one' & \emph{baf=ane=mä=wä}\\
		&&\Recog=\Poss=\Char={\Emph}\\
		&&\\
		rounding of /e/&\emph{zafe} `long ago' & \emph{zafö=wä}\\
		&&long.ago=\Emph\\
		&\emph{etfthme} `overnight' & \emph{etfth=mö=wä}\\
		&&sleep=\Ins=\Emph\\
		\lspbottomrule
	\end{tabularx}
\end{table}	%Vowel harmony caused by \emph{=wä}

The \isi{vowel harmony} does not affect vowels in a closed \isi{syllable}: \emph{kafarwä} `really big' not \textsuperscript{$\ast$}\emph{kafärwä} or \emph{dö kerwä} `really the lizard tail' not \textsuperscript{$\ast$}\emph{dö körwä}. The process is blocked by two intervening consonants. Vowel harmony of this type is restricted to morphophonemics because we do find lexemes where the vowels in question occurs in adjacent syllables, as in \emph{namä} `good' or \emph{dowä} `Wompoo Fruit Dove'.

\subsection{Dissimilation between prefix and verb stem} \label{vowelharmverbstem}

We find a number of verb stems in which the vowel quality of the prefix is raised from /æ/ to /e/. This occurs only in inflections which build on the \isi{restricted stem}, i.e. it is the prefix vowel which encodes the dual versus non-dual contrast. The vowel /æ/ marks usually \isi{non-dual}, whereas /a/ or \isi{zero} mark dual number. See {\S}\ref{roots-and-temp} for stem types and {\S}\ref{prerootdual} for a description of pre-stem dual marking. Dissimilation targets the non-dual /æ/ and raises it to /e/. The trigger is the first vowel of the verb stem. Raising takes place when the first vowel is either /a/ or /æ/; for two verb stems it is /œ/. Some examples are: \emph{mar-} `see', \emph{far-} `set off', \emph{faf-} `hold' and \emph{wär-} `crack, happen', \emph{rä-} `be, do', \emph{räs-} `erect', \emph{söbäth-} `ascend' and \emph{sörfäth-} `descend'.\footnote{The majority of Komnzo verbs have two verb stems, a restricted and an extended stem (\S\ref{roots-and-temp}). I list the restricted stems here, because the first vowel of the stem is relevant here. Elsewhere in this grammar, I use the extended stem or the nominalisation to refer to verbs. Therefore, I provide the respective extended verb stems here: \emph{mar-} `see', \emph{fark-} `set off', \emph{fa-} `hold', \emph{wä-} `crack, happen', \emph{rä-} `be', \emph{räz-} `erect', \emph{mrä-} `stroll', \emph{thfä-} `jump', \emph{thkäfak-} `start', \emph{sog-} `ascend', \emph{rsör-} `descend'.} Thus, for verbs like \emph{marasi} the non-dual of a \isi{recent past} perfective is not realised as \textsuperscript{$\ast$}\emph{zämar} but \emph{zemar} `he looked at himself'. Depending on \isi{syllabification} and intervening prefixes, the trigger vowel in the verb stem and the prefix can be separated by another \isi{syllable}. In most cases, this is a \isi{syllable} created by \isi{epenthesis}. Verb stems like \emph{mräs-} `stroll', \emph{thfär-} `jump' and \emph{thkäf-} `start' have an \isi{epenthetic vowel} after the first consonant in their nominalisations, for example \emph{m.rä.z.si} `stroll'. In the inflected verb form, the initial consonant is syllabified as a coda: \emph{zemräs} `he strolled around'. If the \isi{venitive} prefix \emph{n-} is added to the inflection, the trigger vowel and prefix vowel are separated by another \isi{syllable}, but the raising still takes place: \emph{ze.nm.räs} `he strolled towards here'. The raising pattern described here applies to inflections of various TAM categories (irrealis, imperatives, iteratives). They all share the use of the \isi{restricted stem} and, consequently the encoding of duality takes place in the vowel of the prefix.

A special case is the \isi{copula} \emph{rä-}. Although highly irregular in many ways, it follows the dissilimation pattern just described. What is special about the copula is that the \isi{past} suffix \emph{-a} triggers the same kind of raising in the stem of the copula. Thus, we find \emph{erera} `they were' instead of \textsuperscript{$\ast$}\emph{erära}. Without the the past suffix, raising takes not place: \emph{erä} `they are'.

Raising of the prefix vowel is a morphophonemic process, not a general phonological process. For example, we do find lexemes where /æ/ and /a/ occur in adjacent syllables, as in (\emph{atätö} `tree species', \emph{mätraksi} `bring out'). The same is true for /æ/ and /æ/ in adjacent syllables, as in (\emph{krätär} `tree species', \emph{thäfäm} `ripples'). Moreover, the /æ/ vowel is not raised to /e/ in verb inflections that build on the \isi{extended stem}. Consider the \Stnsg{} \emph{e-} and the \Tsg.\F{} \emph{w-} of the alpha \isi{prefix series} (\S\ref{personprefsection}). The \isi{valency} changing prefix \emph{a-} follows in the next slot and it merges with these two prefixes, i.e. they are realised as \emph{ä-} and \emph{wä-}, respectively. However, the /æ/ vowel in the prefixes is not raised to \emph{e-} if the first vowel of the stem is /æ/. For example, the verb \emph{fänzsi} `show' is realised as \emph{wäfänzr} `s/he shows her' and not \textsuperscript{$\ast$}\emph{wefänzr}. One reason for this might be that raising the vowel to /e/ would neutralise the \isi{valency} changing prefix \emph{a-}. Another explanation might be that the raising pattern developed together with pre-stem dual marking, which is found only with \isi{restricted stem} ({\S}\ref{prerootdual}). Restricted stems in turn do not combine with the prefixes of the alpha series ({\S}\ref{tamprefixseries}), which explains why these are not affected.

\subsection[Approximant  ↔ high vowel]{Approximant $\leftrightarrow$ high vowel} \label{approxhighvowel}

In two different parts of the verbal inflectional paradigm, a change from the approximants to high vowels ([w] $\rightarrow$ [u]/[ü], and [y] $\rightarrow$ [i]) and the reverse is found.

All of the verbal proclitics consist only of a consonant, e.g. the \isi{immediate past} \emph{n=} or the three \isi{deictic} proclitics \emph{z=} {\Prox}, \emph{b=} \Med{}, and \emph{f=} {\Dist}. These are cliticised to otherwise fully inflected verbs. In most cases, this creates an extra \isi{syllable} word-initially, as in \emph{b.ŋa.trak.wr} `s/he falls there'. Some of the verb prefixes in the alpha series begin with an \isi{approximant}, for example (\emph{wo-} \Fsg{}, \emph{w-} \Tsg.\F{}, and \emph{y-} \Tsg.\Masc{}). When the clitics are attached to these prefixes, the approximants are realised as high vowels: \emph{u-} \Fsg{}, \emph{ü-} \Tsg.\F{}, and \emph{i-} \Tsg.\Masc{}. A few examples are given in (\ref{approx-vow-2}-\ref{approx-vow-1}).

\begin{exe}
	\ex \emph{burera}\\
	\gll b=wo-rä-ra\\
	\Med{}=\Fsg.\Alph-\Cop.{\Ndu}-\Pst{}\\
	\trans `I was there.'
	\label{approx-vow-2}

\newpage 
	\ex \emph{zimithgr}\\
	\gll z=y-mi-thgr\\
	{\Prox}=\Tsg.\Masc.\Alph-hang-\Stat.{\Ndu}\\
	\trans `It hangs here.'
	\label{approx-vow-3}

	\ex \emph{zürugr}\\
	\gll z=w-rugr\\
	{\Prox}=\Tsg.\F.\Alph-sleep.{\Ndu}\\
	\trans `She sleeps here.'
	\label{approx-vow-1}
\end{exe}

Another change which involves high vowels and approximants is attested only for [u] $\leftrightarrow$ [w]. The formatives of the beta-2 prefix series (\Betatwo) end in a [u] vowel, for example \emph{ku-} \Fsg{}, \emph{su-} \Tsg.\Masc{}, \emph{thu-} \Stnsg{}. The \isi{valency} changing prefix \emph{a-} occurs in the following slot, for example \emph{ku-a-} `for me', \emph{su-a-} `for him', \emph{thu-a-} `for you/them'. In its presence, the [u] vowel becomes part of an onset consonant cluster and is realised as a high back \isi{approximant} [w]. An example is given in (\ref{approx-vow-4}-\ref{approx-vow-5}).

\begin{exe}
	\ex \emph{thufsinzr}\\
	\gll thu-fsi-nzr-\Zero{}\\
	\Stnsg{}.\Betatwo{}-count.{\Ext}-{\Ndu}-\Stsg{}\\
	\trans `S/he counted them.'
	\label{approx-vow-4}

	\ex \emph{thwafsinzr}\\
	\gll thu-a-fsi-nzr-\Zero{}\\
	\Stnsg{}.\Betatwo{}-{\Vc}-count.{\Ext}-{\Ndu}-\Stsg{}\\
	\trans `S/he counted for them.'
	\label{approx-vow-5}
\end{exe}

\section{Loanwords and loanword phonology} \label{loanword-phonology}

A number of speech sounds are restricted to \isi{loanword}s. These are the voiced oral stops [b], [d], and [g], the \isi{lateral} \isi{approximant} [l] and a few diphthongs. The ``donor languages'' of almost all loanwords found in Komnzo are either \ili{English} or \ili{Hiri Motu}. Only few loanwords come from \ili{Bahasa} Indonesia, for example the terms for introduced fish species: \emph{ikan lele} `Clarias batrachas', \emph{mujair} `Oreochromis mossambicus', \emph{gastor} `Channa striata'. An increasing number of people start to learn the third offical language of Papua New Guinea - Tok Pisin - and sometimes expressions like \emph{maski} `nevermind' can be heard among younger Komnzo speakers. Otherwise Tok Pisin plays only a minor role in loanwords.

From the degree of indigenisation of loanwords we can distinguish at least two periods: an early phase which lasted until the 1960s and a second phase from that time until today. The boundary between the two periods is rather fuzzy. The first period was characterised by \ili{English} speaking patrol officers and officials who visited the area for very short periods. The second period began with the opening of a Mission school in Rouku in the mid 1960s. At the beginning, the language of instruction was \ili{Hiri Motu}. In the 1970s the school was moved to Morehead and since then, the language of instruction is \ili{English}. We find linguistic evidence for the two periods. Loanwords from the first period have undergone indigenisation in order to adapt to Komnzo \isi{phonology}. Loans which entered the language during the second period are much closer to the original \ili{English} or Motu pronunciation. An example is the word \emph{doctor}. While it is pronounced [dokta] nowadays, some older speakers still use a second variant \emph{nzokta} [ⁿdzokta] which they report was common in their parent's and grandparent's generation.

Words from the first period are: \emph{frayn misin} [ɸɾajə̆n mısın] `plane, flying machine', \emph{kas raba} [kas ɾaᵐba] `gas lamp', \emph{dis} [ⁿdi:s] `dish, plate', \emph{damaki} [ⁿdamakı] `dynamite'. We find regular correspondences of \ili{English} phonemes mapping onto Komnzo \isi{phonology}. The bilabial stop [p] becomes a bilabial fricative [ɸ] in \emph{frayn misin}, but in a cluster with the bilabial \isi{nasal} [m] in \emph{kas raba} it becomes a prenasalised voiced bilabial stop [ᵐb]. The \isi{velar} voiced stop [g], also in \emph{kas raba}, comes out as a voiceless \isi{velar} stop [k]. The \isi{lateral} \isi{approximant} [l] in \ili{English} \emph{flying} becomes an \isi{alveolar} tap or trill [ɾ $\sim$ r] in Komnzo \emph{frayn} and again in \emph{kas raba}. The \ili{English} \isi{diphthong} [a͡i] in `dynamite' is monophthongised in \emph{damaki}. The voiced \isi{alveolar} stop [d] becomes prenasalised [ⁿd] in \emph{damaki} and \emph{dis}\footnote{There is no explanation for the change from English [t] > Komnzo [k] in \emph{damaki} `dynamite'.}. In the same word, the post-\isi{alveolar} fricative [ʃ] turns into an \isi{alveolar} fricative [s]. However, there are too few loans from this early period to make a systematic comparison of all \ili{English} phonemes in different environments.

The second period, which lasts until today, is characterised by loan phonemes. Indigenisation is found to a lesser degree. The second period is also characterised by the influx of loans from \ili{Hiri Motu}. We find loan phonemes in the oral voiced stops [b], [d] and [g], as in \emph{bara} `paddle', \emph{durua} `help', \emph{dibura} `prisoner', \emph{gunana} `place name'\footnote{\emph{Gunana} means `the former (one)' in Hiri Motu. In Komnzo, it designates a place `where old Rouku used to be' as informants put it. A new hamlet was founded there a few years ago.} from \ili{Hiri Motu}, and \emph{baisikol} `bicycle' from \ili{English}. Note that the \ili{English} \isi{diphthong} [a͡ı] is retained and not monophthongised and the \isi{lateral} \isi{approximant} [l] also does not change.

There are two correspondences which we find in both periods. The first is between the voiceless bilabial stop [p] in \ili{English} and the voiceless bilabial fricative [ɸ] in Komnzo. The second correspondence is between the \isi{lateral} \isi{approximant} [l] and the \isi{alveolar} trill/flap [ɾ $\sim$ r]. In the early period, [l] was changed in all environments, but in the second period this only occurs in [pl] clusters in \ili{English}. Elsewhere, [l] is taken over into Komnzo as a loan phoneme. We have seen some examples from the first period above. Examples from the second period are: \emph{fren} `plane', \emph{fenzil} `pencil', and \emph{sosfen} `saucepan'.

\section{Orthography development}\label{orthographydev}

There is no writing tradition in Komnzo, but most people can read and write in one of the official languages, namely \ili{English} and Motu. The mission school, which was based at Rouku during the 1960s, operated in Motu, but today \ili{English} is the teaching language at the primary school in Morehead. Thus, reading and writing in Komnzo has not been promoted in the past. As a consequence, literacy in one's mother tongue is an alien concept for most Komnzo speakers.

The first attempt to develop an \isi{orthography} for Komnzo was during an alphabet workshop organised by Marco and Alma Bouvé at Morehead Station in 2000.\footnote{The workshop was supported by the Summer Institute of Linguistics (SIL).} It brought together representatives from a dozen villages. The two representatives from Rouku were Greg Marua and Wendy Yasii. When I began my work in Rouku, this \isi{orthography} was not used except for a few words that were written on the blackboard in the elementary school. Regrettably, the Rouku elementary school has been dysfunctional since 2010. During my fieldwork I have organised two \isi{orthography} meetings. The outcome of these meetings was the Komnzo Language Council, which includes representatives of all clans. The language council has remained an abstract administrative body overseeing my work. In practice, I concentrated most translation and elicitation work on 4-5 interested individuals. Together, we have revised the \isi{orthography} several times. \tabref{orthogcons} and Figure \ref{orthogvow} show the differences between the \isi{orthography} from the workshop in 2000 and the current \isi{orthography}. Changes are shown with an arrow ($\rightarrow$).

\begin{table}
\caption{Comparison of orthographies: consonants}
\label{orthogcons}\small
	\begin{tabularx}{\textwidth}{Q ccc@{~}C@{~}cc@{~}c@{}}
		\lsptoprule
		& {bilabial}& {dental} & {alveolar} & {palato-alveolar}	& {palatal} & {velar} & {labio-velar} \\ \midrule
		{stop} \& && \multicolumn{2}{c}{}&&&&\\
		{affricate}	&b $\rightarrow$ $\square$& \multicolumn{2}{c}{t}& ts $\rightarrow$ z&& k & $\square$ $\rightarrow$ kw \\
		&&&&&&&\\
		{prenasalised} && &&&&&\\
		\mbox{stop \& affricate} & mb $\rightarrow$ b & & nt $\rightarrow$ d & nj $\rightarrow$ nz && n\th $\rightarrow$ g & $\square$ $\rightarrow$ gw\\
		&&&&&&&\\
		{fricative} 	& f	& th & s &&&&\\
		&&&&&&&\\
		{nasal} & m && n &&& ng $\rightarrow$ ŋ & \\
		&&&&&&&\\
		{lateral} &&& r &&&&\\
		&&&&&&&\\
		{semivowel} &&&&&y && w\\
		\lspbottomrule
	\end{tabularx}
\end{table}%Comparison of orthographies: consonants

\begin{figure}
\centering
{
\begin{vowel}[simple]
	\putvowel{i}{0,3\vowelhunit}{0,4\vowelvunit}
	\putvowel{ú}{1,3\vowelhunit}{0,4\vowelvunit}
	\putvowel{e}{0,85\vowelhunit}{1,5\vowelvunit}
	\putvowel{\^{o}}{1,7\vowelhunit}{1,5\vowelvunit}
	\putvowel{á}{1,6\vowelhunit}{2,55\vowelvunit}
	\putvowel{u}{4\vowelhunit}{0,4\vowelvunit}
	\putvowel{a}{2,9\vowelhunit}{2,7\vowelvunit}
	\putvowel{é}{2,7\vowelhunit}{1,3\vowelvunit}
	\putvowel{o}{4\vowelhunit}{1,5\vowelvunit}
	\putvowel{ó}{3,4\vowelhunit}{2,1\vowelvunit}
\end{vowel}
} $\longrightarrow$
{
\begin{vowel}[simple]
	\putvowel{i}{0,3\vowelhunit}{0,4\vowelvunit}
	\putvowel{ü}{1,3\vowelhunit}{0,4\vowelvunit}
	\putvowel{e}{0,85\vowelhunit}{1,5\vowelvunit}
	\putvowel{ö}{1,7\vowelhunit}{1,5\vowelvunit}
	\putvowel{ä}{1,6\vowelhunit}{2,55\vowelvunit}
	\putvowel{u}{4\vowelhunit}{0,4\vowelvunit}
	\putvowel{a}{2,9\vowelhunit}{2,7\vowelvunit}
	\putvowel{é}{2,7\vowelhunit}{1,3\vowelvunit}
	\putvowel{o}{4\vowelhunit}{1,5\vowelvunit}
\end{vowel}
}%
\caption{Comparison of orthographies: vowels} \label{orthogvow}
\end{figure}%Comparison of orthographies: vowels