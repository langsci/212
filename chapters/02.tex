% !TEX root =  ../main.tex

\chapter{Phonology} \label{cha:Phonology}
\vspace{-0,2cm}
In this chapter I describe the phonological system of Komnzo. The chapter begins with the segmental \isi{phonology} of consonants in \S{}\ref{consonant-segments} and vowels in \S{}\ref{vowelsegments}. Each section contains a list of \isi{minimal pair}s which establish the phonemic status of the segments. As Komnzo \isi{phonology} is characterised by widespread \isi{epenthesis}, a discussion of the non-phonemic status of \isi{schwa} is given in \S{}\ref{schwa-as-non-phoneme}. Regular phonological processes are described in \S{}\ref{regular-phon-processes}. I address Komnzo phonotactics in \S{}\ref{syllable-and-phonotactics}. This section consists of a description of the \isi{syllable} structure (\S{}\ref{syllstruc}), \isi{consonant clusters} (\S{}\ref{consonantclusters}), \isi{syllabification} (\S{}\ref{syllabificationandepenthesis}), \isi{minimal word} constraints (\S{}\ref{minwordconstraints}) and \isi{stress} (\S{}\ref{stress}). Morphophonology is addressed in \S{}\ref{morphophonology}. The chapter closes with a discussion of loanwords in \S{}\ref{loanword-phonology} and an account of the development of the \isi{orthography} in \S{}\ref{orthographydev}.
\vspace{-0,3cm}

\section{Consonant phonemes} \label{consonant-segments}

Table \ref{consinv} gives an overview of the consonant phonemes in Komnzo. %The corresponding graphemes are given in angled brackets.

\begin{table}[H]
\caption[Consonant phoneme inventory]{Consonant phoneme inventory}
\label{consinv}
	\begin{tabular}{cccccccc}
		\lsptoprule
		& \scriptsize{bilabial}& \scriptsize{dental} & \scriptsize{alveolar} & \scriptsize{palato-alveolar}	& \scriptsize{palatal} & \scriptsize{velar} & \scriptsize{labio-velar} \\ \midrule
		% \& && \multicolumn{2}{c}{}&&&&\\
		\scriptsize{stop}/\scriptsize{affr.}&& \multicolumn{2}{l}{t̪$\sim$t}& ts&& k & k\super{w}\\
		&& \multicolumn{2}{l}{\footnotesize{<t>}}& \footnotesize{<z>}&& \footnotesize{<k>} & \footnotesize{<kw>}\\
		\scriptsize{pren. stop} & \super{m}b && \super{n}d & \super{n}dz && \super{ŋ}g & \super{ŋ}g\super{w}\\
		\scriptsize{/affr.}&\footnotesize{<b>} && \footnotesize{<d>} & \footnotesize{<nz>} && \footnotesize{<g>} & \footnotesize{<gw>}\\\\
		\scriptsize{fricative} & ɸ & ð & s &&&&\\
		& \footnotesize{<f>} & \footnotesize{<th>} & \footnotesize{<s>} &&&&\\
		\scriptsize{nasal} & m && n &&& ŋ &\\
		& \footnotesize{<m>} && \footnotesize{<n>} &&& \footnotesize{<ŋ>} &\\
		\scriptsize{lateral} &&& r$\sim$ɾ &&&&\\
		&&& \footnotesize{<r>} &&&&\\
		\scriptsize{semivowel} &&&&& j && w\\
		&&&&& \footnotesize{<y>} && \footnotesize{<w>}\\
		\lspbottomrule
	\end{tabular}
\end{table} %phoneme inventory - consonants

\subsection{Obstruents} \label{obstruents}

Obstruents in Komnzo are divided into stops, affricates and \isi{fricatives}. The stops and affricates belong to a chain of pairings of oral and prenasalised phonemes at four places of articulation: \isi{alveolar}, palato-alvealor, \isi{velar} and labio-\isi{velar}. The symmetry is broken at the bilabial place of articulation. The bilabial oral stop is lacking from the phoneme inventory. Since it occurs only in \ili{English} loanwords and a handful of ideophones, I consider it a loan phoneme. As I will show below, the bilabial fricative /f/ can be regarded as the structural counterpart of the prenasalised bilabial stop.\\

In the following section, I describe the oral and prenasalised stops, labialised \isi{velar} stops, affricates and \isi{fricatives}.

\subsubsection{Stops} \label{stopss}

There are two voiceless stops (/t/ and /k/) and three prenasalised stops (/b/, /d/, and /g/). The voiceless stops are slightly aspirated, but aspiration is not phonemic in Komnzo. The two labialised \isi{velar} stops and the two affricates follow the same pairing of voiceless and prenasalised manner of articulation, but these will be discussed in separate sections below.\\

All stops occur in word-initial, medial, and final position. In only a small number of lexical items, the bilabial /b/ occurs word-finally. This phoneme is also deviant because it lacks a voiceless counterpart. There is evidence from \isi{loanword} \isi{phonology} (\S{}\ref{loanword-phonology}) and from surrounding \ili{Tonda} languages that the bilabial fricative /f/ occupies the same structural slot in the opposition of voiceless and prenasalised stops.\\

There is almost no allophonic variation with the stop series, but the prenasalised stops are affected by final \isi{devoicing} (\S{}\ref{final-devoicing-section}). The /t/ phoneme varies between dental and \isi{alveolar} points of articulation. In onset clusters where C\textsubscript{2} is /r/, /t/ is always \isi{alveolar}. Elsewhere, it varies more or less freely.

\begin{figure}[H]
  $\mbox{/t/ $\rightarrow$} \left\{
    \begin{array}{l}
      \parbox{4cm}{[t] / \textsubscript{$\sigma$}[\_ɾ} \parbox{2cm}{\emph{traksi}} \parbox{3cm}{[tɾakə̆si]} \parbox{3cm}{`fall'} \\
      \parbox{4cm}{\hfill} \parbox{2cm}{\hfill} \parbox{3cm}{\hfill} \parbox{3cm}{\hfill} \\
      \parbox{4cm}{\hfill} \parbox{2cm}{\emph{tüf}} \parbox{3cm}{[tʏɸ] $\sim$ [t̪ʏɸ]} \parbox{3cm}{`soft ground'} \\
	  \parbox{4cm}{[t]$\sim$[t̪] / elsewhere} \parbox{2cm}{\emph{rata}} \parbox{3cm}{[ɾata] $\sim$ [ɾat̪a]} \parbox{3cm}{`ladder'} \\
	  \parbox{4cm}{\hfill} \parbox{2cm}{\emph{kwot}} \parbox{3cm}{[k\super{w}ɔ̆t] $\sim$ [k\super{w}ɔ̆t̪]} \parbox{3cm}{`properly'} \\
    \end{array}
  \right.$
\end{figure}%t
\begin{figure}[H]
  $\mbox{/k/ $\rightarrow$} \left\{
    \begin{array}{l}
      \parbox{3,75cm}{\hfill} \parbox{2cm}{\emph{kata}} \parbox{3cm}{[kata]} \parbox{3cm}{`bamboo knife'} \\
	  \parbox{3,75cm}{[k]} \parbox{2cm}{\emph{fokam}} \parbox{3cm}{[ɸokam]} \parbox{3cm}{`grave'} \\
	  \parbox{3,75cm}{\hfill} \parbox{2cm}{\emph{safak}} \parbox{3cm}{[saβak]} \parbox{2,5cm}{`saratoga'} \\
    \end{array}
  \right.$
\end{figure}%k
\begin{figure}[H]
  $\mbox{/b/ $\rightarrow$} \left\{
    \begin{array}{l}
	  \parbox{3,75cm}{[\super{m}p] / \_]\textsubscript{$\sigma$}} \parbox{2cm}{\emph{gb}} \parbox{3cm}{[\super{ŋ}gə̆\super{m}p]} \parbox{3cm}{`black palm'} \\
      \parbox{4cm}{\hfill} \parbox{2cm}{\hfill} \parbox{3cm}{\hfill} \parbox{3cm}{\hfill} \\
      \parbox{3,75cm}{[\super{m}b] / elsewhere} \parbox{2cm}{\emph{bone}} \parbox{3cm}{[\super{m}bone]} \parbox{3cm}{\Ssg{}.\Poss{}} \\
	  \parbox{3,75cm}{\hfill} \parbox{2cm}{\emph{gaba}} \parbox{3cm}{[\super{ŋ}ga\super{m}ba]} \parbox{3cm}{`storage yam'} \\
    \end{array}
  \right.$
\end{figure}%b
\begin{figure}[H]
  $\mbox{/d/ $\rightarrow$} \left\{
    \begin{array}{l}
	  \parbox{3,75cm}{[\super{n}t] / \_]\textsubscript{$\sigma$}} \parbox{2cm}{\emph{kd}} \parbox{3cm}{[kə̆\super{n}t]} 	\parbox{3cm}{`star'} \\
      \parbox{4cm}{\hfill} \parbox{2cm}{\hfill} \parbox{3cm}{\hfill} \parbox{3cm}{\hfill} \\
      \parbox{3,75cm}{[\super{n}d] / elsewhere} \parbox{2cm}{\emph{deya}} \parbox{3cm}{[\super{n}deja]} \parbox{3cm}{`tree wallaby'} \\
	  \parbox{3,75cm}{\hfill} \parbox{2cm}{\emph{rdiknsi}} \parbox{3cm}{[ɾə̆\super{n}dikə̆nsi]} \parbox{3cm}{`tie around'} \\
    \end{array}
  \right.$
\end{figure}%d
\begin{figure}[H]
  $\mbox{/g/ $\rightarrow$} \left\{
    \begin{array}{l}
	  \parbox{3,75cm}{\super{ŋ}k] / \_]\textsubscript{$\sigma$}} \parbox{2cm}{\emph{nag}} \parbox{3cm}{[na\super{ŋ}k]} \parbox{3cm}{`grass skirt'} \\
      \parbox{4cm}{\hfill} \parbox{2cm}{\hfill} \parbox{3cm}{\hfill} \parbox{3cm}{\hfill} \\
      \parbox{3,75cm}{[\super{ŋ}g] / elsewhere} \parbox{2cm}{\emph{gau}} \parbox{3cm}{[\super{ŋ}ga͡u]} \parbox{4cm}{`night heron'} \\
	  \parbox{3,75cm}{\hfill} \parbox{2cm}{\emph{sagara}} \parbox{3cm}{[sa\super{ŋ}gara]} \parbox{3cm}{proper name} \\
    \end{array}
  \right.$
\end{figure}%g

\subsubsection{Labialised velar stops} \label{labvelarstops}

The labialised \isi{velar} stops /kw/ and /gw/ show no allophonic variation due to their restricted distribution. Both occur only in \isi{syllable} onsets, not in the coda. Consequently, we do not find these phonemes in word final position.\footnote{In the neighbouring language \ili{Nama} which belongs to the \ili{Nambu} subgroup, labialised velar stops may occur in coda position, as in [auk\super{w}] `morning'.}

\begin{figure}[H]
  $\mbox{/kw/ $\rightarrow$} \left\{
    \begin{array}{l}
	  \parbox{4cm}{[k\super{w}] / \textsubscript{$\sigma$}[\_} \parbox{2cm}{\emph{kwan}} \parbox{2cm}{[k\super{w}an]} \parbox{3cm}{`shout, voice'} \\
	  \parbox{4cm}{\hfill} \parbox{2cm}{\emph{ysokwr}} \parbox{2cm}{[jə̆sok\super{w}ə̆ɾ]} \parbox{3cm}{`rainy season'} \\
    \end{array}
  \right.$
\end{figure}%kw
\begin{figure}[H]
  $\mbox{/gw/ $\rightarrow$} \left\{
    \begin{array}{l}
	  \parbox{4cm}{[\super{ŋ}g\super{w}] / \textsubscript{$\sigma$}[\_}	\parbox{2cm}{\emph{gwä}} \parbox{2cm}{[\super{ŋ}g\super{w}æ]} \parbox{3cm}{`mosquito'} \\
	  \parbox{4cm}{\hfill} \parbox{2cm}{\emph{fagwa}} \parbox{2cm}{[ɸa\super{ŋ}g\super{w}a]} \parbox{3cm}{`width'} \\
    \end{array}
  \right.$
\end{figure}%gw

I will argue in favour of an analysis whereby the labialised \isi{velar} stops are complex phonemes rather than a sequence of two phonemes (\isi{velar} stop + high back vowel /u/ or \isi{velar} stop + /w/). This argument is based on two lines of evidence: onset \isi{consonant clusters} and \isi{reduplication} patterns.\\

Onset clusters are restricted to two consonants (C\textsubscript{1}C\textsubscript{2}V). If clusters occur, C\textsubscript{2} may only be /r/ or /w/ (\S{}\ref{syllabificationandepenthesis}). For this argument, only the /r/ is relevant. We do find words in Komnzo which have an initial labialised \isi{velar} stop (voiceless or prenasalised) in such a cluster, for example: \emph{kwras} `Brolga' or \emph{gwra} `MacCulloch's Rainbowfish'. If /kw/ and /gw/ were to be analysed as clusters of two phonemes, a separate \isi{syllable} template (CCCV) would be required.\\

We find full and partial \isi{reduplication} in Komnzo (\S{}\ref{nomreduplication}). Full \isi{reduplication} involves repeating the whole word: \emph{yam} `footprint, custom, event' $\rightarrow$ \emph{yamyam} `little feast'. More commonly found is partial \isi{reduplication} where only the first consonant of the initial \isi{syllable} is copied: \emph{zbär} `night' $\rightarrow$ \emph{zzbär} [tsə̆tsə̆\super{m}bæɾ] `dusk, twilight'. Note that the domain of partial \isi{reduplication} does not extend further than the first consonant. Thus, we get \emph{frasi} `hunger' $\rightarrow$ \emph{ffrasi} [ɸə̆ɸɾasi] `appetite, hunger', but not \textsuperscript{$\ast$}\emph{frfrasi} [ɸɾə̆ɸɾasi]. If the labialised \isi{velar} stops comprise two separate phonemes, we would expect that in partial \isi{reduplication} only the \isi{velar} stop is copied without the \isi{semivowel}. On the contrary, we find that the whole phoneme is copied as in \emph{kwayan} `light' $\rightarrow$ \emph{kwkwayan} [k\super{w}ə̆k\super{w}ajan] $\sim$ [kuk\super{w}ajan] `flickering light, dimmed light', but not \textsuperscript{$\ast$}\emph{kkwayan} [kə̆k\super{w}ajan].\\

\subsubsection{Affricates} \label{affricates}

The two consonant phonemes with the highest frequency are the affricates (/z/ and /nz/) which seem to give Komnzo its characteristic fricative sound. Both affricates occur initially, medially and finally showing some allophonic variation. They are palatalised before front vowels as in \emph{zi} [tʃı:] `pain' and \emph{nzikaka} [\super{n}dʒıkaka] `Whistling Kite'. In all other environments they are \isi{alveolar}. There is some degree of variation between speakers. Some speakers always palatalise, while most speakers follow the allophonic rules as formalised below. The prenasalised \isi{affricate} is affected by final \isi{devoicing} (\S{}\ref{final-devoicing-section}).\\

\begin{figure}[H]
  $\mbox{/z/ $\rightarrow$} \left\{
    \begin{array}{l}
      \parbox{4,2cm}{[tʃ] / \_V\textsubscript{\textsc{+front}}} \parbox{2,5cm}{\emph{zena}} \parbox{2,5cm}{[tʃena]} \parbox{3cm}{`now'} \\
      \parbox{4,2cm}{\hfill} \parbox{2,5cm}{\emph{ezi}} \parbox{2,5cm}{[ʔetʃi]} \parbox{3cm}{`morning'} \\
      \parbox{4,2cm}{\hfill} \parbox{2,5cm}{\hfill} \parbox{2,5cm}{\hfill} \parbox{3cm}{\hfill} \\
      \parbox{4,2cm}{[ts] / elsewhere} \parbox{2,5cm}{\emph{zane}} \parbox{2,5cm}{[tsane]} \parbox{3cm}{\Dem:\Prox{}} \\
      \parbox{4,2cm}{\hfill} \parbox{2,5cm}{\emph{mazo}} \parbox{2,5cm}{[matso]} \parbox{3cm}{`ocean'} \\
      \parbox{4,2cm}{\hfill} \parbox{2,5cm}{\emph{müz}} \parbox{2,5cm}{[mʏ.ts]} \parbox{3cm}{`phallocrypt'} \\
    \end{array}
  \right.$
\end{figure}%z
\begin{figure}[H]
  $\mbox{/nz/ $\rightarrow$} \left\{
    \begin{array}{l}
      \parbox{4cm}{[\super{n}dʒ] / \_V\textsubscript{\textsc{+front}}} \parbox{2,5cm}{\emph{nzigfu}} \parbox{2,5cm}{[\super{n}dʒi\super{ŋ}gɸu]} \parbox{2,5cm}{`rain stone'} \\
      \parbox{4cm}{\hfill} \parbox{2,5cm}{\emph{snzä}} \parbox{2,5cm}{[sə̆\super{n}dʒæ]} \parbox{2,5cm}{`crayfish'} \\
      \parbox{4cm}{\hfill} \parbox{2,5cm}{\hfill} \parbox{2,5cm}{\hfill} \parbox{2,5cm}{\hfill} \\
      \parbox{4cm}{{[\super{n}ts] / \_]\textsubscript{$\sigma$}}} \parbox{2,5cm}{\emph{mnz}} \parbox{2,5cm}{[mə̆\super{n}ts]} \parbox{2,5cm}{`house'} \\
      \parbox{4cm}{\hfill} \parbox{2,5cm}{\hfill} \parbox{2,5cm}{\hfill} \parbox{2,5cm}{\hfill} \\
      \parbox{4cm}{[\super{n}dz] / elsewhere} \parbox{2,5cm}{\emph{nzun}} \parbox{2,5cm}{[\super{n}dzun]} \parbox{2,5cm}{\Fsg.\Dat{}} \\
      \parbox{4cm}{\hfill} \parbox{2,5cm}{\emph{rnzam}} \parbox{2,5cm}{[rə̆\super{n}dzam]} \parbox{2,5cm}{`how many'} \\
    \end{array}
  \right.$
\end{figure}%nz

\subsubsection{Fricatives} \label{fricatives}

There are three \isi{fricatives} at the bilabial, dental and \isi{alveolar} places of articulation. The dental fricative is voiced while the other two are voiceless. Consequently, only the dental fricative is affected by final \isi{devoicing}. The bilabial fricative has a voiced allophone which occurs intervocalically. Although voiced in most environments, the dental fricative is affected by final \isi{devoicing} (\S{}\ref{final-devoicing-section}). The \isi{alveolar} fricative is always voiceless in all environments. These rules are formalised below.

\begin{figure}[H]
  $\mbox{/f/ $\rightarrow$} \left\{
    \begin{array}{l}
      \parbox{4cm}{[β] / V\_V} \parbox{2,5cm}{\emph{zafazafa}} \parbox{2,5cm}{[tsaβatsaβa]} \parbox{3cm}{`vine stick'} \\
      \parbox{4cm}{\hfill} \parbox{2,5cm}{\hfill} \parbox{2,5cm}{\hfill} \parbox{3cm}{\hfill} \\
      \parbox{4cm}{[ɸ] / elsewhere} \parbox{2,5cm}{\emph{fid}} \parbox{2,5cm}{[ɸı\super{n}t]} \parbox{3cm}{`bushrope'} \\
	  \parbox{4cm}{\hfill} \parbox{2,5cm}{\emph{zarfa}} \parbox{2,5cm}{[tsaɾɸa]} \parbox{3cm}{`ear'} \\
	  \parbox{4cm}{\hfill} \parbox{2,5cm}{\emph{karaf}} \parbox{2,5cm}{[kaɾaɸ]} \parbox{3cm}{`paddle'} \\
    \end{array}
  \right.$
\end{figure}%ɸ
\begin{figure}[H]
  $\mbox{/th/ $\rightarrow$} \left\{
    \begin{array}{l}
	  \parbox{3,78cm}{[θ] / \_]\textsubscript{$\sigma$}} \parbox{2,5cm}{\emph{süsübäth}} \parbox{2,5cm}{[sʏsʏ\super{m}bæθ]} \parbox{3cm}{`darkness'} \\
      \parbox{3,78cm}{\hfill} \parbox{2,5cm}{\hfill} \parbox{2,5cm}{\hfill} \parbox{3cm}{\hfill} \\
      \parbox{3,78cm}{[ð] / elsewhere} \parbox{2,5cm}{\emph{thamin}} \parbox{2,5cm}{[ðamin]} \parbox{3cm}{`tongue'} \\
	  \parbox{3,78cm}{\hfill} \parbox{2,5cm}{\emph{ŋatha}} \parbox{2,5cm}{[ŋaða]} \parbox{3cm}{`dog'} \\
    \end{array}
  \right.$
\end{figure}%ðɸ
\begin{figure}[H]
  $\mbox{/s/ $\rightarrow$} \left\{
    \begin{array}{l}
      \parbox{2,5cm}{\hfill} \parbox{2,5cm}{\emph{saisai}} \parbox{2,5cm}{[sa͡isa͡i]} \parbox{3cm}{`drizzle (n)'} \\
	  \parbox{2,5cm}{[s]} \parbox{2,5cm}{\emph{fisor}} \parbox{2,5cm}{[ɸisoɾ]} \parbox{3cm}{`turtle'} \\
	  \parbox{2,5cm}{\hfill} \parbox{2,5cm}{\emph{fis}} \parbox{2,5cm}{[ɸi.s]} \parbox{3cm}{`husband'} \\
    \end{array}
  \right.$
\end{figure}%s

\subsection{Nasals} \label{nasals}

There are \isi{nasal} stops at three places of articulation: bilabial, \isi{alveolar}, and \isi{velar}. These three show differences in their frequency and distribution. The \isi{velar} \isi{nasal} /ŋ/ occurs only word initially, while bilabial /m/ and \isi{alveolar} /n/ are found initially, medially and finally. There is no allophonic variation with the \isi{nasals}.

\begin{figure}[H]
  $\mbox{/m/ $\rightarrow$} \left\{
    \begin{array}{l}
      \parbox{2,5cm}{\hfill} \parbox{2,5cm}{\emph{mifum}} \parbox{2,5cm}{[miβum]} \parbox{3cm}{`nose ornament'} \\
	  \parbox{2,5cm}{[m]} \parbox{2,5cm}{\emph{zimu}} \parbox{2,5cm}{[tʃimu]} \parbox{3cm}{`snot'} \\
	  \parbox{2,5cm}{\hfill} \parbox{2,5cm}{\emph{thm}} \parbox{2,5cm}{[ðə̆m]} \parbox{3cm}{`nose'} \\
    \end{array}
  \right.$
\end{figure}%m
\begin{figure}[H]
  $\mbox{/n/ $\rightarrow$} \left\{
    \begin{array}{l}
      \parbox{2,5cm}{\hfill} \parbox{2,5cm}{\emph{no}} \parbox{2,5cm}{[no:]} \parbox{3cm}{`water, rain'} \\
	  \parbox{2,5cm}{[n]} \parbox{2,5cm}{\emph{mane}} \parbox{2,5cm}{[mane]} \parbox{3cm}{`who' (\Abs)} \\
	  \parbox{2,5cm}{\hfill} \parbox{2,5cm}{\emph{minmin}} \parbox{2,5cm}{[minmin]} \parbox{3cm}{`Emerald Dove'} \\
    \end{array}
  \right.$
\end{figure}%n
\begin{figure}[H]
  $\mbox{/ŋ/ $\rightarrow$} \left\{
    \begin{array}{l}
	  \parbox{2,6cm}{[ŋ] / \textsubscript{\textsc{word}}[\_} \parbox{2,5cm}{\emph{ŋazi}} \parbox{2,5cm}{[ŋatʃi]} 	\parbox{3cm}{`coconut'} \\
    \end{array}
  \right.$
\end{figure}%ŋ

\subsection{Trill, tap - /r/} \label{trilltap}

The \isi{alveolar} trill /r/ is often realised as a single tap [ɾ] depending on speech rate and speaker. In onset \isi{consonant clusters} where /r/ is occupying C\textsubscript{2} position, it is always tapped. Elsewhere the trill and the tap are in free variation. Word finally /r/ may also become voiceless. This variation between [ɾ] and [ɾ̥] seems to be conditioned by age. Older speakers use the voiceless variant more frequently.

\begin{figure}[H]
  $\mbox{/r/ $\rightarrow$} \left\{
    \begin{array}{l}
      \parbox{3,7cm}{[ɾ]$\sim$[ɾ̥] / \_]\textsubscript{\textsc{word}}} \parbox{2cm}{\emph{msar}} \parbox{3,5cm}{[mə̯saɾ] $\sim$ [mə̯saɾ̥]} \parbox{3cm}{`green ant'} \\
      \parbox{3,7cm}{\hfill} \parbox{2cm}{\hfill} \parbox{3,5cm}{\hfill} \parbox{3cm}{\hfill} \\
	  \parbox{3,7cm}{[ɾ] / \textsubscript{$\sigma$}[C\_} \parbox{2cm}{\emph{frasi}} \parbox{3,5cm}{[ɸɾasi]} \parbox{3cm}{`hunger'} \\
	  \parbox{3,7cm}{\hfill} \parbox{2cm}{\hfill} \parbox{3,5cm}{\hfill} \parbox{3cm}{\hfill} \\
      \parbox{3,7cm}{[r]$\sim$[ɾ] / elsewhere} \parbox{2cm}{\emph{rnz}} \parbox{3,5cm}{[rə̯\super{n}ts] $\sim$ [ɾə̯\super{n}ts]} \parbox{3cm}{`ember'} \\
	  \parbox{3,7cm}{\hfill} \parbox{2cm}{\emph{ŋare}} \parbox{3,5cm}{[ŋare] $\sim$ [ŋaɾe]} \parbox{3cm}{`woman'} \\
    \end{array}
  \right.$
\end{figure}%r

\subsection{Approximants} \label{approximants}

The two approximants /w/ and /y/ occur in initial, medial and final position. In final position, they may be realised as a short offglide or become part of a \isi{diphthong}. For both approximants, but especially for the palatal /y/, we find only a handful of lexical items where they do occur word finally.\\

\begin{figure}[H]
  $\mbox{/w/ $\rightarrow$} \left\{
    \begin{array}{l}
      \parbox{3,4cm}{[\_͡u]$\sim$[\_\super{w}] / V\_]\textsubscript{$\sigma$}} \parbox{1,7cm}{\emph{daw}} \parbox{3cm}{[\super{n}da͡u] $\sim$ [\super{n}da\super{w}]} \parbox{3cm}{`garden'} \\
      \parbox{3,4cm}{\hfill} \parbox{1,7cm}{\hfill} \parbox{3cm}{\hfill} \parbox{3cm}{\hfill} \\
	  \parbox{3,4cm}{[w] / elsewhere} \parbox{1,7cm}{\emph{wm}} \parbox{3cm}{[wə̆m]} \parbox{3cm}{`stone, gravel'} \\
	  \parbox{3,4cm}{\hfill} \parbox{1,7cm}{\emph{fewa}} \parbox{3cm}{[ɸewa]} \parbox{3cm}{`odour, stench'} \\
    \end{array}
  \right.$
\end{figure}%w
\begin{figure}[H]
  $\mbox{/y/ $\rightarrow$} \left\{
    \begin{array}{l}
      \parbox{3,4cm}{[\_͡ı]$\sim$[\_\super{j}] / V\_]\textsubscript{$\sigma$}} \parbox{1,7cm}{\emph{fäy}} \parbox{3cm}{[ɸæ͡ı] $\sim$ [ɸæ\super{j}]} \parbox{3cm}{`payment'} \\
      \parbox{3,4cm}{\hfill} \parbox{1,7cm}{\hfill} \parbox{3cm}{\hfill} \parbox{3cm}{\hfill} \\
	  \parbox{3,4cm}{[j] / elsewhere} \parbox{1,7cm}{\emph{yusi}} \parbox{3cm}{[jusi]} \parbox{3cm}{`grass'} \\
	  \parbox{3,4cm}{\hfill} \parbox{1,7cm}{\emph{nzöyar}} \parbox{3cm}{[\super{n}dʒœjaɾ]} \parbox{3cm}{`bowerbird'} \\
    \end{array}
  \right.$
\end{figure}%y

There are a number of reasons why the two approximants are analysed as consonants rather than high vowels which alternate according to their environment. Evidence comes from case allomorphy and phonotactics. In stem final position /w/ and /y/ select the same \isi{allomorph} of the \isi{locative} case as other consonants. This can be seen in the word \emph{daw} [\super{n}da͡u] $\sim$ [\super{n}da\super{w}] `garden' which selects \emph{=en} as its \isi{locative} case marker, thus forming \emph{dawen} [\super{n}dawen] `in the garden'. Words which end in a vowel select the \emph{=n} \isi{allomorph} of the \isi{locative} case. Furthermore, the rules of \isi{syllabification} (\S{}\ref{syllabificationandepenthesis}) treat these two phonemes like consonants. Thus, we find examples like \emph{ys} [jı̆s] `thorn' and \emph{ky} [kə̆\super{j}] `yam type' where \isi{epenthesis} occurs after or before /w/ and /y/ respectively.

\subsection{Minimal pairs for Komnzo consonants} \label{minimalpairsconsonants}

The following \isi{minimal pair}s and near \isi{minimal pair}s in Table \ref{minpaircon} illustrate the phonemic contrast between consonants in initial, medial and final position.

\clearpage
\begin{table}
\caption{Minimal pairs of consonant phonemes}
\begin{tabularx}{\textwidth}{lllll}
\label{minpaircon}\\
		\lsptoprule
		\textsc{segments}& \textsc{word}& \textsc{phonemic}& \textsc{phonetic}& \textsc{gloss}\\\midrule
		%\endfirsthead
 		\textsc{segments}& \textsc{word}& \textsc{phonemic}& \textsc{phonetic}& \textsc{gloss}\\\midrule
 		%\endhead
		/kw/ - /k/ & \emph{kwafar} & /kwa.far/ & [k\super{w}aβaɾ] &place name\\
		& \emph{kafar} & /ka.far/ & [kaβaɾ] &`big'\\
		&&&&\\
		& \emph{sakwr} & /sa.kwr/ & [sak\super{w}ə̆ɾ] &`he hit him'\\
		& \emph{sakr} & /sa.kr/ & [sakə̆ɾ] &`mustard vine'\\
		&&&&\\
		& \emph{kwath} & /kwath/ & [k\super{w}aθ]&`crow'\\
		& \emph{kath} & /kath/ & [kaθ]&`ankle'\\
		&&&&\\
		/gw/ - /g/ & \emph{gwra} & /gwra/ & [\super{ŋ}g\super{w}ra:] & `rainbowfish'\\
		& \emph{gra} & /gra/ & [\super{ŋ}gra:] & `tree type'\\
		&&&&\\
		/kw/ - /w/ & \emph{kwath} & /kwath/ & [k\super{w}aθ]&`crow'\\
		& \emph{wath} & /wath/ & [waθ]&`dance (n)'\\
		&&&&\\
		& \emph{kwf} & /kwf/ & [k\super{w}ə̆ɸ]&`stone club'\\
		& \emph{wf} & /wf/ & [wə̆ɸ]&`shirt, blouse'\\
		&&&&\\
		/gw/ - /w/ & \emph{gwth} & /gwth/ & [\super{ŋ}gwə̆θ]&`nest'\\
		& \emph{wth} & /wth/ & [wə̆θ]&`faeces'\\
		&&&&\\
		/k/ - /w/ & \emph{kath} & /kath/ & [kaθ]&`ankle'\\
		& \emph{wath} & /wath/ & [waθ]&`dance (n)'\\
		&&&&\\
		/f/ - /w/ & \emph{far} & /far/ & [ɸaɾ] & `housepost'\\
		& \emph{war} & /war/ & [waɾ] & `top layer'\\
		&&&&\\
		& \emph{kafar} & /ka.far/ & [kaβaɾ] & `big'\\
		& \emph{kawar} & /ka.war/ & [kawaɾ] & pers. name\\
		&&&&\\
		& \emph{zafe} & /za.fe/ & [tsaβe] & `old'\\
		& \emph{zawe} & /za.we/ & [tsawe] & `right (side)'\\
		&&&&\\
		& \emph{tfitfi} & /t.fi.t.fi/ & [tə̆βitə̆βi] & `whirlwind'\\
		& \emph{twitwi} & /t.wi.t.wi/ & [tə̆witə̆wi] & `bird type'\\
		%&&&&\\
		/s/ - /t/ & \emph{süfr} & /sü.fr/ & [sʏɸə̆r] & `tree type'\\
		& \emph{tüfr} & /tü.fr/ & [tʏɸə̆r] & `many'\\
		&&&&\\
		& \emph{kisr} & /ki.sr/ & [kitə̆r] & `lizard type'\\
		& \emph{kitr} & /ki.tr/ & [kisə̆r] & `pandanus'\\
		&&&&\\
		& \emph{wsws} & /ws.ws/ & [wə̆swə̆s] & `grass type'\\
		& \emph{wtwt} & /wt.wt/ & [wə̆twə̆t] & `itchy'\\
		&&&&\\
		/s/ - /th/ & \emph{sirsir} & /sir.sir/ & [sirsir] & `glider'\\
		& \emph{thirthir} & /thir.thir/ & [ðirðir] & `pig tusk'\\
		&&&&\\
		& \emph{bis} & /bis/ & [\super{m}bi:s] & `bird type'\\
		& \emph{bith} & /bith/ & [\super{m}bi:θ] & `honey bee'\\
		&&&&\\
		& \emph{mus} & /mus/ & [mu:s] & `leech'\\
		& \emph{muth} & /muth/ & [mu:θ] & `(sago) grub'\\
		&&&&\\
		/s/ - /z/ & \emph{si} & /si/ & [si:] & `eye'\\
		& \emph{zi} & /zi/ & [tʃi:] & `pain'\\
		&&&&\\
		& \emph{srminz} & /sr.minz/ & [sə̆rmints] & `rainbow'\\
		& \emph{zrminz} & /zr.minz/ & [tsə̆rmints] & `roots'\\
		&&&&\\
		& \emph{ksi kar} & /k.si kar/ & [kə̆si kar] & `savannah'\\
		& \emph{kzi} & /k.zi/ & [kə̆tʃi] & `barktray'\\
		&&&&\\
		& \emph{fs} & /fs/ & [ɸə̆s] & `fish type'\\
		& \emph{fz} & /fz/ & [ɸə̆ts] & `forest'\\
		&&&&\\
		/th/ - /t/ & \emph{thruthru} & /thru.thru/ & [ðruðru] & `bamboo type'\\
		& \emph{trutru} & /tru.tru/ & [trutru] & `stream'\\
		&&&&\\
		& \emph{füth} & /füth/ & [ɸʏθ] & `rotten tuber'\\
		& \emph{füt} & /füt/ & [ɸʏt] & `pouch'\\
		&&&&\\
		&&&&\\
		/th/ - /r/ & \emph{thusi} & /thu.si/ & [ðusi] & `fold (v.t.)'\\
		& \emph{rusi} & /ru.si/ & [ɾusi] & `shoot (v.t.)'\\
		&&&&\\
		& \emph{bthan} & /b.than/ & [\super{m}bə̆ðan] & `magic'\\
		& \emph{bran} & /b.ran/ & [\super{m}bə̆ɾan] & `line-up'\\
		&&&&\\
		& \emph{yathizsi} & /ya.thi.z.si/ & [jaðitsə̆si] & `die'\\
		& \emph{yarizsi} & /ya.ri.z.si/ & [jaɾitsə̆si] & `hear, listen'\\
		&&&&\\
		& \emph{zithzith} & /zith.zith/ & [tʃiθtʃiθ]& `slickness'\\
		& \emph{zirzir} & /zir.zir/ & [tʃiɾtʃiɾ]& `wetness'\\
		&&&&\\
		& \emph{wath} & /wath/ & [waθ] & `dance (n)'\\
		& \emph{war} & /war/ & [waɾ] & `top layer'\\
		&&&&\\
		/r/ - /t/ & \emph{rar} & /rar/ & [ɾaɾ] & `for what'\\
		& \emph{tar} & /tar/ & [taɾ] & `friend'\\
		&&&&\\
		& \emph{ŋarr} & /ŋa.rr/ & [ŋaɾə̆ɾ] & `bandicoot'\\
		& \emph{ŋatr} & /ŋa.tr/ & [ŋatə̆ɾ] & `rope'\\
		&&&&\\
		& \emph{ft} & /ft/ & [ɸə̆t] & `dead tree'\\
		& \emph{fr} & /fr/ & [ɸə̆ɾ] & `palm stem'\\
		&&&&\\
		/r/ - /z/ & \emph{rinaksi} & /ri.na.k.si/ & [ɾinakə̆si] & `pour'\\
		& \emph{zinaksi} & /zi.na.k.si/ & [tʃinakə̆si] & `put down'\\
		&&&&\\
		& \emph{wari} & /wa.ri/ & [waɾi] & `plant type'\\
		& \emph{wazi} & /wa.zi/ & [watʃi] & `side'\\
		&&&&\\
		& \emph{mür} & /mür/ & [mʏɾ] & `grass type'\\
		& \emph{müz} & /müz/ & [mʏts] & `phallocrypt'\\
		&&&&\\
		/b/ - /m/ & \emph{bith} & /bith/ & [\super{m}biθ] & `honey bee'\\
		& \emph{mith} &	/mith/ & [miθ] & `face'\\
		&&&&\\
		&&&&\\
		& \emph{bä} & /bä/	& [\super{m}bæ:] & \Second.\Abs{} \\
		& \emph{mä}	& /mä/	& [mæ:] & `where' \\
		&&&&\\
		& \emph{züb} & /züb/ & [tʃʏ\super{m}b] & `depth'\\
		& \emph{züm} & /züm/ & [tʃʏm] & `centipede'\\
		&&&&\\
		/d/ - /n/ & \emph{dasi} & /da.si/ & [\super{n}dasi] & `bulge'\\
		& \emph{nasi} & /na.si/	& [nasi] & `long yam'\\
		&&&&\\
		& \emph{badabada} & /ba.da.ba.da/ & [\super{m}ba\super{n}da\super{m}ba\super{n}da]& `ancestor'\\
		& \emph{bana} & /ba.na/ & [\super{m}bana]& `pitiful'\\
		&&&&\\
		& \emph{kd} & /kd/ & [kə̆nt] & `star'\\
		& \emph{kn} & /kn/ & [kə̆n] & `yam type'\\
		&&&&\\
		/g/ - /ŋ/ & \emph{gathagatha} & /ga.tha.ga.tha/ & [ŋgaðaŋgaða] & `bad'\\
		& \emph{ŋathaŋatha} & /ŋa.tha.ŋa.tha/ & [ŋaðaŋaða] & `quoll'\\
		&&&&\\
		& \emph{game} & /ga.me/ & [\super{ŋ}game] & `tongs'\\
		& \emph{ŋame} & /ŋa.me/ & [ŋame] & `mother'\\
		&&&&\\
		/m/ - /n/ & \emph{mä} & /mä/ & [mæ:] & `where'\\
		& \emph{nä} & /nä/ & [næ:] & `some'\\
		&&&&\\
		& \emph{mawan} & /ma.wan/ & [mawan] & `tree type'\\
		& \emph{nawan} & /na.wan/ & [nawan] & `waterhole'\\
		&&&&\\
		/nz/ - /d/ & \emph{nzga} & /nz.ga/ & [\super{n}dzə̆\super{ŋ}ga]&`vagina'\\
		& \emph{dga} & /d.ga/ & [\super{n}də̆\super{ŋ}ga]&`gills'\\
		&&&&\\
		& \emph{ŋanz} & /ŋanz/ & [ŋa\super{n}ts] & `planting row'\\
		& \emph{ŋad} & /ŋad/ & [ŋa\super{n}t] & `rope'\\
		&&&&\\
		& \emph{ymnz} & /y.mnz/ & [jə̆mə̆\super{n}ts] & place name\\
		& \emph{ymd} & /y.md/ & [jə̆mə̆\super{n}t] & `bird'\\
		&&&&\\
		&&&&\\
		/nz/ - /n/ & \emph{nzä} & /nzä/ & [\super{n}dʒæ:] & \Fsg.\Abs{}\\
		& \emph{nä} & /nä/ & [næ:]& `some'\\
		&&&&\\
		& \emph{gonz} & /gonz/ & [\super{ŋ}gɔnts] & `place name'\\
		& \emph{gon} & /gon/ & [\super{ŋ}gɔn] & `water lily'\\
		&&&&\\
		/b/ - /f/ & \emph{bä} & /bä/ &[\super{m}bæ:]& \Second{}.\Abs{}\\
		& \emph{fä} & /fä/ &[ɸæ:]& \Dist{}\\
		&&&&\\
		& \emph{bira} & /bi.ra/ & [\super{m}biɾa] & `axe'\\
		& \emph{fira} & /fi.ra/ & [ɸiɾa] & `betelnut'\\
		&&&&\\
		& \emph{bis} & /bis/ & [\super{m}bi:s] & `bird type'\\
		& \emph{fis} & /fis/ & [ɸi:s] & `husband'\\
		&&&&\\
		/d/ - /t/ & \emph{düfr} & /dü.fr/ & [\super{n}dʏɸə̆ɾ] & `headdress'\\
		& \emph{tüfr} & /tü.fr/ & [tʏɸə̆ɾ]& `plenty'\\
		&&&&\\
		& \emph{drari} & /dra.ri/ & [\super{n}dɾaɾi] & `container'\\
		& \emph{trari} & /tra.ri/ & [tɾaɾi] & `strong man'\\
		&&&&\\
		& \emph{kadakada} & /ka.da.ka.da/ & [ka\super{n}daka\super{n}da]&`yamcake'\\
		& \emph{katakata} & /ka.ta.ka.ta/ & [katakata]&`grass type'\\
		&&&&\\
		& \emph{sd} & /sd/ & [sə̆\super{n}t]&`yam type'\\
		& \emph{st} & /st/ & [sə̆t]&`plant type'\\
		&&&&\\
		/nz/ - /z/ & \emph{nzä} & /nzä/ & [\super{n}dʒæ:] & \Fsg.\Abs{}\\
		& \emph{zä} & /zä/ & [tʃæ:] & \Prox{}\\
		&&&&\\
		& \emph{nzanza} & /nza.nza/ & [\super{n}dza\super{n}dza] & `insect type'\\
		& \emph{zaza} & /za.za/ & [tsatsa] & `carrying'\\
		&&&&\\
		& \emph{nzr} & /nzr/ & [\super{n}dzə̆ɾ] & `leftover'\\
		& \emph{zr} & /zr/ & [tsə̆ɾ] & `tooth'\\
		&&&&\\
		&&&&\\
		& \emph{rbänzsi} & /r.bä.nz.si/ & [ɾə̆\super{m}bæ\super{n}dzə̆si] & `prohibit'\\
		& \emph{rbäzsi} & /r.bä.z.si/ & [ɾə̆\super{m}bætsə̆si] & `untie'\\
		&&&&\\
		/g/ - /k/ & \emph{gd} & /gd/ & [\super{ŋ}gə̆\super{n}t]&`mud'\\
		& \emph{kd} & /kd/ & [kə̆\super{n}t]&`star'\\
		&&&&\\
		& \emph{kafar} & /ka.far/ & [kaβaɾ] & `big'\\
		& \emph{gafar} & /ga.far/ & [\super{ŋ}gaβaɾ] & `fish type'\\%[-0.5ex]
		&&&&\\
		& \emph{gursi} & /gur.si/ & [\super{ŋ}guɾsi]&`break off'\\
		& \emph{kursi} & /kur.si/ & [kuɾsi]&`split'\\%[-0.5ex]
		&&&&\\
		& \emph{tag} & /tag/ & [ta\super{ŋ}k]&`type of bee'\\
		& \emph{tak} & /tak/ & [tak]&`pandanus'\\%[-0.5ex]
		&&&&\\
		& \emph{srag} & /srag/ & [sɾa\super{ŋ}k]&pers. name\\
		& \emph{srak} & /srak/ & [sɾak]&`boy'\\%[-0.5ex]
		&&&&\\
		/w/ - /y/ & \emph{yarsi} & /yar.si/ & [jaɾsi] &`tired'\\
		& \emph{warsi} & /war.si/ & [waɾsi] &`chew'\\%[-0.5ex]
		&&&&\\
		& \emph{yf} & /yf/ & [jə̆ɸ] &`name'\\
		& \emph{wf} & /wf/ & [wə̆ɸ] &`shirt'\\%[-0.5ex]
		&&&&\\
		& \emph{yttünzr} & /yt.tü.nzr/ & [jə̆ttʏ\super{n}dzə̆ɾ] &`paints him'\\
		& \emph{wttünzr} & /wt.tü.nzr/ & [wə̆ttʏ\super{n}dzə̆ɾ] &`paints her'\\%[-0.5ex]
		&&&&\\
		& \emph{fäw} & /fäw/ & [ɸæ͡u] & `arrow shaft'\\
		& \emph{fäy} & /fäy/ & [ɸæ͡i] & `payment'\\
		\lspbottomrule
\end{tabularx}
\end{table}

%\isi{minimal pair}s - consonant phonemes

\section{Vowel phonemes} \label{vowelsegments}

Table \ref{vowelinv} and Figure \ref{vowelinvspace} below give an overview of the vowel phonemes. Komnzo vowels divide the articulatory space into four levels of height (high, mid, mid-low, and low) and draw a distinction between front and back vowels. Additionally, for front vowels, there is a phonemic distinction between rounded and unrounded vowels. In Figure \ref{vowelinvspace} IPA symbols are employed, whereas Table \ref{vowelinv} lists the corresponding graphemes. Note that I include the epenthetic \isi{schwa} in parentheses. This is because there is some evidence that \isi{schwa} constitutes an marginal phoneme word-finally. That being said, in all other occurences it is created by \isi{epenthesis} (\S{}\ref{schwa-as-non-phoneme}).

\begin{figure}
		%\caption[Komnzo vowels]{Komnzo vowels}
\centering
{
	\begin{vowel}[simple]
		\putvowel{i$\sim$ı}{0,3\vowelhunit}{0,4\vowelvunit}
		\putvowel{y$\sim$ʏ}{1,3\vowelhunit}{0,4\vowelvunit}
		\putvowel{e}{0,85\vowelhunit}{1,5\vowelvunit}
		\putvowel{ø$\sim$œ}{1,7\vowelhunit}{1,5\vowelvunit}
		\putvowel{æ}{1,6\vowelhunit}{2,55\vowelvunit}
		\putvowel{u}{4\vowelhunit}{0,4\vowelvunit}
		\putvowel{ɐ$\sim$a}{2,9\vowelhunit}{2,7\vowelvunit}
		\putvowel{(ə)}{2,7\vowelhunit}{1,3\vowelvunit}
		\putvowel{ɔ}{4\vowelhunit}{1,5\vowelvunit}
		%\putvowel{(ó)}{3,4\vowelhunit}{2,1\vowelvunit}
	\end{vowel}
}%
\caption{Komnzo vowel space}
\label{vowelinvspace}
\end{figure}%Vowel space

\begin{table}
\caption{Vowel phoneme inventory}
\label{vowelinv}
	\begin{tabular}{lcccc}
		\lsptoprule
		&\multicolumn{2}{c}{front}&central&back \\
		&\footnotesize{unrounded}&\footnotesize{rounded}&& \\
		\midrule
		high&i&ü&&u\\
		mid&e&ö&(é)&o\\
		mid-low&ä&&&\\
		low&&&a&\\
		\lspbottomrule
	\end{tabular}
\end{table}%Vowel phoneme inventory

Nasal vowels are rather marginal in Komnzo. There are only two words in which we find \isi{nasal} vowels. These are the \isi{conjunction} \emph{a} [ã:] `and' and \emph{o} [ɔ̃] `or'. Both have a second, much rarer variant with an initial \isi{velar} \isi{nasal} \emph{ŋa} [ŋa:] and \emph{ŋo} [ŋɔ:]. This suggests that \isi{nasalisation} of the vowel is caused by the loss of the preceding \isi{velar} \isi{nasal}. Nasalisation is not phonemic in Komnzo.\\

There are no diphthongs in Komnzo. All diphthongs which occur on a phonetic level end in high offglides. These are analysed as allophones of the two approximants /w/ and /y/ in coda position (\S{}\ref{approximants}). In the practical \isi{orthography} these are sometimes written as diphthongs, e.g. <ai> or <au>.\footnote{This is an individual decision based on the speakers' preferences.} Two words which exemplify this are \emph{saisai} /say.say/ `drizzle' and \emph{kaukau} /kaw.kaw/ `Mouth Almighty'.

\subsection{Phonetic description and allophonic distribution of vowels} \label{phonetic-description-vowels}

There is free variation between the following allophones, that is respectively of /i/, /ü/, /u/, /e/, /ö/, /o/, and /a/:

\begin{table}
\caption{Vowel allophones}
\label{allovowel}
	\begin{tabular}{lll}
		\lsptoprule
		phoneme&description&allophones\\\midrule
		/i/ &high front unrounded vowel &$\rightarrow$ [i]$\sim$[ı]\\
		/ü/ &high front rounded vowel &$\rightarrow$ [y]$\sim$[ʏ]\\
		/u/ &high back rounded vowel &$\rightarrow$ [u]$\sim$[ʊ]\\
		/e/ &mid front unrounded vowel &$\rightarrow$ [e]$\sim$[ɛ]\\
		/ö/ &mid front rounded vowel &$\rightarrow$ [ø]$\sim$[œ]\\
		/o/ &mid back rounded vowel &$\rightarrow$ [o]$\sim$[ɔ]\\
		/a/ &low central unrounded vowel &$\rightarrow$ [a]$\sim$[ɐ]\\
		/ä/ &low front unrounded vowel &$\rightarrow$ [æ]\\
		\lspbottomrule
	\end{tabular}
\end{table}%Vowel allophones

There is no phonemic contrast between short and long vowels. However, vowels tend to be longer in monosyllabic roots, especially if the monosyllable is light/open, e.g. \emph{nzä} [\super{n}dʒæ:] `I'. This process of vowel lengthening is caused by \isi{minimal word} conditions in combination with \isi{syllable} weight as will be described in \S{}\ref{syllstruc} and \S{}\ref{minwordconstraints}.

\subsubsection{Allophones of /o/}\label{allo-o}

There is further allophonic variation for /o/ which is related to vowel lengthening. In heavy, closed syllables, /o/ is realised as a short, centralised, rounded vowel [ɞ̆], whereas in light, open syllables it is realised as a mid back rounded vowel of normal length [ɔ]. Two words which show this allophonic variation are the language name \emph{Komnzo} /kom.nzo/ [kɞ̆m\super{n}dzɔ] and \emph{komon} /ko.mon/ [kɔmɞ̆n] `maybe'. We find the two allophones [ɞ̆] and [ɔ] conditioned by \isi{syllable} weight in the syllables of the two words respectively. There are two rules which may override this allophonic distribution. The first is a \isi{minimal word} constraint which produces [ɔ] even in closed syllables if the root is monosyllabic (see \S{}\ref{minwordconstraints}). The second rule overrides \isi{syllable} weight and the impact of the \isi{minimal word} constraint. After the labio-\isi{velar} \isi{approximant} (/w/) and the two labialised-\isi{velar} stops (/kw/ and /gw/) /o/ is always realised as short, centralised, rounded vowel [ɞ̆]. Leaving the influences of the \isi{minimal word} constraint to \S{}\ref{minwordconstraints}, we can formalise these observations in the following rule:

\begin{figure}
  $\mbox{/o/ $\rightarrow$} \left\{
    \begin{array}{l}
      \parbox{3cm}{[ɞ̆] / \_C]\textsubscript{$\sigma$}} \parbox{2cm}{\emph{emoth}} \parbox{2,5cm}{/e.moth/} \parbox{2,5cm}{[ʔe:mɞ̆θ]} \parbox{3cm}{`girl'} \\
      \parbox{3cm}{\hfill} \parbox{2cm}{\emph{ymorymor}} \parbox{2,5cm}{/y.mor.y.mor/} \parbox{2,5cm}{[jə̆mɞ̆ɾjə̆mɞ̆ɾ]} \parbox{3cm}{`desire'} \\
      \parbox{3cm}{\hfill} \parbox{2cm}{\emph{thomgsi}} \parbox{2,5cm}{/thom.g.si/} \parbox{2,5cm}{[ðɞ̆m\super{ŋ}gə̆si]} \parbox{3cm}{`help'} \\
      \parbox{3cm}{\hfill} \parbox{2cm}{\hfill} \parbox{2,5cm}{\hfill} \parbox{3cm}{\hfill} \\
	  \parbox{3cm}{{[ɔ] / \_]\textsubscript{$\sigma$}}} \parbox{2cm}{\emph{nibo}} \parbox{2,5cm}{/ni.bo/} \parbox{2,5cm}{[ni\super{m}bɔ]} \parbox{3cm}{`six'} \\
	  \parbox{3cm}{\hfill} \parbox{2cm}{\emph{dokre}} \parbox{2,5cm}{/do.kre/} \parbox{2,5cm}{[\super{n}dɔkɾe]} \parbox{6cm}{`frog'} \\
      \parbox{3cm}{\hfill} \parbox{2cm}{\hfill} \parbox{2,5cm}{\hfill} \parbox{3cm}{\hfill} \\
	  \parbox{3cm}{[ɞ̆] / C\textsubscript{+labio-velar}\_}	\parbox{2cm}{\emph{kwosi}} \parbox{2,5cm}{/kwo.si/} \parbox{2,5cm}{[k\super{w}ɞ̆si]} \parbox{3cm}{`dead'} \\
	  \parbox{3cm}{\hfill} \parbox{2cm}{\emph{woku}} \parbox{2,5cm}{/wo.ku/} \parbox{2,5cm}{[wɞ̆ku]} \parbox{6cm}{`skin'} \\
    \end{array}
  \right.$
\end{figure}%allophones of /o/

There are some irregularities with these rules when it comes to other bilabial consonants, like /f/. There is \emph{fofot} [ɸɔɸɞ̆t] `single child' which follows the rule, but there are a handful of words which do not follow the rule, like: \emph{fothr} [ɸɞ̆ðə̆r] `eucalyptus type' or \emph{fokufoku} [ɸɞ̆kuɸɞ̆ku] `small patch of vegetation'.
\vspace{-.2cm}

\subsubsection{Analytic problems with /ö/}\label{probl-oe}

The vowel /ö/ [œ] poses a problem because there are no \isi{minimal pair}s between /ö/ and some of its immediate neighbours (/e/, /o/, /ä/) in the corpus. There are \isi{minimal pair}s between /ö/ and /i/, /ü/, /u/, /a/. The lack of \isi{minimal pair}s with the former group along with the effects of \isi{vowel harmony} (see \S{}\ref{vowharmwae}) invite an analysis in which /ö/ is a variant of other phonemes, for example: a rounded allophone of /e/ or a fronted allophone of /o/. However, no conditioning environment (e.g. \isi{vowel harmony} or quality of adjacent consonants) can be established. The main problem lies in the fact, that occurences of /ö/ are much rarer than all other vowels.\footnote{Amongst the 1700 entries in the dictionary, only 30 contain /ö/. Compare this number with 730 for /a/. This is a conservative count in which singletons and reduplicates as well as simple forms and compounds are only counted once.} For the current description, /ö/ is set up as an independent vowel phoneme. Further research will have to settle this question.
\vspace{-.2cm}

\subsection{The non-phonemic status of schwa} \label{schwa-as-non-phoneme}

The most frequent vowel in Komnzo is a short \isi{schwa} [ə̆]. I will argue here that this is not a phoneme, but that it is inserted through \isi{epenthesis} in order to create a \isi{syllable} nucleus where there is none underlyingly. That being said, I will make an argument at the end of this section that \isi{schwa} can be analysed as a marginal or emerging phoneme in word final context. The rules of \isi{epenthesis} will be laid out in \S{}\ref{syllabificationandepenthesis}.\\

Epenthetic vowels are known from many Papuan languages. The best documented case is certainly \ili{Kalam} \citep{Biggs:1963wk, Pawley:1966wj, Blevins:2010ee}, but epenthetic vowels have been described for other languages of the Yam family, e.g. \ili{Nen} \citep{Evans:ji}. In Komnzo, the main arguments for \isi{schwa} as an \isi{epenthetic vowel} rather than a phoneme come from syllabicity alternations, the predictability of \isi{schwa}, and its restricted distribution.\\

Syllabicity alternations which cause changes in the place of \isi{schwa} insertion are influenced by affixation. Two examples are the verb \emph{ttüsi} [tə̆tʏsi] `print, paint' and the noun \emph{fzenz} [ɸə̆tʃe\super{n}ts] `wife'. In both stems \isi{schwa} occurs in the first \isi{syllable}. When we inflect the verb with an undergoer prefix, the first consonant is syllabified as a coda and \isi{schwa} needs to be inserted in a different position: \emph{yttünzr} [jə̆ttʏ\super{n}dzə̆ɾ] `s/he paints him'. When we add a \isi{possessive} prefix to \emph{fzenz}, e.g.: \emph{bufzenz} [\super{m}buɸtʃe\super{n}ts] `your wife', again the first consonant of the stem becomes a coda. In this case \isi{schwa} disappears entirely because the \isi{possessive} prefix ends in a vowel. It follows that \isi{schwa} cannot be present in the underlying representation of these two lexemes.\\

Schwa has a very restricted distribution compared to specified vowels. It does not occur word initially and it is very limited word finally. I will show below that word-final schwas should be analysed as a marginal phoneme. Elsewhere \isi{schwa} is entirely predictable and therefore not represented in the \isi{orthography} of Komnzo. The rules of \isi{schwa} insertion are discussed as part of \isi{syllabification} and possible \isi{consonant clusters} (\S{}\ref{syllabificationandepenthesis}). There are many roots in Komnzo which lack specified vowels altogether.\footnote{Among 1700 entries in the dictionary, we find 105 without specified vowels. The number of entries in which the epenthetic vowel occurs together with specified vowels is much higher.} A few examples are: \emph{mnz} [mə̆\super{n}ts] `house', \emph{zfth} [tsə̆ɸə̆θ] `base, reason', and \emph{ggrb} [\super{ŋ}gə̆\super{ŋ}gə̆ɾə̆\super{m}p] `small, unripe coconut'. The quality of the \isi{epenthetic vowel} shows only little variation. In almost all enviroments it is realised as a mid central vowel of very short duration [ə̆]. However, there is one exception. If the \isi{epenthetic vowel} is inserted preceding the two approximants /y/ and /w/ it is realised as a high front or high back vowel respectively, as in: \emph{nyak} [nĭjak] `we go' and \emph{thwak} [ðŭwak] `shoulder'.\\

There is one caveat to the analysis of \isi{schwa} as epenthetic. It cannot be predicted in word-final context. Although word-final \isi{schwa} is very rare in terms of types, it cannot be dismissed as the aberrant behaviour of a few lexical items. This is because it is not rare at all in terms of tokens. For example, word-final \isi{schwa} shows up in the verb morphology (\Fsg{} \emph{-é}), case marking (\Erg.\Nsg{} \emph{=é}) and in the \isi{adjectivaliser} \emph{-thé}. The latter could be historically related to the \isi{similative} case marker (\emph{=thatha}). For the first singular suffix on verbs, I argue in \S{}\ref{personsuffsection}, that this is the result of vowel reduction (a>ə), because neighbouring varieties have a corresponding \emph{-a} suffix. Moreover, the first \isi{person} suffix \emph{-é} disappears if other suffixal material is added to the verb. This is also found with some of the lexical items. For example, if \emph{kayé} `yesterday' is marked with a \isi{temporal} \isi{possessive} case (\emph{=thamane}), word-final \isi{schwa} disappears: \emph{kaythamane dagon} `yesterday's food'. This does not happen with full vowels, e.g. \emph{ezithamane dagon} `food from the morning' from \emph{ezi} `morning'. Thus, I analyse \isi{schwa} in word-final contexts as a marginal phoneme, which emerged or is emerging from vowel reduction. In these word-final cases \isi{schwa} is represented orthographically by <\emph{é}>.

\subsection{Minimal pairs for Komnzo vowels} \label{minimalpairsvowels}

The following \isi{minimal pair}s and near \isi{minimal pair}s illustrate the phonemic contrasts between vowels. Each vowel phoneme is set apart from its immediate neighbours in the vowel space. Each vowel phoneme is contrasted with the \isi{epenthetic vowel}, i.e. the absence of a specified vowel (\Zero{}). Some combinations are redundant (e.g.: /i/ - /e/ and /e/ - /i/) and not repeated in the table.

%{\small%
\begin{table} 
\caption{Minimal pairs of vowel phonemes}
\begin{tabularx}{\textwidth}{lllll}
\label{minpairvow}\\
		\lsptoprule
		\textsc{segments}&\textsc{word}&\textsc{phonemic}&\textsc{phonetic}&\textsc{gloss}\\ \midrule
		%\endfirsthead
 		\textsc{segments}&\textsc{word}&\textsc{phonemic}&\textsc{phonetic}&\textsc{gloss}\\ \midrule
 		%\endhead
		/i/ - /u/ & \emph{mith} & /mith/ & [miθ] & `face'\\
		& \emph{muth} & /muth/ & [muθ] & `(sago) grub'\\
		&&&&\\
		& \emph{grigri} & /gri.gri/ & [\super{ŋ}gɾı\super{ŋ}gɾı] & `maggots'\\
		& \emph{gru} & /gru/ & [\super{ŋ}gɾu:] & `shooting star'\\
		&&&&\\
		/i/ - /ü/ & \emph{minzaksi} &/mi.nza.k.si/ & [mi\super{n}dzakə̆si] & `paint (vt.)'\\
		& \emph{münzaksi} &/mü.nza.k.si/ & [mʏ\super{n}dzakə̆si] & `allow'\\
		&&&&\\
		& \emph{di} & /di/ & [\super{n}di:] & `back of head'\\
		& \emph{düdü} & /dü.dü/ & [\super{n}dʏ\super{n}dʏ] & `in good shape'\\
		&&&&\\
		/i/ - /e/ & \emph{si} & /si/ & [si:] & `eye'\\
		& \emph{se}	& /se/ & [se:] & `torch'\\
		&&&&\\
		& \emph{bi} & /bi/ & [\super{m}bi:] & `sago'\\
		& \emph{be}	& /be/ & [\super{m}be:] & \Ssg.\Erg{}\\
		&&&&\\
		/i/ - /ö/ & \emph{di} & /di/ & [\super{n}di:] & `back of head'\\
		& \emph{dö} & /dö/ & [\super{n}dœ:] & `monitor lizard'\\
		&&&&\\
		/i/ - \Zero{} & \emph{biribiri} & /bi.ri.bi.ri/ & [\super{m}biɾi\super{m}biɾi] & `plant type'\\
		& \emph{bribri} & /b.ri.b.ri/&   [\super{m}bə̆ɾi\super{m}bə̆ɾi] & `weeding'\\
		&&&&\\
		& \emph{with} & /with/ &[wiθ] & `banana'\\
		& \emph{wth} & /wth/ &[wə̆θ]&	`faeces'\\
		&&&&\\
		& \emph{fis} & /fis/ & [ɸis] & `husband'\\
		& \emph{fs} & /fs/ & [ɸə̆s] & `fish type'\\
		&&&&\\
		/u/ - /i/ & \multicolumn{4}{l}{see above /i/ - /u/}\\
		&&&&\\
		/u/ - /ü/ & \emph{futhfuth} & /futh.futh/ & [ɸuθɸuθ] & `scrapes'\\
		& \emph{füthfüth} & /füth.füth/ & [ɸʏθɸʏθ] & `hatched bird'\\
		&&&&\\
		& \emph{but} & /but/ & [\super{m}but] & `kava sticks'\\
		& \emph{büt} & /büt/ & [\super{m}bʏt] & `amputated limb'\\
		&&&&\\
		& \emph{rusi} & /ru.si/ & [ɾusi] & `shoot (vt.)'\\
		& \emph{rüsi} & /rü.si/ & [ɾʏsi] & `rain (v.)'\\
		&&&&\\
		/u/ - /o/ & \emph{muramura} & /mu.ra.mu.ra/ & [muɾamuɾa] & `medicine'\\
		& \emph{moramora} & /mo.ra.mo.ra/ & [mɔɾamɔɾa] & `tree type'\\
		&&&&\\
		& \emph{muth} & /muth/ & [muθ] & `(sago) grub'\\
		& \emph{moth} & /moth/ & [mɞ̆θ] & `path'\\
		&&&&\\
		& \emph{tru} & /tru/ & [tɾu:] & `palm type'\\
		& \emph{tro} & /tro/ & [tɾɔ:] & `python type'\\
		&&&&\\
		/u/ - \Zero{} & \emph{kursi} & /kur.si/ & [kuɾsi] & `split (vt.)'\\
		& \emph{krsi}  & /kr.si/  &	[kə̆ɾsi] & `block (vt.)'\\
		%&&&&\\
		& \emph{kut} & /kut/ & [kut] & `trap'\\
		& \emph{kt} & /kt/ & [kə̆t] & `grass type'\\
		&&&&\\
		& \emph{fuk} & /fuk/ & [ɸuk] & `in a group'\\
		& \emph{fk} & /fk/ & [ɸə̆k] & `buttocks'\\
		&&&&\\
		/ü/ - /i/ & \multicolumn{4}{l}{see above /i/ - /ü/}\\
		/ü/ - /u/ & \multicolumn{4}{l}{see above /u/ - /ü/}\\
		&&&&\\
		/ü/ - /e/ & \emph{fünz} & /fünz/ & [ɸʏ\super{n}ts] & `arm muscles'\\
		& \emph{fenz} & /fenz/ & [ɸe\super{n}ts] & `puss'\\
		&&&&\\
		/ü/ - /ö/ & \emph{nümä} & /nü.mä/ & [nʏmæ] & `one week away'\\
		& \emph{nömä} & /nö.mä/ & [nœmæ] & `yamcake'\\
		&&&&\\
		& \emph{düdü} & /dü.dü/ & [\super{n}dʏ\super{n}dʏ] & `in good shape'\\
		& \emph{dödö} & /dö.dö/ & [\super{n}dœ\super{n}dœ] & `plant type'\\
		&&&&\\
		/ü/ - \Zero{} & \emph{sün} & /sün/ & [sʏn] & `dirt, dust'\\
		& \emph{sn} & /sn/ & [sə̆n] & `yam type'\\
		&&&&\\
		& \emph{tüfr} & /tü.fr/ & [tʏɸə̆ɾ] & `plenty'\\
		& \emph{tfrtfr} & /t.fr.t.fr/ & [tə̆ɸə̆ɾtə̆ɸə̆ɾ] & `tree type'\\
		&&&&\\
		/e/ - /i/ & \multicolumn{4}{l}{see above /i/ - /e/}\\
		/e/ - /ü/ & \multicolumn{4}{l}{see above /ü/ - /e/}\\
		/e/ - /ö/ & \multicolumn{4}{l}{not attested}\\
		&&&&\\
		/e/	- /o/ & \emph{fethaksi} & /fe.tha.k.si/ & [ɸeðakə̆si] & `dip in'\\
		& \emph{fothaksi} & /fo.tha.k.si/ & [ɸɔðakə̆si] & `take off (bag)'\\
		&&&&\\
		& \emph{game} & /ga.me/ & [\super{ŋ}game] & `tongs'\\
		& \emph{gamo} & /ga.mo/ & [\super{ŋ}gamɔ] & `magic spell'\\
		&&&&\\
		/e/ - /a/ & \emph{yem} & /yem/ & [jem] & `cassowary'\\
		& \emph{yam} & /yam/ & [jam] & `event'\\
		&&&&\\
		& \emph{fetr} & /fe.tr/ & [ɸetə̆ɾ] & `dangerous'\\
		& \emph{fatr} & /fa.tr/ & [ɸatə̆ɾ] & `shoulder'\\
		&&&&\\
		& \emph{gwra} & /gwra/ & [\super{ŋ}g\super{w}ra:] & `fish type'\\
		& \emph{gwre} & /gwre/ & [\super{ŋ}g\super{w}re:] & `bird type'\\
		&&&&\\
		/e/ - /ä/ & \emph{erbänzé} & /e.r.bä.nzé/ & [ʔeɾə̆\super{m}bæ\super{n}tsə̆] & `I untie them'\\
		& \emph{ärbänzé} & /ä.r.bä.nzé/ & [ʔæɾə̆\super{m}bæ \super{n}tsə̆] & `I untie for them'\\
		&&&&\\
		& \emph{fenz} & /fenz/ & [ɸe\super{n}ts] & `puss'\\
		& \emph{fänz} & /fenz/ & [ɸæ\super{n}ts] & `proper name'\\
		&&&&\\
		& \emph{nze} & /nze/ & [\super{n}dʒe:] & \Fsg.\Erg\\
		& \emph{nzä} & /nzä/ & [\super{n}dʒæ:] & \Fsg.\Abs\\
		&&&&\\
		/e/ - \Zero{} & \emph{menz} & /menz/ & [me\super{n}ts] & `story man'\\
		& \emph{mnz} & /mnz/ & [mə̆\super{n}ts] & `house'\\
		&&&&\\
		& \emph{fethaksi} & /fe.tha.k.si/ & [ɸeðakə̆si] & `dip in'\\
		& \emph{fthaksi} & /f.tha.k.si/ & [ɸə̆ðakə̆si] & `take from fire'\\
		&&&&\\
		& \emph{ŋakwire} & /ŋa.kwi.re/ & [ŋak\super{w}iɾe] & `we run'\\
		& \emph{ŋakwiré} & /ŋa.kwi.ré/ & [ŋak\super{w}iɾə̆] & `I run'\\
		&&&&\\
		/ä/ - /e/ & \multicolumn{4}{l}{see above /e/ - /ä/}\\
		&&&&\\
		/ä/ - /a/ & \emph{näbi} & /nä.bi/ & [næ\super{m}bi] & `one'\\
		& \emph{nabi} & /na.bi/ & [na\super{m}bi] & `bow, bamboo'\\
		&&&&\\
		& \emph{fätr} & /fä.tr/ & [ɸætə̆ɾ] & `left'\\
		& \emph{fatr} & /fa.tr/ & [ɸatə̆ɾ] & `shoulder'\\
		&&&&\\
		& \emph{mafä} & /ma.fä/ & [maɸæ] & `with whom'\\
		& \emph{mafa} & /ma.fa/ & [maɸa] & `who'\\
		&&&&\\
		/ä/ - /ö/ & \multicolumn{4}{l}{not attested}\\
		&&&&\\
		/ä/ - /o/ & \emph{bärbär} & /bär.bär/&[\super{m}bæɾ\super{m}bæɾ]&`half'\\
		& \emph{bor} & /bor/ & [\super{m}bɞ̯ɾ]&`rat'\\
		&&&&\\
		& \emph{nä} & /nä/ & [næ:] & `some'\\
		& \emph{no} & /no/ & [nɔ:] & `water'\\
		&&&&\\
		/ä/ - \Zero{} & \emph{fäk}& /fäk/& [ɸæk]& `jaw'\\
		& \emph{fk}&/fk/&[ɸə̆k]&`buttocks'\\
		&&&&\\
		& \emph{märmär}&/mär.mär/&[mæɾmæɾ]&`slope'\\
		& \emph{mrmr}&/mr.mr/&[mə̆ɾmə̆ɾ]&`inside'\\
		&&&&\\
		& \emph{bnä}&/b.nä/&[\super{m}bə̆næ]&`with you'\\
		& \emph{bné}&/b.né/&[\super{m}bə̆nə̆]&\Snsg.\Erg{}\\
		&&&&\\
		/a/ - /ä/ & \multicolumn{4}{l}{see above /ä/ - /a/}\\
		/a/ - /e/ & \multicolumn{4}{l}{see above /e/ - /a/}\\
		&&&&\\
		/a/ - /ö/ & \emph{namä} & /na.mä/ & [namæ] & `good'\\
		& \emph{nömä} & /nö.mä/ & [nœmæ] & `yamcake'\\
		&&&&\\
		/a/ - /o/ & \emph{zan} & /zan/ & [tsan] & `fight'\\
		& \emph{zon} & /zon/ & [tsɔn] & `plant type'\\
		&&&&\\
		& \emph{karfa} & /kar.fa/ & [kaɾɸa] & `from village'\\
		& \emph{karfo} & /kar.fo/ & [kaɾɸɔ] & `to village'\\
		&&&&\\
		& \emph{far} & /far/ & [ɸaɾ] & `house post'\\
		& \emph{for} & /for/ & [ɸɞ̯ɾ] & `riverbank'\\
		&&&&\\
		/a/ - \Zero{} & \emph{ngath} & /n.gath/ & [nə̆\super{ŋ}gaθ] & `friend'\\
		& \emph{ngth} & /n.gth/ & [nə̆\super{ŋ}gə̆θ] & `young sibling'\\
		&&&&\\
		& \emph{tharthar} & /thar.thar/ & [ðaɾðaɾ] & `next to'\\
		& \emph{thrthr} & /thr.thr/ & [ðə̆ɾðə̆ɾ] & `intestines'\\
		&&&&\\
		&&&&\\
		& \emph{mar} & /mar/ & [maɾ] & `pandanus type'\\
		& \emph{mr} & /mr/ & [mə̆ɾ] & `brain'\\
		&&&&\\
		& \emph{sakwra} & /sa.kw.ra/ & [sak\super{w}ə̆ɾa]&`I hit him' (\Pst{})\\
		& \emph{sakwré} & /sa.kw.ré/ & [sak\super{w}ə̆ɾə̆]&`I hit him' (\Rpst{})\\
		&&&&\\
		/o/ - /e/ & \multicolumn{4}{l}{see above /e/ - /o/}\\
		/o/ - /ö/ & \multicolumn{4}{l}{not attested}\\
		/o/ - /a/ & \multicolumn{4}{l}{see above /a/ - /o/}\\
		/o/ - /ä/ & \multicolumn{4}{l}{see above /ä/ - /o/}\\
		/o/ - /u/ & \multicolumn{4}{l}{see above /u/ - /o/}\\
		&&&&\\
		/o/ - \Zero{} & \emph{borsi} & /bor.si/ & [\super{m}bɞ̯ɾsi]&`laugh'\\
		& \emph{brsi} & /br.si/ & [\super{m}bə̆ɾsi]&`scoop water'\\
		&&&&\\
		& \emph{fothaksi} & /fo.tha.k.si/ & [ɸɔðakə̆si] & `take off'\\
		& \emph{fthaksi} & /f.tha.k.si/ & [ɸə̆ðakə̆si] & `take from fire'\\
		&&&&\\
		& \emph{rgosi} & /r.go.si/ & [ɾə̆\super{ŋ}gɔsi] & `poke through'\\
		& \emph{rgsi} & /r.g.si/ & [ɾə̆\super{ŋ}gə̆si] & `wear clothes'\\
		&&&&\\
		& \emph{monz} & /monz/ & [mɔ\super{n}ts] & `trench, ditch'\\
		& \emph{mnz} & /mnz/ & [mə̆\super{n}ts] & `house'\\
		&&&&\\
		& \emph{nzigom} & /nzi.gom/ & [\super{n}dʒi\super{ŋ}gɞ̯m] & `chain smoker'\\
		& \emph{nzigm} & /nzi.gm/ & [\super{n}dʒi\super{ŋ}gə̆m] & `stickyness'\\
		\lspbottomrule
\end{tabularx}
\end{table}

%}

\section{Regular phonological processes} \label{regular-phon-processes}

\subsection{Gemination} \label{gemination-section}

Gemination occurs with a subset of the consonantal phonemes (/t/, /k/, /f/, /th/, /m/, /n/, and /r/). We find geminates in medial, heterosyllabic \isi{consonant clusters} where the rules of \isi{syllabification} specify that no \isi{epenthetic vowel} needs to be inserted (see \S{}\ref{syllabificationandepenthesis}). Phonetically, geminates are characterised by a prolonged realisation of \isi{fricatives}, \isi{nasals}, and \isi{alveolar} trill. Geminate stops are realised with a delayed release of the airflow. Although gemination is caused by affixation in most cases, I discuss the topic here rather than as a morphophonemic rule because we also find monomorphemic roots with geminates. The examples in Table \ref{geminates} provide some attested examples from the corpus. In some of the examples, we find \isi{minimal pair}s based on gemination as can be seen in the rightmost column.


\begin{table}
\caption{Geminate consonants}
\label{geminates}
	\begin{tabular}{lll}
		\lsptoprule
		\textsc{segment} & \textsc{geminate} & \textsc{non-geminate} \\ \midrule
		/t/ & \emph{yttünzr} `s/he paints him' & n/a \\
		&&\\
		& \emph{yakkarä} `quickly' & \emph{yakarä} `in tears'\\
		/k/ & yak=karä & ya=karä \\
		& walk=\Prop{} & cry=\Prop{}\\
		&&\\
		& \emph{yamme} `through this event' & \emph{yame} `mat' \\
		/m/ & yam=me & \\
		& event=\Ins{} & \\
		&&\\
		& \emph{fammäre} `without thinking' & n/a \\
 	   	& fam=märe & \\
 	   	& thoughts=\Priv{} & \\
		&&\\
		& \emph{yannor} `he shouts hither' & \emph{yanor} `he shouts'\\
		/n/ & ya-n-nor & ya-nor \\
		& \Tsg{}.\Masc{}-\Venit{}-shout & \Tsg{}.\Masc{}-shout\\
		&&\\
 	   	& \emph{fiyaffa} `from the hunt' & n/a \\
		/f/ & fiyaf=fa & \\
		& hunt=\Abl{} & \\
		&&\\
		/th/ & \emph{yththagr} `it is sticking (on sth.)' 	& n/a \\
		&&\\
		/r/ & \emph{firra} `place name' & \emph{fira} `betelnut'\\
		& \emph{kwrro} `Blue-winged Kookaburra' & n/a \\
		\lspbottomrule
	\end{tabular}
\end{table}%Geminate consonants

Gemination is not attested for complex consonants, including the prenasalised stops (/b/, /d/, and /g/) as well as the two affricates (/z/ and /nz/) and /s/. Gemination is not relevant for the labialised \isi{velar} stops (/kw/ and /gw/) and the \isi{velar} \isi{nasal} (/ŋ/) because these do not occur in coda position.

\subsection{Final-devoicing} \label{final-devoicing-section}

The process of final \isi{devoicing}, naturally, affects only those consonants which (i) occur in final position (excluding non-final: /kw/, /gw/ and /ŋ/) and (ii) are voiced in all other environments (excluding voiceless: /t/, /k/, /f/, /s/, and /z/). The \isi{nasal} stops and the approximants are also not affected by final \isi{devoicing}. This leaves us with the following phonemes which are targetted by final \isi{devoicing}: /b/, /d/, /g/, /nz/, /th/, and /r/.\\

The domain of final \isi{devoicing} is the \isi{syllable}. For example, in words where /nz/ occurs in onset position, it is always voiced: \emph{nzafar} [\super{n}dzaɸaɾ] `sky' and \emph{knzun} [kə̆\super{n}dzun] `parallel'. If /nz/ occurs in final position, it is always voiceless: \emph{mnz} [mə̆\super{n}ts] `house'. We find evidence in suffixation and encliticisation that the process is targetting the right edge of the \isi{syllable} rather than the word. \emph{Mnz} [mə̆\super{n}ts] `house' may take the vowel initial \isi{locative} \isi{enclitic} \emph{=en} in which case /nz/ occurs in onset position and is voiced: \emph{mnzen} [mə̆\super{n}dzen] `in the house'. This contrasts with the consonant initial formatives \emph{=fa} (\Abl) and \emph{-wä} (\Emph). In both cases /nz/ is syllabified in coda position and is voiceless: \emph{mnzfa} [mə̆\super{n}tsɸa] `from the house' and \emph{mnzwä} [mə̆\super{n}tswæ] `really the house' . We can formalise final \isi{devoicing} in the following rule:

\begin{figure}
\centering
$\mbox{/b/, /d/, /g/, /nz/, /th/ $\rightarrow$} \left\{
\begin{array}{l}
  \parbox{3,75cm}{[-voiced] / \_]\textsubscript{$\sigma$}} \\
\end{array}
\right.$
\end{figure}%final devoicing rule

The only excepion is /r/, where final \isi{devoicing} occurs only word-finally. However, final \isi{devoicing} of /r/ is optional and more commonly found with older speakers.

\subsection{Glottal stop insertion} \label{glottal-stop-insertion-section}

There are only few lexemes in Komnzo which are vowel initial.\footnote{Among the 1700 entries in the dictionary, there are 54 vowel initial lexemes: /a/ (21), /e/ (17), /o/ (8), /ä/ (4), /u/ (3), /i/ (1). Three of these are loanwords.} In addition, the non-singular undergoer prefix for second/third person in one of the five prefix series is also vowel initial. However, vowel initial words are a marginal pattern in Komnzo and with one exception, which I describe below, word-medial syllables without onsets are not found. A possible explanation for the occurence of vowel initial words in Komnzo is contact with the \ili{Nambu} languages to the east.\\

For this marginal pattern we find a rule of \isi{glottal stop} insertion as in: \emph{ebar} [ʔe\super{m}baɾ] `head' or \emph{ettünzr} [ʔettʏ\super{n}dzə̆ɾ] `s/he paints them'. This rule is restricted to word-initial environment, because the rules of \isi{syllabification} maximise onsets in almost all cases (see \S{}\ref{syllabificationandepenthesis}). There is only one exception. Word-medial \isi{glottal stop} insertion occurs with the vowel initial possessive suffix \emph{-ane}. When the \isi{possessive} is suffixed to a word which ends in a vowel, a \isi{glottal stop} is inserted at the morpheme boundary. An example is \emph{kabe} `man' $\rightarrow$ \emph{kabeane} [ka\super{m}beʔane] `of the man'.

\section{The syllable and phonotactics} \label{syllable-and-phonotactics}

The phonotactics of Komnzo are best described in terms of the \isi{syllable}. My description of the \isi{syllable} is influenced by Blevins \citeyearpar{Blevins:1995tt}. I begin by outlining different \isi{syllable} templates and the constraints which help to define them (\S{}\ref{syllstruc}). I provide evidence for the internal structure of the \isi{syllable}. Consonant clusters are shown in \S\ref{consonantclusters}. It follows a step-by-step analysis of \isi{syllabification} and \isi{epenthesis} (\S{}\ref{syllabificationandepenthesis}). The section closes with a discussion of the \isi{minimal word} (\S\ref{minwordconstraints}) and \isi{stress} (\S{}\ref{stress}).

\subsection{Syllable structure} \label{syllstruc}

The template for the maximal \isi{syllable} in Komnzo is [CCVC]\textsubscript{$\sigma$}. The minimal \isi{syllable} is [CV]\textsubscript{$\sigma$} and in a more restricted environment [V]\textsubscript{$\sigma$}. Thus, a \isi{syllable} maximally consists of an onset, which may or may not be complex, a nucleus and a simple coda. Three constraints help to define the possible representations of the \isi{syllable} in Komnzo:

\begin{enumerate}
	\item Onsets are obligatory in word-medial and final position. There is a constraint against vowels in onset position: \textsuperscript{$\ast$}\textsubscript{$\sigma$}[V. The only position where we find vowels in onsets is word-initially, but this is a marginal pattern. If the process of \isi{syllabification} produces vowel initial words, a \isi{glottal stop} fills the onset position (see \S{}\ref{glottal-stop-insertion-section}). Word-internal or word-final syllables never lack a consonantal onset.
	\item Syllables may have complex onsets with a maximal number of two adjacent consonants: \textsubscript{$\sigma$}[CC. There are constraints on the phonemes involved in CC onset clusters. (see \S{}\ref{tautosyllabiccc})
	\item Syllables may only have a simple coda: C]\textsubscript{$\sigma$}. Post-vocalic consonsant clusters are always heterosyllabic, never tautosyllabic: \textsuperscript{$\ast$}CC]\textsubscript{$\sigma$}. There are a number of constraints on the possibilities of heterosyllabic \isi{consonant clusters} (see \S{}\ref{heterosyllabiccc}).
\end{enumerate}

From the three constraints given above, we can now derive the following possible \isi{syllable} types: CV, CVC, CCV, CCVC. Word-initially, we also find V and VC. Figure \ref{syllableinternal} presents the \isi{syllable} in Komnzo as a binary branching construct.

\begin{figure}
	\centering
		\Tree[.$\sigma$
		 [.onset
		   [.{\vline height 1.3em} (C\textsubscript{1})\footnotemark \hspace{0,1cm} (C\textsubscript{2}) ]
		 ]
		 [.rhyme
		   %[.(C\textsubscript{$\sigma$})
		      [.nucleus V ]
		     [.coda (C\textsubscript{3}) ] ]
		 %]
		]
	\caption{The internal structure of the syllable}\label{syllableinternal}
\end{figure}%\isi{syllable} structure
\footnotetext{Syllables without consonantal onsets are restricted to word initial environments. In this case, a phonological rule states that a glottal stop is inserted (\S{}\ref{glottal-stop-insertion-section}).}

A branching \isi{syllable} is chosen over a flat structure because there is evidence for the rhyme as a separate node of which nucleus and coda are subnodes. Such evidence includes the different shapes and constraints for onset and coda. Onsets may be complex. Codas can only be simple. Onsets are obligatory in almost all cases while codas are optional. Onsets and rhyme combine freely, thus capturing the generalisation that onsets rarely influence the nucleus. All consonant phonemes may appear in a simple onset (C\textsubscript{1}). There are some restrictions, but these are internal to the onset (see \S{}\ref{tautosyllabiccc}). The coda position (C\textsubscript{3}) on the other hand is more limited as to which consonant phonemes may appear. The labialised \isi{velar} stops /kw/ and /gw/ and the \isi{velar} \isi{nasal} /ŋ/ never appear in a coda.\\

The strongest evidence for an independent rhyme comes from \isi{syllable} weight which impacts on vowel length of the nucleus. If there is a specified vowel in the nucleus, the vowel will become long in open/light syllables, and it will become short in closed/heavy syllables. This affects different vowels to varying degrees. We find a good example of this in the distribution of the two allophones of /o/ which are [ɔ] and [ɞ̆]. In the language name \emph{Komnzo} /kom.nzo/ [kɞ̆m\super{n}dzɔ] the first vowel is very short (although stressed) and the second vowel is of normal length. It follows that \isi{syllable} weight influences the length (and sometimes quality) of the vowel in the nucleus. The shortening or lengthening of nuclei may be overridden by \isi{minimal word} constraints (see \S{}\ref{minwordconstraints}), but these rules hold for all polysyllabic roots. Consequently, we require reference to the rhyme as an independent subnode of the \isi{syllable}.

\subsection{Consonant clusters} \label{consonantclusters}

We find tautosyllabic and heterosyllabic \isi{consonant clusters} in Komnzo. These have very different restrictions in their possibilities.

\subsubsection{Tautosyllabic clusters} \label{tautosyllabiccc}

Tautosyllabic clusters are restricted to the onset of a \isi{syllable}, no more than two consonants may occur and they only involve a subset of the phonemes. In a \textsubscript{$\sigma$}[C\textsubscript{1}C\textsubscript{2} template, C\textsubscript{2} may only be /r/ or /w/.\\

In a cluster with /r/ we find all consonant phonemes except for the three \isi{nasal} stops (\textsuperscript{$\ast$}\textsubscript{$\sigma$}[mr, \textsuperscript{$\ast$}\textsubscript{$\sigma$}[nr, \textsuperscript{$\ast$}\textsubscript{$\sigma$}[ŋr) and the approximants (\textsuperscript{$\ast$}\textsubscript{$\sigma$}[wr and \textsuperscript{$\ast$}\textsubscript{$\sigma$}[yr) and /r/ itself (\textsuperscript{$\ast$}\textsubscript{$\sigma$}[rr). This points to an explanation in terms of a sonority hierarchy in which \isi{nasal} and approximants are more sonorous than the trill/tap. Some examples of \textsubscript{$\sigma$}[Cr clusters are \emph{brüzi} `catfish type', \emph{frar} `small fishtrap', \emph{krüfr} `cold', \emph{gru} `shooting star', \emph{kwras} `Brolga', \emph{srima kabe} `scout, spy', \emph{thruthru} `bamboo type', \emph{trisi} `scratch (v)', \emph{zra} `swamp'.\\

In a cluster with /w/ the restrictions on C\textsubscript{1} are more severe and roots in which it is attested are rare. We only find the following phonemes in C\textsubscript{1} position: /k/, /g/, /z/, /nz/, /th/, and /s/. The first two phonemes in the list pose a problem because one has find a distinction between a Cw cluster and the labialised \isi{velar} stops /kw/ and /gw/. This is impossible to do for lexemes, but we find some evidence in a morphophonemic rule in \S{}\ref{approxhighvowel} where the vowel /u/ is realised as [w] and becomes part of a \textsubscript{$\sigma$}[Cw cluster. Some examples of lexemes with \textsubscript{$\sigma$}[Cw onset clusters are: \emph{swäyé} `anchoring place', \emph{zwäf} `luke-warm', \emph{bzwär} [\super{m}bə̆zwæɾ] `place name'.

\subsubsection{Heterosyllabic clusters} \label{heterosyllabiccc}

Heterosyllabic clusters are much harder to pin down because - as we will see in \S{}\ref{syllabificationandepenthesis} below - there are syllabicity alternations where a coda consonant may become an onset by inserting epenthetic \isi{schwa} after which it breaks up the cluster. I will label the two consonants involved C\textsubscript{a} (the coda of the first \isi{syllable}) and C\textsubscript{b} (the onset of the following \isi{syllable}).\\

We find that where C\textsubscript{a} and C\textsubscript{b} are identical the consonants are never broken up but always realised as geminates. The attested \isi{geminate} patterns are described as a phonological rule in \S{}\ref{gemination-section}. These patterns exclude a number of logically possible geminates: labialised \isi{velar} stops (/kw/ and /gw/), \isi{velar} \isi{nasal} (/ŋ/), and all the prenasalised phonemes (/b/, /d/, /g/, and /nz/).\footnote{The labialised velar stop and the velar nasal may not occur as C\textsubscript{a} because these never occur in coda position.} Other heterosyllabic clusters are rather unrestricted. Table \ref{heterosyllcctable} presents the possible cluster types in Komnzo and Table \ref{heterosyllcctableexamples} lists examples of these types.

\begin{table}
\caption{Heterosyllabic consonant clusters}
\label{heterosyllcctable}
	\begin{tabular}{lcccccccc}
		\lsptoprule
		&&\textsc{oral}&\textsc{pren.}&&&&&\textsc{lab-}\\
		& /r/ & \textsc{stop} & \textsc{stop}\footnotemark & \textsc{nasal} & \textsc{affr.} & \textsc{fric.} & \textsc{approx.} & \textsc{velar}\\ \midrule
		%&&&&&&&&\\
		/r/ & \checkmark & \checkmark & n/a & \checkmark & \checkmark & \checkmark & \checkmark  & \checkmark\\%[1.5ex]
		\textsc{oral stop} & n/a & \checkmark & n/a & \checkmark & n/a & \checkmark & \checkmark  & \checkmark\\%[1.5ex]
		\textsc{pren. stop} & n/a & \checkmark & n/a & \checkmark & n/a & \checkmark & \checkmark  & n/a\\%[1.5ex]
		\textsc{nasal} & \checkmark & \checkmark & \checkmark & \checkmark & \checkmark & \checkmark & \checkmark  & \checkmark\\%[1.5ex]
		\textsc{affr.} & n/a & \checkmark & n/a & \checkmark & n/a & \checkmark & \checkmark & n/a\\%[1.5ex]
		\textsc{fric.} &  n/a & \checkmark & n/a & \checkmark & \checkmark & \checkmark & \checkmark  & \checkmark\\%[1.5ex]
		\textsc{approx.} &  n/a & \checkmark & n/a & \checkmark & \checkmark & \checkmark & n/a  & n/a\\%[1.5ex]
		\textsc{lab-velar} & n/a & n/a& n/a& n/a& n/a& n/a& n/a& n/a\\%[1.5ex]
		\lspbottomrule
	\end{tabular}
\end{table}%Heterosyllabic \isi{consonant clusters}
\footnotetext{The column and line labelled `prenasal' includes prenasalised stops and the prenasalised affricate.}
\clearpage
\begin{table}
\label{heterosyllcctableexamples}
\caption{Examples of attested heterosyllabic consonant clusters}
\begin{tabularx}{\textwidth}{p{1,2cm}p{1,55cm}lll}
		\lsptoprule
		C\textsubscript{a} & C\textsubscript{b} & \textsc{underlying} & \textsc{phonetic} & \textsc{gloss}\\
		&&\textsc{representation}& \textsc{realisation}&\\ \midrule
		%\endfirsthead
		C\textsubscript{a} & C\textsubscript{b} & \textsc{underlying} & \textsc{phonetic} & \textsc{gloss}\\
		&&\textsc{representation}& \textsc{realisation}&\\ \midrule
		%\endhead
		/r/ & [+\isi{nasal}] & /ke\underline{r.m}a/&[ke\underline{ɾm}a] &`from tail'\\
		&&/t\underline{r.n}ä/ &[tə̆\underline{ɾn}æ] &`palm frond'\\
		&&&&\\
		/r/ &[+oral] & /fo\underline{r.t}u/&[ɸɞ̆\underline{ɾt}u] &`scar'\\
		&& /ke\underline{r.k}o/&[ke\underline{ɾk}o] &`headdress'\\
		&&&&\\
		/r/ &[+affr.]&/z\underline{r.z}ü/&[tsə̆\underline{ɾtʃ}ʏ] &`knee' \\
		&&&&\\
		/r/ &[+fric.] & /wa\underline{r.f}o/&[wa\underline{ɾɸ}ɔ] &`above'\\
		&&/k\underline{r.s}i/&[kə̆\underline{ɾs}i] &`block (v)'\\
		&&/t\underline{r.th}a/&[tə̆\underline{ɾð}a] &`life'\\
		&&&&\\
		/r/&[+approx.]&/ka\underline{r.w}ä.si/&[ka\underline{ɾw}æsi] &`lie (v)'\\
		&&/ya\underline{r.y}om.g.si/&[ja\underline{ɾj}ɞ̆m\super{ŋ}gə̆si] &`scream (v)'\\
		&&&&\\
		/r/&[+lab-vel]&/ŋa.fa\underline{r.kw}.re/&[ŋaɸa\underline{ɾk\super{w}}ə̆ɾe]&`we leave'\\
		&&&&\\
		{[+oral]}&[+oral]&/wä\underline{t.k}u/&[wæ \underline{tk}u]&`pelican'\\
		&&&&\\
		{[+oral]} &[+\isi{nasal}]&/de\underline{k.n}i.ni/&[\super{n}de\underline{kn}ini]&`praying mantis'\\
		&&/r\underline{t.m}aksi/&[ɾə̆\underline{tm}akə̆si]&`cut'\\
		&&&&\\
		{[+oral]} &[+fric.]&/f.r\underline{k.th}é/&[ɸə̆ɾə̆\underline{kð}ə̆]&`red'\\
		&&/e\underline{t.f}th/&[ʔe\underline{tɸ}ə̆θ]&`sleep (n)'\\
		&&&&\\
		{[+oral]} &[+approx.]&/thi\underline{k.y}a.si/&[ði\underline{kj}asi]&`build fence'\\
		&&/zo\underline{k.w}a.si/&[tsɞ̆\underline{kw}asi]&`speech'\\
		&&/mi\underline{t.w}a.si/&[mi\underline{tw}asi]&`swing (v)'\\
		&&&&\\
		{[+oral]} &[+lab-vel]&/ta\underline{t.kw}o.nam/&[ta\underline{tk\super{w}}ɔnam]&`tree type'\\
		&&&&\\
		{[+pren.]}&[+oral]&/g\underline{b.k}a.rä/&[\super{ŋ}gə̆\underline{\super{m}bk}aɾæ]&`with pandanus'\\
		&&&&\\
		{[+pren.]}&[+\isi{nasal}]&/ŋa\underline{d.m}e/&[ŋa\super{n}tme]&`with rope'\\
		%&&&&\\
		{[+pren.]}&[+fric.]&/ba\underline{d.f}o/&[\super{m}ba\super{n}tɸɔ]&`to the ground'\\
		&&&&\\
		{[+pren.]}&[+approx.]&/m\underline{nz.w}ä/&[mə̆\underline{\super{n}tsw}æ]&`house (\Emph)'\\
		&&&&\\
		{[+\isi{nasal}]}&/r/&/ni\underline{n.r}r/&[ni\underline{nɾ}ə̆ɾ]&`with us'\\
		&&&&\\
		{[+\isi{nasal}]} &[+oral]&/a\underline{m.k}f/&[ʔa\underline{mk}ə̆ɸ]&`breath'\\
		&&/thu\underline{n.t}.nä.gwr/&[ðu\underline{nt}ə̆næ\super{ŋ}gwə̆ɾ]&`he lost them'\\
		&&&&\\
		{[+\isi{nasal}]} &[+\isi{nasal}]&/ka\underline{n.m}otha/&[ka\underline{nm}ɔða]&`river snake'\\
		&&&&\\
		{[+\isi{nasal}]} &[+pren.]&/yar.yo\underline{m.g}.si/&[jaɾjɞ̆\underline{m\super{ŋ}g}ə̆si]&`scream (v)'\\
		&&/ku\underline{m.d}a/&[ku\underline{m\super{n}d}a]&`basket'\\
		&&/kä\underline{n.b}rim/&[kæ\underline{n\super{m}b}ɾim]&`come here!'\\
		&&&&\\
		{[+\isi{nasal}]} &[+affr.]&/sa\underline{n.z}in/&[sa\underline{ntʃ}in]&`put him down!'\\
		&&&&\\
		{[+\isi{nasal}]} &[+fric.]&/za\underline{n.f}r/&[tsa\underline{nɸ}ə̆ɾ]&`far'\\
		&&/ka\underline{m.th}a.tha/&[ka\underline{mð}aða]&`like a bone'\\
		&&&&\\
		{[+\isi{nasal}]} &[+approx.]&/nze.n\underline{m.w}ä/&[\super{n}dʒenə̆\underline{mw}æ]&`for us (\Emph)'\\
		&&&&\\
		{[+\isi{nasal}]} &[+lab-vel]&/ŋa\underline{n.kw}ir/&[ŋa\underline{nk\super{w}}ir]&`run hither'\\
		&&&&\\
		{[+affr.]} &[+oral]&/e\underline{z.k}n.wr/&[ʔe\underline{tsk}ə̆nwə̆ɾ]&`he moves them'\\
		&&&&\\
		{[+affr.]} &[+\isi{nasal}]&/kä\underline{z.n}ob/&[kæ\underline{tsn}ɞ̆\super{m}p]&`drink (it)!'\\
		&&&&\\
		{[+affr.]} &[+fric.]&/f\underline{z.f}o/&[ɸə̆\underline{tsɸ}ɔ]&`to forest'\\
		&&&&\\
		{[+affr.]} &[+approx.]&/f\underline{z.w}ä/&[ɸə̆\underline{tsw}æ]&`forest (\Emph)'\\
		&&&&\\
		{[+fric.]} &[+oral]&/m\underline{nz.w}ä/&[mə̆\underline{\super{n}tsw}æ]&`house (\Emph)'\\
		&&&&\\
		{[+fric.]} &[+affr.]&/bu\underline{f.z}enz/&[\super{m}bu\underline{ɸtʃ}e\super{n}ts]&`your wife'\\
		&&&&\\
		{[+fric.]} &[+fric.]&/e\underline{f.th}ar/&[ʔe\underline{ɸð}aɾ]&`dry season'\\
		&&/fü\underline{s.f}üs/&[ɸʏ\underline{sɸ}ʏs]&`wind'\\
		&&&&\\
		{[+fric.]} &[+approx.]&/nz\underline{f.w}i.yak/&[\super{n}tsə̆\underline{ɸw}Ijak]&`we walked'\\
		&&/na\underline{f.w}ä/&[na\underline{ɸw}æ]&`they (\Emph{})'\\
		&&/fi\underline{th.w}o.g.si/&[ɸi\underline{θw}ɔ\super{ŋ}gə̆si]&`take out'\\
		&&&&\\
		{[+fric.]} &[+lab-vel]&/ma\underline{th.kw}i/&[ma\underline{θk\super{w}}i]&`personal name'\\
		&&/y.ra.k\underline{th.kw}a/&[jə̆rakə̆\underline{θk\super{w}}a]&`he put on top'\\
		&&&&\\
		{[+approx.]} &[+oral]&/fa\underline{w.k}a.rä/&[ɸa\underline{\super{w}k}aɾæ]&`with payment'\\
		&&&&\\
		{[+approx.]} &[+\isi{nasal}]&/fa\underline{w.m}a/&[ɸa\underline{\super{w}m}a]&`from payment'\\
		&&&&\\
		{[+approx.]} &[+affr.]&/bä\underline{w.z}ö/&[\super{m}bæ\underline{\super{w}tʃ}œ]&`paperbark'\\
		&&&&\\
		{[+approx.]} &[+fric.]&/w\underline{y.th}k/&[wə̆\underline{\super{j}ð}ə̆k]&`comes to end'\\
		\lspbottomrule
\end{tabularx}
\end{table} 

%Examples of attested heterosyllabic \isi{consonant clusters}

We can make a number of observations from Table \ref{heterosyllcctableexamples} above. The prenasalised phonemes do occur in C\textsubscript{a} as well as C\textsubscript{b}. In the latter case, C\textsubscript{a} may only be another \isi{nasal} as in: \emph{ku\underline{md}a} [ku\underline{m\super{n}d}a] `basket', \emph{ku\underline{mg}si} [ku\underline{m\super{ŋ}g}ə̆si] `smell (v)', \emph{d\underline{mg}u} [\super{n}də̆\underline{m\super{ŋ}g}u] `waterhole', \emph{ti\underline{ngw}ä} [ti\underline{n\super{ŋ}g\super{w}}æ] `tree type'. If C\textsubscript{a} is a phoneme other than a \isi{nasal}, the cluster will be broken up: \emph{ga\underline{rd}a} [\super{ŋ}ga\underline{ɾə̆\super{n}d}a] `canoe', \emph{ä\underline{thg}am} [ʔæ\underline{ðə̆\super{ŋ}g}am] `Parinari nonda', \emph{th\underline{fg}arwrmth} [ðə̆\underline{ɸə̆\super{ŋ}g}aɾwə̆ɾə̆mə̆θ] `they were breaking them'. There are no attested cases of a prenasalised phoneme in C\textsubscript{b} with a homorganic \isi{nasal} in C\textsubscript{a}, i.e. /m/+/b/, /n/+/nz/, /n/+/d/.\\

There are only few clusters which involve /r/ in the C\textsubscript{b} position. This is caused by maximizing onsets during \isi{syllabification}, which creates complex onsets clusters of the type Cr. As a consequence, the only heterosyllabic clusters with /r/ in C\textsubscript{b} position are the ones which are illegal as onset clusters (e.g. \textsuperscript{$\ast$}\textsubscript{$\sigma$}[mr, \textsuperscript{$\ast$}\textsubscript{$\sigma$}[nr, \textsuperscript{$\ast$}\textsubscript{$\sigma$}[rr). In other words, because \textsuperscript{$\ast$}\textsubscript{$\sigma$}[nr is illegal as an onset, we do find it as a heterosyllabic cluster (\emph{ninrr} /ni\underline{n.r}r/ [ni\underline{nɾ}ə̆ɾ] `with us'). Likewise, because \textsubscript{$\sigma$}[fr is a legal onset cluster, we never find it as a heterosyllabic cluster.\\

We do find heterosyllabic clusters which involve /w/ in C\textsubscript{b} position and a \isi{velar} (prensalised) stop in  C\textsubscript{a} position. Evidence that these clusters are indeed heterosyllabic as opposed to an instantiation of the labialised \isi{velar} stop /kw/ and /gw/ comes two sources. First, we find examples like \emph{zokwasi} [tsɞ̆kwasi] `speech' where the short, centralised allophone of /o/ shows that /k/ is the coda of a closed \isi{syllable}. Compare this with the discussion of /o/ (\S{}\ref{phonetic-description-vowels}) and the discussion of \isi{syllable} weight (\S{}\ref{syllstruc}). Secondly, verb stems ending in /k/ and /g/ select the \emph{-wr} \isi{allomorph} of the non-dual suffix (\S{}\ref{allomorphdualsuffix}). Consequently, heterosyllabic clusters /k.w/ and /g.w/ as well as the complex phonemes /kw/ and /gw/ are required for an adequate description of the phonological system.
\vspace{.2cm}
\subsection{Syllabification and epenthesis} \label{syllabificationandepenthesis}
\vspace{.2cm}
Syllable structure is generally understood not to be defined at the underlying representation (\citealt[221]{Blevins:1995tt}). Hence, we do not find \isi{minimal pair}s based on syllabicity in Komnzo. As was explained in \S{}\ref{schwa-as-non-phoneme} above, \isi{schwa} is not a phoneme but an \isi{epenthetic vowel} inserted in order to break up \isi{consonant clusters}. There is some degree of free variation in syllabicity and \isi{schwa} insertion. An example is the word \emph{mrn} `family, clan' with the \isi{locative} suffix \emph{-en}. The resulting word \emph{mrnen} `in the family' may be realised either /mr.nen/ [mə̆ɾnen] or /m.r.nen/ [mə̆ɾə̆nen]. There is no phonemic contrast and speakers find it difficult to perceive the difference in syllabicity.\\

The process of \isi{syllabification} will be outlined here in the form of three ordered rules which predict \isi{epenthesis} and \isi{syllable} structure:

\begin{enumerate}
	\item Associate each specified vowel with a \isi{syllable} nucleus.
	\item Establish and maximise onsets in accordance with \isi{syllable} templates (See constraint number 2 in \S{}\ref{syllstruc} on onset clusters). A phonological rule will insert a \isi{glottal stop} if there is no consonantal onset in word initial position (see \S{}\ref{glottal-stop-insertion-section}).
	\item Break-up unsyllabified consonants with epenthetic vowels:
	\begin{enumerate}
		\item Exception: suffixes which allow no other \isi{syllabification} than inserting the \isi{epenthetic vowel} in final position. This includes the \isi{adjectivaliser} \emph{-thé}, non-singular ergative case marker \emph{-yé} and the first singular actor verb suffix \emph{-é}.
		\item Elsewhere: proceed from right to left breaking up \isi{consonant clusters}.
		\item After each \isi{schwa} insertion, establish codas in accordance with possible heterosyllabic \isi{consonant clusters}. Otherwise, maximise onsets. Exception: word-initial segments are always recognised as onsets.
		\item The \isi{epenthetic vowel} is [ŭ] and [ı̆] if followed by heterosyllabic /w/ and /y/ respectively. In all other instances it is [ə̆].
	\end{enumerate}
\end{enumerate}

The process of \isi{syllabification} attempts to map the minimal \isi{syllable} CV onto the underlying representation. The rules give preference to onsets rather than codas. Consequently, we do not find vowel initial syllables word-medially or word-finally.\\

I have modelled the process of \isi{syllabification} as being divided into two steps. Syllables which contain full vowels are recognised first and in a second step epenthetic vowels are inserted to break up unsyllabified \isi{consonant clusters}. This algorithm proceeds backwards (from right to left) and inserts epenthetic schwas between unsyllabified consonants to create \isi{syllable} nuclei. The insertion ensures that onsets are maximised. After each onset, the processs checks against the list of possible heterosyllabic \isi{consonant clusters} (see \S{}\ref{heterosyllabiccc}) whether another insertion occurs right away or only after a coda has been recognised. In the latter case, it `jumps' one consonant and breaks up the next pair of unsyllabified consonants. An exception is the word initial position where the segment is automatically recognised as an onset. The rules ensure that no word-initial \isi{schwa} insertion occurs. The direction (right to left) explains why we find \isi{schwa} never in word-final position. There are only a handful of lexemes in which \isi{schwa} is attested word-finally.\\

The direction is important in order to explain forms like \emph{wonrsoknwr} [wɞ̆nə̆ɾsɔkə̆nwə̆ɾ]\footnote{The allophone [ɞ̆] of the phoneme /o/ occurs here not because this might be a closed syllable, but because it follows a labio-velar approximant (see \S{}\ref{phonetic-description-vowels})} `s/he is bothering me' which is syllabified /wo.nr.so.kn.wr/. The algorithm is applied from right to left which is why the cluster /r.s/ is first recognised as a possible heterosyllabic consonant cluster. After this recognition, \isi{schwa} is inserted between /n/ and /r/. If the process was applied from left to right, one would expect that /n.r/ is first recognised as a possible heterosyllabic cluster and \isi{schwa} would be inserted between /r/ and /s/ which yields the incorrect form \textsuperscript{$\ast$}/won.r.so.kn.wr/. As pointed out above, there is some degree of optionality. In elicitation, informants accepted \isi{schwa} insertion in both places [wɞ̆nə̆ɾə̆sɔkə̆nwə̆ɾ]. This might be an artefact introduced by elicitation, because in fluent speech this hardly ever occurs.\\

The algorithm specifies that \isi{schwa} is inserted between consonants disregarding possible onset clusters (\S{}\ref{syllstruc}) whereas syllables with specified vowels maximise their onsets and produce onset clusters. Indeed, we do not find the possible onset clusters Cr or Cw with epenthetic vowels. There are only two exceptions for Cr. The first is the verb \emph{frmnzsi} /frm.nz.si/ `fix, prepare' in which the onset cluster /fr/ is never broken up even if the verb is fully inflected: \emph{yafrmnzr} /ya.frm.nzr/ `s/he prepares him'. The second exception occurs with all verbs in a specific inflection: Word-initially, the irrealis prefix \emph{ra-} becomes part of an onset cluster with the \isi{undergoer} prefix. This cluster only contains an \isi{epenthetic vowel} if (i) the restricted verb stem is used and (ii) the verb is marked for dual number: \emph{thrthbth} [ðɾə̆ðə̆\super{m}bə̆θ] `they put them inside'.\footnote{This verb is glossed as: th-r-\Zero{}-thb-th \Stnsg-\Irr-\Ndu-put.inside.\Rs{}-\Stnsg{} It it a rare inflection because three things have to come together: irrealis mood, restricted verb stem, dual number marker (which is a zero-morpheme in this case).}\\

In Figure \ref{syll001}-5 below, I present four examples spelling out the algorithm step by step:

\begin{figure}
\caption{Syllabification of \emph{kwark} `deceased'}
\label{syll001}
\begin{mdframed}
$\begin{array}{l}
	\vspace{0,2cm}
	\parbox{3cm}{/kwark/} \parbox{5cm}{underlying representation}\vspace{-0,2cm}\\
	\vspace{0,2cm}
	\parbox{3cm}{\hfill}\parbox{4cm}{\hspace{1cm}$\downarrow$}\vspace{-0,2cm}\\
	\vspace{0,2cm}
	\parbox{3cm}{/kw\textsubscript{$\sigma$}[\underline{a}]rk/} \parbox{15cm}{Rule 1: Associate each specified vowel with a nucleus.}\vspace{-0,2cm} \\
	\vspace{0,2cm}
	\parbox{3cm}{\hfill}\parbox{4cm}{\hspace{1cm}$\downarrow$}\vspace{-0,2cm}\\
	\vspace{0,2cm}
	\parbox{3cm}{/\textsubscript{$\sigma$}[\underline{kwa}]rk/} \parbox{15cm}{Rule 2: Maximise onsets.}\vspace{-0,2cm} \\
	\parbox{3cm}{\hfill} \parbox{15cm}{$\rightarrow$ establishes the \isi{syllable} \textsubscript{$\sigma$}[kwa]} \\
	\vspace{0,2cm}
	\parbox{3cm}{\hfill}\parbox{4cm}{\hspace{1cm}$\downarrow$}\vspace{-0,2cm}\\
	\vspace{0,2cm}
	\parbox{3cm}{/\textsubscript{$\sigma$}[kwa]\textsubscript{$\sigma$}[\underline{rk}]/} \parbox{10cm}{Rule 3b: Break up consonant clusters.}\vspace{-0,2cm} \\
	\parbox{3cm}{\hfill} \parbox{10cm}{$\rightarrow$ \isi{schwa} is inserted between /r/ and /k/ and creates a} \\
	\parbox{3cm}{\hfill} \parbox{10cm}{CVC syllable} \\
	\vspace{0,2cm}
	\parbox{3cm}{\hfill}\parbox{4cm}{\hspace{1cm}$\downarrow$}\vspace{-0,2cm}\\
	\vspace{0,2cm}
	\parbox{3cm}{/kwa.rk/} 	\parbox{12cm}{syllabified form: [k\super{w}aɾə̆k]}
\end{array}$
\end{mdframed}
\end{figure}%kwark
\begin{figure}
\caption{Syllabification of \emph{yanthugwr} `s/he tricks him here'}\label{syll002}
\begin{mdframed}
$\begin{array}{l}
	\vspace{0,2cm}
	\parbox{3,5cm}{/yanthugwr/} \parbox{5cm}{underlying representation}\vspace{-0,2cm}\\
	\vspace{0,2cm}
	\parbox{3,5cm}{\hfill}\parbox{4cm}{\hspace{1cm}$\downarrow$}\vspace{-0,2cm}\\
	\vspace{0,2cm}
	\parbox{3,5cm}{/y\textsubscript{$\sigma$}[\underline{a}]nth\textsubscript{$\sigma$}[\underline{u}]gwr/} 	\parbox{9cm}{Rule 1: Associate each specified vowel with a nucleus.}\vspace{-0,2cm} \\
	\vspace{0,2cm}
	\parbox{3,5cm}{\hfill}\parbox{4cm}{\hspace{1cm}$\downarrow$}\vspace{-0,2cm}\\
	\parbox{3,5cm}{/\textsubscript{$\sigma$}[\underline{ya}]n\textsubscript{$\sigma$}[\underline{thu}]gwr/} 	\parbox{9cm}{Rule 2: Maximise onsets.} \\
	\parbox{3,5cm}{\hfill} 	\parbox{9cm}{$\rightarrow$ establishes the syllables \textsubscript{$\sigma$}[ya] and \textsubscript{$\sigma$}[thu]} \\
	\vspace{0,2cm}
	\parbox{3,5cm}{\hfill}\parbox{4cm}{\hspace{1cm}$\downarrow$}\vspace{-0,2cm}\\
	\vspace{0,2cm}
	\parbox{3,5cm}{/\textsubscript{$\sigma$}[ya]n\textsubscript{$\sigma$}[thu]g\textsubscript{$\sigma$}[\underline{wr}]/} 	\parbox{9cm}{Rule 3b: Break up consonant clusters.} \vspace{-0,2cm}\\
	\parbox{3,5cm}{\hfill} 	\parbox{9cm}{$\rightarrow$ \isi{schwa} is inserted between /w/ and /r/}\\
	\vspace{0,2cm}
	\parbox{3,5cm}{\hfill}\parbox{4cm}{\hspace{1cm}$\downarrow$}\vspace{-0,2cm}\\
	\vspace{0,2cm}
	\parbox{3,5cm}{/\textsubscript{$\sigma$}[ya]n\textsubscript{$\sigma$}[thu\underline{g}]\textsubscript{$\sigma$}[\underline{w}r]/} 	\parbox{9cm}{Rule 3c: Establish codas.}\vspace{-0,2cm} \\
	\parbox{3,5cm}{\hfill} 	\parbox{9cm}{$\rightarrow$ /g.w/ is possible} \\
	\parbox{3,5cm}{\hfill} 	\parbox{9cm}{$\rightarrow$ /g/ becomes a coda of the preceding syllable} \\
	\vspace{0,2cm}
	\parbox{3,5cm}{\hfill}\parbox{4cm}{\hspace{1cm}$\downarrow$}\vspace{-0,2cm}\\
	\vspace{0,2cm}
	\parbox{3,5cm}{/\textsubscript{$\sigma$}[ya\underline{n}]\textsubscript{$\sigma$}[\underline{th}ug]\textsubscript{$\sigma$}[wr]/} 	\parbox{9cm}{Rule 3c: Establish codas.}\vspace{-0,2cm}\\
	\parbox{3,5cm}{\hfill} 	\parbox{9cm}{$\rightarrow$ /n.th/ is possible}\\
	\parbox{3,5cm}{\hfill} 	\parbox{9cm}{$\rightarrow$ /n/ becomes coda of the preceding syllable} \\
	\vspace{0,2cm}
	\parbox{3,5cm}{\hfill}\parbox{4cm}{\hspace{1cm}$\downarrow$}\vspace{-0,2cm}\\
	\vspace{0,2cm}
	\parbox{3,5cm}{/yan.thug.wr/} 	\parbox{9cm}{syllabified form: [janðu\super{ŋ}gwə̆ɾ]}\vspace{-0,2cm}
\end{array}$
\end{mdframed}
\end{figure}%yanthugwr
\begin{figure}
\vspace{-0,7cm}
\caption{Syllabification of \emph{zwäfiyokwé} `I finished sth. for her'}\label{syll004}
\begin{mdframed}
$\begin{array}{l}
	\vspace{0,1cm}
	\parbox{4,3cm}{/zwäfiyokw\textsubscript{$\sigma$}[é]/} \parbox{8,7cm}{underlying representation: final \isi{schwa} (\Fsg{}) is}\vspace{-0,1cm}\\
	%\parbox{4,1cm}{\hfill} \parbox{8,7cm}{actor suffix, which is prespecified as nucleus}\\
	\parbox{4,3cm}{\hfill} \parbox{8,7cm}{prespecified as nucleus}\vspace{-0,1cm}\\
	\vspace{0,1cm}
	\parbox{4,3cm}{\hfill}\parbox{4cm}{\hspace{1cm}$\downarrow$}\vspace{-0,1cm}\\
	\vspace{0,1cm}
	\parbox{4,3cm}{/zw\textsubscript{$\sigma$}[\underline{ä}]f\textsubscript{$\sigma$}[\underline{i}]y\textsubscript{$\sigma$}[\underline{o}]kw\textsubscript{$\sigma$}[é]/} 	\parbox{8,7cm}{Rule 1: Associate each specified vowel with a} \vspace{-0,2cm}\\
	\parbox{4,3cm}{\hfill} 	\parbox{8,7cm}{nucleus.} \\
	\vspace{0,1cm}
	\parbox{4,3cm}{\hfill}\parbox{4cm}{\hspace{1cm}$\downarrow$}\vspace{-0,1cm}\\
	\vspace{0,1cm}
	\parbox{4,3cm}{/\textsubscript{$\sigma$}[\underline{zwä}]\textsubscript{$\sigma$}[\underline{fi}]\textsubscript{$\sigma$}[\underline{yo}]k\textsubscript{$\sigma$}[\underline{wé}]/} 	\parbox{8,7cm}{Rule 2: Maximise onsets.}\vspace{-0,1cm}\\
	\parbox{4,3cm}{\hfill} 	\parbox{8,7cm}{$\rightarrow$ establishes: \textsubscript{$\sigma$}[zwä], \textsubscript{$\sigma$}[fi], \textsubscript{$\sigma$}[yo], \textsubscript{$\sigma$}[wé]}\vspace{-0,1cm}\\
	\vspace{0,1cm}
	\parbox{4,3cm}{\hfill}\parbox{4cm}{\hspace{1cm}$\downarrow$}\vspace{-0,1cm}\\
	\vspace{0,1cm}
	\parbox{4,3cm}{/\textsubscript{$\sigma$}[zwä]\textsubscript{$\sigma$}[fi]\textsubscript{$\sigma$}[yo\underline{k}]\textsubscript{$\sigma$}[\underline{w}é]/} 	\parbox{8,7cm}{Rule 3c: Establish codas.}\vspace{-0,1cm} \\
\parbox{4,3cm}{\hfill}\parbox{8,7cm}{$\rightarrow$ /k.w/ is possible} \\
\parbox{4,3cm}{\hfill}\parbox{8,7cm}{$\rightarrow$ /k/ becomes coda of the preceding syllable} \\
	\vspace{0,1cm}
	\parbox{4,3cm}{\hfill}\parbox{4cm}{\hspace{1cm}$\downarrow$}\vspace{-0,1cm}\\
	\vspace{0,1cm}
	\parbox{4,3cm}{/zwä.fi.yok.wé/} 	\parbox{8,7cm}{syllabified form: [tswæɸıjɔkwə̆]}\vspace{-0,2cm}
\end{array}$
\end{mdframed}
\end{figure}%zwäfiyokwé
\begin{figure}
\caption{Syllabification of \emph{skrifzenz} `Skri's wife'}\label{syll003}
\begin{mdframed}
$\begin{array}{l}
	\vspace{0,1cm}
	\parbox{3,3cm}{/skrifzenz/} \parbox{5cm}{underlying representation}\\
	\vspace{0,1cm}
	\parbox{3,3cm}{\hfill}\parbox{4cm}{\hspace{1cm}$\downarrow$}\\
	\vspace{0,1cm}
	\parbox{3,3cm}{/skr\textsubscript{$\sigma$}[\underline{i}]fz\textsubscript{$\sigma$}[\underline{e}]nz/} 	\parbox{9,5cm}{Rule 1: Associate each specified vowel with a nucleus.} \\
	\vspace{0,1cm}
	\parbox{3,3cm}{\hfill}\parbox{4cm}{\hspace{1cm}$\downarrow$}\\
	\vspace{0,1cm}
	\parbox{3,3cm}{/s\textsubscript{$\sigma$}[\underline{kri}]f\textsubscript{$\sigma$}[\underline{ze}]nz/} 	\parbox{9,5cm}{Rule 2: Maximise onsets.} \\
	\parbox{3,3cm}{\hfill} 	\parbox{9,5cm}{$\rightarrow$ establishes: \textsubscript{$\sigma$}[\underline{kri}], \textsubscript{$\sigma$}[\underline{ze}]} \\
	\vspace{0,1cm}
	\parbox{3,3cm}{\hfill}\parbox{4cm}{\hspace{1cm}$\downarrow$}\\
	\vspace{0,1cm}
	\parbox{3,3cm}{/s\textsubscript{$\sigma$}[kri]f\textsubscript{$\sigma$}[ze\underline{nz}]/} 	\parbox{9,5cm}{Rule 3c: Establish codas.} \\
	\parbox{3,3cm}{\hfill} 	\parbox{9,5cm}{$\rightarrow$ no cluster with /nz/} \\
	\parbox{3,3cm}{\hfill} 	\parbox{9,5cm}{$\rightarrow$ /nz/ becomes the coda of the preceding syllable} \\
	\vspace{0,1cm}
	\parbox{3,3cm}{\hfill}\parbox{4cm}{\hspace{1cm}$\downarrow$}\\
	\vspace{0,1cm}
	\parbox{3,3cm}{/s\textsubscript{$\sigma$}[kri\underline{f}]\textsubscript{$\sigma$}[\underline{z}enz]/} 	\parbox{9,5cm}{Rule 3c: Establish codas.} \\
	\parbox{3,3cm}{\hfill} 	\parbox{9,5cm}{$\rightarrow$ /f.z/ is possible} \\
	\parbox{3,3cm}{\hfill} 	\parbox{9,5cm}{$\rightarrow$ /f/ becomes the coda of the preceding \isi{syllable}.} \\
	\vspace{0,1cm}
	\parbox{3,3cm}{\hfill}\parbox{4cm}{\hspace{1cm}$\downarrow$}\\
	\vspace{0,1cm}
	\parbox{3,3cm}{/\textsubscript{$\sigma$}[\underline{s}]\textsubscript{$\sigma$}[\underline{k}rif]\textsubscript{$\sigma$}[zenz]/} 	\parbox{9,5cm}{Rule 3b: Break up consonant clusters.} \\
	\parbox{3,3cm}{\hfill} 	\parbox{9,5cm}{$\rightarrow$ \isi{schwa} is inserted between /s/ and /k/} \\
	\vspace{0,1cm}
	\parbox{3,3cm}{\hfill}\parbox{4cm}{\hspace{1cm}$\downarrow$}\\
	\vspace{0,1cm}
	\parbox{3,3cm}{/s.krif.zenz/} 	\parbox{9,5cm}{syllabified form: [sə̆kɾiɸtʃe\super{n}ts]}
\end{array}$
\end{mdframed}
\end{figure}%skrifzenz

\subsection{Minimal word} \label{minwordconstraints}

We find some constraints on the minimal size of a word in Komnzo. I will describe this here, because the \isi{minimal word} helps to explain a number of phenomena. It has an impact on allophonic variation of /o/ (see \S{}\ref{phonetic-description-vowels}), vowel length in general, and \isi{epenthesis}.\\

Compared to polysyllables, monosyllabic roots have a slightly longer vowel if they are closed syllables and a very long vowel if they consist of an open \isi{syllable}. This is relevant for roots with specified vowels only, not for roots with an \isi{epenthetic vowel}. Three examples are: \emph{fk} [ɸə̆k] `buttocks', \emph{fäk} [ɸæk] `jaw', and \emph{fä} [ɸæ:] `there (\Dist{})'. In moraic theory, we could rephrase the \isi{minimal word} constraint as: ``Words with specified vowels need to be at least two morae long''.\\

We saw in \S{}\ref{phonetic-description-vowels} that the phoneme /o/ has two allophones: a short centralised rounded vowel [ɞ̯] which occurs in closed syllables and a rounded back vowel [ɔ] which occurs in open syllables. I employed this phenomenon above in \S{}\ref{syllstruc} to justify the need of \isi{syllable} weight as a concept. As for the phoneme /o/, in monosyllabic roots the difference between these \isi{syllable} types is suspended and we do find [ɔ] in closed syllables as in: \emph{gon} [\super{ŋ}gɔn] `hips' or \emph{rot} [ɾɔt] `fence type'. Thus, the \isi{minimal word} constraint overrides these allophonic rules. The constraint applies at the root level and not the level of the inflected word. For example, we find [ɔ] instead of [ɞ̆] in the verb \emph{thorsi} [ðɔɾsi] `put inside' because \emph{thorsi} is multimorphemic (\emph{thor-} `put inside' + \emph{-si} \Nmlz{}). With polysyllabic roots, this is not the case and the two variants of /o/ follow the allophonic rule as was layed out in \S{}\ref{phonetic-description-vowels}. An example is: \emph{thomonsi} [ðɔmɞ̯nsi], which consists of (\emph{thomon-} `pile up firewood' + \emph{-si} \Nmlz{}).\\

The \isi{minimal word} constraint impacts on \isi{syllabification} because there are two variants for monosyllabic roots of the type CrV(C). These kinds of roots may be realised with a lengthened vowel in the nucleus. Alternatively an \isi{epenthetic vowel} may be inserted to break up the onset cluster thus creating a disyllabic form. In this case the specified vowel is of normal length and \isi{stress} does not shift to the initial \isi{epenthetic vowel} but remains with the specified vowel. Examples are: \emph{srak} [sɾak] $\sim$ [sə̆ɾak] `boy' and \emph{zra} [tsɾa:] $\sim$ [tsə̆ɾa] `swamp'.

\subsection{Stress} \label{stress}

Stress is a syllable-level phenomenon in Komnzo. A stressed \isi{syllable} is marked by a clearer pronunciation, higher intensity and sometimes higher pitch. Vowel length is not an acoustic correlate of \isi{stress} and even the \isi{epenthetic vowel} (a short \isi{schwa}) is frequently stressed. That being said, specified vowels usually become more centralised and shortened in word-final position which is always unstressed.\\

The domain of primary \isi{stress} (marked by ˈ in the examples) is the initial \isi{syllable} of a word. There are a number of exceptions to initial \isi{stress} which I will describe below. Secondary \isi{stress} (marked by ˌ in the examples) carries little function in Komnzo and it is often hard to distinguish from unstressed syllables. Secondary \isi{stress} is absent in bi- and trisyllabic words. Only few roots have more than three syllables and none have more than four. An example of a four-\isi{syllable} root is \emph{ˈngeˌmäku} /n.ge.mä.ku/ [nə̆\super{ŋ}gemæku] `term of address between foster parent and real parent'. It follows, that all words with more than four syllables are polymorphemic. For example, inflected verbs often comprise more than four syllables as in: \emph{ˈkwamnzokˌwrmth} /kwam.nzok.w.r.mth/ [k\super{w}am\super{n}dzɞ̆kwə̆rə̆mə̆θ] `They were dancing.'\\

There are some exceptions to initial \isi{stress}. For example, in partial \isi{reduplication} (\S{}\ref{nomreduplication}) the first \isi{syllable} is unstressed as in: \emph{rˈrokar} /r.ro.kar/ `things'. In full \isi{reduplication}, we find initial \isi{stress} \emph{ˈrokarˈrokar} as with the corresponding singleton form \emph{ˈrokar}. A second environment in which the first \isi{syllable} is unstressed are inflected verbs with a \isi{proclitic}. An example is the form \emph{bŋatrakwr} /b.ˈŋa.trak.wr/ `s/he falls there'. The \isi{proclitic} \emph{b=} (\Med{}) is added on an `outer layer' to the otherwise fully inflected verb. Cases like partial \isi{reduplication} and verbal proclitics should be seen as exceptions to the rule of initial \isi{stress}.\\

Stress is assigned from left to right. Words of up to four syllables construct a disyllabic trochee foot. In Table \ref{stresspattern} below, I present templatic \isi{stress} patterns for words between two and four syllables of length.

\begin{table}
\caption{Stress patterns of words with two to four syllables}
\label{stresspattern}
	\begin{tabular}{llll}
		\lsptoprule
		\textsc{syllable}&\textsc{example}&\textsc{phonetic}&\textsc{gloss}\\
		\textsc{structure}&&&\\
		\midrule
		ˈ$\sigma$$\sigma$& ˈ\emph{nzäthe} &[\super{n}dʒæðe]& `namesake'\\
		&ˈ\emph{ebar}& [ʔe\super{m}baɾ]& `head'\\
		&ˈ\emph{nzrm}& [\super{n}dʒə̆rə̆m]& `flower'\\
		&&&\\
		ˈ$\sigma$$\sigma$$\sigma$& ˈ\emph{kafara} &[kaβaɾa]& `river pandanus'\\
		&ˈ\emph{bägwrm}& [bæ\super{ŋ}g\super{w}ə̆rə̆m]& `butterfly'\\
		&ˈ\emph{krbu}& [kə̆rə̆\super{m}bu]& `swelling'\\
		&&&\\
		ˈ$\sigma$$\sigma$ˌ$\sigma$$\sigma$& ˈ\emph{nänzüth}ˌ\emph{zsi} &[næ\super{n}dʒʏθtsə̆si]& `cover with soil/mud'\\
		& ˈ\emph{kuku}ˌ\emph{fasi}&[kukuɸasi]&`Grey Shrike-trush'\\
		& ˈ\emph{kde}ˌ\emph{wawa}&[kə̆\super{n}dewawa]&`firefly'\\
		\lspbottomrule
	\end{tabular}
\end{table}%Stress patterns of words with two to four syllables

Words with more than four syllables vary in their assignment of secondary \isi{stress}. Most five-\isi{syllable} words assign secondary \isi{stress} to the third \isi{syllable}, but some assign it to the fourth. Most six-\isi{syllable} and seven-\isi{syllable} words assign secondary \isi{stress} to the fourth \isi{syllable}, thus, constructing a tri-syllabic foot, but there are also exceptions. Variation in words with more than four syllables might be explained in terms of open vs. closed syllables, or in terms of specified vs. \isi{epenthetic vowel} nucleus. The nature of secondary \isi{stress} in Komnzo remains to be investigated in more detail.

\section{Morphophonemic Processes} \label{morphophonology}

The following section addresses \isi{morphophonemic processes} which occur through affixation or cliticisation.

\subsection{Vowel harmony after \emph{-wä}} \label{vowharmwae}

The emphasiser suffix \emph{-wä} attaches to \isi{nominal}s. Affixation of \emph{-wä} causes a change in the quality of the vowel of the preceding \isi{syllable} regardless whether this \isi{syllable} is part of the root or another suffix. Depending on the vowel quality its impact can be described as fronting or rounding. Some examples are given in Table \ref{vowelharmwae}.

\begin{table}
\caption{Vowel harmony caused by \emph{=wä}}
\label{vowelharmwae}
	\begin{tabular}{lll}
		\lsptoprule
		\textsc{process}&\textsc{example}& \textsc{example with} \emph{=wä} \\ \midrule
		fronting of /o/&\emph{karfo} `to the village' & \emph{kar=fö=wä} \\
		&&village=\Abl{}=\Emph{}\\
		&\emph{bobo} `towards there' & \emph{bobö=wä}\\
		&&\Med{}.\All{}=\Emph{}\\
		&&\\
		raising of /a/&\emph{nima} `this way' & \emph{nimä=wä}\\
		&&like.this=\Emph{}\\
		&\emph{bafanema} `because of that one' & \emph{baf=ane=mä=wä}\\
		&&\Recog=\Poss=\Char=\Emph{}\\
		&&\\
		rounding of /e/&\emph{zafe} `long ago' & \emph{zafö=wä}\\
		&&long.ago=\Emph\\
		&\emph{etfthme} `overnight' & \emph{etfth=mö=wä}\\
		&&sleep=\Ins=\Emph\\
		\lspbottomrule
	\end{tabular}
\end{table}	%Vowel harmony caused by \emph{=wä}

The \isi{vowel harmony} does not affect vowels in a closed \isi{syllable}: \emph{kafarwä} `really big' not \textsuperscript{$\ast$}\emph{kafärwä} or \emph{dö kerwä} `really the lizard tail' not \textsuperscript{$\ast$}\emph{dö körwä}. The process is blocked by two intervening consonants. Vowel harmony of this type is restricted to morphophonemics because we do find lexemes where the vowels in question occurs in adjacent syllables, as in \emph{namä} `good' or \emph{dowä} `Wompoo Fruit Dove'.

\subsection{Dissimilation between prefix and verb stem} \label{vowelharmverbstem}

We find a number of verb stems in which the vowel quality of the prefix is raised from /ä/ to /e/. This occurs only in inflections which build on the \isi{restricted stem}, i.e. it is the prefix vowel which encodes the dual versus non-dual contrast. The vowel /ä/ marks usually \isi{non-dual}, whereas /a/ or \isi{zero} mark dual number. see \S{}\ref{roots-and-temp} for stem types and \S{}\ref{prerootdual} for a description of dual marking. Dissimilation targets the non-dual /ä/ and raises it to /e/. The trigger is the first vowel of the verb stem. Raising takes place when the first vowel is either /a/ or /ä/, for two verb stems it is /ö/. Some examples are: \emph{mar-} `see', \emph{far-} `set off', \emph{faf-} `hold' and \emph{wär-} `crack, happen', \emph{rä-} `be, do', \emph{räs-} `erect', \emph{söbäth-} `ascend' and \emph{sörfäth-} `descend'.\footnote{The majority of Komnzo verbs have two verb stems, a restricted and an extended stem (See \S\ref{roots-and-temp}). I list the restricted stems here, because the first vowel of the stem is relevant here. Elsewhere in this grammar, I use the extended stem or the nominalisation to refer to verbs. Therefore, I provide the respective extended verb stems here: \emph{mar-} `see', \emph{fark-} `set off', \emph{fa-} `hold', \emph{wä-} `crack, happen', \emph{rä-} `be', \emph{räz-} `erect', \emph{mrä-} `stroll', \emph{thfä-} `jump', \emph{thkäfak-} `start', \emph{sog-} `ascend', \emph{rsör-} `descend'.} Thus, for verbs like \emph{marasi} the non-dual of a \isi{recent past} perfective is not realised as \textsuperscript{$\ast$}\emph{zämar} but \emph{zemar} `he looked at himself'. Depending on \isi{syllabification} and intervening prefixes, the trigger vowel in the verb stem and the prefix can be separated by another \isi{syllable}. In most cases, this is a \isi{syllable} created by \isi{epenthesis}. Verb stems like \emph{mräs-} `stroll', \emph{thfär-} `jump' and \emph{thkäf-} `start' have an \isi{epenthetic vowel} after the first consonant in their nominalisations, for example \emph{mräzsi} /m.rä.z.si/ `stroll'. In the inflected verb form, the initial consonant is syllabified as a coda: \emph{zemräs} `he strolled around' (syllabified as /zem.räs/). If the \isi{ventive} prefix \emph{n-} is added to the inflection, trigger vowel and prefix vowel are separated by another \isi{syllable}, but this does not affect the raising: \emph{zenmräs} `he strolled towards here' (syllabified as /zen.m.räs/). The raising pattern described here applies to inflections of various TAM categories (irrealis, imperatives, iteratives). They all share the use of the \isi{restricted stem} and, consequently the fact that the vowel in the prefix encodes duality.\\

A special case is the \isi{copula} \emph{rä-}. Although highly irregular in many ways, it follows the dissilimation pattern just described. What is special about the copula is that the \isi{past} suffix \emph{-a} triggers the same kind of raising in the stem of the copula. Thus, we find \emph{erera} `they were' instead of \textsuperscript{$\ast$}\emph{erära}.\\

Raising of the prefix vowel is a morphophonemic process, not a general phonological process. For example, we do find lexemes where /ä/ and /a/ occur in adjacent syllables (\emph{atätö} `tree type' (Pouteria sp), \emph{mätraksi} `bring out'); the same goes for /ä/ and /ä/ (\emph{krätär} `tree type' (Oriocalis sp), \emph{thäfäm} `ripples'). Moreover, the /ä/ vowel is not raised to /e/ in verb inflections that build on the \isi{extended stem}. Consider the \Stnsg{} \emph{e-} and the \Tsg.\F{} \emph{w-} of the alpha \isi{prefix series}. When the \isi{valency} changing prefix \emph{a-} is added to the inflection, these two formatives are realised as \emph{ä-} and \emph{wä-} respectively (see \S{}\ref{personprefsection}). However, the /ä/ vowel in these formatives is not raised to \emph{e-} in inflected verb forms, for example \emph{wäfänzr} `he shows her' and not \textsuperscript{$\ast$}\emph{wefänzr}. One reason for this might be that raising the vowel to /e/ would neutralise the \isi{valency} changing prefix \emph{a-}. Another explanation might be that the raising pattern developed together with pre-stem dual marking, which is only found with \isi{restricted stem}. Restricted stems in turn do not combine with the prefixes of the alpha series (see \S{}\ref{personprefsection}), which explains why these are not affected.

\subsection[Approximant  ↔ high vowel]{Approximant $\leftrightarrow$ high vowel} \label{approxhighvowel}

In two different parts of the verbal inflectional paradigm, a change from the approximants to high vowels ([w] $\rightarrow$ [u] or [ü], and [y] $\rightarrow$ [i]) and the reverse from [u] to [w] is found.\\

All of the verbal proclitics consist only of a consonant, e.g. the \isi{immediate past} \emph{n=} or the three \isi{deictic} proclitics \emph{z=} \Prox{}, \emph{b=} \Med{}, and \emph{f=} \Dist{}. These are cliticised to otherwise fully inflected verbs. In most cases, this creates an extra \isi{syllable} word initially as in \emph{bŋatrakwr} /b.ŋa.trak.wr/ `s/he falls there'. Some of the verb prefixes in the alpha series begin with an \isi{approximant} (\emph{wo-} \Fsg{}, \emph{w-} \Tsg.\F{}, and \emph{y-} \Tsg.\Masc{}). If the clitics are attached to these forms the high approximants are realised as high vowel: \emph{u-} \Fsg{}, \emph{ü-} \Tsg.\F{}, and \emph{i-} \Tsg.\Masc{}. A few examples are given in (\ref{approx-vow-2}-\ref{approx-vow-1}) below.

\begin{exe}
	\ex \emph{burera}\\
	\gll b=wo-rä-ra\\
	\Med{}=\Fsg.\Alph-\Cop.\Ndu{}-\Pst{}\\
	\trans `I was there.'
	\label{approx-vow-2}
	\vspace{.7cm}
	\ex \emph{zimithgr}\\
	\gll z=y-mi-thgr\\
	\Prox{}=\Tsg.\Masc.\Alph-hang-\Stat.\Ndu{}\\
	\trans `It hangs here.'
	\label{approx-vow-3}

	\ex \emph{zürugr}\\
	\gll z=w-rugr\\
	\Prox{}=\Tsg.\F.\Alph-sleep.\Ndu{}\\
	\trans `She sleeps here.'
	\label{approx-vow-1}
\end{exe}

Another change which involves high vowels and approximants is attested only for [u] $\leftrightarrow$ [w]. The formatives of one of the subseries of beta (\Betatwo) end in a [u] vowel, for example \emph{ku-} \Fsg{}, \emph{su-} \Tsg.\Masc{}, \emph{thu-} \Stnsg{}. The \isi{valency} changing prefix \emph{a-} occurs between the beta prefix and the verb stem , for example \emph{ku-a-} `for me', \emph{su-a-} `for him', \emph{thu-a-} `for you/them'. In this case, the [u] becomes part of an onset consonant cluster and is realised as a high back \isi{approximant} [w]. An example is given in (\ref{approx-vow-4}-\ref{approx-vow-5}).

\begin{exe}
	\ex \emph{thufsinzr}\\
	\gll thu-fsi-nzr-\Zero{}\\
	\Stnsg{}.\Betatwo{}-count.\Ext{}-\Ndu{}-\Stsg{}\\
	\trans `S/he counted them.'
	\label{approx-vow-4}

	\ex \emph{thwafsinzr}\\
	\gll thu-a-fsi-nzr-\Zero{}\\
	\Stnsg{}.\Betatwo{}-\Vc{}-count.\Ext{}-\Ndu{}-\Stsg{}\\
	\trans `S/he counted for them.'
	\label{approx-vow-5}
\end{exe}

\section{Loanwords and loanword phonology} \label{loanword-phonology}

A number of speech sounds are restricted to \isi{loanword}s. These are the voiced oral stops [b], [d], and [g], the \isi{lateral} \isi{approximant} [l] and a few diphthongs. The `donor languages' of almost all loanwords found in Komnzo are either \ili{English} or \ili{Hiri Motu}. Only few loanwords come from \ili{Bahasa} Indonesia, for example the terms for introduced fish species: \emph{ikan lele} `Clarias batrachas', \emph{mujair} `Oreochromis mossambicus', \emph{gastor} `Channa striata'. An increasing number of people start to learn the third offical language of Papua New Guinea - Tok Pisin - and sometimes expressions like \emph{maski} `nevermind' can be heard amongst younger Komnzo speakers. Otherwise Tok Pisin plays only a minor role in loanwords.\\

From the degree of indigenisation of loanwords we can distinguish at least two periods: an early phase which lasted until the 1960s and a second phase from that time until today. The boundary between the two periods is rather fuzzy. The first period was characterised by \ili{English} speaking patrol officers and officials who visited the area for very short periods. The second period began with the opening of a Mission school in Rouku in the mid 1960s. At the beginning, the language of instruction was \ili{Hiri Motu}. In the 1970s the school was moved to Morehead and since then, the language of instruction is \ili{English}. We find linguistic evidence for the two periods. Loanwords from the first period have undergone indigenisation in order to adapt to Komnzo \isi{phonology}. Loans which entered the language during the second period are much closer to the original \ili{English} or Motu pronunciation. An example is the word \emph{doctor}. While it is pronounced [dokta] nowadays, some older speakers still use a second variant \emph{nzokta} [\super{n}dzokta] which they report was common in their parent's and grandparent's generation.\\

Words from the first period are: \emph{frayn misin} [ɸɾajə̆n mısın] `plane, flying machine', \emph{kas raba} [kas ɾa\super{m}ba] `gas lamp', \emph{dis} [\super{n}di:s] `dish, plate', \emph{damaki} [\super{n}damakı] `dynamite'. We find regular correspondences of \ili{English} phonemes mapping onto Komnzo \isi{phonology}. The bilabial stop [p] becomes a bilabial fricative [ɸ] in \emph{frayn misin}, but in a cluster with the bilabial \isi{nasal} [m] in \emph{kas raba} it becomes a prenasalised voiced bilabial stop [\super{m}b]. The \isi{velar} voiced stop [g], also in \emph{kas raba}, comes out as a voiceless \isi{velar} stop [k]. The \isi{lateral} \isi{approximant} [l] in \ili{English} \emph{flying} becomes an \isi{alveolar} tap or trill [ɾ $\sim$ r] in Komnzo \emph{frayn} and again in \emph{kas raba}. The \ili{English} \isi{diphthong} [a͡i] in `dynamite' is monophthongised in \emph{damaki}. The voiced \isi{alveolar} stop [d] becomes prenasalised [\super{n}d] in \emph{damaki} and \emph{dis}. In the same word, the post-\isi{alveolar} fricative [ʃ] turns into an \isi{alveolar} fricative [s]. However, there are too few loans from this early period to make a systematic comparison of all \ili{English} phonemes in different environments.\\

The second period which lasts until today is characterised by loan phonemes. Indigenisation is found to a lesser degree. The second period is also characterised by the influx of loans from \ili{Hiri Motu}. We find loan phonemes in the oral voiced stops [b], [d] and [g] as in: \emph{bara} `paddle', \emph{durua} `help', \emph{dibura} `prisoner', \emph{gunana} `place name'\footnote{\emph{Gunana} means `the former (one)' in Hiri Motu. In Komnzo, it designates a place `where old Rouku used to be' as informants put it. A new hamlet was founded there a few years ago.} from \ili{Hiri Motu} and \emph{baisikol} `bicycle' from \ili{English}. Note that the \ili{English} \isi{diphthong} [a͡ı] is retained and not monophthongised and the \isi{lateral} \isi{approximant} [l] also does not change.\\

There are two correspondences which we find in both periods. The first is between the voiceless bilabial stop [p] in \ili{English} and the voiceless bilabial fricative [ɸ] in Komnzo. The second correspondence is between the \isi{lateral} \isi{approximant} [l] and the \isi{alveolar} trill/flap [ɾ $\sim$ r]. It seems, in the early period, [l] was changed in all environments, but the second period this only occurs in [pl] clusters in \ili{English}. Elsewhere, [l] is taken over into Komnzo as a loan phoneme. We have seen some examples from the first period above. Examples from the second period are: \emph{fren} `plane', \emph{fenzil} `pencil', and \emph{sosfen} `saucepan'.

\section{Orthography development}\label{orthographydev}

There is no writing tradition in Komnzo, but most people can read and write in one of the official languages, namely \ili{English} and Motu. The mission school, which was based at Rouku during the 1960's, operated in Motu, but today \ili{English} is the teaching language at the primary school in Morehead. Thus, reading and writing in Komnzo has not been promoted in the past. As a consequence, literacy in one's mother tongue is an alien concept for most Komnzo speakers.\\

The first attempt to develop an \isi{orthography} for Komnzo was during an alphabet workshop organised by Marco and Alma Bouvé at Morehead Station in 2000. It brought together representatives from a dozen villages. The two representatives from Rouku were Greg Marua and Wendy Yasii. When I began my work in Rouku, this \isi{orthography} was not used except for a few words that were written on the blackboard in the elementary school. Regrettably, the Rouku elementary school has been disfunctional since 2010. During my fieldwork I have organised two \isi{orthography} meetings. The outcome of these meetings was the Komnzo Language Council which includes representatives of all clans. The language council has remained an abstract administrative body overseeing my work. In practice, I concentrated most translation and elicitation work on 4-5 interested individuals. Together, we have revised the \isi{orthography} several times. Table \ref{orthogcons} and Figure \ref{orthogvow} show the differences between the \isi{orthography} from the workshop in 2000 and the current \isi{orthography}. Changes are shown with an arrow ($\rightarrow$).

{\renewcommand{\tabcolsep}{2,3pt}
\begin{table}
\caption{Comparison of orthographies: consonants}
\label{orthogcons}
	\begin{tabular}{p{1,5cm}ccccccc}
		\lsptoprule
		& \scriptsize{bilabial}& \scriptsize{dental} & \scriptsize{alveolar} & \scriptsize{palato-alveolar}	& \scriptsize{palatal} & \scriptsize{velar} & \scriptsize{labio-velar} \\ \midrule
		\scriptsize{stop} \& && \multicolumn{2}{c}{}&&&&\\
		\scriptsize{affricate}	&b $\rightarrow$ n/a& \multicolumn{2}{c}{t}& ts $\rightarrow$ z&& k & n/a $\rightarrow$ kw \\%[-0.5ex]
		&&&&&&&\\
		\scriptsize{prenasalised} && &&&&&\\
		\scriptsize{stop \&} & mb $\rightarrow$ b & & nt $\rightarrow$ d & nj $\rightarrow$ nz && n\th $\rightarrow$ g & n/a $\rightarrow$ gw\\
		\scriptsize{affricate}&&&&&&&\\%[-0.5ex]
		&&&&&&&\\
		\scriptsize{fricative} 	& f	& th & s &&&&\\%[-0.5ex]
		&&&&&&&\\
		\scriptsize{nasal} & m && n &&& ng $\rightarrow$ ŋ & \\%[-0.5ex]
		&&&&&&&\\
		\scriptsize{lateral} &&& r &&&&\\%[-0.5ex]
		&&&&&&&\\
		\scriptsize{semivowel} &&&&&y && w\\
		\lspbottomrule
	\end{tabular}
\end{table}}%Comparison of orthographies: consonants
%}

\begin{figure}
\centering
{
\begin{vowel}[simple]
	%\caption[Komnzo vowels]{Komnzo vowels}
	\putvowel{i}{0,3\vowelhunit}{0,4\vowelvunit}
	\putvowel{ú}{1,3\vowelhunit}{0,4\vowelvunit}
	\putvowel{e}{0,85\vowelhunit}{1,5\vowelvunit}
	\putvowel{\^{o}}{1,7\vowelhunit}{1,5\vowelvunit}
	\putvowel{á}{1,6\vowelhunit}{2,55\vowelvunit}
	\putvowel{u}{4\vowelhunit}{0,4\vowelvunit}
	\putvowel{a}{2,9\vowelhunit}{2,7\vowelvunit}
	\putvowel{é}{2,7\vowelhunit}{1,3\vowelvunit}
	\putvowel{o}{4\vowelhunit}{1,5\vowelvunit}
	\putvowel{ó}{3,4\vowelhunit}{2,1\vowelvunit}
\end{vowel}
} $\longrightarrow$
{
\begin{vowel}[simple]
	%\caption[Komnzo vowels]{Komnzo vowels}
	\putvowel{i}{0,3\vowelhunit}{0,4\vowelvunit}
	\putvowel{ü}{1,3\vowelhunit}{0,4\vowelvunit}
	\putvowel{e}{0,85\vowelhunit}{1,5\vowelvunit}
	\putvowel{ö}{1,7\vowelhunit}{1,5\vowelvunit}
	\putvowel{ä}{1,6\vowelhunit}{2,55\vowelvunit}
	\putvowel{u}{4\vowelhunit}{0,4\vowelvunit}
	\putvowel{a}{2,9\vowelhunit}{2,7\vowelvunit}
	\putvowel{é}{2,7\vowelhunit}{1,3\vowelvunit}
	\putvowel{o}{4\vowelhunit}{1,5\vowelvunit}
	%\putvowel{ó}{3,4\vowelhunit}{2,1\vowelvunit}
\end{vowel}
}%
\caption{Comparison of orthographies: vowels} \label{orthogvow}
\end{figure}%Comparison of orthographies: vowels