%!TEX root = ../main.tex

\chapter{Tense, aspect and mood} \label{TAMpalooza}

\section{Introduction} \label{TAMintro}

Tense, \isi{aspect} and \isi{mood} is the most complex set of grammatical categories in the \isi{verb} inflection, both in the way the categories are encoded and in the number of distinctions that can be expressed. Morphologically, there are 18 categories, which may be supplemented by a set of TAM particles. There are four morphological \isi{tense} values (\isi{non-past}, \isi{immediate past}, \isi{recent past} and \isi{past}), four \isi{aspect} values (\isi{perfective}, \isi{imperfective}, \isi{durative} and \isi{iterative}) and three \isi{mood} values (indicative, \isi{imperative} and \isi{irrealis}).

I will begin this section with an overview of the morphological material that is involved in TAM inflection. Most of these building blocks and the idiosyncrasies in their behaviour have been addressed in the preceding chapter and I will refer to these sections where appropriate. In the following, I will focus on the \isi{combinatorics} of the morphemes and stems (\S\ref{combitam}), the impact of clitics and particles (\S\ref{tam-particles-sec}) and the semantics of the resulting TAM categories (\S\ref{TAMsemantics}). Aspect in Komnzo can at best be somewhat misleadingly captured with the traditional definition of \isi{perfective} versus \isi{imperfective}, which is often based on the completion of an event. Although I employ these labels, it should be noted that the \isi{perfective} focusses more on the left edge of the event (inceptive) or expresses a momentaneous quality (punctual). With that in mind, I defer the discussion of the semantics of TAM to the end of this chapter (\S\ref{TAMsemantics}).

\section{The combinatorics of TAM} \label{combitam}

The most basic element of TAM inflection is the distinction between an extended (\Ext) and a restricted stem (\Rs). Both types are attested for almost every \isi{verb} lexeme ({\S}\ref{roots-and-temp}). {\Ext} and {\Rs} stems differ in their templates with respect to dual marking ({\S}\ref{dualextrs}) and in the possible combinations with the five \isi{prefix series} \Alph, \Bet, \Betaone, \Betatwo{} and \Gam{} ({\S}\ref{combinatoricsextrs}). In addition to the five series, the \isi{irrealis} prefix \emph{ra-} and the \isi{immediate past} \isi{proclitic} \emph{n=} are involved in TAM marking. The suffixal material includes a \isi{past} suffix (\emph{-a}) and a \isi{durative} suffix (\emph{-m}) and a special actor suffix series for the imperatives. \tabref{TAMpalooza1} gives a full overview of the TAM categories and the way these are built up from the listed morphological material. An important distinction in the verb template, not expressed in \tabref{TAMpalooza1}, is the difference between post-stem dual marking with {\Ext} stems and pre-stem dual marking with {\Rs} stems. This was described in detail in {\S}\ref{number-subsec}.

The combinations in \tabref{TAMpalooza1} illustrate a feature of Komnzo morphology that reverberates throughout the verb inflection: the distribution of exponents. In other words, a grammatical category is encoded and manipulated by formatives that are scattered across the verb template. On the flip side of this phenomenon, most formatives lack a clear grammatical meaning, or have multiple grammatical functions depending on the combinatorics. Thus, they have to be glossed in an abstract manner. However, there are degrees of morpheme underspecification. For example, two morphemes in \tabref{TAMpalooza1} can be assigned an unambiguous grammatical meaning. These are the \isi{irrealis} prefix \emph{ra-} and the \isi{past} suffix \emph{-a}. The \emph{-a} morpheme only occurs in \isi{past} \isi{tense} inflections, and the label \Pst{} is a sufficient gloss for the \emph{-a} suffix. However, the \emph{a-} suffix is insufficient to describe the tense value ``\isi{past}'' because other morphs, e.g. the \isi{prefix series}, are required to form a \isi{past} \isi{tense}. A second group of morphemes is underspecified in the following way: they fulfil several functions, either simultaneously or in different morphological contexts. For example, the \isi{durative} suffix \emph{-m} encodes \isi{durative} \isi{aspect}, but it also ``pushes back'' the \isi{tense} value. Thus, when suffixed to a \isi{non-past} (\isi{imperfective}), it will produce a \isi{recent past} (\isi{durative}), and when suffixed to a \isi{recent past} (\isi{imperfective}), it will produce a \isi{past} (\isi{durative}). One option would be to label it \isi{durative}/\isi{backshifting} suffix. However, in imperatives the \emph{-m} suffix pushes the tense values ``forward'', producing a delayed \isi{imperative} (`do X a little later'), and duration is not part of its meaning. Furthermore, the \emph{-m} suffix may occur with perfectives as a means of backgrounding an event, again without encoding duration. Thus, the choice of the glossing label (\Dur) for the \emph{-m} suffix is somewhat arbitrary, and we could just as well label it ``\isi{tense} shifting'' or ``backgrounding''. For a third group of morphemes, especially the five \isi{prefix series}, all attempts to assign a grammatical meaning to them is futile and we have to draw on abstract labels like \Alph{}, \Bet{} and \Gam{}.

\begin{sidewaystable}
	\caption{The combinatorics TAM marking}
	\label{TAMpalooza1}
\fittable{
\begin{tabular}{lllc|c|c|c|c|c|c|}
	\lsptoprule
	\multicolumn{3}{c}{\multirow{3}{*}{\textsc{tam} value}}	&\multicolumn{1}{c}{}&\multicolumn{1}{c}{clitic} &\multicolumn{1}{c}{prefix series}&\multicolumn{1}{c}{\textsc{irr} prefix}&\multicolumn{1}{c}{stem type}&\multicolumn{1}{c}{\textsc{tam} suffix}&\multicolumn{1}{c}{\textsc{imperative} suffix}\\
	&&&\multicolumn{1}{c}{}&\multicolumn{1}{c}{\emph{n=}}&\multicolumn{1}{c}{\Alph, \Bet, \Betaone, \Betatwo, \Gam}&\multicolumn{1}{c}{\emph{ra-}}&\multicolumn{1}{c}{{\Ext}}&\multicolumn{1}{c}{\Pst{} (\emph{-a})}&\multicolumn{1}{c}{{\Imp}/\Ssg (\emph{-é})}\\
	&&&\multicolumn{1}{c}{}&\multicolumn{1}{c}{}&\multicolumn{1}{c}{}&\multicolumn{1}{c}{}&\multicolumn{1}{c}{{\Rs}}&\multicolumn{1}{c}{\Dur{} (\emph{-m})}&\multicolumn{1}{c}{\Snsg \emph{-e}}\\\midrule
	\multicolumn{9}{c}{}\\\cline{6-6}\cline{8-8}
	non-past &imperfective&indicative&\multicolumn{1}{c}{}&&\Alph{}&&\Ext&\multicolumn{2}{c}{}\\\cline{5-6}\cline{8-8}
	immediate-past &imperfective&indicative&&\emph{n=}&\Alph{}&&\Ext&\multicolumn{2}{c}{}\\\cline{5-6}\cline{8-9}
	immediate-past &durative&indicative&&\emph{n=}&\Alph{}&&\Ext&\emph{-m}&\multicolumn{1}{c}{}\\\cline{5-6}\cline{8-9}
	recent-past &imperfective&indicative&\multicolumn{1}{c}{}&&\Betaone{} or \Betatwo&&\Ext&\multicolumn{2}{c}{}\\\cline{6-6}\cline{8-9}
	recent-past &durative&indicative&\multicolumn{1}{c}{}&&\Alph&&\Ext&\emph{-m}&\multicolumn{1}{c}{}\\\cline{6-6}\cline{8-9}
	recent-past &perfective&indicative&\multicolumn{1}{c}{}&&\Gam{}&&\Rs&\multicolumn{2}{c}{}\\\cline{6-6}\cline{8-9}
	past &imperfective&indicative&\multicolumn{1}{c}{}&&\Alph&&\Ext&\emph{-a}&\multicolumn{1}{c}{}\\\cline{6-6}\cline{8-9}
	past &durative&indicative&\multicolumn{1}{c}{}&&\Betaone{} or \Betatwo&&\Ext&\emph{-m}&\multicolumn{1}{c}{}\\\cline{6-6}\cline{8-9}
	past &perfective&indicative&\multicolumn{1}{c}{}&&\Gam&&\Rs&\emph{-a}&\multicolumn{1}{c}{}\\\cline{6-6}\cline{8-9}
	past&iterative&indicative&\multicolumn{1}{c}{}&&\Betaone{} or \Betatwo&&\Rs&\multicolumn{2}{c}{}\\\cline{6-6}\cline{8-9}
	past&iterative/durative&indicative&\multicolumn{1}{c}{}&&\Betaone{} or \Betatwo&&\Rs&\emph{-m}&\multicolumn{1}{c}{}\\\cline{6-9}
	n/a&imperfective&irrealis&\multicolumn{1}{c}{}&&\Bet&\emph{ra-}&\Ext&\multicolumn{2}{c}{}\\\cline{6-9}
	n/a&durative&irrealis&\multicolumn{1}{c}{}&&\Bet&\emph{ra-}&\Ext&\emph{-m}&\multicolumn{1}{c}{}\\\cline{6-9}
	n/a&perfective&irrealis&\multicolumn{1}{c}{}&&\Bet&\emph{ra-}&\Rs&\multicolumn{2}{c}{}\\\cline{6-8}\cline{10-10}
	n/a&imperfective&imperative&\multicolumn{1}{c}{}&&\Bet&&\Ext&\multicolumn{1}{c|}{}&{\Imp}\\\cline{6-6}\cline{8-8}\cline{10-10}
	n/a&perfective&imperative&\multicolumn{1}{c}{}&&\Bet&&\Rs&\multicolumn{1}{c|}{}&{\Imp}\\\cline{6-6}\cline{8-10}
	future&imperfective&imperative&\multicolumn{1}{c}{}&&\Bet&&\Ext&\emph{-m}&{\Imp}\\\cline{6-6}\cline{8-10}
	future&perfective&imperative&\multicolumn{1}{c}{}&&\Bet&&\Rs&\emph{-m}&{\Imp}\\\cline{6-6}\cline{8-10}
	\multicolumn{9}{c}{}\\
	\lspbottomrule
\end{tabular}
}
\end{sidewaystable} 

Not all logically possible combinations of morphs are grammatically acceptable. For example, the \Alph{} and \Gam{} \isi{prefix series} only combine with {\Ext} and {\Rs} stems, respectively, but not vice versa. Likewise, the \isi{past} suffix \emph{-a} and the \isi{durative} suffix \emph{-m} are mutually exclusive and a \isi{verb} form with both is rejected as ungrammatical. Third, the \isi{irrealis} prefix \emph{ra-} only combines with the \Bet{} prefixes and not with the other \isi{prefix series}. Lastly, the \isi{immediate past} \isi{clitic} \emph{n=} can only attach to a verb form which employs the \Alph{} \isi{prefix series}, not to the other combinations. We can conclude from this observation that the combinatorial space is not fully exhausted, i.e. not all logically possible combinations of the morphological material are actually employed. Such a system is not surprising because all natural languages evolve incrementally without an overall design. What is remarkable about Komnzo in particular and the \ili{Yam languages} in general is the fact that so many combinations are employed. In other words, the genius of the verb morphology lies in its extensive exploitation of combinations.

In the following section, I will describe the functions and some of the distributional characteristics of the morphemes in \tabref{TAMpalooza1}.

\clearpage 
\subsection{The prefix series} \label{tamprefixseries}

The five \isi{prefix series} \Alph, \Bet, \Betaone, \Betatwo, \Gam{} were briefly addressed in {\S}\ref{personprefsection}. The table from page \pageref{perspref} is reproduced as \tabref{perspref2}.

\begin{table}
\caption{TAM prefixes}
\label{perspref2}
	\begin{tabularx}{\textwidth}{XXXXXl}
		\lsptoprule
		{gloss} &\Alph &\Bet &\Betaone &\Betatwo	&\Gam\\\midrule
		\Fsg &\emph{wo-} &\emph{kw-} &\emph{ku-} &\emph{kwof-} &\emph{zu-}\\
		\Fnsg &\emph{n-} &\emph{nz-} / \emph{nzn-} &\emph{nzu-} &\emph{nzf-} &\emph{nzn-}\\
		\Ssg &\emph{n-} &\emph{nz-} / \emph{gn-} &\emph{gu-} &\emph{gf-} &\emph{nzn-}\\
		\Tsg.\F &\emph{w-} &\emph{z-} &\emph{zu-} &\emph{zf-} &\emph{z-}\\
		\Tsg.\Masc &\emph{y-} &\emph{s-} &\emph{su-} &\emph{sf-} &\emph{s-}	\\
		\Stnsg &\emph{e-} &\emph{th-} &\emph{thu-} &\emph{thf-} &\emph{th-}\\
		\M &\emph{ŋ-} &\emph{k-} &\emph{kw-} &\emph{kf-} &\emph{z-}\\
		\lspbottomrule
	\end{tabularx}
\end{table}%TAM prefixes

The \Alph{} prefixes combine only with the extended stem. They are used to encode \isi{non-past} (\ref{ex276}), \isi{recent past} \isi{durative} (\ref{ex275}) and \isi{past} \isi{imperfective} (\ref{ex277}). Example (\ref{ex276}) comes from a hunting story, where the narrator meets a spiritual being in the forest. In (\ref{ex275}), the speaker reports an incident from a neighbouring village involving a young boy who was attacked by a sorcerer in his yam garden. Example (\ref{ex277}), is from an interview about the customs around the sister-exchange marriage system.

\begin{exe}
	\ex \emph{``nzä maf \textbf{wonrsoknwr}?''}\\
	\glll nzä maf wo-n-rsokn-wr-\Zero{}\\
	\Fsg.{\Abs} who.{\Erg} \Fsg.\Alph-\Venit-bother.\Ext-\Ndu-\Stsg{}\\
	~ ~ {\footnotesize \Stsg:\Sbj>\Fsg:\Obj:\Nonpast:\Ipfv:\Venit/bother}\\
	\trans ```Who bothers me here?''' \Corpus{tci20111119-03}{ABB \#165}
	\label{ex276}
\end{exe}
\begin{exe}
	\ex \emph{fthé zöfthamen zamatho frk komnzo zä \textbf{wtnägwrmo}.}\\
	\glll fthé zöftha=thamen z-a-math-o-\Zero{} frk komnzo zä\\
	when first=\Temp.{\Loc} \M.\Gam-\Ndu-run.\Rs-\Andat-{\Sg} blood only {\Prox}\\
	~ ~ {\footnotesize \Sg:\Sbj:\Rpst:\Pfv:\Andat/run} ~ ~ ~\\
	\sn
	\glll w-tnäg-wr-m-o-\Zero\\
	\Tsg.\F.\Alph-lose.\Ext-\Ndu-\Dur-\Andat-{\Sg}\\
	{\footnotesize \Sg:\Sbj>\Tsg.\F:\Obj:\Rpst:\Dur:\Andat/lose}\\
	\trans `At first, when he started to run, he was just losing blood here.'\\ \Corpus{tci20130901-04}{YUK \#40}
	\label{ex275}
\end{exe}
\begin{exe}
	\ex \emph{nzun etha nzüthamöwä \textbf{warnzürwrath} wath.}\\
	\glll nzun etha nzüthamöwä wo-a-rnzür-wr-a-th wath\\
	\Fsg.{\Dat} three times \Fsg.\Alph-\Vc-dance.\Ext-\Ndu-\Pst-\Stnsg{} dance\\
	~ ~ ~ {\footnotesize \Stpl:\Sbj>\Fsg:\Io:\Pst:\Ipfv/dance} ~\\
	\trans `They danced three times for me.'\Corpus{tci20120805-01}{ABB \#769}
	\label{ex277}
\end{exe}

If the \isi{proclitic} \emph{n=} is attached to a \isi{verb} employing the \Alph{} prefixes, the resulting inflection is either \isi{immediate past} imperfective (\ref{ex273}) or \isi{immediate past} \isi{durative} (\ref{ex274}) depending on the suffixal material. In other words, the \isi{immediate past} is built from verbs inflected for \isi{non-past}. This is preserved in the integrated glossing style, because the \emph{n=} is analysed as a \isi{clitic}. The \emph{n=} is related to the \isi{imminent} \isi{particle} \emph{n} ({\S}\ref{imminentm}). Example (\ref{ex273}) sums up a story about the origin of the Morehead people. In (\ref{ex274}), the speaker talks about competitive yam cultivation and how older people assess a young man's status by the number and size of his crop.

\begin{exe}
	\ex \emph{trikasi mane \textbf{nŋatrikwé} fof ... ŋafynm ... badafa ane fof ŋanritakwa fof.}\\
	\glll trik-si mane n=ŋ-a-trik-w-é fof (.) ŋafe=nm (.)\\
	tell-{\Nmlz} which \Immpst=\M.\Alph-\Vc-tell.\Ext-\Ndu-\Fsg{} {\Emph} (.) father=\Dat.{\Nsg} (.)\\
	~ ~ {\footnotesize \Immpst=\Fsg:\Sbj:\Nonpast:\Ipfv/tell} ~ ~ ~ ~\\
	\sn
	\glll bada=fa ane fof ŋ-a-n-ritak-w-a-\Zero{} fof\\
	ancestor={\Abl} {\Dem} {\Emph} \M.\Alph-\Vc-\Venit-pass.\Ext-\Ndu-\Pst-{\Sg} {\Emph}\\
	~ ~ ~ {\footnotesize \Stsg:\Sbj:\Pst:\Ipfv:\Venit/pass} ~\\
	\trans `The story which I have just told passed from the ancestors to (our) fathers.'\\ \Corpus{tci20131013-01}{ABB \#403-405}
	\label{ex273}
\end{exe}
\begin{exe}
	\ex \emph{fthé bone kafarwä \textbf{nefathwrmth} ``eh yabun zane!'' wtrikaräsü we gnrärm.}\\
	\glll fthé bone kafar=wä n=e-fath-wr-m-th eh\\
	when \Ssg.{\Poss} big={\Emph} \Immpst=\Stnsg.\Alph-hold.\Ext-\Ndu-\Dur-\Stnsg{} eh\\
	~ ~ ~ {\footnotesize \Immpst=\Stpl:\Sbj>\Stpl:\Obj:\Nonpast:\Dur/hold} ~\\
	\sn
	\glll yabun zane wtri=karä=sü we gn-rä-r-m\\
	big \Dem:{\Prox} fear=\Prop=\Etc{} also \Ssg.\Bet-\Cop-\Ndu-\Dur{}\\
	~ ~ ~ ~ {\footnotesize \Ssg:\Sbj:\Futimp:\Ipfv/be}\\
	\trans `When they have just held your big (yam tubers) and say: ``Hey, that (is) a big one!'' then you have to be afraid!' \Corpus{tci20120805-01}{ABB \#378-380}
	\label{ex274}
\end{exe}

The \Bet{} series is split into a basic series \Bet{} and two related series \Betaone{} and \Betatwo. The basic \Bet{} series is used for all the non-tensed categories like the \isi{irrealis} (\ref{ex330}) and the imperatives (\ref{ex278}). Example (\ref{ex330}) comes from a procedural text about fish baskets and the speaker explains how the fish gets trapped inside. In (\ref{ex278}), the narrator took over the role of a character in a stimulus picture task.

\begin{exe}
	\ex \emph{watik, fthé \textbf{kranbrigwrth} keke kwa zba we \textbf{krämätroth}.}\\
	\glll watik fthé k-ra-n-brig-wr-th keke kwa zba we\\
	then when \M.\Bet-\Irr.\Vc-\Venit-return.\Ext-\Ndu-\Stnsg{} {\Neg} {\Fut} \Prox.{\Abl} also\\
	~ ~ {\footnotesize \Stpl:\Sbj:\Irr:\Ipfv:\Venit/return} ~ ~ ~ ~\\
	\sn
	\glll k-rä-mätr-o-th\\
	\M.\Bet-\Irr.\Vc.\Ndu-exit.\Rs-\Andat-\Nsg{}\\
	{\footnotesize \Pl:\Sbj:\Irr:\Pfv:\Andat/exit}\\
	\trans `Well, when they turn around, they will not escape from here.'\\ \Corpus{tci20120906}{SKK \#45}
	\label{ex330}
\end{exe}
\begin{exe}
	\ex \emph{``bné \textbf{käznobe}! nzä keke miyo worä.''}\\
	\glll bné k-ä-znob-e nzä keke miyo wo-rä\\
	\Snsg.{\Erg} \M.\Bet-\Ndu.\Vc-drink.\Rs-\Snsg.{\Imp} \Fsg.{\Abs} {\Neg} desire \Fsg.\Alph-\Cop.\Ndu\\
	~ {\footnotesize \Spl:\Sbj:\Imp:\Pfv/drink} ~ ~ ~ {\footnotesize \Fsg:\Sbj:\Nonpast:\Ipfv/be}\\
	\trans ```You drink! I don't want to.''' \Corpus{tci20111004}{RMA \#282}
	\label{ex278}
\end{exe}

\tabref{perspref2} shows that there are two formatives for the first person \isi{non-singular} (\emph{nz-} and \emph{nzn-}) as well as the second \isi{singular} (\emph{nz-} and \emph{gn-}) of the \Bet{} series. For the first \isi{person} \isi{non-singular}, \emph{nz-} is used for \isi{irrealis} (\ref{ex283}) and \emph{nzn-} for the imperatives (\ref{ex284}). In example (\ref{ex283}), the speaker explains how a kundu drum is carved and prepared. Example (\ref{ex284}) is taken from a conversation by the fire that involved a lot of hearsay information. In conclusion, the speaker tells the two addressees to go to Morehead and clarify the rumours.

\begin{exe}
	\ex \emph{fiyafr \textbf{nzrayak} tauri woku thoraksir.}\\
	\glll fiyaf=r nz-ra-yak tauri woku thorak-si=r\\
	hunting={\Purp} \Fnsg.\Bet-\Irr-walk.\Ext.{\Ndu} wallaby skin search-\Nmlz={\Purp}\\
	~ {\footnotesize \Fpl:\Sbj:\Irr:\Ipfv/walk} ~ ~ ~\\
	\trans `We will go hunting and search for wallaby skin.' \Corpus{tci20120824}{KAA \#64}
	\label{ex283}
\end{exe}
\begin{exe}
	\ex \emph{kanbrime! ... aneme nzenm \textbf{nznatrife}!}\\
	\glll k-a-n-brim-e (.) ane=me nzenm\\
	\M.\Bet-\Vc.\Du-\Venit-return.\Rs-\Snsg.{\Imp} (.) \Dem={\Ins} \Fnsg.{\Dat}\\
	{\footnotesize \Sdu:\Sbj:\Imp:\Pfv:\Venit/return} ~ ~ ~\\
	\sn
	\glll nzn-a-trif-e\\
	\Fnsg.\Bet-\Vc.\Du-tell.\Rs-\Snsg.{\Imp}\\
	{\footnotesize \Sdu:\Sbj>\Fdu:\Obj:\Imp:\Pfv/tell}\\
	\trans `You come back and tell us about it!' \Corpus{tci20130901-04}{RNA \#162}
	\label{ex284}
\end{exe}

For the second person singular, the situation is more complicated. The \emph{gn-} formative is used for the imperatives of prefixing verbs, where the prefix encodes the \isi{imperative} mood and the addressee simultanously (\ref{ex285}). The second \isi{non-singular} prefix is \emph{th-} for all inflections that involve the \Bet{} series. Note that for ambifixing verbs in the \isi{imperative}, there is no overt marking of second \isi{person} in the prefix because this would then be \isi{reflexive} (`X yourself!') or auto-\isi{benefactive} (`X for yourself!'). As pointed out in {\S}\ref{middletemplatesubsection}, reflexives and auto-benefactives are expressed in a \isi{middle} template. Hence, the first verb in example (\ref{ex284}) could be translated as a \isi{reflexive} (`return yourselves!').

\begin{exe}
	\ex \emph{ezi \textbf{gnyako}!}\\
	\glll ezi gn-yak-o\\
	morning \Ssg.\Bet.\Imp-walk.\Ext.\Ndu-\Andat\\
	~ {\footnotesize \Ssg:\Sbj:\Imp:\Ipfv:\Andat/walk}\\
	\trans `You go there in the morning!' \Corpus{tci20120906}{MAB \#31}
	\label{ex285}
\end{exe}

The second formative for the second singular in \tabref{perspref2} (\emph{nz-}) is used for \isi{irrealis} inflection of prefixing and ambifixing verbs. Interestingly, only the second \isi{person} singular of ambifixing verbs does not employ the \isi{irrealis} prefix \emph{ra-} in the \isi{irrealis} inflection (\ref{ex286}). If it is a prefixing verb, the \isi{irrealis} prefix \emph{ra-} is employed (\ref{ex287}).\footnote{Both verbs in this example are deponent employing, the valency change prefix \emph{a-} without a change in valency. The second verb \emph{yak} `walk' is only deponent when it employs the venitive marker, meaning `come', not when it is neutral or andative `walk', `go away'.} Example (\ref{ex286}) is taken from a procedural text in which the speaker shows me how to manufacture two children's toys. In (\ref{ex287}), the malignant protagonist invites a stranger to stay with her.

\begin{exe}
	\ex \emph{gräthé znsä rä ... thrma \textbf{nzasämiré} bun.}\\
	\glll grä-thé znsä rä (.) thrma nz-a-sämir-é\\
	slow-{\Adlzr} work \Tsg.\F.\Cop.{\Ndu} (.) later \Ssg.\Bet-\Vc.\Ndu-whisper.\Rs-\Fsg{}\\
	~ ~ {\footnotesize \Tsg.\F:\Sbj:\Nonpast:\Ipfv/be} ~ ~ {\footnotesize \Fsg:\Sbj>\Ssg:\Io:\Irr:\Pfv/whisper}\\
	\sn
	\gll bun\\
	\Ssg.\Dat\\
	\trans `It is easy work ... I will teach you later.' \Corpus{tci20120914}{RNA \#50-51}
	\label{ex286}
\end{exe}
\begin{exe}
	\ex \emph{nima zräzigrm ``awe nzone moba \textbf{nzranyak}?''}\\
	\glll nima z-rä-zigr-m awe nzone moba\\
	{\Quot} \Tsg.\F.\Bet-\Irr.\Vc.\Ndu-look.around.\Rs-\Dur{} come \Fsg.{\Poss} where.{\Abl}\\
	~ {\footnotesize \Tsg.\F:\Sbj:\Irr:\Pfv/look.around} ~ ~ ~\\
	\sn
	\glll nz-ra-n-yak\\
	\Ssg.\Bet-\Irr.\Vc-\Venit-walk.\Ext.\Ndu\\
	{\footnotesize \Ssg:\Sbj:\Irr:\Ipfv:\Venit/walk}\\
	\trans `She looks around and says, ``Come my friend! Where are you coming from?''' \Corpus{tci20120901-01}{MAK \#74}
	\label{ex287}
\end{exe}

The \Betaone{} and \Betatwo{} series are used for \isi{recent past} \isi{imperfective} (\ref{ex280}), \isi{past} \isi{durative} (first verb in \ref{ex279}) and \isi{past} \isi{iterative} (second verb in \ref{ex279}). In example (\ref{ex279}), the speaker talks about his experiences at the Rouku mission school in the 1960s.

\begin{exe}
	\ex \emph{kayé ama zuzir \textbf{zfyak}.}\\
	\glll kayé ama zuzi=r zf-yak\\
	yesterday mother fishing={\Purp} \Tsg.\F.\Betatwo-walk.\Ext.\Ndu\\
	~ ~ ~ {\footnotesize \Tsg.\F:\Sbj:\Rpst:\Ipfv/walk}\\
	\trans `Yesterday, mother went fishing.' \Corpus{tci20111107-03}{RNA \#40}
	\label{ex280}
\end{exe}
\begin{exe}
	\ex \emph{teste \textbf{nzwasäminzrm} bobomr kwarikwari efoth ... sokoro \textbf{kfäbth}}\\
	\glll teste nzu-a-sämi-nzr-m-\Zero{} bobomr {kwarikwari} efoth\\
	thursday \Fnsg.\Betaone-\Vc-whisper.\Ext-\Ndu-\Dur-\Stsg{} until {midday} sun\\
	~ {\footnotesize \Stsg:\Sbj>\Fpl:\Io:\Pst:\Dur/teach} ~ ~ ~\\
	\sn
	\glll (.) sokoro kf-ä-bth-\Zero{}\\
	(.) school \M.\Betatwo-\Vc.\Ndu-finish.\Rs-\Stsg{}\\
	~ ~ {\footnotesize \Stsg:\Sbj:\Pst:\Iter/finish}\\
	\trans `On Thursday, he was teaching us until midday and then school always ended (for the week).' \Corpus{tci20120904-02}{MAB \#14}
	\label{ex279}
\end{exe}

These two \isi{prefix series} are derived from the \Bet{} series by adding an element to it. For \Betaone{}, this is the vowel /u/ and, for \Betatwo{}, it is the consonant /ɸ/. The only exceptions are the first \isi{person} singular and the second \isi{person} singular formatives (see \tabref{perspref2}). In a different analysis, the /u/ and /ɸ/ elements could be described as separate morphemes. Like the prefixes, these two morphemes would then have to receive an abstract label. Such an analysis would reduce the number of \isi{prefix series} to three. Under the current analysis, there are three main series and two subseries. I retain the current analysis, but I do not see either as being more elegant or more parsimonious than the other. More important is the question regarding the difference between \Betaone{} and \Betatwo{}, which, for the moment, is unsettled. I will briefly discuss two possible explanations.

First, the difference might be understood in terms of sociolinguistic variation, i.e. the use of either variant is determined by an individual's linguistic biography. Although all Komnzo speakers are multilingual, the strongest influence comes from two close varieties, namely \ili{Wära} and \ili{Anta}. In my preliminary survey of the surrounding varieties, I found that \Betaone{} and \Betatwo{} exist in \ili{Wära} as well as \ili{Anta}. My impressionistic view is that the \Betatwo{} \isi{prefix series} occurs much more frequently than \Betaone. However, comparative work and documentation on both varieties is needed.

A second explanation is a true difference in meaning. Although \Betaone{} and \Betatwo{} are almost always interchangeable without a clear change in meaning, there are some hints that semantics may play a role. For example, the \isi{copula} can only take \Betatwo{} and not \Betaone{}, and the same is true for the verb \emph{yak} `walk' (\ref{ex280}). Only when the \isi{copula} is used in an ambifixing template, both \Betaone{} and \Betatwo{} are possible. However, in an ambifixing template the copula cannot be translated as `be', but instead functions as a \isi{light verb} with the meaning `do'. For other verbs, \Betaone{} and \Betatwo{} are interchangeable. This observation leads me to believe that the \Betatwo{} prefixes encode either a longer duration of the event or a greater degree of affectedness of the participants. However, targeted elicitation and close observation of natural texts did not lead to a clear pattern along these lines. Informants found it hard to give a characterisation or translation of the difference and often contradicted each other or themselves. For now I will leave this question open for future research.

The \Gam{} prefixes are used for the perfectives: the \isi{recent past} \isi{perfective} (\ref{ex281}) and the \isi{past} \isi{perfective} (\ref{ex282}). Example (\ref{ex281}) comes from a spontaneous conversation in the yam garden when a friend happened to pass by on his bicycle. Example (\ref{ex282}) describes a dance that took place in the nearby settlement of Forzitho.

\begin{exe}
	\ex \emph{watik, zä zf \textbf{zamse} bä \textbf{nznäthor}.}\\
	\glll watik zä zf z-a-ms-e bä nzn-ä-thor\\
	then {\Prox} {\Imm} \M.\Gam-\Vc.\Du-sit.\Rs-{\Fnsg} \Ssg{} \Ssg.\Gam-\Ndu-arrive.\Rs\\
	~ ~ ~ {\footnotesize \Fdu:\Sbj:\Rpst:\Pfv/sit} ~ {\footnotesize \Ssg:\Sbj:\Rpst:\Pfv/arrive}\\
	\trans `Then, we two sat down and you arrived.' \Corpus{tci20130823-06}{CAM \#31}
	\label{ex281}
\end{exe}
\begin{exe}
	\ex \emph{wati, mane änyaka forzitho wath \textbf{sathaifath}.}\\
	\glll wati mane e-a-n-yak-a forzitho wath\\
	then which \Stnsg.\Alph-\Vc-\Venit-walk.\Ext.\Ndu-\Pst{} forzitho dance\\
	~ ~ {\footnotesize \Stpl:\Sbj:\Pst:\Ipfv:\Venit/walk} ~ ~\\
	\sn
	\glll s-a-thayf-a-th\\
	 \Tsg.\Masc.\Gam-\Ndu-bring.out.\Rs-\Pst-\Stnsg\\
	 {\footnotesize \Stpl:\Sbj>\Tsg.\Masc:\Obj:\Pst:\Pfv/bring.out}\\
	\trans `Well, those who came to Forzitho brought the dance out (to the village square).' \Corpus{tci20120909-06}{KAB \#25}
	\label{ex282}
\end{exe}

\subsection{The irrealis prefix \emph{ra-}} \label{irrealisra}

The \isi{irrealis} prefix \emph{ra-} is used for the \isi{imperfective}, \isi{perfective} and \isi{durative} \isi{irrealis} inflections. We have seen examples of all three \isi{aspect} values in (\ref{ex286}) and (\ref{ex287}). Example (\ref{ex286}) showed that the only place in the paradigm where the \isi{irrealis} prefix \emph{ra-} is not used is the second \isi{person} singular of an ambifixing \isi{verb}.

The interaction of the \isi{irrealis} prefix with the \isi{valency} changing prefix \emph{a-} and pre-stem dual marking is explained in {\S}\ref{prerootdual}. In that section, I pointed out that the \isi{irrealis} prefix \emph{ra-} overrides the \isi{valency} changing prefix \emph{a-} to the effect that the absence versus presence of the \isi{valency} changing prefix is neutralised. For verb forms which employ the extended stem, this \isi{neutralisation} is complete. For verb forms which employ the restricted stem, there are small changes in the pre-stem duality marking pattern ({\S}\ref{prerootdual}).  In these cases, only the \isi{case} frame indicates whether the \isi{undergoer} argument is a direct \isi{object}, such as the absolutive \isi{case} on \emph{szsi} `calling' in (\ref{ex288}), or an indirect object, such as the dative \isi{case} on \emph{ŋatha} in (\ref{ex289}). Both examples are taken from the same hunting story in which the narrator talks about his usual routines when going on a hunting expedition.

\begin{exe}
	\ex \emph{ŋathar foba \textbf{szsi} \textbf{threthkäfé}}\\
	\glll ŋatha=r foba sz-si th-rä-thkäf-é\\
	dog={\Purp} \Dist.{\Abl} call.out-{\Nmlz} \Stnsg.\Bet-\Irr.\Ndu-start.\Rs-\Fsg\\
	~ ~ ~ {\footnotesize \Fsg:\Sbj>\Stpl:\Obj:\Irr:\Pfv/start}\\
	\trans `From there, I started calling out for the dogs.' \Corpus{tci20111119-03}{ABB \#63}
	\label{ex288}
\end{exe}
\begin{exe}
	\ex \emph{watik wamnza \textbf{ŋathanm} biskar mni \textbf{threthkäfé}}\\
	\glll watik wo-a-m-nz-a ŋatha=nm biskar mni\\
	then \Fsg.\Alph-\Vc-sit.\Ext-\Ndu-\Pst{} dog=\Dat.{\Nsg} cassawa fire\\
	~ {\footnotesize \Fsg:\Sbj:\Pst:\Ipfv/sit} ~ ~ ~\\
	\sn
	\glll th-rä-thkäf-é\\
	\Stnsg.\Bet-\Irr.\Ndu-start.\Rs-\Fsg\\
	{\footnotesize \Fsg:\Sbj>\Stpl:\Obj:\Irr:\Pfv/start}\\
	\trans `Then I sat and started to cook the cassava for the dogs.' \Corpus{tci20111119-03}{ABB \#73}
	\label{ex289}
\end{exe}

\subsection{The past suffix \emph{-a}} \label{pastsuffixa}

The position of the \isi{past} suffix \emph{-a} within the suffixing subsystem is described in {\S}\ref{personsuffsection}. The \isi{past} suffix \emph{-a} is employed for two TAM categories: the \isi{past} \isi{imperfective} (\ref{ex290}) and the \isi{past} \isi{perfective} (\ref{ex291}). Example (\ref{ex290}) is taken from a text on oral history of the Morehead district. The narrator talks about conficts caused by an alleged sorcerer in the 1940s. The second example (\ref{ex291}) comes from a much more recent event, where aa woman is talking about camping at the Morehead river and going fishing only a week before the recording was made.

\begin{exe}
	\ex \emph{watik gathagatha zokwasi fä \textbf{ykonath}.}\\
	\glll watik {gathagatha} zokwasi fä y-ko-n-a-th\\
	then {bad} words {\Dist} \Tsg.\Masc.\Alph-speak.\Ext-\Du-\Stnsg{}\\
	~ ~ ~ ~ {\footnotesize \Stdu:\Sbj>\Tsg.\Masc:\Obj:\Pst:\Ipfv/speak}\\
	\trans `Then, they cursed him there.' \Corpus{tci20131013-02}{ABB \#102}
	\label{ex290}
\end{exe}
\begin{exe}
	\ex \emph{\textbf{zukorath} ``mama, bä bana ketharuf! zuzi käzir!''}\\
	\glll zu-\Zero{}-kor-a-th mama bä bana k-ä-tharuf-\Zero{}\\
	\Fsg.\Gam-\Du-speak.\Rs-\Pst-\Stnsg{} mother \Ssg{} poor \M.\Bet-\Vc.\Ndu-enter.\Rs-\Ssg.{\Imp}\\
	{\footnotesize \Stdu:\Sbj>\Fsg:\Obj:\Pst:\Pfv/speak} ~ ~ ~ {\footnotesize \Ssg:\Sbj:\Imp:\Pfv/enter}\\
	\sn
	\glll zuzi k-ä-zir-\Zero\\
	fishing.line \M.\Bet-\Vc.\Ndu-throw.\Rs-\Ssg.{\Imp}\\
	~ {\footnotesize \Ssg:\Sbj:\Imp:\Pfv/throw}\\
	\trans `They said to me: ``Mama, get on (the canoe) and throw the fishing line!'''\\ \Corpus{tci20120922-25}{ALK \#7-8}
	\label{ex291}
\end{exe}

\subsection{The durative suffix \emph{-m}}\label{durativesuffixm}

The \isi{durative} suffix \emph{-m} is described in {\S}\ref{personsuffsection} with regard to its position in the suffixing subsystem. It is employed for \isi{durative} \isi{aspect}, which expresses an ongoing event in the \isi{immediate past}\footnote{The immediate past occurs with a low frequency in the text corpus and, consequently, there is only a handful of examples in the immediate past durative.\footnotesize Example (\ref{ex274}) on page \pageref{ex274} is one of these.}, \isi{recent past} (\ref{ex293}), \isi{past} (\ref{ex292}) and \isi{irrealis} (\ref{ex294}). In example (\ref{ex293}), the speaker reports on how he fought a bushfire in his garden the preceding day. Example (\ref{ex292}) is taken from a story about rain-making magic which the narrator acquired and practiced in his youth. The \isi{irrealis} example (\ref{ex294}) is taken from a conversation about local customs surrounding the sister-exchange system.

\begin{exe}
	\ex \emph{wthzak zane \textbf{ŋanrsirwrmth}.}\\
	\glll wthzak zane ŋ-a-n-rsir-wr-m-th\\
	sole \Dem:{\Prox} \M.\Alph-\Vc-\Venit-burn.\Ext-\Ndu-\Dur-\Stnsg\\
	~ ~ {\footnotesize \Stpl:\Sbj:\Rpst:\Dur:\Venit/burn}\\
	\trans `The soles of my feet here were burning.' \Corpus{tci20120922-24}{MAA \#63}
	\label{ex293}
\end{exe}
\begin{exe}
	\ex \emph{grigri zä \textbf{kwasogwrmth}.}\\
	\glll grigri zä kw-a-sog-wr-m-th\\
	maggot {\Prox} \M.\Betatwo-\Vc-ascend.\Ext-\Ndu-\Dur-\Stnsg\\
	~ ~ {\footnotesize \Stpl:\Sbj:\Pst:\Dur/ascend}\\
	\trans `The maggots were climbing up here.' \Corpus{tci20110810-01}{MAB \#71}
	\label{ex292}
\end{exe}
\begin{exe}
	\ex \emph{fäms fthé \textbf{krakwinmth} ... fäms fämsnzo ...}\\
	\glll fäms fthé k-ra-kwi-n-m-th (.) fäms\\
	exchange.man when \M.\Bet-\Irr.\Vc-argue.\Ext-\Du-\Dur-\Stnsg{} (.) exchange.man\\
	~ ~ {\footnotesize \Stdu:\Sbj:\Irr:\Ipfv/argue} ~\\
	\sn
	\gll fäms=nzo (.)\\
	exchange.man={\Only} (.)\\
	\trans `When exchange men are fighting ... exchange man (against) exchange man ...' \Corpus{tci20120805-01}{ABB \#460}
	\label{ex294}
\end{exe}

Part of the function of the \isi{durative} suffix is to shift back the \isi{tense}. If we remove the \emph{-m} suffix from a \isi{verb} inflected for the \isi{recent past} \isi{durative} (\ref{ex293}) or \isi{past} \isi{durative} (\ref{ex292}), the resulting form would be a \isi{non-past} \isi{imperfective} and \isi{recent past} \isi{imperfective}, respectively. Figure \ref{backshiftdur} shows this with the verb \emph{songsi} from example (\ref{ex292}).

\begin{figure}[h]
	\begin{tabularx}{.75\textwidth}{|l|l|l|}
		\cline{1-1} \cline{3-3}
		\Nonpast:\Ipfv &\multirow{2}{*}{$\rightarrow$}& \Nonpast:\Ipfv\emph{-m} = \Rpst:\Dur\\
		\emph{ŋasogwr} `S/he climbs.' && \emph{ŋasogwrm} `S/he was climbing.'\\
		\cline{1-1} \cline{3-3}
		\multicolumn{3}{c}{}\\
		\cline{1-1} \cline{3-3}
		\Rpst:\Ipfv &\multirow{2}{*}{$\rightarrow$}& \Rpst:\Ipfv\emph{-m} = \Pst:\Dur\\
		\emph{kwasogwr} `S/he climbed.' && \emph{kwasogwrm} `S/he had been climbing.'\\
		\cline{1-1} \cline{3-3}
	\end{tabularx}
\caption{The backshifting function of the durative suffix \emph{-m}}
\label{backshiftdur}
\end{figure}%The \isi{backshifting} function of the \isi{durative} suffix \emph{-m}

The \isi{durative} suffix can also attach to an \isi{iterative} inflection, in which case the iteration of the event is streched over a longer duration, as in (\ref{ex295}) and (\ref{ex296}). In (\ref{ex295}), the speaker talks about the first fire which destroyed the world inhabited by humans. In (\ref{ex296}), the speaker describes how the people used to avoid a particular place during the early and late hours of the day because it was inhabited by a story man.

\begin{exe}
	\ex \emph{zfth mni nä kayé \textbf{zwäsmth} kidn.}\\
	\glll zfth mni nä kayé zu-ä-s-m-th kidn\\
	base fire some yesterday \Tsg.\F.\Betaone-\Ndu-call.\Rs-\Dur-\Stnsg{} kidn\\
	~ ~ ~ ~ {\footnotesize \Stpl:\Sbj>\Tsg:\Obj:\Pst:\Iter:\Dur/call} ~\\
	\trans `They always used to call the eternal fire Kidn.' \Corpus{tci20120909-06}{KAB \#55}
	\label{ex295}
\end{exe}
\begin{exe}
	\ex \emph{kwamonegwrmth e efoth fthé zbo warfo \textbf{kwänkorm} fthé kwarafinzrmth zä zerä.}\\
	\glll kw-a-moneg-wr-m-th e efoth fthé zbo warfo\\
	\M.\Betaone-\Vc-wait.\Ext-\Ndu-\Dur-\Stnsg{} until sun when \Prox.{\All} above\\
	{\footnotesize \Stpl:\Sbj:\Pst:\Dur/wait} ~ ~ ~ ~ ~\\
	\sn
	\glll kw-ä-n-kor-m-\Zero{} fthé kw-a-rafi-nzr-m-th\\
	\M.\Betaone-\Vc.\Ndu-\Venit-become.\Rs-\Dur-\Stsg{} when \M.\Betaone-\Vc-paddle.\Ext-\Ndu-\Dur-\Stnsg{}\\
	{\footnotesize \Stsg:\Sbj:\Pst:\Iter:\Dur:\Venit/become} ~ {\footnotesize \Stpl:\Sbj:\Pst:\Dur/paddle}\\
	\sn
	\glll zä z=e-rä\\
	{\Prox} \Prox=\Stnsg.\Alph-\Cop.{\Ndu}\\
	~ {\footnotesize \Prox=\Stpl:\Sbj:\Nonpast:\Ipfv/be}\\
	\trans `They were waiting until the sun always reached highest point and then they paddled here.' \Corpus{tci20120922-19}{DAK \#13}
	\label{ex296}
\end{exe}

The \isi{durative} suffix \emph{-m} can be suffixed to perfective verbs in the \isi{recent past}, \isi{past} and \isi{irrealis}. In this case, the event is only \isi{backgrounded} without encoding a longer duration. However, these inflections are so rare that, at least for the \isi{recent past} and \isi{past} tenses, they are not attested in the corpus. For the \isi{irrealis} \isi{perfective} with the \isi{durative} suffix, there are a handful of examples. In (\ref{ex297}), the speaker talks about an old procedure for punishment which involved striking the culprit with a yam tuber over the head.\footnote{I will show the backgrounded status of the perfective verb in the unified gloss line with \Bg{}, as in the examples below. In the maximally segmented gloss line, I will continue to use the durative label \Dur{}.}

\begin{exe}
	\ex \emph{nasime \textbf{sräkwrmth} ebaren ``ah, miyatha käkor bä monwä zbrigwé!''}\\
	\glll nasi=me s-rä-kwr-m-th ebar=en ah miyatha \\
	long.yam={\Ins} \Tsg.\Masc.\Bet-\Irr.\Ndu-hit.\Rs-\Dur-\Stnsg{} head={\Loc} ah knowledge\\
	~ {\footnotesize \Stpl:\Sbj>\Tsg.\Masc:\Obj:\Irr:\Pfv:\Bg/hit} ~ ~ ~\\
	\sn
	\glll k-ä-kor-\Zero{} bä mon=wä z-brig-w-é\\
	\M.\Bet-\Ndu-become.\Rs-\Ssg.{\Imp} \Second.{\Abs} how={\Emph} \Tsg.\F.\Bet-return.\Ext-\Ndu-\Ssg.\Imp\\
	{\footnotesize \Ssg:\Sbj:\Imp:\Pfv/become} ~ ~ {\footnotesize \Ssg:\Sbj>\Tsg.\F:\Obj:\Imp:\Ipfv/return}\\
	\trans `They would hit him on the head with the long yam (and say) ``Now you come up with a plan to pay this back!''' \Corpus{tci20120805-01}{ABB \#236-240}
	\label{ex297}
\end{exe}

Irrespective of perfectivity, the \isi{durative} suffix on any \isi{irrealis} inflection can have a far \isi{future} interpretation. In examples (\ref{ex298}) and (\ref{ex299}), it is clear from the context that the event is set in the \isi{future} and the \emph{-m} on the verb indicates that the event is further in the \isi{future} (as opposed to an \isi{irrealis} form without the \emph{-m} suffix). In (\ref{ex298}), the speaker showed me an old method of tying a bowstring. He then speculates whether and when these old practices will vanish. Example (\ref{ex299}) is taken from a conversation about yam cultivation during which the speaker complains about young people's lack of interest in gardening.

\begin{exe}
	\ex \emph{ni miyamr mä kwa kräbth mane ... mrnen \textbf{kräbthmo} frthé}\\
	\glll ni miyamr mä kwa k-rä-bth-\Zero{} mane (.) mrn-en\\
	{\Fnsg} ignorance where {\Fut} \M.\Bet-\Irr.\Vc.\Ndu-finish.\Rs-\Stsg{} which (.) clan-{\Loc}\\
	~ ~ ~ ~ {\footnotesize \Stsg:\Sbj:\Irr:\Pfv/finish} ~ ~ ~\\
	\sn
	\glll k-rä-bth-m-o-\Zero{} frthé\\
	\M.\Bet-\Irr.\Vc.\Ndu-finish.\Rs-\Dur-\Andat-{\Sg} when\\
	{\footnotesize \Sg:\Sbj:\Irr:\Pfv:\Bg:\Andat/finish} ~\\
	\trans `We do not know where it will finish ... in which generation it will finish.'\\ \Corpus{tci20130914-01}{KAB \#43-44}
	\label{ex298}
\end{exe}
\begin{exe}
	\ex \emph{nzä miyamr thrma ra \textbf{sranathrmth} ... nagayé}\\
	\glll nzä miyamr thrma ra s-ra-na-thr-m-th\\
	\Fsg.{\Abs} ignorance later what \Tsg.\Masc.\Bet-\Irr-eat.\Ext-\Ndu-\Dur-\Stnsg{}\\
	~ ~ ~ ~ {\footnotesize \Stpl:\Sbj>\Tsg.\Masc:\Obj:\Irr:\Ipfv:\Bg/eat}\\
	\sn
	\gll (.) nagayé\\
	(.) children\\
	\trans `I do not know what the children will eat later.' \Corpus{tci20120805-01}{ABB \#577}
	\label{ex299}
\end{exe}

If the \isi{durative} suffix is attached to a \isi{verb} in the \isi{imperative} mood, it encodes a delayed or \isi{future} \isi{imperative} (`do X a little later!').\footnote{I gloss the future imperative with \Futimp{} in the unified gloss line.} The \isi{future} \isi{imperative} is also a rare inflection, and we have seen one text example in (\ref{ex274}) on page \pageref{ex274}. In example (\ref{ex300}), the speaker describes how competitive yam harvesting took place in the old days. After harvesting and sorting, a piece of rattan was used to measure the size of the largest tubers. This measurement was then sent to the competitors as a sign of one's superior gardening skills.

\begin{exe}
 	\ex \emph{wati, ŋatr \textbf{thärifthm} nafanmedbo!}\\
 	\glll wati ŋatr th-ä-rifth-m-\Zero{} nafanme=dbo\\
 	then rattan \Stnsg.\Bet-\Ndu-send.\Rs-\Dur-\Ssg.{\Imp} \Tnsg=\All.{\Sg}\\
	~ ~ {\footnotesize \Ssg:\Sbj>\Stpl:\Obj:\Futimp:\Pfv/send} ~\\
 	\trans `Then, you send the measure string to them!' \Corpus{tci20120805-01}{ABB \#402}
 	\label{ex300}
\end{exe}

\subsection{The imperative suffixes}\label{imperativesuffix}

The formatives of the \isi{imperative} actor suffix series were given in \tabref{perssuffimp} on page \pageref{perssuffimp}, where I pointed out the syncretism with the first \isi{person} indicative actor suffixes and the second \isi{person} \isi{imperative} suffixes, as well as the fact that the second singular suffix differs between \isi{perfective} and \isi{imperfective} imperatives. I refer the reader to section {\S}\ref{personsuffsection} for further information.

Here I describe the morphology of imperatives for the prefixing template. Prefixing verbs as defined here encode their single \isi{participant} in the prefix. We saw in \tabref{perspref2} on page \pageref{perspref2} that imperatives are formed with the \Bet{} \isi{prefix series}. For prefixing verbs, the formatives are \emph{gn-} (\Ssg.\Imp) and \emph{th-} (\Snsg.\Imp). A further suffix is added to prefixing verbs only. Consider example (\ref{ex301}) in which the speaker quotes himself talking to his wife. The \isi{imperative} inflected \isi{verb} is marked with an \emph{-é} suffix which resembles the actor suffix of an ambifixing \isi{imperfective} \isi{imperative} (\Ssg.\Imp) or of an ambifixing indicative of any \isi{aspect} class (\Fsg). In the morphological context of prefixing imperatives, this \emph{-é} does not encode a \isi{person} value, as can be seen in example (\ref{ex302}) where the number of the addressee argument is \isi{plural}. In other words, the \emph{-é} suffix looks like a \isi{person}/\isi{number} suffix, but with prefixing verbs it is inert to those categories and it only encodes \isi{imperative} mood.

\begin{exe}
	\ex \emph{bä znrä. zä \textbf{gnamnzé} kwot e nzä kränbrimé!}\\
	\glll bä z=n-rä zä gn-a-m-nz-é kwot e nzä\\
	\Second.{\Abs} \Prox=\Ssg.\Alph-\Cop.{\Ndu} {\Prox} \Ssg.\Bet-\Vc-sit.\Ext-\Ndu-{\Imp} properly until \Fsg.{\Abs}\\
	~ {\footnotesize \Prox=\Ssg:\Sbj:\Nonpast:\Ipfv/be} ~ {\footnotesize \Ssg:\Sbj:\Imp:\Ipfv/sit} ~ ~ ~\\
	\sn
	\glll k-rä-n-brim-é\\
	\M.\Bet-\Irr.\Vc.\Ndu-\Venit:return.\Rs-\Fsg{}\\
	{\footnotesize \Fsg:\Sbj:\Irr:\Pfv:\Venit/return}\\
	\trans `Now you are here. You stay here until I return.' \Corpus{tci20130823-06}{STK \#221}
	\label{ex301}
\end{exe}
\begin{exe}
	\ex \emph{... zbär fiyafr mane eyak famäsü \textbf{thyaké}!}\\
	\glll (.) zbär fiyaf=r mane e-yak fam=ä=sü\\
	(.) night hunting={\Purp} who \Stnsg.\Alph-walk.\Ext.{\Ndu} thought=\Assoc=\Etc{}\\
	~ ~ ~ ~ {\footnotesize \Stpl:\Sbj:\Nonpast:\Ipfv/walk} ~\\
	\sn
	\glll th-yak-é\\
	\Stnsg.\Bet-walk.\Ext.\Ndu-\Imp\\
	{\footnotesize \Stpl:\Sbj:\Imp:\Ipfv/walk}\\
	\trans `You (boys) who go hunting at night must be careful!' \Corpus{tci20130901-04}{RNA \#27}
	\label{ex302}
\end{exe}

The \emph{-é} formative for imperatives, regardless of whether it occurs on prefixing or ambifixing verbs, shows the same idiosyncrasies as the first \isi{person} \isi{singular} suffix \emph{-é}, which is described in {\S}\ref{personsuffsection}. For example, it disappears when other suffixes are added, as we saw in example (\ref{ex285}) on page \pageref{ex285}, where the \emph{-é} suffix does not appear because of the andative suffix \emph{-o}.

\section{The TAM particles}\label{tam-particles-sec}

The rich system of TAM categories in Komnzo can be further supplemented by a set of preverbal particles. These include the \isi{future} \emph{kwa}, the \isi{habitual} \emph{nomai}, the \isi{potential} \emph{kma}, the \isi{iamitive} \emph{z}\footnote{I adopt the term \emph{iamitive} from Olsson (\citeyear{Olsson:2013vn}), who has coined the term based on Latin \emph{iam} `already'.}, the \isi{apprehensive} or prohibitive \emph{m} and the \isi{imminent} \emph{n}. The latter two are related to the \isi{deictic} \isi{proclitic} \emph{m=} and the \isi{immediate past} \emph{n=}. These particles interact with the numerous TAM categories and there are only few limitations on the possible combinations.

\subsection{The imminent particle \emph{n}}\label{imminentm}

The \isi{imminent} \isi{particle} \emph{n} expresses the point in time just before the event takes place, usually without implying that it actually happened. This often gets translated by informants as `try to do X' or `be about to do X'. Both interpretations, the intentional and the imminent one, are possible and difficult to separate. In example (\ref{ex307}), the speaker showed me how to weave a fish basket. He says that he will try and fetch me when the work is finished because he does not know whether or not it will be successful.\footnote{Indeed, he never came and showed me the finished fish basket because I had already left the village. But he proudly presented it to me in the following year.}

\begin{exe}
	\ex \emph{\textbf{n} thrma \textbf{nzänmesé} ... fthé zräbthé zane kafar.}\\
	\glll n thrma nz-ä-n-mes-é (.) fthé z-rä-bth-é \\
	{\Imn} later \Ssg.\Bet-\Ndu-\Venit-fetch.\Rs-\Fsg{} (.) when \Tsg.\F.\Bet-\Irr.\Ndu-finish.\Rs-\Fsg{}\\
	~ ~ {\footnotesize \Fsg:\Sbj>\Ssg:\Obj:\Irr:\Pfv:\Venit/fetch} ~ ~ {\footnotesize \Fsg:\Sbj>\Tsg.\F:\Obj:\Irr:\Pfv/finish}\\
	\sn
	\gll zane kafar\\
	\Dem:{\Prox} big\\
	\trans `Later I will try and fetch you, when I have finished that big (basket).'\\ \Corpus{tci20120906}{SKK \#18}
	\label{ex307}
\end{exe}

The \isi{imminent} \isi{particle} can occur with inflections of different TAM categories. The important part of its semantic contribution is twofold: (i) the point in time before the event and (ii) the fact that the action has not yet been carried out or \textendash{} in most cases \textendash{} is not or was not carried out. Example (\ref{ex308}) is taken from a headhunting story in which two men are about to kill a young woman when they realise that the rest of their headhunting party has left already.\footnote{The word \emph{ngemäku} in the example is an address term between two people where one has adopted the child of the other.}

\begin{exe}
	\ex \emph{\textbf{n} \textbf{zfrnmth} di kam garsir ``awkwot! ngemäku, kabe matak erä!"}\\
	\glll n zf-r-n-m-th di kam gar-si=r\\
	{\Imn} \Tsg.\F.\Betatwo-do.\Ext-\Du-\Dur-\Stnsg{} back.of.head bone break-\Nmlz={\Purp}\\
	~ {\footnotesize \Stdu:\Sbj>\Tsg.\F:\Obj:\Pst:\Dur/do} ~ ~ ~\\
	\sn
	\glll awkwot ngemäku kabe matak e-rä\\
	interjection foster.parent man nothing \Stnsg.\Alph-\Cop.\Ndu\\
	~ ~ ~ ~ {\footnotesize \Stpl:\Sbj:\Nonpast:\Ipfv/be}\\
	\trans `They were about to break her neck. (He said:) ``Oh no, my friend, all the people have left!''' \Corpus{tci20111119-01}{ABB \#151-152}
	\label{ex308}
\end{exe}

There is an overlap in the semantics of the \isi{proclitic} \emph{n=} which encodes \isi{immediate past} and the \isi{imminent} \isi{particle} \emph{n}. I pointed out earlier that the \isi{immediate past} \isi{clitic} attaches to a verb which is otherwise inflected for \isi{non-past}. Thus, it marks a point in time immediately before the present. The \isi{particle} \emph{n} occurs in front of verb forms of different TAM categories, marking a point in time immediately before the event. The semantic difference is in the implication as to whether or not the event was actually carried out. In the case of the \isi{immediate} \isi{clitic}, the event has happened, but with the \isi{particle} \emph{n} this is not the case. The difference between the two also lies in formal criteria. The \isi{particle} \emph{n} is syntactically independent in that it occurs free (\ref{ex307}), or occur directly in front of the verb, where it is hard to say whether it is a \isi{proclitic} or an independent element (\ref{ex308}). On the other hand, the \isi{immediate} \isi{clitic} \emph{n=} is always bound to the verb.

Speakers of Komnzo who have been brought up in a \ili{Wära}-speaking family, and most young speakers of all backgrounds, have replaced the \isi{immediate past} \isi{proclitic} \emph{n=} with its \ili{Wära} equivalent \emph{nz=}. This change only affects the \isi{proclitic} and not the \isi{imminent} \isi{particle} \emph{n}.

\subsection{The apprehensive particle \emph{m}}\label{apprehensivem}

I point out in {\S}\ref{deicticcliticssection} that among the \isi{deictic} proclitics there is one with a limited distribution. The \emph{m=} \isi{proclitic} can only attach to the copula, in which case it turns the clause into a \isi{question} (`where is X?').\footnote{I will gloss \emph{m} as interrogative (where=) when it attaches to the copula. I will gloss it as apprehensive (\Appr) in all other cases, including the cases where \emph{m} and the potential particle \emph{kma} express a prohibitive.} See example (\ref{ex271}) on page \pageref{ex271}. The \emph{m} \isi{particle} shows more syntactic flexibility as it can procliticise to the \isi{verb} as \emph{m=}, encliticise to the \isi{potential} \isi{particle} in the combination \emph{kma=m} or occur by itself. The latter is only attested through elicitation and there are no corpus examples of independent \emph{m}. Nevertheless, it can be classified as a \isi{particle} and a \isi{clitic}.

The \isi{particle} \emph{m} functions as an \isi{apprehensive} marker. It is attested in the corpus with irrealis, imperative and perfective forms. Example (\ref{ex303}) is from a story about a man who mocked a crowd of dancers by threatening them with a matchbox. They were afraid, as they did not know about matches and lighters.

\begin{exe}
	\ex \emph{krenafthth ``sritüthe! sfafe! kidn mni \textbf{mzärfusir} ... frthe bramöwä ŋarsirwre.''}\\
	\glll k-rä-nafth-th s-\Zero{}-ritüth-e\\
	\M.\Bet-\Irr.\Vc.\Ndu-say.\Rs-\Stnsg{} \Tsg.\Masc.\Bet-\Du-grab.\Rs-\Stnsg{}.{\Imp}\\
	{\footnotesize \Stpl:\Sbj:\Irr:\Pfv/say} {\footnotesize \Sdu:\Sbj>\Tsg.\Masc:\Obj:\Imp:\Pfv/grab}\\
	\sn
	\glll s-\Zero{}-faf-e kidn mni m=z-ä-rfusir-\Zero{}\\
	\Tsg.\Masc.\Bet-\Du-hold.\Rs-\Stnsg{}.{\Imp} kidn fire \Appr=\M.\Gam-\Vc.\Ndu-light.up.{\Rs}-\Stsg{}\\
	{\footnotesize \Sdu:\Sbj>\Tsg.\Masc:\Obj:\Imp:\Pfv/hold} ~ ~ {\footnotesize \Appr=\Stsg:\Sbj:\Rpst:\Pfv/light.up}\\
	\sn
	\glll (.) frthe bramöwä ŋ-a-rsir-wr-e\\
	(.) when all \M.\Alph-\Vc-burn.\Ext-\Ndu-{\Fnsg}\\
	 ~ ~ ~ {\footnotesize \Fpl:\Sbj:\Nonpast:\Ipfv/burn}\\
	\trans `They said: ``Grab him! Hold him! He might ignite the Kidn fire. (That is) when we will all burn.''' \Corpus{tci20120909-06}{KAB \#82}
	\label{ex303}
\end{exe}

In these cases, the \isi{particle} \emph{m} seems to override the TAM value of the verb. In (\ref{ex303}), the verb is in the \isi{recent past} but lacks a the\isi{recent past} reading. Likewise, I often heard the warning \emph{mkätr}\footnote{\parbox{0.02cm}{\hfill}\parbox{6cm}{\emph{mkätr}}\\ \parbox{0.05cm}{\hfill}\parbox{6cm}{m=k-ä-tr-\Zero}\\ \parbox{0.05cm}{\hfill}\parbox{6cm}{\Appr=\M.\Bet-\Vc.\Ndu-fall.\Rs-\Ssg.{\Imp}}} `(watch out) you might fall!', where \emph{m} is attached to an \isi{imperative} form, but lacks an \isi{imperative} reading. Naturally, if \emph{m} occurs with an irrealis form, there is no such conflict. Example (\ref{ex304}) is taken from a story about a bushfire. The speaker explains how he set a small controlled fire in order to stop the wild bushfire from spreading.

\begin{exe}
	\ex \emph{we ane nzefé zaföfé ... we \textbf{mkrärit} we fafä.}\\
	\glll we ane nzefé z-a-föf-é (.) we\\
	also {\Dem} \Fsg.\Erg.{\Emph} \Tsg.\F.\Gam-\Vc.\Ndu-burn.down.\Rs-\Fsg{} (.) also\\
	~ ~ ~ {\footnotesize \Fsg:\Sbj>\Tsg.\F:\Obj:\Rpst:\Pfv/burn.down} ~ ~\\
	\sn
	\glll m=k-rä-rit-\Zero{} we fafä\\
	\Appr=\M.\Bet-\Irr.\Vc.\Ndu-pass.\Rs-\Stsg{} also after.that\\
	{\footnotesize \Appr=\Stsg:\Sbj:\Irr:\Pfv/pass} ~ ~\\
	\trans `I also burned down this (grass) ... (the fire) might cross over later.'\\ \Corpus{tci20120922-24}{MAA \#30-31}
	\label{ex304}
\end{exe}

If \emph{m} occurs with an \isi{imperative} inflected verb and the \isi{potential} \emph{kma}, it functions as a prohibitive. Example (\ref{ex305}) is from the very beginning of a hunting story. The speaker tells his son to be quiet during the recording, while I am setting up the microphone.

\begin{exe}
	\ex \emph{zokwasi wzänzr ... daddyf. \textbf{kmam} \textbf{kanafré}!}\\
	\glll zokwasi w-zä-nzr-\Zero{} (.) daddy=f kma=m\\
	words \Tsg.\F.\Alph-carry.\Ext-\Ndu-\Stsg{} (.) father=\Erg.{\Sg} \Pot={\Appr}\\
	~ {\footnotesize \Stsg:\Sbj>\Tsg.\F:\Obj:\Nonpast:\Ipfv/carry} ~ ~ ~\\
	\sn
	\glll k-a-naf-r-é\\
	\M.\Bet-\Vc-speak.\Ext-\Ndu-\Ssg.\Imp\\
	{\footnotesize \Ssg:\Sbj:\Imp:\Ipfv/speak}\\
	\trans `Daddy is recording the words. You must not talk!' \Corpus{tci20130903-03}{MKW \#3-4}
	\label{ex305}
\end{exe}

In the prohibitive construction, the \isi{particle} \emph{m} is rather flexible. It can attach to the verb as a \isi{proclitic} (\ref{ex306}) or to the \isi{potential} \isi{particle} \emph{kma} as an \isi{enclitic} (\ref{ex305} and \ref{ex383}). What is important for the prohibitive reading is the co-occurence of \emph{m} and \emph{kma} in the clause, not the fact that they are conjoined. Example (\ref{ex306})\footnote{The verb \emph{yak} ‘walk’ is deponent and employs the valency change prefix \emph{a-} without a change in the valency of the verb. It is only deponent when it employs the venitive marker, meaning ‘come’, not when it is neutral or andative, meaning ‘walk’, ‘go away’.} comes from a public speech at a dance in which the speaker tells the audience the rules for the night. Example (\ref{ex383}) is taken from a text about food taboos.

\begin{exe}
	\ex \emph{\textbf{kma} wärir bä \textbf{mgnanyaké} zena zbär zbo!}\\
	\glll kma wäri=r bä m=gn-a-n-yak-é zena zbär zbo\\
	{\Pot} sex={\Purp} \Second.{\Abs} \Appr=\Ssg.\Bet-\Vc-\Venit-walk.\Ext.\Ndu-{\Imp} today night \Prox.\All\\
	~ ~ ~ {\footnotesize \Appr=\Ssg:\Sbj:\Imp:\Ipfv/come} ~ ~ ~\\
	\trans `You must not come here for sex tonight!' \Corpus{tci20121019-04}{ABB \#46}
	\label{ex306}
\end{exe}
\begin{exe}
	\ex \emph{be \textbf{kmam} ŋazikarä \textbf{kathafrakwé}!}\\
	\gll be kma=m ŋazi=karä k-a-thafrak-w-é\\
	\Ssg.{\Erg} \Pot={\Appr} coconut={\Prop} \M.\Bet-\Vc-mix.\Ext-\Ndu-\Ssg.{\Imp}\\
	~ ~ ~ {\footnotesize \Ssg:\Sbj:\Imp:\Ext/mix}\\
	\trans `You must not mix it with coconut' \Corpus{tci20120922-26}{DAK \#12}
	\label{ex383}
\end{exe}

\subsection{The potential particle \emph{kma}}\label{potentialkma}

The \isi{potential} \isi{particle} \emph{kma} can be employed with almost all TAM categories. We saw in {\S}\ref{apprehensivem} that it encodes a prohibitive when it occurs together with imperatives and the \isi{apprehensive} \isi{particle} \emph{m}. This is the only construction in which \emph{kma} and the \isi{imperative} inflections occur together.

The \isi{potential} \isi{particle} \emph{kma} is used to encode various types of speculation and counterfactuality with deontic or epistemic interpretation. Example (\ref{ex331}) is taken from a public speech at a dance, where the guest side has brought too many people, and consequently the host side found it impossible to meet the needs of so many people. The speaker regrets that no proper arrangement has been made prior to the event. Thus, the clause ``it should have been well'' has a clear deontic reading.

\begin{exe}
 	\ex \emph{namä \textbf{kma} nimame zrarenzrm fof ... fthé namä yamme nüfifthakwrme.}\\
 	\glll namä kma nima=me z-ra-re-nzr-m fof (.) fthé namä\\
 	good {\Pot} like.this={\Ins} \Tsg.\F.\Bet-\Irr.\Vc-look.\Ext-\Ndu-\Dur{} {\Emph} (.) when good\\
	~ ~ ~ {\footnotesize \Tsg.\F:\Sbj:\Irr:\Ipfv/look} ~ ~ ~ ~\\
	\sn
	\glll yam=me n=w-fifthak-wr-m-e\\
	custom={\Ins} \Immpst=\Tsg.\F.\Alph-put.down.straight.\Ext-\Ndu-\Dur-{\Fnsg}\\
	~ {\footnotesize \Immpst=\Fpl:\Sbj>\Tsg.\F:\Obj:\Nonpast:\Dur/put.down.straight}\\
	\trans `It should have been well today, if we had straightened things out in a good way.' \Corpus{tci20121019-04}{ABB \#79}
	\label{ex331}
\end{exe}

Example (\ref{ex334}) is taken from an origin myth in which the speaker speculates that one of the protagonists ``must have had a shotgun'', while his brother only had bow and arrow. This is a clear epistemic use of \emph{kma}.

\begin{exe}
	\ex \emph{nafangth \textbf{kma} markai nabikarä sfrärm.}\\
	\glll nafa-ngth kma markai nabi=karä sf-rär-m\\
	\Third.{\Poss}-younger.sibling {\Pot} outsider bow={\Prop} \Tsg.\Masc.\Betatwo-\Cop.\Ndu-\Dur\\
	~ ~ ~ ~ {\footnotesize \Tsg.\Masc:\Sbj:\Pst:\Dur/be}\\
	\trans `His younger brother must have had a shotgun.' \Corpus{tci20131013-01}{ABB \#112}
	\label{ex334}
\end{exe}

\subsection{The future particle \emph{kwa}}\label{futurekwa}

Future tense is marked periphrastically in Komnzo with the \isi{particle} \emph{kwa}, which combines either with the \isi{non-past} (\ref{ex335}) or irrealis inflections (\ref{ex309}).

\begin{exe}
 	\ex \emph{zena \textbf{kwa} \textbf{natrikwé} bun ... no kzima.}\\
 	\glll zena kwa n-a-trik-w-é bun (.) no kzi=ma\\
 	today {\Fut} \Ssg.\Alph-\Vc-tell.\Ext-\Ndu-\Fsg{} \Ssg.{\Dat} (.) rain barktray={\Char}\\
	~ ~ {\footnotesize \Fsg:\Sbj>\Ssg:\Io:\Nonpast:\Ipfv/tell} ~ ~ ~\\
 	\trans `Today, I will tell you about the rain-making barktray.' \Corpus{tci20110810-01}{MAB \#1}
 	\label{ex335}
\end{exe}
\begin{exe}
 	\ex \emph{gb \textbf{kwa} \textbf{thrarfikwr} zba.}\\
 	\glll gb kwa th-ra-rfik-wr zba\\
 	sprout {\Fut} \Stnsg.\Bet-\Irr-grow.\Ext-{\Ndu} \Prox.\Abl\\
	~ ~ {\footnotesize \Stpl:\Sbj:\Irr:\Ipfv/grow} ~\\
 	\trans `The sprouts will grow from here.' \Corpus{tci20120805-01}{ABB \#35}
 	\label{ex309}
\end{exe}

The \isi{future} \isi{particle} can also be used by itself meaning `wait', as in example (\ref{ex310}), where the name of a particular plant has slipped from the speaker's mind.

\begin{exe}
	\ex \emph{\textbf{kwa}! yf kwot keke miyatha worä.}\\
	\glll kwa yf kwot keke miyatha wo-rä\\
	wait name properly {\Neg} knowledge \Fsg.\Alph-\Cop.\Ndu\\
	~ ~ ~ ~ ~ {\footnotesize \Fsg:\Sbj:\Nonpast:\Ipfv/be}\\
	\trans `Wait! I don't quite know that name.' \Corpus{tci20130907-02}{RNA \#609}
	\label{ex310}
\end{exe}

When negated, the \isi{future} \isi{particle} \emph{kwa} can express `not yet', as in example (\ref{ex367}), where the speaker points out that he has not yet heard the name that will be given to a particular person at an upcoming namesake celebration.

\begin{exe}
	\ex \emph{ni miyamr mane zrarä ane kar yf fof. \textbf{keke kwa} kar yf nä zamare fof.}\\
	\glll ni miyamr mane z-ra-rä ane kar yf fof keke kwa\\
	{\Fnsg} ignorance which \Tsg.\F.\Bet-\Irr-\Cop.{\Ndu} {\Dem} village name {\Emph} {\Neg} {\Fut}\\
	~ ~ ~ {\footnotesize \Tsg.\F:\Sbj:\Irr:\Ipfv/be} ~ ~ ~ ~ ~ ~\\
	\sn
	\glll kar yf nä z-a-mar-e fof\\
	village name some \Tsg.\F.\Gam-\Ndu-see-{\Fnsg} {\Emph}\\
	~ ~ ~ {\footnotesize \Fpl:\Sbj>\Tsg.\F:\Obj:\Rpst:\Pfv/see} ~\\
	\trans `We do not know which local name it will be. We have not heard the name yet.' \Corpus{tci20110817-02}{ABB \#58-60}
	\label{ex367}
\end{exe}

Younger speakers of Komnzo are beginning to use the \ili{Wära} equivalent \emph{ka}, which which has a pure velar rather than labiovelar onset.

\subsection{The iamitive particle \emph{z}}\label{iamitivez}

I adopt the term ``\isi{iamitive}'' from Olsson's (\citeyear{Olsson:2013vn}) comparative study of particles that express a perfect. Reesink (\citeyear[184]{Reesink:2009bird}) uses the term ``perspectival \isi{aspect}'', which he adopts from Dik (\citeyear{Dik:1997uj}). Komnzo speakers often translate the \isi{iamitive} \isi{particle} \emph{z} as `already', hence the \isi{gloss} label {\Iam}. An introductory example is given in (\ref{ex311}). This is taken from a recording where two women took me on a plant walk. Example (\ref{ex313}) is the answer to the question in (\ref{ex312}).

\begin{exe}
\ex \label{ex311}
\begin{xlist}
	\ex \emph{zuyak \textbf{z} safäs?}\\
	\glll zuyak z s-a-fäs-\Zero{}\\
	zuyak {\Iam} \Tsg.\Masc.\Gam-\Ndu-show.\Rs-\Stsg{}\\
	~ ~ {\footnotesize \Stsg:\Sbj>\Tsg.\Masc:\Obj:\Nonpast:\Pfv/show}\\
	\trans `Have you shown him zuyak (Rhodania sp) already?'\\ \Corpus{tci20130907-02}{JAA \#44}
	\label{ex312}
	\ex \emph{\textbf{z} fof!}\\
	\gll z fof\\
	{\Iam} {\Emph}\\
	\trans `Yes, (I have) already.' \Corpus{tci20130907-02}{RNA \#121}
	\label{ex313}
\end{xlist}
\end{exe}

Example (\ref{ex311}) shows that the function of the \isi{iamitive} is to express ``current relevance'' of some past event. Consequently, the \isi{particle} may combine with verbs inflected for different TAM categories. Example (\ref{ex311}) shows a verb in \isi{recent past} \isi{perfective}. In (\ref{ex332}), the \isi{iamitive} \isi{particle} is used with a \isi{past} \isi{durative} inflected verb. This combination is rarer, but well-attested in the corpus. In the example, the speaker is explaining which clans settled at which locations. He points out that his clan had already been living in Masu for a while.

\begin{exe}
	\ex \emph{fi fobo thwamnzrm nima ... ni masun \textbf{z} nzwamnzrm.}\\
	\glll fi fobo thu-a-m-nzr-m nima (.) ni masu=n z\\
	\Third.{\Abs} \Dist.{\All} \Stnsg.\Betaone-\Vc-sit.\Ext-\Ndu-\Dur{} like.this (.) {\Fnsg} masu={\Loc} {\Iam}\\
	~ ~ {\footnotesize \Stpl:\Sbj:\Pst:\Dur/sit} ~ ~ ~ ~ ~\\
	\sn
	\glll nzu-a-m-nzr-m\\
	\Fnsg.\Betaone-\Vc-sit.\Ext-\Ndu-\Dur{}\\
	{\footnotesize \Fpl:\Sbj:\Pst:\Dur/sit}\\
	\trans `They lived over there, this way ... and we had already been living in Masu.'\\ \Corpus{tci20120922-08}{DAK \#97-98}
	\label{ex332}
\end{exe}

The \isi{iamitive} \isi{particle} can also be used with a \isi{non-past} inflection. This is often restricted to interrogatives, as in (\ref{ex314}), where the speaker is asking a crowd of people whether they can hear him speaking.

\begin{exe}
	\ex \emph{zbär bä zagrwä ämnzro. \textbf{z} wanrizrth?}\\
	\glll zbär bä zagr=wä e-a-m-nzr-o z\\
	night \Med{} far={\Emph} \Stnsg.\Alph-\Vc-sit.\Ext-\Ndu-\Andat{} {\Iam}\\
	~ ~ ~ {\footnotesize \Stpl:\Sbj:\Nonpast:\Ipfv:\Andat/sit} ~\\
	\sn
	\glll w-a-n-riz-r-th\\
	\Fsg.\Alph-\Vc-\Venit-hear.\Ext-\Ndu-\Stnsg{}\\
	{\footnotesize \Stpl:\Sbj>\Fsg:\Io:\Nonpast:\Ipfv:\Venit/hear}\\
	\trans `Tonight you are sitting too far away. Can you hear me?' \Corpus{tci20121019-04}{SKK \#9}
	\label{ex314}
\end{exe}

The \isi{iamitive} \isi{particle} additionally expresses the completion of an event. Evidence for this come from different observations. First, it can express a current relevance meaning. Secondly, it never combines with \isi{iterative} verbs, which express an ongoing repetition of some event in the past. Thirdly, it marks sequentiality of events in some narratives where the verb form which combines with it seems to be almost a prerequisite to the following verb. Example (\ref{ex333})\footnote{The verb \emph{yak} ‘walk’ is deponent and employs the valency changing prefix \emph{a-} without a change in the valency of the verb. Note that this occurs only with the venitive marker, in which case the verb means ‘come’, not when it is neutral (‘walk’) or marked with the andative (‘go away’).} is a description of a path, which the speaker had taken the previous day. He describes the sequenced stages of his path to the location called Tümgo.

\begin{exe}
	\ex \emph{bä komnzo zwäzik ... ksi karen \textbf{z} kwanyak e zbo zwänthor tümgon.}\\
	\glll bä komnzo zu-ä-zik (.) ksi kar=en z\\
	\Med{} only \Fsg.\Gam-\Ndu-turn.off.{\Rs} (.) bush place={\Loc} {\Iam}\\
	~ ~ {\footnotesize \Fsg:\Sbj:\Rpst:\Pfv/turn.off} ~ ~ ~ ~\\
	\sn
	\glll ku-a-n-yak e zbo zu-ä-n-thor tümgo=n\\
	\Fsg.\Betaone-\Vc-walk.\Ext.{\Ndu} until \Prox.{\All} \Fsg.\Gam-\Ndu-\Venit-arrive.{\Rs} tümgo=\Loc\\
	{\footnotesize \Fsg:\Sbj:\Rpst:\Ipfv:\Venit/walk} ~ ~ {\footnotesize \Fsg:\Sbj:\Rpst:\Pfv:\Venit/arrive} ~\\
	\trans `It turned off (the path) there ... I walked in the bushy place until I arrived here in Tümgo.' \Corpus{tci20120922-24}{MAA \#8-10}
	\label{ex333}
\end{exe}

The \isi{iamitive} \isi{particle} \emph{z} in Komnzo shares a number of semantics characteristics with the forms described by Olsson (\citeyear{Olsson:2013vn}) in his comparative study. The main two characteristics of iamitives cross-linguistically are ``the notion of a “new situation” that holds after a transition'' and ``the consequences that this situation has at reference time for the participants in the speech event'' (\citeyear[43]{Olsson:2013vn}). The former was described above as event completion, and the latter as current relevance. In fact, the \isi{iamitive} \isi{particle} is the main way to express event completion in Komnzo, because the \isi{perfective} \isi{aspect} does not explicitly set this boundary on an event.

There has been much discussion in the literature about the paths of grammaticalisation of perfects, for example in Bybee \& Dahl (\citeyear{Bybee:1989hk}). In Komnzo, the \isi{iamitive} \isi{particle} \emph{z} is formally closest to the \isi{proximal} series of the deictic markers, and one might speculate about these as a source of grammaticalisation (\S\ref{demonstratives-sec}).

\subsection{The habitual particle \emph{nomai}}\label{habitualnomai}

The \isi{habitual} \isi{particle} \emph{nomai} typically combines with \isi{durative} inflections. In example (\ref{ex315}), the cockatoo always warns the protagonist of another man who comes and visits him.

\begin{exe}
	\ex \emph{krara ymd suwägrm maf swatrikwrm \textbf{nomai} nima ``oh, kabe yanyak.''}\\
	\glll krara ymd su-wägr-m maf\\
	cockatoo bird \Tsg.\Masc.\Betaone-be.on.top.\Ndu-\Dur{} who.{\Erg}\\
	~ ~ {\footnotesize \Tsg.\Masc:\Sbj:\Pst:\Dur/be.on.top} ~\\
	\sn
	\glll su-a-trik-wr-m-\Zero{} nomai nima oh kabe\\
	\Tsg.\Masc.\Betaone-\Vc-tell.\Ext-\Ndu-\Dur-\Stsg{} \Hab{} {\Quot} oh man\\
	{\footnotesize \Stsg:\Sbj>\Tsg.\Masc:\Io:\Pst:\Dur/tell} ~ ~ ~ ~\\
	\sn
	\glll y-a-n-yak\\
	\Tsg.\Masc.\Alph-\Venit-walk.\Ext.{\Ndu}\\
	{\footnotesize \Tsg.\Masc:\Sbj:\Nonpast:\Ipfv:\Venit/walk}\\
	\trans `The cockatoo bird used to sit on top (of the tree), and told him always: ``Oh, a man is coming.''' \Corpus{tci20100802}{ABB \#80-82}
	\label{ex315}
\end{exe}

The \isi{habitual} can also combine with verb forms inflected for other TAM categories, such as imperfectives (\ref{ex316}). It only occasionally occurs with perfectives, as in (\ref{ex317}), where the event is negated. In both examples, \emph{nomai} expresses an extended period of time rather that a repeated habit.

\begin{exe}
	\ex \emph{yamnza yamnza ... \textbf{nomai} ... ysokwr tüfr.}\\
	\glll 2x[y-a-m-nz-a] (.) nomai (.) ysokwr tüfr\\
	2x[\Tsg.\Alph-\Vc-sit.\Ext-\Ndu-\Pst] (.) \Hab{} (.) year plenty\\
	{\footnotesize 2x[\Tsg.\Masc:\Sbj:\Pst:\Ipfv/sit]} ~ ~ ~ ~ ~\\
	\trans `He stayed and stayed there for many years.' \Corpus{tci20120904-01}{MAB \#13}
	\label{ex316}
\end{exe}
\begin{exe}
	\ex \emph{keke \textbf{nomai} zämsath.}\\
	\glll keke nomai z-ä-ms-a-th\\
	{\Neg} \Hab{} \M.\Gam-\Vc.\Ndu-sit.\Ext-\Pst-\Stnsg{}\\
	~ ~ {\footnotesize \Stpl:\Sbj:\Pst:\Pfv/sit}\\
	\trans `They did not stay (there) for long.' \Corpus{tci20131013-02}{ABB \#87}
	\label{ex317}
\end{exe}

\section{Some remarks on the semantics of TAM}\label{TAMsemantics}

Following from our description of the morphology and \isi{combinatorics} of TAM in Komnzo, I want to sketch out a coherent picture of the semantics of these categories and their extended uses. Although \isi{tense}, \isi{aspect} and \isi{mood} are intertwined, I will discuss them separately in the following sections.

\subsection{Tense}\label{TAMsemtense}

We saw that Komnzo has three or four morphological tenses depending on the analysis: the \isi{non-past}, the \isi{recent past} and the \isi{past}. The \isi{immediate past} is expressed by a clitic and builds on a verb form inflected for the \isi{non-past}. Future reference is expressed periphrastically with the \isi{particle} \emph{kwa}.

The \isi{temporal} reference of the \isi{immediate past} and the \isi{recent past} overlaps. The \isi{immediate past} is used for events that took place a short while prior to the time of speaking, and it may be used to put extra emphasis on that fact. The \isi{recent past} covers the same period of time, but it reaches further back, usually to the preceding day. Example (\ref{ex374}) is taken from a hunting story, at the end of which the speaker returns home to find one of his dogs. He tells his wife that this is the dog which had disturbed him at the outset of the trip when he was about to cross the Morehead river. He had pushed the dog into the water, whereupon the poor dog ran back to the house. The whole episode in (\ref{ex374}) is set in the same time frame with respect to the moment of speech. Only the `pushing in the water' is expressed in the \isi{immediate past}, while the other two verb forms are in the \isi{recent past}.\footnote{The speaker uses the \emph{nz=} formative of the immediate past clitic. As pointed out in {\S}\ref{imminentm}, this formative is a borrowing from \ili{Wära}. The Komnzo formative is \emph{n=}.}

\begin{exe}
	\ex \emph{nzefe nima ``ane ŋatha bä \textbf{nzwathofikwr} ... watik anema \textbf{nzibrüzé} bobo ... watik ane wtrime fi ŋatha \textbf{zanmath}.''}\\
	\glll nzefe nima ane ŋatha bä nzu-a-thofik-wr-\Zero{} (.) watik\\
	\Fsg.\Erg.{\Emph} {\Quot} {\Dem} dog \Med{} \Fsg.\Betaone-\Vc-disturb.\Ext-\Ndu-\Stsg{} (.) then\\ 
	~ ~ ~ ~ ~ {\footnotesize \Stsg:\Sbj>\Fsg:\Obj:\Rpst:\Ipfv/disturb} ~ ~\\
	\sn
	\glll ane=ma nz=y-brüz-\Zero{}-é bobo (.) watik ane wtri=me\\
	\Dem={\Char} \Immpst=\Tsg.\Masc.\Alph-submerge.\Ext-\Ndu-\Fsg{} \Med.{\All} (.) then {\Dem} fear={\Ins}\\
	~ {\footnotesize \Immpst=\Fsg:\Sbj>\Stsg.\Masc:\Obj:\Nonpast:\Ipfv/submerge} ~ ~ ~ ~ ~\\
	\sn
	\glll fi ŋatha z-a-n-math-\Zero\\
	\Third.{\Abs} dog \M.\Gam-\Venit-run.{\Rs}-\Stsg\\
	~ ~ {\footnotesize \Stsg:\Sbj:\Rpst:\Pfv:\Venit/run}\\
	\trans `I said: ``That dog disturbed me there and therefore I pushed him into the water. Well, full of fear he ran back here.''' \Corpus{tci20130903-03}{MKW \#188}
	\label{ex374}
\end{exe}

The bidirectional time adverbials discussed in {\S}\ref{temporals-sec} help to identify the appropriate time frames for each \isi{tense} value. The term \emph{kayé} expresses a moment in time which is removed by one day from the present time. Thus, \emph{kayé} can mean `tomorrow', when used with a non-past inflection, or it can mean `yesterday', when used with a \isi{recent past}. Events further back in time have to be expressed by the \isi{past} \isi{tense}. Likewise, one cannot use a \isi{recent past} with the time adverbial \emph{nama}, which indicates a point in time that is removed two days from the present time (`day before yesterday' or `day after tomorrow'). In short, the \isi{recent past} reaches back one day, whereas the \isi{past} \isi{tense} covers everything before yesterday, irrespective of whether it happened a week ago or in ancestral time. Example (\ref{ex375}) shows the use of \emph{kayé} and the \isi{recent past}. Example (\ref{ex376}) shows the use of \emph{nama} and the \isi{past} \isi{tense}.\footnote{\emph{Nama} can also be used metaphorically to mean `recently'.}

\begin{exe}
  	\ex \emph{\textbf{kayé} nzä boba \textbf{zenfaré} ... kanathr.}\\
  	\glll kayé nzä boba z-ä-n-far-é (.) kanathr\\
	yesterday \Fsg.{\Abs} \Med.{\Abl} \M.\Gam-\Vc.\Ndu-\Venit-set.off.\Ext-\Fsg{} (.) kanathr\\
 	~ ~ ~ {\footnotesize \Fsg:\Sbj:\Rpst:\Pfv:\Venit/set.off} ~ ~\\
  	\trans `Yesterday, I set off from there towards here ... to Kanathr.'\Corpus{tci20120922-24}{MAA \#1}
  	\label{ex375}
\end{exe}
\begin{exe}
	\ex \emph{zane nane dayr zbo \textbf{nama} mane \textbf{wänyaka} ...}\\
	\gll zane nane dayr zbo nama mane\\
	\Dem:{\Prox} elder.sibling dayr \Prox.{\All} two.days.ago which\\
	\sn
	\glll w-a-n-yak-a (.)\\
	\Tsg.\F.\Alph-\Vc-\Venit-go.\Ext.\Ndu-\Pst{} (.)\\
	{\footnotesize \Tsg.\F:\Sbj:\Pst:\Ipfv:\Venit/go} ~\\
	\trans `The older sister Dayr who came here two days ago ...'\Corpus{tci20130901-04}{RNA \#87}
	\label{ex376}
\end{exe}

Tense values can be used with a pragmatic motivation. It is quite common to foreground events in a narrative by putting them into the \isi{non-past}, even though the story is set in the \isi{recent past} or the \isi{past}. Example (\ref{ex668}) comes from a story that took place in the speaker's youth. In the example clauses, he describes walking with a friend during night time. The two boys rested along the way and smoked tobacco. Although the story is set in the past, only the first and the last verbs in (\ref{ex668}) are inflected in the past \isi{tense} (`walk' in both cases). The `sitting down' and the `setting off' are inflected for irrealis, and are thus tenseless. The rolling of the cigarettes and their smoking is told in the \isi{non-past}, which moves this part to the foreground.

\begin{exe}
	\ex \emph{nyana ttfö bä rä ... bäne ... sazäthi fä kramse sukufa \textbf{eknne} \textbf{änane} boba krafare ... zbär nzfyanm.}\\
	\glll n-yan-a ttfö bä rä (.) bäne (.) sazäthi\\
	\Fnsg.\Alph-walk.\Ext.\Du-\Pst{} creek \Med{} \Tsg.\F.\Cop.{\Ndu} (.) \Recog.{\Abs} (.) sazäthi\\
	{\footnotesize \Fdu:\Sbj:\Pst:\Ipfv/walk} ~ ~ {\footnotesize \Tsg.\F:\Sbj:\Nonpast:\Ipfv/be} ~ ~ ~ ~\\
	\sn
	\glll fä k-ra-ms-e sukufa e-kn-n-e\\
	{\Dist} \M.\Bet-\Irr.\Vc.\Du-sit.\Rs-{\Fnsg} tobacco \Stnsg.\Alph-roll.\Ext-\Du-{\Fnsg}\\
	~ {\footnotesize \Fdu:\Sbj:\Irr:\Pfv/sit} ~ {\footnotesize \Fdu:\Sbj>\Stpl:\Obj:\Nonpast:\Ipfv/roll}\\
	\sn
	\glll e-a-na-n-e boba k-ra-far-e (.) zbär\\ 
	\Stnsg.\Alph-\Vc-eat.\Ext-\Du-{\Fnsg} \Med.{\Abl} \M.\Alph-\Irr.\Vc.\Du-set.off.\Rs-{\Fnsg} (.) night\\
	{\footnotesize \Fdu:\Sbj>\Stpl:\Obj:\Nonpast:\Ipfv/eat} ~ {\footnotesize \Fdu:\Sbj:\Irr:\Pfv/set.off} ~ ~\\
	\sn
	\glll nzf-yan-m\\
	\Fnsg.\Betatwo-walk.\Ext.\Du-\Dur{}\\
	{\footnotesize \Fdu:\Sbj:\Pst:\Dur/walk}\\
	\trans `We walked. There is a creek there (called) Sazäthi. We sat down there, rolled the cigarettes and smoked. We set off from there. We were walking in the night.' \Corpus{tci20210904-01}{MAB \#140-143}
	\label{ex668}
\end{exe}

Future reference can be expressed by \isi{irrealis} or \isi{non-past} inflections combined with the \isi{future} \isi{particle} \emph{kwa}. The main difference between the two strategies seems to lie in the anticipated degree of certainty: the \isi{non-past} inflection is usually used when the speaker is more certain that the event is going to take place.

\subsection{Aspect}\label{TAMsemaspect}

I have labelled the principal aspectual distinction in Komnzo \isi{imperfective} versus \isi{perfective}. Durative \isi{aspect} is understood as a subtype of the \isi{imperfective} and we could label these two as `basic \isi{imperfective}' and `\isi{durative} \isi{imperfective}'. I use the traditional labels \isi{imperfective} and \isi{perfective}, but I want to spell out the particular flavour that Komnzo gives to them.

Traditional accounts of perfectivity often take the completion of an event as a starting point (\citealt[296]{Frawley:1992wi}) or suggests that ``perfectivity indicates the view of a situation as a single whole'' (\citealt[16]{Comrie:1976vd}). In Komnzo, completion does not really play a role in the semantics of the perfective-\isi{imperfective} distinction. The boundary set up by the \isi{perfective} seems to concentrate more on the left edge, i.e. on the beginning of the event. Similar systems are found elsewhere in the Southern New Guinea region, for example in \ili{Marind} (\citealt[41]{Drabbe:1955tm}), \ili{Nama} (\citealt{Siegel:2015bp}) and \ili{Nen} (\citealt{Evans:2015wy}). In Komnzo, the main mechanism for expressing event completion, i.e. to set up a right edge event boundary, is the \isi{iamitive} \isi{particle}, which can occur with verb forms in \isi{perfective}, \isi{imperfective} and \isi{durative} \isi{aspect} ({\S}\ref{iamitivez}). It follows that imperfectivity does not entail that the event is open-ended. Example (\ref{ex370}) is taken from a head hunting story. The \isi{quantifier} \emph{bramöwä} `all' signals that the attack was full-scale and all inhabitants were killed, but the verb form in (\ref{ex370}) is in the \isi{imperfective}.

\begin{exe}
	\ex \emph{watik ebar kabe ane fof thäthora fof ... bramöwä ane fof \textbf{efnzath}}\\
	\glll watik ebar kabe ane fof th-ä-thor-a fof (.) bramöwä ane fof e-fn-nz-a-th\\
	then head man {\Dem} {\Emph} \Stnsg.\Gam-\Ndu-arrive.\Rs-\Pst{} {\Emph} (.) all {\Dem} {\Emph} \Stnsg.\Alph-hit.\Ext-\Ndu-\Pst-\Stnsg{}\\
	~ ~ ~ ~ ~ {\footnotesize \Stpl:\Sbj:\Pst:\Pfv/arrive} ~ ~ ~ ~ ~ {\footnotesize \Stpl:\Sbj>\Stpl:\Obj:\Pst:\Ipfv/hit}\\
	\trans `Then, the head hunter arrived. They killed all of them.'\Corpus{tci20131013-02}{ABB \#143-145}
	\label{ex370}
\end{exe}

Likewise, perfectives do not entail that an event is finished, but rather that it has started or that its duration was of a punctual quality. The latter is shown in the first verb `arrive' in example (\ref{ex370}). The former is shown in example (\ref{ex361}), which is taken from a story about a malignant being. At the end of the story this being tries to escape by entering a bird, but the villagers are quick to shoot down the bird. The entering event in (\ref{ex361}) is expressed in the \isi{perfective}, but the \isi{imminent} \isi{particle} \emph{n} shows that the event has not started yet. Hence, completion of the entering event is not entailed, but excluded. Thus, a literal translation of \emph{n zäthba} would be: `s/he was about to start to enter'.

\begin{exe}
	\ex \emph{brbrnzo fof \textbf{n zäthba} bafen ... ymden fof.}\\
	\glll brbr=nzo fof n z-ä-thb-a-\Zero{} baf=en (.) ymd=en fof\\
	spirit={\Only} {\Emph} {\Imn} \M.\Gam-\Ndu-enter.\Rs-\Pst-\Stsg{} \Recog={\Loc} (.) bird={\Loc} {\Emph}\\
	~ ~ ~ {\footnotesize \Stsg:\Sbj:\Pst:\Pfv/enter} ~ ~ ~ ~\\
	\trans `Only the spirit was about to go inside that one ... inside the bird.'\\ \Corpus{tci20120901-01}{MAK \#193-194}
	\label{ex361}
\end{exe}

Aspect in Komnzo seems to concentrate more on a punctual/inceptive versus ongoing/{\linebreak}stretched-out distinction. I adopt the traditional labels \isi{perfective} for the former and \isi{imperfective} for the latter. The degree to which an event is ``stretched out'' would then decide whether the speaker chooses the \isi{imperfective} or \isi{durative} \isi{aspect}. The basic binary distinction is clearest in the \isi{imperative} forms. The \isi{imperfective} imperatives often encode an ongoing action and, depending on the context, they can be translated as ``keep on X-ing'' or ``do X for some time''. Perfective imperatives, on the other hand, express inception ``start X-ing'' or punctuality ``do X once/quickly''. In example (\ref{ex373}), the speaker has just produced a toy bullroarer from a coconut leaf and shows me how to hold it properly. In (\ref{ex372}), she tells me not to hit anything while swinging, and the \isi{imperative} of `hit' is in the \isi{perfective}.\footnote{This is a conditional construction which frequently employs imperative inflections together with \emph{fthé} `when/if' ({\S}\ref{TAMsemmood} and \S\ref{condiclauses}).} In (\ref{ex371}), she is already swinging the bullroarer, telling me to hold it away from the body. Consequently, all the \isi{imperative} verb forms (`hold', `blow', and `swing') are in the \isi{imperfective}.

\begin{exe}
\ex
\label{ex373}
\begin{xlist}
	\ex
	\emph{fthé \textbf{sakwr} gwonyamen o festhen o wämnen ... keke kwa sranor}.\\
	\glll fthé s-a-kwr-\Zero{} gwonyame=n o festh=en o wämne=n\\
	when \Tsg.\Masc.\Alph-\Ndu-hit.\Rs-\Ssg.{\Imp} clothes={\Loc} or body={\Loc} or tree={\Loc}\\
	~ {\footnotesize \Fsg:\Sbj>\Tsg.\Masc:\Obj:\Imp:\Pfv/hit} ~ ~ ~ ~ ~\\
	\sn
	\glll (.) keke kwa s-ra-nor\\
	(.) {\Neg} {\Fut} \Tsg.\Masc.\Bet-\Irr.\Vc-shout.\Ext\\
	~ ~ ~ {\footnotesize \Tsg.\Masc:\Sbj:\Irr:\Ipfv/shout}\\
	\trans `If you hit it on clothes, body or a tree, it will not make a sound.'
	\label{ex372}
	\ex
	\emph{zagrwä nima \textbf{sfathwé} byé nima \textbf{sfsgwé} ... \textbf{smitwanzé} ... fi kwa yanor.}\\
	\glll zagr=wä nima s-fath-w-é b=\stem{yé} \\
	far={\Emph} like.this \Tsg.\Masc.\Bet-hold.\Ext-\Ndu-\Ssg.{\Imp} \Med=\Tsg.\Masc.\Cop.{\Ndu}\\
	~ ~ {\footnotesize \Ssg:\Sbj>\Tsg.\Masc:\Obj:\Imp:\Ipfv/hold} {\footnotesize \Med=\Tsg.\Masc:\Sbj:\Nonpast:\Ipfv/be}\\
	\sn
	\glll nima s-fsg-w-é (.) s-mitwa-nz-é\\
	like.this \Tsg.\Masc.\Bet-blow.\Ext-\Ndu-\Ssg.{\Imp} (.) \Tsg.\Masc.\Bet-swing.\Ext-\Ndu-\Ssg.{\Imp}\\
	~ {\footnotesize \Ssg:\Sbj>\Tsg.\Masc:\Obj:\Imp:\Ipfv/blow} ~ {\footnotesize \Ssg:\Sbj>\Tsg.\Masc:\Obj:\Imp:\Ipfv/swing}\\
	\sn
	\glll (.) fi kwa y-a-nor\\
	(.) \Third.{\Abs} {\Fut} \Tsg.\Masc.\Alph-\Vc-shout.\Ext.\Ndu\\
	~ ~ ~ \footnotesize {\footnotesize \Tsg.\Masc:\Sbj:\Nonpast:\Ipfv/shout}\\
	\trans `You have to hold it away like this and blow and swing it like this ... (then) it will make a sound.' \Corpus{tci20120914}{RNA \#25-28}
	\label{ex371}
\end{xlist}
\end{exe}

A number of authors have used a scale-based approach to model certain operators which change the structure of predicates (\citealt{Kennedy:2005fy} and \citealt{Kubota:2010bx}). Such an approach is compatible with the TAM system of Komnzo, once we accept that the \isi{imperfective} versus \isi{perfective} distinction highlights different parts of event by manipulating the \isi{temporal} scale. Applied to the Komnzo TAM system, such a model portrays perfectives as a means to (i) set an explicit initial boundary and to (ii) limit the \isi{temporal} scale of the event. (Basic) imperfectives leave this initial boundary implicit, but highlight that the event was carried out for some time \textendash{} a little further along the scale. The \isi{durative} (\isi{imperfective}) increases the \isi{temporal} scale of the event. As shown above, none of these (morphological) aspectual categories sets an explicit boundary at the right edge of the event. The function of event completion is reserved for the \isi{iamitive} \isi{particle}. I will leave the theoretical modelling of the semantics of the Komnzo TAM system for future research.

The theoretical discussion of \isi{aspect} has often focussed on the distinction between viewpoint \isi{aspect} and situation \isi{aspect}.\footnote{See Sasse (\citeyear{Sasse:2002vn}) for a formidable overview of the research on aspect.} Despite all terminological confusion, the former is often called \textsc{aspect}, a term which is employed for ``different ways of viewing the internal constituency of a situation'' (\citealt[3]{Comrie:1976vd}). Situation \isi{aspect}, on the other hand, has often been called \textsc{aktionsart}, and is associated with the internal structure of the event. Thus, situation \isi{aspect} is something objective about the nature of the event, whereas viewpoint \isi{aspect} is subjectively manipulated by the speakers, or as Smith puts it: ``the categories of viewpoint \isi{aspect} are overt, whereas situation \isi{aspect} is expressed in covert categories'' (\citeyear[5]{Smith:1997tq}). We have seen that this does not apply to Komnzo. Aspectual categories, although highly grammaticalised, are based on the situation type rather than on viewpoint, i.e. they are about inception/punctuality, iteration and duration rather than completion. The fact that \isi{aspect} is highly grammaticalised means that the categories are accessible to virtually all verb lexemes. I showed in {\S}\ref{roots-and-temp} that the two stem types ({\Rs} and {\Ext}) are attested for almost all stems. This supports the argument that the notion of an objective internal event structure, which is fed into the inflectional system, plays little role in Komnzo.

As we have seen in the discussion of verbal morphology, a central part of the inflectional system are the two stem types. The labels {\Ext} and {\Rs} of course refer to ``extended in time'' and ``restricted in time'', respectively. All perfectives are built from the {\Rs} stem and all imperfectives are built from the {\Ext} stem. However, a relabelling of the {\Rs} stem as ``\isi{perfective} stem'' and the {\Ext} stem as ``\isi{imperfective} stem'' would be misleading. For example, the {\Rs} stem is employed for \isi{iterative} \isi{aspect}, which is by definition not bounded in time. This contradiction can be resolved by assuming a more transparent contribution of the morphological mechanisms which participate in the \isi{iterative} inflection. As shown in {\S}\ref{combitam} (\tabref{TAMpalooza1}), the \isi{iterative} builds on the {\Rs} stem, but it employs the \Betaone{} or \Betatwo{} \isi{prefix series}, which otherwise only occur with the {\Ext} stem to build imperfectives and duratives. In other words, the \isi{iterative} \isi{aspect} limits the event structure by stem selection and simultaneously spreads out the event structure by the selection of the \isi{prefix series}. This is an interesting scenario, which calls for further comparative research within the \ili{Yam languages} to shed light on the grammaticalisation of \isi{iterative} \isi{aspect}.

\subsection{Mood}\label{TAMsemmood}

There are three modal categories in Komnzo: indicative, \isi{imperative} and \isi{irrealis}. Further nuances can be expressed with the help of particles, especially the \isi{potential} \emph{kma}, the \isi{imminent} \emph{n} and the \isi{apprehensive} \emph{m} ({\S}\ref{tam-particles-sec}). Here, I will only describe some of the ways in which two of the three basic categories \textendash{} the \isi{imperative} and the \isi{irrealis} \textendash{} deviate from their conventional definitions.

Imperatives can be used in a number of ways that fall outside the definition of `giving an order'. In example (\ref{ex318}), the speaker showed me the leaves of a pandanus plant pointing out that I can use the leaves to sleep on. The \isi{imperative} form \emph{gnyaké} `you go' is thus not a command `go without a mat', but more like a conditional `if you go without a mat'. The conditional interpretation also comes from the word \emph{fthé} which means `when' or `at the time when'. This type of conditional construction is an extended use of the \isi{imperative} inflection. Most imperatives are used as commands, and there are conditional constructions without \isi{imperative} inflections.

\begin{exe}
	\ex \emph{yamemäre fthé \textbf{gnyaké} ... etfthar.}\\
	\glll yame=märe fthé gn-yak-é (.) etfth=r\\
	mat={\Priv} when \Stsg.\Bet-walk.\Ext.\Ndu-{\Imp} (.) sleep=\Purp\\
	~ ~ {\footnotesize \Ssg:\Sbj:\Imp:\Ipfv/walk} ~ ~\\
	\trans `If you go without a mat, (this one is) for sleeping.'\\ \Corpus{tci20130907-02}{JAA \#546-547}
	\label{ex318}
\end{exe}

As we have seen in {\S}\ref{irrealisra}, the irrealis is marked by the prefix \emph{ra-}. There is no \isi{realis} marker, hence no realis inflection. Beyond counterfactuality and futurity, the \isi{irrealis} \isi{mood} has a number of semantic extensions in Komnzo. Cross-linguistically \isi{irrealis} \isi{mood} is employed for a wide range of functions, which has led some authors to challenge its validity as a comparative category (\citealt{Bybee:irrealis}). Others have suggested a prototype approach to \isi{irrealis} \isi{mood}, for example Givon (\citeyear[327]{Givon:1994ko}). I will adopt the latter here. Example (\ref{ex381}) and (\ref{ex382}) show the \isi{irrealis} \isi{mood} in its more central functions, counterfactuality and futurity, respectively. Examples (\ref{ex381}) is taken from a headhunting story which involved the speaker's father.\footnote{The example also shows the `relative use' of the immediate past. Although the events in the story happened a long time ago, the speaker uses the immediate past (\emph{niyamnzrm} `He was staying just before') to emphasise that the headhunt took place just after his father had left the village.} Example (\ref{ex382}) is taken from a procedural in which the speaker shows me how to make a toy from a coconut leaf.

\begin{exe}
	\ex \emph{fi fthé niyamnzrm nafäsü kwa \textbf{thräkwrth}.}\\
	\glll fi fthé n=y-a-m-nzr-m nafä=sü kwa\\
	\Third.{\Abs} when \Immpst=\Tsg.\Masc.\Alph-\Vc-sit.\Ext-\Ndu-\Dur{} \Third\Assoc.\Pl=\Etc{} {\Fut}\\
	~ ~ {\footnotesize \Immpst=\Tsg.\Masc:\Sbj:\Nonpast:\Dur/sit} ~ ~\\
	\sn
	\glll th-rä-kwr-th\\
	\Stsg.\Bet-\Irr.\Ndu-hit.\Rs-\Stnsg{}\\
	{\footnotesize \Stpl:\Sbj>\Stpl:\Obj:\Irr:\Pfv/hit}\\
	\trans `If he had stayed, they would have killed him with all the others.'\\ \Corpus{tci20111107-01}{MAK \#80}
	\label{ex381}
\end{exe}
\begin{exe}
	\ex \emph{katan kwa \textbf{sräfiyothé} ... kafar minzü yé.}\\
	\glll katan kwa s-rä-fiyoth-é (.) kafar minzü yé\\
	small {\Fut} \Tsg.\Masc.\Bet-\Vc.\Ndu-make.\Rs-\Fsg{} (.) big very \Tsg.\Masc.\Cop.{\Ndu}\\
	~ ~ {\footnotesize \Fsg:\Sbj>\Tsg:\Obj:\Irr:\Pfv/make} ~ ~ ~ {\footnotesize \Tsg.\Masc:\Sbj:\Nonpast:\Ipfv/be}\\
	\trans `I will make it smaller. This is too big.' \Corpus{tci20120914}{RNA \#41}
	\label{ex382}
\end{exe}

Verbs inflected for irrealis can be used as habituals. This use, especially with past habituals, has been noticed in a cross-linguistic study by Cristofaro (\citeyear{Cristofaro:2004wi}). Example (\ref{ex426}) comes from a procedural about poison-root fishing, which is a common activity during the dry season when the water recedes. The speaker talks about the preparations and the process of poison-root fishing, while his family is busy fishing in the background. All verb forms are in the irrealis \isi{mood}.

\begin{exe}
	\ex \emph{\textbf{thranäbünzrth} ... sam ane mane erä \textbf{threthkäfth} ... \textbf{zranrsrwrth} fof no \textbf{zrerärth} ... \textbf{thranor} ``si rore rore rore!!''}\\
	\glll th-ra-näbü-nzr-th (.) sam ane mane e-rä\\
	\Stnsg.\Bet-\Irr-smash.\Ext-\Ndu-\Stnsg{} (.) liquid {\Dem} which \Stnsg.\Alph-\Cop.{\Ndu}\\
	{\footnotesize \Stpl:\Sbj>\Stpl:\Obj:\Irr:\Ipfv/smash} ~ ~ ~ ~ {\footnotesize \Stpl:\Sbj:\Nonpast:\Ipfv/be}\\
	\sn
	\glll th-rä-thkäf-th (.) z-ra-n-rsr-wr-th\\
	\Stnsg.\Bet-\Irr.\Ndu-start.\Rs-\Stnsg{} (.) \Tsg.\F.\Bet-\Irr-\Venit-squeeze.\Ext-\Ndu-\Stnsg{}\\
	{\footnotesize \Stpl:\Sbj>\Stpl:\Obj:\Irr:\Pfv/start} ~  {\footnotesize \Stpl:\Sbj>\Tsg.\F:\Obj:\Irr:\Ipfv/squeeze}\\
	\sn
	\glll
	fof no z-rä-rä-r-th (.) th-ra-nor\\
	{\Emph} water \Tsg.\F.\Bet-\Irr.\Vc-do.\Ext-\Ndu-\Stnsg{} (.) \Stnsg.\Bet-\Irr-shout.\Ext.{\Ndu}\\
	 ~ ~ {\footnotesize \Stpl:\Sbj>\Tsg.\F:\Io:\Irr:\Ipfv/start} ~ {\footnotesize \Stpl:\Sbj:\Irr:\Ipfv/shout}\\
	\sn
	\gll {si.rore.rore.rore}\\
	{\textsc{interjection}}\\
	\trans `They smash (the sticks). As for the juice that starts coming out, they squeeze it and mix it properly with the water ... and they shout out: ``Si rore rore rore!!''' \Corpus{tci20110813-09}{DAK \#22-23}
	\label{ex426}
\end{exe}

Ir\isi{realis} \isi{mood} is frequently used in narratives which report factual truths. Foley (\citeyear[389]{Foley:2000uh}) points out that Papuan languages often employ the realis-irrealis distinction for pragmatic purposes. In Komnzo, the pragmatic use comes from the alternation between \isi{irrealis} and \isi{realis} inflections especially in event sequencing. In this pattern, the \isi{irrealis} is used for backgrounding. Example (\ref{ex377}) is taken from a hunting story that occured many years ago. The story is told from a first-person perspective, thus, there is no reason to question the factual truth of what is being told. The clauses in (\ref{ex377}) describe a sequence of events: fall asleep > be sleeping > wake up. Only the foregrounded clause (`sleep') is expressed in \isi{realis} (\isi{past} \isi{durative}), whereas the \isi{backgrounded} clauses (`fall asleep' and `wake up') are expressed in \isi{irrealis} (\isi{perfective}). In that sense, the \isi{irrealis} verb forms act as a backgrounding bracket around the foregrounded clause.\footnote{Note that example (\ref{ex668}) on page \pageref{ex668} employs the same bracket-like use of the irrealis inflected verb forms. The only difference is that in (\ref{ex668}), the foregrounded event is in the non-past, whereas in (\ref{ex377}) the foregrounded event is in past durative.}

\begin{exe}
	\ex \emph{\textbf{krämnzeré} efoth etfth \textbf{kwofrugrm} e zizi ... \textbf{krebnafé}.}\\
	\glll k-rä-mnzer-é efoth etfth kwof-rugr-m e\\
	\M.\Bet-\Irr.\Vc.\Ndu-fall.asleep.\Rs-\Fsg{} sun sleep \Fsg.\Betatwo-sleep.\Ext.\Ndu-\Dur{} until\\
	{\footnotesize \Fsg:\Sbj:\Irr:\Pfv/fall.asleep} ~ ~ {\footnotesize \Fsg:\Sbj:\Pst:\Dur/sleep} ~\\
	\sn
	\glll zizi (.) k-rä-bnaf-é\\
	afternoon (.) \M.\Bet-\Irr.\Vc.\Ndu-wake.up.\Rs-\Fsg{}\\
	~ ~ {\footnotesize \Fsg:\Sbj:\Irr:\Pfv/wake.up}\\
	\trans `I fell asleep (for) a daytime nap. I was sleeping until the late afternoon ... and I woke up.' \Corpus{tci20111119-03}{ABB \#31-32}
	\label{ex377}
\end{exe}

The interaction of TAM categories with \isi{information structure} was described by Hopper (\citeyear{Hopper:1979us}). Hartzler describes a similar function of the irrealis \isi{mood} in \ili{Sentani} (\citeyear{Hartzler:1983wm}). I defer the discussion of this topic to {\S}\ref{info-tam-event}, where a detailed analysis is offered, drawing on a longer text segment.